%================================================================================
\section{Shallow Water Equations Test Problem Descriptions}
%================================================================================
%--------------------------------------------------------------------------------
\subsection{1-D Dam Break over Flat Terrain}\label{sec:1d_dam_break_flat}
%--------------------------------------------------------------------------------
This is a 1-D Riemann problem with flat bottom topography, i.e.,
$\bathymetry(x)=0$, taken from \cite{chen2013}.
Typical end times are $\timevalue=0.5$ and $\timevalue=2$.
Table \ref{tab:1d_dam_break_flat} summarizes the test parameters.

\begin{mytable}{1-D Dam Break over Flat Terrain Test Problem Summary}
{1d_dam_break_flat}{l l}
{\emph{Parameter} & \emph{Value}}
Domain              & $\domain = [-5,5]$\\
Initial Conditions  & $(\height,\speed)^T(x,0) = \left\{\begin{array}{l l}
  (3,0)^T & x < 0\\
  (1,0)^T & \text{otherwise}\\
  \end{array}\right.$\\
Boundary Conditions & $(\height,\speed)^T(-5,\timevalue) = (3,0)^T$\\
                    & $(\height,\speed)^T(5,\timevalue) = (1,0)^T$\\
Gravity & $\gravity = 1$\\
Bottom Topography & $\bathymetry(x) = 0$\\
Exact Solution    & Available with exact Riemann solver\\
\end{mytable}
%--------------------------------------------------------------------------------
\subsection{1-D Dam Break over Rectangular Bump}\label{sec:1d_dam_break_bump}
%--------------------------------------------------------------------------------
This is a 1-D dam break problem with a rectangular bump on the
terrain beneath the water surface, taken from \cite{chen2013} and
\cite{vukovic2002}. This bump straddles the dam so that essentially
the initial values for the height $\height$ are defined in 4 regions, so this is
not a classic 2-region Riemann value problem. The bump is nominally in the
region $|x-750|\leq 187.5$, i.e., $x\in(562.5,937.5)$; 
however, the bump profile, i.e., the bathyemetry function $\bathymetry(x)$,
is interpolated to the finite element space.
Typical end times are $\timevalue=15$ and $\timevalue=55$.
Table \ref{tab:1d_dam_break_bump} summarizes the test parameters.

\begin{mytable}{1-D Dam Break over Flat Terrain Test Problem Summary}
{1d_dam_break_bump}{l l}
{\emph{Parameter} & \emph{Value}}
Domain              & $\domain = [0,1500]$\\
Initial Conditions  & $(\height+\bathymetry,\speed)^T(x,0)
  = \left\{\begin{array}{l l}
  (20,0)^T & x < 750\\
  (15,0)^T & \text{otherwise}\\
  \end{array}\right.$\\
Boundary Conditions & $(\height,\speed)^T(-5,\timevalue) = (3,0)^T$\\
                    & $(\height,\speed)^T(5,\timevalue) = (1,0)^T$\\
Gravity & $\gravity = 9.812$\\
Bottom Topography & $\bathymetry(x)
  = \left\{\begin{array}{l l}
  8 & |x-750|\leq 187.5\\
  0 & \text{otherwise}\\
  \end{array}\right.$\\
\end{mytable}
%--------------------------------------------------------------------------------
\subsection{1-D Lake at Rest}\label{sec:1d_lake_at_rest}
%--------------------------------------------------------------------------------
This is a 1-D problem in which a lake sits at rest over a nonzero bottom
topography $\bathymetry(x)$, taken from \cite{fjordholm2011} and \cite{goutal1997}.
The bathymetry function $\bathymetry(x)$ is interpolated to the finite
element space so that the desired constant water level solution
$\approximate{\waterlevel}\equiv\approximate{\height} + \approximate{\bathymetry}$
can be achieved.
A typical end time is $\timevalue=100$.
Table \ref{tab:1d_lake_at_rest} summarizes the test parameters.

\begin{mytable}{1-D Lake at Rest Test Problem Summary}
{1d_lake_at_rest}{l l}
{\emph{Parameter} & \emph{Value}}
Domain              & $\domain = [0,20]$\\
Initial Conditions  & $(\height+\bathymetry,\speed)^T(x,0) = (1,0)^T$\\
Boundary Conditions & Open\\
Gravity & $\gravity = 9.812$\\
Bottom Topography & $\bathymetry(x)
  = \left\{\begin{array}{l l}
  \frac{4-(x-10)^2}{20} & |x - 10| < 2\\
  0                     & \text{otherwise}\\
  \end{array}\right.$\\
Exact Solution    & $(\height+\bathymetry,\speed)^T(x,0) = (1,0)^T$\\
\end{mytable}
%--------------------------------------------------------------------------------
\subsection{1-D Lake at Rest with Perturbation}
  \label{sec:1d_lake_at_rest_perturbed}
%--------------------------------------------------------------------------------
This test problem, taken from \cite{fjordholm2011}, is the same lake-at-rest
problem as in Section \ref{sec:1d_lake_at_rest} but with an initial
perturbation at $x=6$.
Table \ref{tab:1d_lake_at_rest_perturbed} summarizes the test parameters.

\begin{mytable}{1-D Lake at Rest with Perturbation Test Problem Summary}
{1d_lake_at_rest_perturbed}{l l}
{\emph{Parameter} & \emph{Value}}
Domain              & $\domain = [0,20]$\\
Initial Conditions  & $(\height+\bathymetry)(x,0) = \left\{\begin{array}{l l}
                        1.01 & |x-6|<\frac{1}{4}\\
                        1    & \mbox{otherwise}\end{array}\right.$\\
                    & $\speed(x,0) = 0$\\
Boundary Conditions & Open\\
Gravity & $\gravity = 9.812$\\
Bottom Topography & $\bathymetry(x)
  = \left\{\begin{array}{l l}
  \frac{4-(x-10)^2}{20} & |x - 10| < 2\\
  0                     & \text{otherwise}\\
  \end{array}\right.$\\
\end{mytable}
%--------------------------------------------------------------------------------
\subsection{2-D Lake at Rest}\label{sec:2d_lake_at_rest}
%--------------------------------------------------------------------------------
This is a 2-D problem in which a lake sits at rest over a nonzero bottom
topography $\bathymetry(x)$, taken from \cite{fjordholm2011}.
The bathymetry function $\bathymetry(x)$ is interpolated to the finite
element space so that the desired constant water level solution
$\approximate{\waterlevel}\equiv\approximate{\height} + \approximate{\bathymetry}$
can be achieved.
A typical end time is $\timevalue=100$.
Table \ref{tab:2d_lake_at_rest} summarizes the test parameters.

\begin{mytable}{2-D Lake at Rest Test Problem Summary}
{2d_lake_at_rest}{l l}
{\emph{Parameter} & \emph{Value}}
Domain              & $\domain = [0,2]\times[0,1]$\\
Initial Conditions  &
  $(\height+\bathymetry,\velocity)^T(x,y,0) = (1,\mathbf{0})^T$\\
Boundary Conditions & Open\\
Gravity & $\gravity = 9.812$\\
Bottom Topography &
  $\bathymetry(x,y) = 0.8\exp\pr{-5(x-0.9)^2 - 50(y-0.5)^2}$\\
Exact Solution    &
  $(\height+\bathymetry,\velocity)^T(x,0) = (1,\mathbf{0})^T$\\
\end{mytable}
%--------------------------------------------------------------------------------
\subsection{2-D Lake at Rest with Perturbation}
  \label{sec:2d_lake_at_rest_perturbed}
%--------------------------------------------------------------------------------
This test problem, taken from \cite{fjordholm2011}, is the same lake-at-rest
problem as in Section \ref{sec:2d_lake_at_rest} but with an initial
perturbation in the region $x\in[0.1,0.2]$, where the initial height is
increased by 0.01.
Table \ref{tab:2d_lake_at_rest_perturbed} summarizes the test parameters.

\begin{mytable}{2-D Lake at Rest with Perturbation Test Problem Summary}
{2d_lake_at_rest_perturbed}{l l}
{\emph{Parameter} & \emph{Value}}
Domain              & $\domain = [0,2]\times[0,1]$\\
Initial Conditions  & $(\height+\bathymetry)(x,y,0) = \left\{\begin{array}{l l}
                        1.01 & x\in[0.1,0.2]\\
                        1    & \mbox{otherwise}\end{array}\right.$\\
                    & $\velocity(\mathbf{x},0) = \mathbf{0}$\\
Boundary Conditions & Open\\
Gravity & $\gravity = 9.812$\\
Bottom Topography &
  $\bathymetry(x,y) = 0.8\exp\pr{-5(x-0.9)^2 - 50(y-0.5)^2}$\\
\end{mytable}
