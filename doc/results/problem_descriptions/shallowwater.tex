%================================================================================
\section{Shallow Water Equations Test Problem Descriptions}
%================================================================================
%--------------------------------------------------------------------------------
\subsection{1-D Dam Break over Flat Terrain}\label{sec:1d_dam_break_flat}
%--------------------------------------------------------------------------------
This is a 1-D Riemann problem with flat bottom topography, i.e.,
$\bathymetry(x)=0$, taken from \cite{chen2013}.
Typical end times are $\timevalue=0.5$ and $\timevalue=2$.
Table \ref{tab:1d_dam_break_flat} summarizes the test parameters.

\begin{mytable}{1-D Dam Break over Flat Terrain Test Problem Summary}
{1d_dam_break_flat}{l l}
{\emph{Parameter} & \emph{Value}}
Domain              & $\domain = (-5,5)$\\
Initial Conditions  & $(\height,\speed)^T(x,0) = \left\{\begin{array}{l l}
  (3,0)^T & x < 0\\
  (1,0)^T & \text{otherwise}\\
  \end{array}\right.$\\
Boundary Conditions & $(\height,\speed)^T(-5,\timevalue) = (3,0)^T$\\
                    & $(\height,\speed)^T(5,\timevalue) = (1,0)^T$\\
Gravity & $\gravity = 1$\\
Bottom Topography & $\bathymetry(x) = 0$\\
Exact Solution    & Available with exact Riemann solver\\
\end{mytable}
%--------------------------------------------------------------------------------
\subsection{1-D Dam Break over Rectangular Bump}\label{sec:1d_dam_break_bump}
%--------------------------------------------------------------------------------
This is a 1-D dam break problem with a rectangular bump on the
terrain beneath the water surface, taken from \cite{chen2013} and
\cite{vukovic2002}. This bump straddles the dam so that essentially
the initial values for the height $\height$ are defined in 4 regions, so this is
not a classic 2-region Riemann value problem. The bump is nominally in the
region $|x-750|\leq 187.5$, i.e., $x\in(\tilde{x}_{\text{bump,L}},
\tilde{x}_{\text{bump,R}})=(562.5,937.5)$; 
however, the bump edges are chosen to be coincident with the mesh so
that the gradient $\nabla\bathymetry$ need not be evaluated, due to the
discontinuity at the bump's edges. Therefore the actual bump edges
are chosen to be $(x_{\text{bump,L}},x_{\text{bump,R}})$, which
are computed as
\begin{equation}
  x_{\text{bump,L}} = \text{round}\pr{\frac{\tilde{x}_{\text{bump,L}}}
  {\Delta x}}\Delta x
  \eqc \quad
  x_{\text{bump,R}} = \text{round}\pr{\frac{\tilde{x}_{\text{bump,R}}}
  {\Delta x}}\Delta x
  \eqc
\end{equation}
where the $\text{round}()$ function rounds the operand to the nearest integer,
and $\Delta x$ is a uniform mesh size.
Typical end times are $\timevalue=15$ and $\timevalue=55$.
Table \ref{tab:1d_dam_break_bump} summarizes the test parameters.

\begin{mytable}{1-D Dam Break over Flat Terrain Test Problem Summary}
{1d_dam_break_bump}{l l}
{\emph{Parameter} & \emph{Value}}
Domain              & $\domain = (0,1500)$\\
Initial Conditions  & $(\height+\bathymetry,\speed)^T(x,0)
  = \left\{\begin{array}{l l}
  (20,0)^T & x < 750\\
  (15,0)^T & \text{otherwise}\\
  \end{array}\right.$\\
Boundary Conditions & $(\height,\speed)^T(-5,\timevalue) = (3,0)^T$\\
                    & $(\height,\speed)^T(5,\timevalue) = (1,0)^T$\\
Gravity & $\gravity = 9.812$\\
Bottom Topography & $\bathymetry(x)
  = \left\{\begin{array}{l l}
  8 & x_{\text{bump,L}}\leq x\leq x_{\text{bump,R}}\\
  0 & \text{otherwise}\\
  \end{array}\right.$\\
\end{mytable}
%--------------------------------------------------------------------------------
\subsection{1-D Lake at Rest}\label{sec:1d_lake_at_rest}
%--------------------------------------------------------------------------------
This is a 1-D problem in which a lake sits at rest over a nonzero bottom
topography.
$\bathymetry(x)=0$, taken from \cite{fjordholm2011} and \cite{goutal1997}.
A typical end time is $\timevalue=100$.
Table \ref{tab:1d_lake_at_rest} summarizes the test parameters.

\begin{mytable}{1-D Lake at Rest Test Problem Summary}
{1d_lake_at_rest}{l l}
{\emph{Parameter} & \emph{Value}}
Domain              & $\domain = (0,20)$\\
Initial Conditions  & $(\height+\bathymetry,\speed)^T(x,0) = (1,0)^T$\\
Boundary Conditions & $(\height,\speed)^T(0, \timevalue) = (1,0)^T$\\
                    & $(\height,\speed)^T(20,\timevalue) = (1,0)^T$\\
Gravity & $\gravity = 9.812$\\
Bottom Topography & $\bathymetry(x)
  = \left\{\begin{array}{l l}
  \frac{4-(x-10)^2}{20} & |x - 10| < 2\\
  0                     & \text{otherwise}\\
  \end{array}\right.$\\
Exact Solution    & $(\height+\bathymetry,\speed)^T(x,0) = (1,0)^T$\\
\end{mytable}
%--------------------------------------------------------------------------------
\subsection{1-D Lake at Rest with Perturbation}
  \label{sec:1d_lake_at_rest_perturbed}
%--------------------------------------------------------------------------------
This test problem, taken from \cite{fjordholm2011}, is the same lake-at-rest
problem as in Section \ref{sec:1d_lake_at_rest} but with an initial
perturbation at $x=6$.
Table \ref{tab:1d_lake_at_rest_perturbed} summarizes the test parameters.

\begin{mytable}{1-D Lake at Rest with Perturbation Test Problem Summary}
{1d_lake_at_rest_perturbed}{l l}
{\emph{Parameter} & \emph{Value}}
Domain              & $\domain = (0,20)$\\
Initial Conditions  & $(\height+\bathymetry)(x,0) = \left\{\begin{array}{l l}
                        1.01 & |x-6|<\frac{1}{4}\\
                        1    & \mbox{otherwise}\end{array}\right.$\\
                    & $\speed(x,0) = 0$\\
Boundary Conditions & $(\height,\speed)^T(0, \timevalue) = (1,0)^T$\\
                    & $(\height,\speed)^T(20,\timevalue) = (1,0)^T$\\
Gravity & $\gravity = 9.812$\\
Bottom Topography & $\bathymetry(x)
  = \left\{\begin{array}{l l}
  \frac{4-(x-10)^2}{20} & |x - 10| < 2\\
  0                     & \text{otherwise}\\
  \end{array}\right.$\\
\end{mytable}
