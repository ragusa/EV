%%%%%%%%%%%%%%%%%%%%%%%%%%%%%%%%%%%%%%%%%%%%%%%%%%%%%%%%%%%%%%%%%%%%%%%%%%%%%%%%%
\section{Scalar Methodology}
\subsection{Problem Formulation}
%%%%%%%%%%%%%%%%%%%%%%%%%%%%%%%%%%%%%%%%%%%%%%%%%%%%%%%%%%%%%%%%%%%%%%%%%%%%%%%%%
\begin{frame}
\begin{center}
  \Huge{\textcolor{myblue}{Scalar Transport}}
\end{center}
\end{frame}
%%%%%%%%%%%%%%%%%%%%%%%%%%%%%%%%%%%%%%%%%%%%%%%%%%%%%%%%%%%%%%%%%%%%%%%%%%%%%%%%%
\begin{frame}
\frametitle{Scalar Problem Formulation}

\begin{itemize}
  \item Consider a linear conservation law model
    ($\consflux(\scalarsolution)=\velocity\scalarsolution$):
    \begin{equation}
      \pd{\scalarsolution}{t} + \nabla\cdot(\velocity\scalarsolution)
      + \reactioncoef(\x)\scalarsolution\xt = \scalarsource\xt \eqc
    \end{equation}
    where $\reactioncoef(\x)\ge 0$ and $\scalarsource\xt\ge 0$.
  \item Provide initial conditions and some boundary
    condition, such as Dirichlet:
   \begin{equation}
      u(\x,0) = u^0(\x) \quad \forall \x\in \domain
   \end{equation}
   \begin{equation}
      u\xt = u^{\text{inc}}(\x) \quad \forall \x\in \incomingdomainboundary
   \end{equation}
   \item CFEM solution:
   \begin{equation}
      \approximatescalarsolution\xt = \sum\limits_{j=1}^N U_j(t) \testfunction_j(\x) \eqc
      \quad \testfunction_j(\x)\in P^1_h
   \end{equation}
\end{itemize}

\end{frame}
%%%%%%%%%%%%%%%%%%%%%%%%%%%%%%%%%%%%%%%%%%%%%%%%%%%%%%%%%%%%%%%%%%%%%%%%%%%%%%%%
\subsection{Discretization}
%%%%%%%%%%%%%%%%%%%%%%%%%%%%%%%%%%%%%%%%%%%%%%%%%%%%%%%%%%%%%%%%%%%%%%%%%%%%%%%%
\begin{frame}
\frametitle{Some Notation}

\begin{minipage}{0.49\textwidth}
  \begin{itemize}
    \item Let \hlorange{$\support\ij$} be the shared support of test functions
      $\testfunction_i$ and $\testfunction_j$.
    \item Let \hlorange{$\cellindices(\support\ij)$} be the set of indices of
      cells in $\support\ij$.
  \end{itemize}
  \def\pointsize{2pt}

\begin{tikzpicture}[
  scale=1]

\coordinate (p1) at (1,1);
\coordinate (p2) at (2.1,0.4);
\coordinate (p3) at (3.3,0.4);
\coordinate (p4) at (4.3,0.9);
\coordinate (p5) at (5.2,1.3);
\coordinate (p6) at (5.8,2.1);
\coordinate (p7) at (5.8,3.5);
\coordinate (p8) at (5.2,4);
\coordinate (p9) at (4.5,4.8);
\coordinate (p10) at (3.6,5);
\coordinate (p11) at (2.7,4.5);
\coordinate (p12) at (1.6,4.6);
\coordinate (p13) at (0.8,3.5);
\coordinate (p14) at (1.2,2.5);
\coordinate (p15) at (2,1.5);
\coordinate (p16) at (3.3,1.2);
\coordinate (p17) at (4,1.9);
\coordinate (p18) at (5,2.2);
\coordinate (p19) at (5,3.3);
\coordinate (p20) at (4.2,3.8);
\coordinate (p21) at (3.3,3.6);
\coordinate (p22) at (2,3.7);
\coordinate (i) at (3,2.5);
\coordinate (j) at (4,3);

\fill (p1) circle (\pointsize);
\fill (p2) circle (\pointsize);
\fill (p3) circle (\pointsize);
\fill (p4) circle (\pointsize);
\fill (p5) circle (\pointsize);
\fill (p6) circle (\pointsize);
\fill (p7) circle (\pointsize);
\fill (p8) circle (\pointsize);
\fill (p9) circle (\pointsize);
\fill (p10) circle (\pointsize);
\fill (p11) circle (\pointsize);
\fill (p12) circle (\pointsize);
\fill (p13) circle (\pointsize);
\fill (p14) circle (\pointsize);
\fill (p15) circle (\pointsize);
\fill (p16) circle (\pointsize);
\fill (p17) circle (\pointsize);
\fill (p18) circle (\pointsize);
\fill (p19) circle (\pointsize);
\fill (p20) circle (\pointsize);
\fill (p21) circle (\pointsize);
\fill (p22) circle (\pointsize);
\fill (i) circle (\pointsize);
\fill (j) circle (\pointsize);

\draw (p1) -- (p2);
\draw (p1) -- (p14);
\draw (p1) -- (p15);
\draw (p2) -- (p3);
\draw (p2) -- (p15);
\draw (p2) -- (p16);
\draw (p3) -- (p4);
\draw (p3) -- (p16);
\draw (p4) -- (p5);
\draw (p4) -- (p16);
\draw (p4) -- (p17);
\draw (p4) -- (p18);
\draw (p5) -- (p6);
\draw (p5) -- (p18);
\draw (p6) -- (p7);
\draw (p6) -- (p18);
\draw (p6) -- (p19);
\draw (p7) -- (p8);
\draw (p7) -- (p19);
\draw (p8) -- (p9);
\draw (p8) -- (p19);
\draw (p8) -- (p20);
\draw (p9) -- (p10);
\draw (p9) -- (p20);
\draw (p10) -- (p11);
\draw (p10) -- (p20);
\draw (p10) -- (p21);
\draw (p11) -- (p12);
\draw (p11) -- (p21);
\draw (p11) -- (p22);
\draw (p12) -- (p13);
\draw (p12) -- (p22);
\draw (p13) -- (p14);
\draw (p13) -- (p22);
\draw (p14) -- (p15);
\draw (p14) -- (p22);
\draw (p15) -- (i);
\draw (p15) -- (p16);
\draw (p15) -- (p22);
\draw (p16) -- (i);
\draw (p16) -- (p17);
\draw (p17) -- (i);
\draw (p17) -- (j);
\draw (p17) -- (p18);
\draw (p18) -- (j);
\draw (p18) -- (p19);
\draw (p19) -- (j);
\draw (p19) -- (p20);
\draw (p20) -- (j);
\draw (p20) -- (p21);
\draw (p21) -- (i);
\draw (p21) -- (j);
\draw (p21) -- (p22);
\draw (p22) -- (i);
\draw (i) -- (j);

\draw[thick, draw=black, fill=red, fill opacity=0.25] (p15)--(p16)--(p17)--(j)
  --(p21)--(p22)--cycle;
\draw[thick, draw=black, fill=blue, fill opacity=0.25] (p17)--(p18)--(p19)--(p20)
  --(p21)--(i)--cycle;

\node at ($(i)-(0.25,0)$) {$i$};
\node at ($(j)+(0.2,0.25)$) {$j$};

\node at (2.5,3) {\Large $\textcolor{gray!20!red}{\support_i}$};
\node at (3.5,2.8) {\Large $\textcolor{violet}{\support\ij}$};
\node at (4.5,2.5) {\Large $\textcolor{gray!20!blue}{\support_j}$};

\end{tikzpicture}

\end{minipage}
\begin{minipage}{0.49\textwidth}
  \begin{itemize}
    \item Let \hlorange{$\indices(\cell)$} be the set of DoF indices on cell $\cell$.
    \item Let \hlorange{$n_\cell$} be the number of DoFs on cell $\cell$.
  \end{itemize}
  \begin{tikzpicture}[
  scale=1]

\input{content/diagrams/mesh.tex}

\draw[draw=none, fill=red, fill opacity=0.5] (p15)--(p16)--(p17)--(j)
  --(p21)--(p22)--cycle;
\draw[draw=none, fill=blue, fill opacity=0.5] (p17)--(p18)--(p19)--(p20)
  --(p21)--(i)--cycle;
\draw[draw=none, fill=yellow, fill opacity=0.5] (p16)--(i)--(j)--(p18)
  --(p4)--cycle;

\node[black] at ($0.333*(i) + 0.333*(j) + 0.333*(p17)$) {\Large $K$};

\fill[black] (i) circle (\pointsize);
\fill[black] (j) circle (\pointsize);
\fill[black] (p17) circle (\pointsize);

\end{tikzpicture}

\end{minipage}

\end{frame}
%%%%%%%%%%%%%%%%%%%%%%%%%%%%%%%%%%%%%%%%%%%%%%%%%%%%%%%%%%%%%%%%%%%%%%%%%%%%%%%%
\begin{frame}
\frametitle{Discretization}

\begin{itemize}
   \item Discretizing with forward Euler (FE) in time and \hlorange{Galerkin CFEM}
      in space gives
   \begin{equation}
      \consistentmassmatrix\frac{\solutionvector^{n+1}-\solutionvector^n}
        {\dt} + \ssmatrix\solutionvector^n = \ssrhs^n \eqc
   \end{equation}
   where
   \begin{equation}
     M\ij^C \equiv \int\limits_{\support\ij}
       \testfunction_i(\x) \testfunction_j(\x) d\x \eqc
   \end{equation}
   \begin{equation}
     A\ij \equiv \int\limits_{\support\ij}\left(
       \mathbf{v}\cdot\nabla\testfunction_j(\x) +
		\reactioncoef(\x)\testfunction_j(\x)\right)\testfunction_i(\x) d\x \eqc
   \end{equation}
   \begin{equation}
      b_i^n \equiv \int\limits_{S_i} q(\x,t^n)\testfunction_i(\x) d\x \eqp
   \end{equation}
\end{itemize}

\end{frame}
%%%%%%%%%%%%%%%%%%%%%%%%%%%%%%%%%%%%%%%%%%%%%%%%%%%%%%%%%%%%%%%%%%%%%%%%%%%%%%%%
\begin{frame}
\frametitle{Temporal Discretization}

\begin{itemize}
  \item \hlorange{Forward Euler (FE)}:
    \begin{equation}
      \consistentmassmatrix\frac{\solutionvector^{n+1}-\solutionvector^n}
        {\dt} + \ssmatrix\solutionvector^n = \ssrhs^n \eqc
    \end{equation}
  \item \hlorange{Strong Stability Preserving Runge-Kutta (SSPRK) Methods}:
    %\begin{itemize}
    %  \item Can be expressed as a combination of FE-like steps:
    %\end{itemize}
\begin{subequations}
\begin{align}
  & \RKstagesolution^0 = \solutionvector^n \\
%  & \RKstagesolution^i = \gamma_i \solutionvector^n + \zeta_i \sq{
%      \RKstagesolution^{i-1}
%      + \dt\mathbf{G}(t^n+c_i\dt, \RKstagesolution^{i-1})}
  & \RKstagesolution^i = \gamma_i \solutionvector^n + \zeta_i \bar{\solutionvector}^i
    \eqc \quad 
      \consistentmassmatrix\frac{\bar{\solutionvector}^i-\RKstagesolution^{i-1}}
        {\dt} + \ssmatrix\RKstagesolution^{i-1} = \ssrhs(t_i) \eqc
    %\eqc \quad
    %i = 1,\ldots,s \eqc
    \\
  & \solutionvector^{n+1} = \RKstagesolution^s
\end{align}
\end{subequations}
  where $t_i \equiv t^n+c_i\dt$, and $\{\gamma_i,\zeta_i,c_i\}_{i=1}^s$
  depend on the method.
  \item \hlorange{Theta Method}:
\begin{subequations}
\begin{equation}
  \consistentmassmatrix\frac{\solutionvector^{\timeindex+1}
    - \solutionvector^\timeindex}{\timestepsize}
    + \ssmatrix\solutionvector^\theta
    = \ssrhs^\theta
\end{equation}
\begin{equation}
  \solutionvector^\theta \equiv (1-\theta)\solutionvector^n
    + \theta\solutionvector^{n+1}
  \eqc \quad
  \ssrhs^\theta \equiv (1-\theta)\ssrhs^n
    + \theta\ssrhs^{n+1}
\end{equation}
\end{subequations}
    where $0\leq\theta\leq1$.
\end{itemize}

\end{frame}
%%%%%%%%%%%%%%%%%%%%%%%%%%%%%%%%%%%%%%%%%%%%%%%%%%%%%%%%%%%%%%%%%%%%%%%%%%%%%%%%
\subsection{Boundary Conditions}
%%%%%%%%%%%%%%%%%%%%%%%%%%%%%%%%%%%%%%%%%%%%%%%%%%%%%%%%%%%%%%%%%%%%%%%%%%%%%%%%
\begin{frame}
\frametitle{Boundary Conditions}

\begin{itemize}
  \item The following 3 methods for imposing incoming flux boundary conditions
    for node $i$ are considered:
    \begin{enumerate}
      \item \hlorange{Strongly impose}: \emph{replace} equation $i$ with the equation
        $U_i = u^{\textup{inc}}(\x_i)$.
      \item \hlorange{Weakly impose}: evaluate incoming boundary fluxes 
        in equation $i$ with values $u^{\textup{inc}}(\x_j)$ instead of values $U_j$.
      \item Weakly impose with \hlorange{boundary penalty}: \emph{add} to
        equation $i$ a multiple $\alpha_i$ of the equation $U_i = u^{\textup{inc}}(\x_i)$,
        where $\alpha_i$ should be large enough such that this \emph{penalty}
        equation dominates the original equation $i$.
    \end{enumerate}
  \item The choice of incoming flux boundary condition will later to
    be shown to have consequences on conservation for FCT.
\end{itemize}

\end{frame}
