%%%%%%%%%%%%%%%%%%%%%%%%%%%%%%%%%%%%%%%%%%%%%%%%%%%%%%%%%%%%%%%%%%%%%%%%%%%%%%%%%
\subsection{FCT Scheme}
%%%%%%%%%%%%%%%%%%%%%%%%%%%%%%%%%%%%%%%%%%%%%%%%%%%%%%%%%%%%%%%%%%%%%%%%%%%%%%%%%
\begin{frame}
\frametitle{FCT Antidiffusive Flux Definition}

\begin{itemize}
  \item The antidiffusive fluxes $\correctionfluxvector_i$ are defined such that
    \begin{equation}
      \lumpedmassentry
        \frac{\textcolor{secondarycolorheavy}{\solutionvector_i^{H}}
          -\solutionvector_i^n}{\dt}
        + \sum_j\gradiententry\cdot\consfluxinterpolant_j^n
        + \sum_j\diffusionmatrixletter\ij^{L,n}\solutionvector_j^n
        = \ssrhs_i^n + \textcolor{secondarycolorheavy}{\correctionfluxvector_i}
      \eqc
    \end{equation}
   \item Subtracting the high-order scheme equation from this gives the
      definition of $\correctionfluxvector$:
      \begin{equation}
        \correctionfluxvector_i \equiv
          \lumpedmassentry
            \frac{\solutionvector_i^{H}-\solutionvector_i^n}{\dt}
          -\sum_j\consistentmassentry
            \frac{\solutionvector_j^{H}-\solutionvector_j^n}{\dt}
          +\sum_j(\diffusionmatrixletter\ij^{L,n}-\diffusionmatrixletter\ij^{H,n})
            \solutionvector_j^n
      \end{equation}
   \item Decomposing $\correctionfluxvector$ into internodal fluxes
      $\correctionfluxmatrix\ij$ such that $\sum_j \correctionfluxmatrix\ij =
      \correctionfluxvector_i$,
      \begin{equation}
        \correctionfluxmatrix\ij = -M\ij^C\pr{
            \frac{\solutionvector_j^{H}-\solutionvector_j^n}{\dt}
            -\frac{\solutionvector_i^{H}-\solutionvector_i^n}{\dt}
          }
          + (D\ij^{L,n}-D\ij^{H,n})(\solutionvector_j^n-\solutionvector_i^n) \eqp
      \end{equation}
\end{itemize}

\end{frame}
%%%%%%%%%%%%%%%%%%%%%%%%%%%%%%%%%%%%%%%%%%%%%%%%%%%%%%%%%%%%%%%%%%%%%%%%%%%%%%%%%
\begin{frame}
\frametitle{FCT Limitation Process for Systems}

\begin{itemize}
  \item Bounds to impose on FCT solution are unclear for systems case:
    \begin{itemize}
      \item For scalar case, a DMP was used, but no DMP exists for systems.
      \item The invariant domain property is not so straightforward to enforce.
    \end{itemize}
  \item One may want to impose physical bounds on some non-conservative
    set of variables.
  \item Limitation of conservative variables may not satisfy these bounds.
  \item Consider the following sets of variables for the 1-D SWE:
    \begin{itemize}
      \item \textcolor{secondarycolorheavy}{Conservative}:
        $\vectorsolution \equiv [\height,\height\velocityx]\transpose$
      \item \textcolor{secondarycolorheavy}{Primitive}:
        $\check{\vectorsolution} \equiv [\height,\velocityx]\transpose$
      \item \textcolor{secondarycolorheavy}{Characteristic}:
        $\hat{\vectorsolution} \equiv [\velocityx-2\speedofsound,
          \velocityx+2\speedofsound]\transpose$,
          where $\speedofsound\equiv\sqrt{\gravity\height}$
    \end{itemize}
  \item Results from literature suggest that limitation on a non-conservative
    set of variables may produce superior results.
\end{itemize}

\end{frame}
%%%%%%%%%%%%%%%%%%%%%%%%%%%%%%%%%%%%%%%%%%%%%%%%%%%%%%%%%%%%%%%%%%%%%%%%%%%%%%%%%
\begin{frame}
\frametitle{Limitation on Non-Conservative Variables}

\begin{itemize}
  \item Consider some non-conservative set of variables
    $\hat{\vectorsolution} = \mathbf{T}^{-1}(\vectorsolution)\vectorsolution$.
    \begin{itemize}
      \item For \textcolor{secondarycolorheavy}{conservative} variables,
        $\mathbf{T}(\vectorsolution)=\mathbb{I}$.
      \item For \textcolor{secondarycolorheavy}{characteristic} variables,
        $\mathbf{T}(\vectorsolution)$ is the matrix with right eigenvectors
        of the Jacobian matrix $\partial\consfluxvector/\partial\vectorsolution$
        as its columns.
    \end{itemize}
  \item The FCT scheme for limitation of conservative variables is
    \begin{equation}
      \lumpedmassentry
        \frac{\solutionvector_i^{n+1}-\solutionvector_i^n}{\dt}
        + \sum_j\gradiententry\cdot\consfluxinterpolant_j^n
        + \sum_j\diffusionmatrixletter\ij^{L,n}\solutionvector_j^n
        = \ssrhs_i^n + \sum_j\limitermatrix\ij\odot\correctionfluxmatrix\ij \eqp
    \end{equation}
  \item Applying a local transformation $\mathbf{T}^{-1}(\solutionvector_i^n)$ gives
    \begin{equation}
      \lumpedmassentry
        \frac{\hat{\solutionvector}_i^{n+1}-\hat{\solutionvector}_i^n}{\dt}
        + \sum_j\gradiententry\cdot\hat{\consfluxinterpolant}_j^n
        + \sum_j\diffusionmatrixletter\ij^{L,n}\hat{\solutionvector}_j^n
        = \hat{\ssrhs}_i^n + \sum_j\limitermatrix\ij\odot\hat{\correctionfluxmatrix}\ij
        \eqc
    \end{equation}
    where accents denote transformed quantities, for example,
    \begin{equation}
      \hat{\solutionvector}_j^k
        = \mathbf{T}^{-1}(\solutionvector_i^n)\solutionvector_j^k
      \eqc \quad
      \hat{\correctionfluxmatrix}\ij
        = \mathbf{T}^{-1}(\solutionvector_i^n)\correctionfluxmatrix\ij
      \eqp
    \end{equation}
\end{itemize}

\end{frame}
%%%%%%%%%%%%%%%%%%%%%%%%%%%%%%%%%%%%%%%%%%%%%%%%%%%%%%%%%%%%%%%%%%%%%%%%%%%%%%%%%
\begin{frame}
\frametitle{Limiting Coefficients}

\begin{itemize}
  \item Choose $\limitermatrix\ij$ to satisfy some solution bounds
    $\hat{\solutionvector}_{i}^\pm$ and then define corresponding
    antidiffusive flux bounds $\hat{\mathbf{\limitedfluxbound}}_i^\pm$:
    \begin{equation}
      \hat{\mathbf{\limitedfluxbound}}^-_i \leq
        \sum\limits_j \limitermatrix\ij\odot\hat{\correctionfluxmatrix}\ij \leq
        \hat{\mathbf{\limitedfluxbound}}^+_i
      \quad\Rightarrow\quad
      \hat{\solutionvector}_i^- \leq
        \hat{\solutionvector}_i^{n+1} \leq
        \hat{\solutionvector}_i^+ \quad \forall i \eqp
    \end{equation}
  \item Performing some algebra gives the definition
    \begin{equation}
      \hat{\mathbf{\limitedfluxbound}}_i^\pm \equiv
        \lumpedmassentry\frac{\hat{\solutionvector}_i^\pm
          -\hat{\solutionvector}_i^n}{\dt}
        + \sum_j\gradiententry\cdot\hat{\consfluxinterpolant}_j^n
        + \sum_j\diffusionmatrixletter\ij^{L,n}\hat{\solutionvector}_j^n
        - \hat{\ssrhs}_i^n \eqp
    \end{equation}
  \item Limiting coefficients are computed just as for scalar case, for
    each solution component $\componentindex$:
    $\limiterletter\ij^{\componentindex}$.
  \item Literature suggests limiting coefficients may require
    \hlorange{synchronization}, e.g.,
    \begin{equation}
      \limiterletter\ij^\componentindex \mapsfrom
        \min\limits_\ell\limiterletter\ij^\ell \quad \forall\componentindex \eqp
    \end{equation}
\end{itemize}

\end{frame}
