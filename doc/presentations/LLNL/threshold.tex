\documentclass{article}

\usepackage{graphicx}   % for graphics
\usepackage{subcaption} % for subplots
\usepackage{rotating}   % for landscape figure

% Input
%------------------------------------------------------------------------------
% path to figures directory
\newcommand{\FiguresDir}{../images}

% type of figure: "figure" for portrait or "sidewaysfigure" for landscape
\newcommand{\FigureType}{figure}
%------------------------------------------------------------------------------

\begin{document}

\newcommand{\MyFigure}[1]{
\begin{\FigureType}[htb]
   \centering
   \begin{subfigure}{0.45\textwidth}
      \includegraphics[width=1.0\textwidth]{\FiguresDir/EV_#1_threshold.png}
      \caption{EV}
   \end{subfigure}
   \begin{subfigure}{0.45\textwidth}
      \includegraphics[width=1.0\textwidth]{\FiguresDir/EVFCT_#1_threshold.png}
      \caption{EV-FCT}
   \end{subfigure}
   \caption{Comparison of solutions with 16384 cells for the normally-incident
     absorber test problem using #1 time discretization, showing
     threshold violations: \emph{blue} indicates solution values greater than the
     incoming flux value, and \emph{red} indicates negative solution values.}
\end{\FigureType}
}

\MyFigure{FE}

\MyFigure{SSPRK33}

\end{document}
