Evaluating row $i$ of Eqn.~\ref{eq:loworderscheme} and rearranging,

\[
   U_i^{L,n+1} = U_i^n - \frac{\Delta t}{m_i}\sum\limits_j U_j^n A^L_{i,j}
      + \frac{\Delta t}{m_i}b_i,
\]

\noindent
where $m_i$ is the $i$th element of the lumped mass matrix.
Rearranging this equation,

\[
   U_i^{L,n+1} = \left(1-\frac{\Delta t}{m_i}A^L_{i,i}\right)U_i^n - \frac{\Delta t}{m_i}
      \sum\limits_{j\ne i} U_j^n A^L_{i,j} + \frac{\Delta t}{m_i}b_i.
\]

\noindent
First it will be proved that the off-diagonal elements of the low-order steady
state matrix $\mathbf{A}^L = \mathbf{A} + \mathbf{D}^L$ are non-positive.
Bounding $D_{i,j}^L=\sum\limits_{K\subset S_{i,j}}\nu_K^L b_K(\varphi_j, \varphi_i)$
for $j \ne i$:

\begin{eqnarray*}
   D_{i,j}=\sum\limits_{K\subset S_{i,j}}\nu_K^L b_K(\varphi_j, \varphi_i)
   & = & \sum\limits_{K\subset S_{i,j}} \max\limits_{k\ne \ell\in \mathcal{I}(K)}
      \frac{\max(0,A_{k,\ell})}
		{-\sum\limits_{T\subset S_{k,\ell}} b_T(\varphi_{\ell}, \varphi_k)}
      b_K(\varphi_j,\varphi_i)\\
   & \le & \sum\limits_{K\subset S_{i,j}} \frac{\max(0,A_{i,j})}
		{-\sum\limits_{T\subset S_{i,j}} b_T(\varphi_j, \varphi_i)}b_K(\varphi_j,\varphi_i)\\
   & =   & -\max(0,A_{i,j}) \frac{\sum\limits_{K\subset S_{i,j}}
      b_K(\varphi_j,\varphi_i)}
		{\sum\limits_{T\subset S_{i,j}} b_T(\varphi_j, \varphi_i)}\\   
   & = & -\max(0,A_{i,j})\\
   & \le & -A_{i,j}
\end{eqnarray*}

\noindent
Thus for $j \ne i$ (i.e., the off-diagonal elements), 
\begin{eqnarray*}
	A^L_{i,j} & = & A_{i,j} + D_{i,j}^L\\
   A^L_{i,j} & \le & A_{i,j} - A_{i,j}\\
   A^L_{i,j} & \le & 0.
\end{eqnarray*}

\noindent
Given that the CFL condition $\Delta t \leq \frac{m_i}{A_{i,i}^L}$ is satisfied,
which corresponds to $1-\frac{\Delta t}{m_i}A^L_{i,i} \ge 0$, and since the off-diagonal
elements of $\mathbf{A}^L$, are non-positive. Thus, the following inequality is
able to be applied:

\[
   U_i^{L,n+1} \le
   U_{\max,i}^n\left(1-\frac{\Delta t}{m_i}\sum\limits_j A^L_{i,j}\right)
      + \frac{\Delta t}{m_i}b_i,
\]

\noindent
and similarly for the lower bound.

%---------------------------------------------------------------------
\section{Proof that the FCT Scheme Satisfies the Discrete Maximum Principle}
\label{ap:fct_dmp}

To prove that the FCT scheme satisfies the discrete maximum principle bounds,
first the limited antidiffusion flux $\sum\limits_j \alpha_{i,j}F_{i,j}$
will be shown to be bounded. Note some properties of definitions given in Eqns. \ref{eq:P_defs},
\ref{eq:Q_defs}, \ref{eq:R_defs}, and \ref{eq:L_defs}:

\begin{gather*}
   P_i^+ \geq 0, \qquad P_i^- \leq 0,\\
   Q_i^+ \geq 0, \qquad Q_i^- \leq 0,\\
   0 \leq R_i^\pm \leq 1,\\
   0 \leq \alpha_{i,j} \leq 1.
\end{gather*}

\noindent
Now it is shown that the limited antidiffusion flux $\sum\limits_j \alpha_{i,j}F_{i,j}$ is bounded by the $Q_i^\pm$:

\[
   \sum\limits_j \alpha_{i,j}F_{i,j}
      \leq \sum\limits_{j:F_{i,j}\geq 0} \alpha_{i,j}F_{i,j}
      = \sum\limits_{j:F_{i,j}\geq 0} \min(R_i^+,R_j^-)F_{i,j}
      \leq \sum\limits_{j:F_{i,j}\geq 0} R_i^+ F_{i,j}.
\]

\noindent
For the case $P_i^+ = 0$,

\[
   \sum\limits_{j:F_{i,j}\geq 0} R_i^+ F_{i,j} = 0 \leq Q_i^+
\]

\noindent
For the case $P_i^+ \ne 0$,

\[
   \sum\limits_{j:F_{i,j}\geq 0} R_i^+ F_{i,j}
   \leq \sum\limits_{j:F_{i,j}\geq 0}\frac{Q_i^+}{P_i^+} F_{i,j}
   = \frac{Q_i^+}{P_i^+} \sum\limits_{j:F_{i,j}\geq 0} F_{i,j}
   = \frac{Q_i^+}{P_i^+} \sum\limits_{j:F_{i,j}\geq 0} \max(0,F_{i,j})
   = Q_i^+.
\]

\noindent
Thus,

\[
   \sum\limits_j \alpha_{i,j}F_{i,j} \leq Q_i^+.
\]

\noindent
From Eqn.~\ref{eq:FCTscheme}, the limited antidiffusion flux is expressed as

\[
   \sum\limits_j\alpha_{i,j}F_{i,j} =
   m_i\frac{U_i^{n+1}-U_i^n}{\Delta t}
   + \sum\limits_j A_{i,j}^L U_j^n
   - b_i.
\]

\noindent
Using the bound for $\sum\limits_j\alpha_{i,j}F_{i,j}$ and the definition
of $Q_i^\pm$ given in Eqn. \ref{eq:Q_defs},

\[
m_i\frac{W_i^- -U_i^n}{\Delta t}
   + \sum\limits_j A_{i,j}^L U_j^n
   - b_i
\le m_i\frac{U_i^{n+1}-U_i^n}{\Delta t}
   + \sum\limits_j A_{i,j}^L U_j^n
   - b_i
\le m_i\frac{W_i^+ -U_i^n}{\Delta t}
   + \sum\limits_j A_{i,j}^L U_j^n
   - b_i,
\]

\noindent
Rearranging, the discrete maximum principle is proved:

\[
   W_i^-\le U_i^{n+1}\le W_i^+.
\]
