% !TEX root = ../paper.tex

\subsubsection{Solution Bounds}\label{sec:solution_bounds}

The integral form of the transport equation can be derived using the method
of characteristics. Consider a frame of reference moving with the radiation field so that position is a function of time, resulting in a family of characteristic curves
(since the transport equation is linear, these curves are straight lines)
$\x(t)$ that solve the following ODE:
\begin{equation}
  \ddt{\x}=v\di\eqc \quad \x(0)=\x_0\eqp
\end{equation}
Then taking the time derivative of $u(\x(t),t)$ gives
\begin{align}
  \ddt{u} & = \ppt{u} + \nabla\cdot(u(\x(t),t))\ddt{\x}\\
    & =  \ppt{u} + \nabla\cdot(u(\x(t),t))\ddt{\x}
\end{align}
Finally, combining this with Equation \eqref{eq:scalar_model} and solving
the resulting ODE gives the integral transport equation \cite{glasstone}:
\begin{multline}\label{eq:integral_transport}
   u(\x,t) = u_0(\x - v t\di) e^{-\int\limits_0^t
    \sigma(\x - v(t -t')\di)v dt'}\\
    + \int\limits_0^t q(\x - v(t -t')\di,t') e^{-\int\limits_{t'}^t
    \sigma(\x - v(t -{t''})\di)v d{t''}} v dt' \eqp
\end{multline}

% %================================================================================
% \section{Introduction}
% %================================================================================
% In this section, an analytic local maximum principle is derived for
% scalar conservation laws having a constant, linear flux $\consfluxscalar$,
% i.e., $\consfluxscalar = \velocity u$ with $\nabla\cdot(\velocity u) =
% \velocity\cdot\nabla u$, where $\velocity$ is the constant velocity field. This
% analysis is valid for radiation transport, where the constant velocity field is
% $\velocity=v\directionvector$, with $v$ being the radiation speed.
%
% The analytic DMPs are derived using the method of characteristics, whereby
% paths in the $x-t$ plane are found, along which the governing PDE becomes an ODE.
% This is simple for the case of constant linear transport because in this case
% the characteristics are constant.
%
% %================================================================================
% \section{Integral Form of the Linear Transport Equation}
% %================================================================================
% \begin{theorem}{Integral Form of the Linear Transport Equation}{}
%    An implicit solution to the initial value problem
%    \begin{equation}\label{PDE}
%      \frac{1}{v}\ppt{\scalarsolution}
%        + \directionvector\cdot\nabla\scalarsolution(\x,t)
%         + \sigma(\x)\scalarsolution(\x,t)
%         = q(\x,t),
%      \qquad \scalarsolution(\x,0) = \scalarsolution_0(\x)
%    \end{equation}
%    is the following:
%    \begin{equation}\label{eq:integral_form}
%       \scalarsolution(\x,t) = \scalarsolution_0(\x - v t\directionvector)
%          e^{-\int\limits_0^t \sigma(\x - v(t -t')\directionvector)v dt'} +
%          \int\limits_0^t q(\x - v(t -t')\directionvector,t')
%            e^{-\int\limits_{t'}^t\sigma(\x
%              - v(t -\bar{t})\directionvector)v d\bar{t}} v dt' \eqp
%    \end{equation}
% \end{theorem}
%
% \begin{proof}
%    This proof will proceed by using the method of characteristics. The position
%    $\x$ will be regarded as a function of time: $\x=\x(t)$.
%    The characteristic $\x(t)$ is the solution of the following initial value problem:
%    \[
%       \frac{d\x}{dt} = v\directionvector \eqc \qquad \x(0) = \x_0 \eqc
%    \]
%    which is
%    \[
%       \x(t) = \x_0 + v t\directionvector \eqp
%    \]
%    Taking the derivative of $\scalarsolution(\x(t),t)$ gives
%    \begin{eqnarray*}
%       \frac{d\scalarsolution}{dt} & = &\ppt{\scalarsolution}
%         + \nabla\cdot\scalarsolution(\x(t),t) \frac{d\x}{dt}\\
%         & = & \ppt{\scalarsolution}
%         + v\directionvector\cdot\nabla\scalarsolution(\x(t),t) \eqc
%    \end{eqnarray*}
%    which when combined with the PDE in Equation \eqref{PDE}, gives
%    \begin{equation}\label{eq:pre_integrating_factor}
%       \ddt{\scalarsolution} + v\sigma(\x(t))\scalarsolution(\x(t),t)
%         = vq(\x(t),t) \eqp
%    \end{equation}
%    This is a 1st-order linear ODE, which may be solved using an integrating factor
%    \[
%       \mu(t)=e^{\int\limits_0^t\sigma(\x(t'))v dt'}.
%    \]
%    Multiplying both sides of Equation \eqref{eq:pre_integrating_factor}
%    by this integrating factor and using the product rule,
%    \[
%       \frac{d}{dt}\left[\scalarsolution(\x(t),t)\mu(t)\right]
%         = vq(\x(t),t) \mu(t) \eqc
%    \]
%    and integrating from $0$ to $t$ gives
%    \[
%       \scalarsolution(\x(t),t)\mu(t)-\scalarsolution(\x(0),0)\mu(0) =
%          \int\limits_0^t q(\x(t'),t') \mu(t') v dt' \eqp
%    \]
%    Simplifying,
%    \[\begin{split}
%       \scalarsolution(\x(t),t) &= \scalarsolution(\x(0),0)
%          e^{-\int\limits_0^t \sigma(\x(t'))v dt'} +
%          \left(\int\limits_0^t q(\x(t'),t')
%            e^{\int\limits_0^{t'}\sigma(\x(\bar{t}))v d\bar{t}}
%            v dt'\right)
%          e^{-\int\limits_0^tv\sigma(\x(t'))dt'} \eqc\\
%       &= \scalarsolution(\x(0),0)
%          e^{-\int\limits_0^t \sigma(\x(t'))v dt'} +
%          \int\limits_0^t q(\x(t'),t')
%            e^{-\int\limits_{t'}^t\sigma(\x(\bar{t}))v d\bar{t}}
%            v dt' \eqp
%    \end{split}\]
%    Finally, expressing $\x(t)$ in terms of $\x$, $v$, $\directionvector$,
%    and $t$ gives
%    \begin{multline*}
%       \scalarsolution(\x,t) = \scalarsolution_0(\x - v t\directionvector)
%          e^{-\int\limits_0^t \sigma(\x - v(t -t')\directionvector)
%            v dt'}\\
%          + \int\limits_0^t q(\x - v(t -t')\directionvector,t')
%            e^{-\int\limits_{t'}^t\sigma(\x
%              - v(t -\bar{t})\directionvector)v d\bar{t}}
%              v dt'\eqp \qed
%    \end{multline*}
% \end{proof}
% %================================================================================
% \section{Local Maximum Principles\label{sec:local_max_principles}}
% %================================================================================
% Before giving an analytic local discrete maximum principle, a local maximum
% principle applying to a general region is given by the following theorem.
%
% \begin{theorem}[thm:analytic_max_principle]{Analytic Local Maximum Principle}
%    Let $L(\x,\tau)$ be the line segment that spans between
%    $\x-v\tau\directionvector$ and $\x$:
%    \begin{equation}
%       L(\x,\tau)\equiv \left\{\mathbf{y}\in\mathbb{R}^d : \mathbf{y}
%          = \x-v t\directionvector \eqc \qquad t\in(0,\tau) \right\} \eqp
%    \end{equation}
%    See Figure \ref{fig:neighborhood} for an illustration.
%    The following local maximum principle is valid for the solution to the
%    problem given by Equation \eqref{PDE}:
%    \begin{subequations}\label{eq:local_max_principle}
%    \begin{equation}
%       \scalarsolution_{\text{min}} \le \scalarsolution(\x,\tau)
%         \le \scalarsolution_{\text{max}} \eqc
%    \end{equation}
%    \begin{equation}
%       \scalarsolution_{\text{min}}
%         \equiv \left\{\begin{array}{l l}
%           %\scalarsolution_{\min,N}^0
%           \scalarsolution_0(\x - v \tau\directionvector)
%              e^{-v\tau\sigma_{\max,L}}
%             + \frac{q_{\min,L}}{\sigma_{\max,L}}
%              (1 - e^{-v\tau\sigma_{\max,L}}) \eqc
%           & \sigma_{\max,L} \ne 0 \\
%           %\scalarsolution_{\min,L}^0
%           \scalarsolution_0(\x - v \tau\directionvector)
%             + v\tauq_{\min,L} \eqc
%           & \sigma_{\max,L} = 0
%         \end{array}\right.\eqc
%    \end{equation}
%    \begin{equation}
%       \scalarsolution_{\text{max}}
%         \equiv \left\{\begin{array}{l l}
%           %\scalarsolution_{\max,N}^0
%           \scalarsolution_0(\x - v \tau\directionvector)
%             e^{-v\tau\sigma_{\min,L}}
%             + \frac{q_{\max,L}}{\sigma_{\min,L}}
%             (1 - e^{-v\tau\sigma_{\min,L}}) \eqc
%           & \sigma_{\min,L} \ne 0 \\
%           %\scalarsolution_{\max,N}^0
%           \scalarsolution_0(\x - v \tau\directionvector)
%             + v\tauq_{\max,L} \eqc
%           & \sigma_{\min,L} = 0
%         \end{array}\right.\eqc
%    \end{equation}
%    \begin{equation}
%      \sigma_{\min,L}\equiv\min\limits_{\mathbf{y}\in L(\x,\tau)}
%        \sigma(\mathbf{y}) \eqc \quad
%      \sigma_{\max,L}\equiv\max\limits_{\mathbf{y}\in L(\x,\tau)}
%        \sigma(\mathbf{y}) \eqc
%    \end{equation}
%    \begin{equation}
%      q_{\min,L}\equiv\min\limits_{\mathbf{y}\in L(\x,\tau)}
%        q(\mathbf{y}) \eqc \quad
%      q_{\max,L}\equiv\max\limits_{\mathbf{y}\in L(\x,\tau)}
%        q(\mathbf{y}) \eqp
%    \end{equation}
%    \end{subequations}
% \end{theorem}
% %-------------------------------------------------------------------------------
% \begin{figure}[htb]
%    \centering
%      \begin{tikzpicture}[
  scale=1]

\def\pointsize{2pt}
\def\radius{2}

\coordinate (mycenter) at (0,0);
\fill (mycenter) circle (\pointsize);
\draw[draw=red, fill=red, fill opacity=0.25] (mycenter) circle (\radius);
\draw (mycenter) node[below] {$\x$};
\draw[-latex] (mycenter) -- (-30:\radius) node[pos=0.5,sloped,above] {$\speed\tau$};
\draw[blue,line width=1.5pt] (mycenter) -- (210:\radius)
  node[pos=0.5,sloped,above] {$L(\x,\tau)$};
\draw[dashed] (mycenter) -- (30:\radius);
\node[red] at ($0.5*(0,\radius)$) {$N(\x,\tau)$};
\coordinate (dircenter) at (1.2*\radius,0.3*\radius);
\draw[|-latex,shift=(dircenter)] (0,0) -- (30:0.75)
  node[pos=0.5,sloped,above] {$\di$};

\end{tikzpicture}

%       \caption{Illustration of Neighborhoods $L(\x,\tau)$ and $N(\x,\tau)$}
%    \label{fig:neighborhood}
% \end{figure}
% %-------------------------------------------------------------------------------
% \begin{proof}
%    Rewriting Equation \eqref{eq:integral_form} with $t=\tau$ gives
%    \begin{multline*}
%       \scalarsolution(\x,\tau) = \scalarsolution_0(\x - v\tau\directionvector)
%          e^{-\int\limits_0^\tau \sigma(\x
%            - v(\tau -t')\directionvector)v dt'}\\
%          +
%          \int\limits_0^\tau q(\x - v(\tau -t')\directionvector,t')
%          e^{-\int\limits_{t'}^\tau \sigma(\x
%          - v(\tau -\bar{t})\directionvector)v d\bar{t}}v dt' \eqp
%    \end{multline*}
%    One can bound the first term in the right-hand-side of Equation
%    \eqref{eq:integral_form} by considering
%    the maximum and minimum cross section on the line segment $L(\x,\tau)$
%    for the lower and upper bounds, respectively:
%    \[
%       %\scalarsolution_{\min,L}^0
%       \scalarsolution_0(\x - v\tau\directionvector)
%         e^{-v\tau\sigma_{\max,L}} \le
%       \scalarsolution_0(\x - v\tau\directionvector)
%         e^{-\int\limits_0^\tau \sigma(\x
%            - v(\tau -t')\directionvector)v dt'} \le
%       %\scalarsolution_{\max,L}^0
%       \scalarsolution_0(\x - v\tau\directionvector)
%         e^{-v\tau\sigma_{\min,L}} \eqp
%    \]
%    The source term can be bounded as follows:
%    \begin{align*}
%      \scalarsolution_q
%       & \equiv
%          \int\limits_0^\tau q(\x - v(\tau -t')\directionvector,t')
%          e^{-\int\limits_{t'}^\tau \sigma(\x
%            - v(\tau -\bar{t})\directionvector)v d\bar{t}} v dt'\\
%       & \le
%          q_{\max,L}\int\limits_0^\tau
%          e^{-\int\limits_{t'}^\tau \sigma(\x
%            - v(\tau -\bar{t})\directionvector)v d\bar{t}}v dt'\\
%       & \le
%          q_{\max,L}\int\limits_0^\tau
%          e^{-\sigma_{\min,L}\int\limits_{t'}^\tau v d\bar{t}}v dt'\\
%       & =
%          q_{\max,L} \int\limits_0^\tau
%          e^{-v(\tau-t')\sigma_{\min,L}}v dt'\\
%       & =
%          q_{\max,L}e^{-v\tau\sigma_{\min,L}}
%          \int\limits_0^\tau e^{\sigma_{\min,L}v t'}v dt'\\
%       & =
%          \left\{\begin{array}{l l}
%             \frac{q_{\max,L}}{\sigma_{\min,L}}
%               (1 - e^{-v\tau\sigma_{\min,L}}) \eqc
%                & \sigma_{\min,L} \ne 0\\
%             v\tau q_{\max,L} \eqc & \sigma_{\min,L} = 0
%             \end{array}\right.
%    \end{align*}
%    A similar analysis is performed for the lower bound.
%    Putting the two components together gives the bounds given by Equation
%    \eqref{eq:local_max_principle}.\qed
% \end{proof}
% %-------------------------------------------------------------------------------
%
% This result gives relatively tight solution bounds; however, its use as
% solution bounds for FCT may prove difficult in practice (especially for
% multi-dimensional problems), as one must
% compute the solution at the point $\x-v\tau\di$ and must be able
% to evaluate the minimum and maximum of the reaction coefficients and
% sources on the line segment $L(\x_i,\tau)$.
% The following corollary loosens the solution bounds for use in a
% more simple implementation of solution bounds for FCT. It considers
% not just the upstream line segment of length $v\tau$, but the
% sphere of radius $v\tau$ centered at $\x_i$.
%
% %-------------------------------------------------------------------------------
% \begin{corollary}[cly:loose_analytic_max_principle]
%   {Loose Analytic Local Maximum Principle}
% Let $N(\x,\tau)$ denote the sphere centered at $\x$ with radius
% $v\tau$, as shown in Figure \ref{fig:neighborhood}:
%    \begin{equation}\label{eq:neighborhood}
%       N(\x,\tau)\equiv\left\{\mathbf{y}\in\mathbb{R}^d :
%          \|\mathbf{y} - \x\| \le v\tau\right\} \eqp
%    \end{equation}
% The following, looser, local maximum principle is valid for the solution to the
% problem given by Equation \eqref{PDE}:
% \begin{subequations}\label{eq:loose_local_max_principle}
%    \begin{equation}
%       \scalarsolution_{\text{min}} \le \scalarsolution(\x,\tau)
%         \le \scalarsolution_{\text{max}} \eqc
%    \end{equation}
%    \begin{equation}
%       \scalarsolution_{\text{min}}
%         \equiv \left\{\begin{array}{l l}
%           \scalarsolution_{\min,N}^0 e^{-v\tau\sigma_{\max,N}}
%             + \frac{q_{\min,N}}{\sigma_{\max,N}}
%              (1 - e^{-v\tau\sigma_{\max,N}}) \eqc
%           & \sigma_{\max,N} \ne 0 \\
%           \scalarsolution_{\min,N}^0
%             + v\tauq_{\min,N} \eqc
%           & \sigma_{\max,N} = 0
%         \end{array}\right.\eqc
%    \end{equation}
%    \begin{equation}
%       \scalarsolution_{\text{max}}
%         \equiv \left\{\begin{array}{l l}
%           \scalarsolution_{\max,N}^0 e^{-v\tau\sigma_{\min,N}}
%             + \frac{q_{\max,N}}{\sigma_{\min,N}}
%             (1 - e^{-v\tau\sigma_{\min,N}}) \eqc
%           & \sigma_{\min,N} \ne 0 \\
%           \scalarsolution_{\max,N}^0
%             + v\tauq_{\max,N} \eqc
%           & \sigma_{\min,N} = 0
%         \end{array}\right.\eqc
%    \end{equation}
%    \begin{equation}
%      \scalarsolution_{\min,N}^0 \equiv \min\limits_{\mathbf{y}\in N(\x,\tau)}
%        \scalarsolution(\mathbf{y},0) \eqc \quad
%      \scalarsolution_{\max,N}^0 \equiv \max\limits_{\mathbf{y}\in N(\x,\tau)}
%        \scalarsolution(\mathbf{y},0) \eqc
%    \end{equation}
%    \begin{equation}
%      \sigma_{\min,L}\equiv\min\limits_{\mathbf{y}\in L(\x,\tau)}
%        \sigma(\mathbf{y}) \eqc \quad
%      \sigma_{\max,L}\equiv\max\limits_{\mathbf{y}\in L(\x,\tau)}
%        \sigma(\mathbf{y}) \eqc
%    \end{equation}
%    \begin{equation}
%      q_{\min,L}\equiv\min\limits_{\mathbf{y}\in L(\x,\tau)}
%        q(\mathbf{y}) \eqc \quad
%      q_{\max,L}\equiv\max\limits_{\mathbf{y}\in L(\x,\tau)}
%        q(\mathbf{y}) \eqp
%    \end{equation}
% \end{subequations}
% \end{corollary}
% %-------------------------------------------------------------------------------
% \begin{proof}
% Because $\x-v\tau\di\in N(\x,\tau)$,
% \[
%   \scalarsolution_0(\x - v\tau\directionvector) \geq \scalarsolution_{\min,N}^0
%   \eqc \quad
%   \scalarsolution_0(\x - v\tau\directionvector) \leq \scalarsolution_{\max,N}^0
%   \eqp
% \]
% Because $L(\x,\tau)\subset N(\x,\tau)$ (see Figure \ref{fig:neighborhood}),
% the following is true:
% \[
%   q_{\min,N} \leq q_{\min,L}
%   \eqc \quad
%   q_{\max,N} \geq q_{\max,L}
%   \eqc
% \]
% \[
%   \sigma_{\min,N} \leq \sigma_{\min,L}
%   \eqc \quad
%   \sigma_{\max,N} \geq \sigma_{\max,L}
%   \eqp
% \]
% Applying these inequalities to Equation \eqref{eq:local_max_principle}
% proves Equation \eqref{eq:loose_local_max_principle}.\qed
% \end{proof}
% %-------------------------------------------------------------------------------
%
% The following theorem applies Corollary \ref{cly:loose_analytic_max_principle} to derive
% an analytic discrete maximum principle for radiation transport.
%
% %-------------------------------------------------------------------------------
% \begin{theorem}[thm:analytic_dmp]{Analytic Discrete Maximum Principle}
\noindent
If the time step size $\dt$ satisfies the condition
\begin{equation}\label{eq:cfl_analytic_dmp}
  v\dt \leq h_{\min} \eqc \quad h_{\min} \equiv \min\limits_K h_K \eqc
\end{equation}
where $h_K$ is the diameter of cell $K$, then the following discrete
solution bounds apply:
\begin{subequations}\label{eq:solution_bounds}
  \begin{equation}
      U^-_i \le U_i^{n+1} \le U^+_i \eqc
  \end{equation}
where
  \begin{equation}
      U^-_i
        \equiv \left\{\begin{array}{l l}
          U_{\min,i}^n e^{-v\dt\sigma_{\max,i}}
            + \frac{q_{\min,i}}{\sigma_{\max,i}}
            (1 - e^{-v\dt\sigma_{\max,i}}) \eqc
          & \sigma_{\max,i} \ne 0 \\
          U_{\min,i}^n
            + v\dt q_{\min,i} \eqc
          & \sigma_{\max,i} = 0
        \end{array}\right.\eqc
  \end{equation}
and
  \begin{equation}
      U^+_i
        \equiv \left\{\begin{array}{l l}
          U_{\max,i}^n e^{-v\dt\sigma_{\min,i}}
            + \frac{q_{\max,i}}{\sigma_{\min,i}}
            (1 - e^{-v\dt\sigma_{\min,i}}) \eqc
          & \sigma_{\min,i} \ne 0 \\
          U_{\max,i}^n
            + v\dt q_{\max,i} \eqc
          & \sigma_{\min,i} = 0
        \end{array}\right.\eqp
  \end{equation}
The other quantities used in the above expressions are:
  \begin{equation}
    U_{\max,i}^n \equiv\max\limits_{j\in\indices(S_i)}U_j^n \eqc \quad
    U_{\min,i}^n \equiv\min\limits_{j\in\indices(S_i)}U_j^n \eqc
  \end{equation}
  \begin{equation}
    \sigma_{\max,i} \equiv\max\limits_{\x\in S_i}\sigma(\x) \eqc \quad
    \sigma_{\min,i} \equiv\min\limits_{\x\in S_i}\sigma(\x) \eqc
  \end{equation}
  \begin{equation}
    q_{\max,i} \equiv\max\limits_{\x\in S_i}q(\x) \eqc \quad
    q_{\min,i} \equiv\min\limits_{\x\in S_i}q(\x) \eqp
  \end{equation}
\end{subequations}
Note the time step size condition given by Equation \eqref{eq:cfl_analytic_dmp}
implies that when using CFL numbers greater than 1 with implicit time
discretizations, these bounds no longer apply. Similar bounds can be derived
for $v\dt > h_{min}$; however, these bounds for a node $i$ will no longer
only depend on the solution values of the immediate neighbors of $i$; instead, a larger neighborhood
must be used in the bounds, making the local solution bounds wider and thus less
restrictive and arguably less useful in the FCT algorithm. This represents a
significant disadvantage for implicit FCT, not only because the converged
FCT solution could contain more undesirable features but also because the wider
bounds typically result in a greater number of nonlinear iterations because
of the increased freedom in the limiting coefficients.

Steady-state FCT solution bounds can be inferred from Equation \eqref{eq:solution_bounds}
by making the substitution $v\dt\rightarrow s$, where $0\leq s \leq h_{min}$. This
restriction of $s$ similarly ensures that only the nearest neighbors of $i$ are
needed for the solution bounds of $i$. Steady-state FCT unfortunately suffers
many of the same drawbacks as implicit FCT because like implicit FCT, its
solution bounds are implicit and thus change with each iteration.
% \end{theorem}
% \begin{proof}
% Due to the CFL condition, Equation \eqref{eq:cfl_analytic_dmp}, the support of
% test function $i$ is a superset of the neighborhood $N(\x_i)$ defined by
% Equation \eqref{eq:neighborhood}: $N(\x_i)\subsetS_i$. Thus
% for an arbitrary function of space $f(\x)$,
% \[
%   \max\limits_{\x\in S_i}f(\x)
%     \geq \max\limits_{\x\in N(\x_i)}f(\x) \eqc \quad
%   \min\limits_{\x\in S_i}f(\x)
%     \leq \min\limits_{\x\in N(\x_i)}f(\x) \eqc
% \]
% and
% \[
%   \approximatescalarsolution_{\max,S_i}
%     \geq \approximatescalarsolution_{\max,N} \eqc \quad
%   \approximatescalarsolution_{\min,S_i}
%     \leq \approximatescalarsolution_{\min,N} \eqp
% \]
% Since $\approximatescalarsolution$ is a convex combination of nodal solution
% values, the local extremum are obtained only at nodal values:
% \[
%   \approximatescalarsolution_{\max,S_i}
%     = U_{\max,i} \eqc \quad
%   \approximatescalarsolution_{\min,S_i}
%     = U_{\min,i} \eqp \qed
% \]
% \end{proof}
% %-------------------------------------------------------------------------------
%
% The following corollary extends the analytic discrete maximum principle given
% in Theorem \ref{thm:analytic_dmp} to the steady-state case and is given
% without proof, as it follows the same logic as Theorem \ref{thm:analytic_dmp}.
%
% %-------------------------------------------------------------------------------
% \begin{corollary}{Analytic Steady-State Discrete Maximum Principle}
% If one uses a parameter $s$ such that $s \leq \celldiameter_{\min}$, where
% $\celldiameter_{\min}$ is defined by Equation \eqref{eq:cfl_analytic_dmp}, then
% the following analytic discrete maximum principle bounds apply to the
% steady-state problem:
% \begin{subequations}\label{eq:analyticDMP_ss}
%   \begin{equation}
%       U^-_i \le U_i
%         \le U^+_i \eqc
%   \end{equation}
%   \begin{equation}
%       U^-_i
%         \equiv \left\{\begin{array}{l l}
%           U_{\min,i} e^{-s\sigma_{\max,i}}
%             + \frac{q_{\min,i}}{\sigma_{\max,i}}
%             (1 - e^{-s\sigma_{\max,i}}) \eqc
%           & \sigma_{\max,i} \ne 0 \\
%           U_{\min,i}
%             + sq_{\min,i} \eqc
%           & \sigma_{\max,i} = 0
%         \end{array}\right.\eqc
%   \end{equation}
%   \begin{equation}
%       U^+_i
%         \equiv \left\{\begin{array}{l l}
%           U_{\max,i} e^{-s\sigma_{\min,i}}
%             + \frac{q_{\max,i}}{\sigma_{\min,i}}
%             (1 - e^{-s\sigma_{\min,i}}) \eqc
%           & \sigma_{\min,i} \ne 0 \\
%           U_{\max,i}
%             + sq_{\max,i} \eqc
%           & \sigma_{\min,i} = 0
%         \end{array}\right.\eqp
%   \end{equation}
% \end{subequations}
% \end{corollary}
%
% \begin{remark}
% In practice, one can approximate the maximum/minimum operations
% by taking the maximum/minimum over quadrature points: e.g.,
% $\max\limits_{\x\in S_i} \approx \max\limits_{\x\in Q(S_i)}$,
% where $Q(S_i)$ is the set of quadrature points in $S_i$.
% \end{remark}
