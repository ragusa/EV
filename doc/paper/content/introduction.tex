\tcr{Leave this comment. I want to re-read it again, after modifications. This introduction
is very important}

The transport equation, or linear Boltzmann equation, describes the
transport of particles interacting with a background medium \cite{glasstone}.  Some
of its applications include the modeling of nuclear reactors, radiation
therapy, astrophysical applications, radiation shielding, and high energy density physics
\cite{glasstone,radiotherapy,astrophysics_textbook,lewis,laser_plasmas}.
This paper focuses on solution techniques applicable to the first-order
form of the transport equation, recalled below in Equation~\eqref{eq:transport_scalar}. The
transport equation is a particle balance statement in a six-dimensional phase-space volume
where $\x$ denotes the particle's position, $\di$ its direction of flight, and $E$ its energy:
\begin{equation}\label{eq:transport_scalar}
  \frac{1}{v(E)}\ppt{\aflux} + \di\cdot\nabla\aflux\xdet
    + \totalxsec\xet\aflux\xdet = \Qtot\xdet
  \eqp
\end{equation}
$\Qtot\xdet$ denotes the total particle source gains in an infinitesimal phase-space volume
due to particle scattering, extraneous source of particles (if any), and
fission sources (in the case of neutron transport in multiplying media):
%\begin{multline}
%  \Qtot\xdet \equiv
%       \int\limits_0^\infty dE'\int\limits_{4\pi}d\di'
%      \scatteringxsec(\x,E'\rightarrow E,\di'\rightarrow\di,t)\aflux(\x,\di',E',t) \\
%+ \Qext\xdet
%    + \frac{\chi(E)}{4\pi}\int\limits_0^\infty dE'  \nu(\x,E',t)\fissionxsec(\x,E',t)
%    \int\limits_{4\pi}d\di'\aflux(\x,\di',E',t)
%  \eqp
%\end{multline}
\begin{equation}
  \Qtot\xdet \equiv \Qsca\xdet + \Qext\xdet + \Qfis\xdet
  \eqp
\end{equation}
The source terms $\Qsca$ and $\Qfis$ linearly depend on the solution variable, the angular flux denoted by $\aflux$. Only simple configurations are amenable to an analytical solution of Equation~\eqref{eq:transport_scalar}.
In most cases of relevance, the transport equation must be solved numerically; this involves discretization
of the phase-space and the use of iterative techniques.
One common angular discretization is the discrete-ordinate or $S_N$ method
\cite{glasstone,lewis,duderstadt}; it is a collocation method whereby the transport equation is solved only along discrete directions $\di_d$
($1 \le d \le n_{\di}$, with $n_{\di}$ the total number of discrete directions). One of the
main advantages of the $S_N$ technique is that it enables an iterative approach, called source iteration
in the transport literature \cite{glasstone,lewis,duderstadt}, to resolve
both the particle's streaming and interaction processes and the scattering events as follows:
%equations is that the are an attractive form of the transport equation because
%the $S_N$ equations can be decoupled by using iterative techniques for the
%scattering source, an approach called source iteration \cite{}:
\begin{equation} \label{eq:SI}
  \frac{1}{v}\ppt{\aflux_d^{(\ell)}}
    + \di_{d}\cdot\nabla\aflux_d^{(\ell)}
    + \totalxsec\aflux_d^{(\ell)} = Q_{\textup{tot},d}^{(\ell-1)} \quad \forall d \eqc
\end{equation}
where $\ell$ is the iteration index and $\aflux_d^{(\ell)}(\x,E,t) = \aflux^{(\ell)}(\xdet)$.
Hence, a system of $n_{\di}$ decoupled equations are to be solved
at a given source iteration index $\ell$. This allows solution techniques for scalar conservation
laws to be leveraged in solving Equations~\eqref{eq:SI}. For brevity, the discrete-ordinate subscript $d$
will be omitted hereafter and our model transport equation will consist in one of the $n_{\di}$ transport
equations for a given fixed source (right-hand-side).

Traditionally, the spatial discretization method for the $S_N$
equations has been a Discontinuous Galerkin finite element method (DGFEM) with upwinding
\cite{Lesaint1974,Reed_Hill_1973}. Here, however,
a Continuous Galerkin finite element method (CGFEM) is applied.
Some recent work by Guermond and Popov \cite{guermond_ev} on
solution techniques for conservation laws with CGFEM addresses some
of the main disadvantages of CGFEM versus DGFEM, including the formation
of spurious oscillations. The purpose of the present paper is to demonstrate a proof of concept
for the application of such solution techniques to the transport equation.
Furthermore, some or all of the methodology explored in this paper may be
later extended to DGFEM as well; see, for instance, Zingan et al. \cite{zingan_2013}
where the techniques of Guermond and Popov \cite{guermond_ev} have been ported to DGFEM schemes.

One of the main objectives of this paper is to present a method that precludes
the formation of spurious oscillations and the negativities that result from
these oscillations; these issues have been a long-standing issue in the
numerical solution of the transport equation \cite{lanthrop}.
Not only are these negativities physically incorrect
(a particle's distribution density must be non-negative), but they can cause
simulations to terminate prematurely, for example in radiative transfer where
the radiation field is nonlinearly coupled to a material energy equation.
Many attempts to remedy the negativities in
transport solutions rely on ad-hoc fix-ups, such as the set-to-zero fix-up for the
classic diamond difference scheme \cite{lewis}. Recent work by Hamilton
introduced a similar fix-up for the linear discontinuous finite element
method (LDFEM) that conserves local balance and preserves third-order accuracy.
Walters and Wareing developed characteristic methods \cite{walters_NC}, but
Wareing later notes that these characteristic methods are difficult to
implement and offers a nonlinear positive spatial differencing scheme
known as the exponential discontinuous scheme \cite{wareing}.
Maginot has recently developed a consistent set-to-zero (CSZ) LDFEM
method \cite{maginot}, as well as a non-negative method for bilinear
discontinuous FEM \cite{maginot_mc2015}\cite{maginot_2017}.

In fluid dynamics, tradiational approaches to remedy the issue of spurious oscillations include
the flux-corrected transport (FCT) algorithm, introduced in 1973 as
the SHASTA algorithm for finite difference discretizations
by Boris and Book \cite{borisbook}, where it was applied to linear discontinuities
and gas dynamic shock waves. To the best of our knowledge, these FCT techniques have not been
applied to the particle transport equation.
The main idea of the FCT algorithm is to blend a
low-order scheme having desirable properties with a high-order scheme which may
lack these properties.
Zalesak improved methodology of the algorithm and introduced a fully
multi-dimensional limiter \cite{zalesak}.
Parrott and Christie extended the algorithm to the finite element method
on unstructured grids \cite{parrott}, thus beginning the FEM-FCT methodology.
L\"{o}hner et. al. applied FEM-FCT to the Euler and Navier-Stokes equations and
began using FCT with complex geometries \cite{lohner}.
Kuzmin and M\:{o}ller introduced an algebraic FCT approach for scalar conservation
laws \cite{kuzmin_FCT} and later introduced a general-purpose FCT scheme, which
is designed to be applicable to both steady-state and transient problems \cite{kuzmin_general}.
In these FEM-FCT works and others \cite{moller_2008,kuzmin_failsafe,kuzmin_closepacking},
the high-order scheme used in the FCT algorithm was the Galerkin finite element
method, but this work uses the entropy viscosity method developed by Guermond
and others \cite{guermond_ev}.

Recent work by Guermond and Popov addresses the issue of spurious oscillations
for general conservation laws by using artificial dissipation based on
local entropy production, a method known as entropy viscosity \cite{guermond_ev}.
The idea of entropy viscosity is to enforce an entropy inequality on the weak solution,
and thus filter out weak solutions containing spurious oscillations. However,
entropy viscosity solutions may still contain spurious
oscillations, albeit smaller in magnitude, and consequently negativities
are not precluded. To circumvent this deficiency, Guermond proposed using
the entropy viscosity method in conjunction with the FCT
algorithm \cite{guermond_secondorder}; the high-order scheme component in FCT,
traditionally the unmodified Galerkin scheme, is replaced with the entropy
viscosity scheme.
For the low-order
scheme, Guermond also introduced
a discrete maximum principle (DMP) preserving (and positivity-preserving)
scheme for scalar
conservation laws \cite{guermond_firstorder}.

The algorithm described in this paper takes a similar approach to the algorithm
described in the work by Guermond and Popov for scalar conservation laws,
but is extended to allow application to the transport equation,
which does not fit the precise definition of a conservation law but is instead a
balance law since it includes sinks and sources, namely the reaction term
$\totalxsec\aflux$ and the source term $\Qtot$. The presence
of these terms is also a novelty in the context of the FCT algorithm.
In addition, much of the work on FCT to date has been for fully explicit time
discretizations. Because speeds in radiation transport (such as the speed of light)
are so large, implicit and steady-state time discretization are important
considerations, given the CFL time step size restriction for fully explicit
methods. Thus this paper also considers implicit and steady-state FCT.

This paper is organized as follows. Section \ref{sec:preliminaries} gives
some preliminaries such as the problem formulation and discretization.
Recall that the FCT algorithm uses a low-order scheme and a high-order scheme.
Section \ref{sec:low} presents the low-order scheme, Section \ref{sec:high}
presents the high-order scheme (which is based on entropy viscosity),
and Section \ref{sec:fct} presents the FCT scheme that combines the two. Section
\ref{sec:results} presents results for a number of test problems, and
Section \ref{sec:conclusions} gives conclusions.
