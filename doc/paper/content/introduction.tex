2-4 pages
\begin{itemize}
\item applications of transport equation
\begin{itemize}
\item reactors, medical, astrophys, HEDP (high energy density physics ....)
\item  in this paper, we focus on solution techniques applicable to the first-order form of the TE discretized with DO (Sn) in angle. Sn because it decouples directions when using iteration techniques for scattering source (references DO, SI)
\item usually, Sn transport people use DG (Lessaint/Raviart 1972, Reed/Hill 1973). Here, we opted for CFEM: (1) proof of concept of FCT-CFEM for transport abundant lit on this (Kuzmin), (2) recent work by JLG/Po on stabilization technique for conserva laws used CFEM, (3) once demonstrated, there are ways to pass to DG (Zingan LO, EV ingredients are there, FCT-DG  FCT algebraic ???)
\end{itemize}

\item previous work on positivity-preservation
\begin{itemize}
\item transport community: fix-ups, Maginots CSZ, BCSZ, Warsa LANL, 
\item 
\item 
\end{itemize}
\item FCT: hydro, old idea, new spin algebraic FCT Kuzmin. DMP. What is FCT: Lo/HO. novelty: HO -> EV
\item entropy viscosity.
\item a lot of FCT out there is EXPLICIT. in radiation transport, SS, Impl. new and needs to be investigated
\item emphasis transport = advection + reaction + source term. new as well in FCT world.
\end{itemize}

outline paper:




The transport equation, also called the Boltzmann equation, describes the
transport of particles or waves through some background media and some
of its applications include nuclear reactors, atmospheric science, radiation
therapy, astrophysics, radiation shielding, and high energy density physics.
In this paper, focus is on solution techniques applicable to the first-order
form of the transport equation, discretized in angle with discrete ordinates,
which gives what is commonly called the $S_N$ equations:
\begin{equation}
  \frac{1}{v(E)}\ppt{\aflux} + \di\cdot\nabla\aflux\xdet
    + \totalxsec\xet\aflux\xdet = \Qtot\xdet
  \eqc
\end{equation}
where $\Qtot\xdet$ denotes the sum of the extraneous source, prompt and delayed
fission sources, and scattering source:
\begin{multline}
  \Qtot\xdet \equiv \Qext\xdet\\
    + \frac{\chi_\text{p}(E)}{4\pi}\int\limits_0^\infty
      dE'\nu_\text{p}(\x,E',t)\fissionxsec(\x,E',t)\phi(\x,\di,E',t)
    + \sum\limits_{i=1}^{n_\text{d}}\frac{\chi_{\text{d},i}(E)}{4\pi}\lambda_i C_i\xt\\
    + \int\limits_0^\infty dE'\int\limits_{4\pi}d\di'
      \scatteringxsec(\x,E'\rightarrow E,\di'\rightarrow\di,t)\aflux(\x,\di',E',t)
  \eqp
\end{multline}
The $S_N$ equations are an attractive form of the transport equation because
the $S_N$ equations can be decoupled by using iterative techniques for the
scattering source, an approach called source iteration \cite{glasstone}:
\begin{equation}
  \frac{1}{\speed}\ppt{\angularflux^{(\ell)}}
    + \directionvector\cdot\nabla\angularflux^{(\ell)}
    + \totalcrosssection\angularflux^{(\ell)} = \radiationsource^{(\ell-1)} \eqc
\end{equation}
where $\ell$ is the iteration index. The decoupling of the equations allows
scalar solution techniques to be leveraged.
Traditionally, the preferred spatial discretization method for the $S_N$
equations is the Discontinuous Galerkin finite element method (DGFEM)
\cite{Lesaint1974}\cite{Reed_Hill_1973}. Here, however, the
Continuous Galerkin finite element method (CGFEM) is applied. There
has been some recent work by Guermond and Popov \cite{guermond_ev} on
solution techniques for conservation laws with CGFEM, which addresses some
of the main disadvantages of CGFEM versus DGFEM, including the formation
of spurious oscillations. This work aims to demonstrate a proof of concept
for the application of these solution techniques to the transport equation.
Furthermore, some or all of the methodology explored in this paper may be
later extended to DGFEM as well \cite{zingan_2013}.





