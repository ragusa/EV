The transport equation, also called the Boltzmann equation, describes the
transport of particles or waves through some background media and some
of its applications include nuclear reactors, atmospheric science, radiation
therapy, astrophysics, radiation shielding, and high energy density physics.
In this paper, focus is on solution techniques applicable to the first-order
form of the transport equation, discretized in angle with discrete ordinates,
which gives what is commonly called the $S_N$ equations:
\begin{equation}\label{eq:transport_scalar}
  \frac{1}{v(E)}\ppt{\aflux} + \di\cdot\nabla\aflux\xdet
    + \totalxsec\xet\aflux\xdet = \Qtot\xdet
  \eqc
\end{equation}
where $\Qtot\xdet$ denotes the sum of the extraneous source, prompt and delayed
fission sources, and scattering source:
\begin{multline}
  \Qtot\xdet \equiv \Qext\xdet\\
    + \frac{\chi_\text{p}(E)}{4\pi}\int\limits_0^\infty
      dE'\nu_\text{p}(\x,E',t)\fissionxsec(\x,E',t)\phi(\x,\di,E',t)
    + \sum\limits_{i=1}^{n_\text{d}}\frac{\chi_{\text{d},i}(E)}{4\pi}\lambda_i C_i\xt\\
    + \int\limits_0^\infty dE'\int\limits_{4\pi}d\di'
      \scatteringxsec(\x,E'\rightarrow E,\di'\rightarrow\di,t)\aflux(\x,\di',E',t)
  \eqp
\end{multline}
The $S_N$ equations are an attractive form of the transport equation because
the $S_N$ equations can be decoupled by using iterative techniques for the
scattering source, an approach called source iteration \cite{glasstone}:
\begin{equation}
  \frac{1}{v}\ppt{\aflux^{(\ell)}}
    + \di\cdot\nabla\aflux^{(\ell)}
    + \totalxsec\aflux^{(\ell)} = \Qtot^{(\ell-1)} \eqc
\end{equation}
where $\ell$ is the iteration index. The decoupling of the equations allows
scalar solution techniques to be leveraged.
Traditionally, the preferred spatial discretization method for the $S_N$
equations is the Discontinuous Galerkin finite element method (DGFEM)
\cite{Lesaint1974}\cite{Reed_Hill_1973}. Here, however, the
Continuous Galerkin finite element method (CGFEM) is applied. There
has been some recent work by Guermond and Popov \cite{guermond_ev} on
solution techniques for conservation laws with CGFEM, which addresses some
of the main disadvantages of CGFEM versus DGFEM, including the formation
of spurious oscillations. This work aims to demonstrate a proof of concept
for the application of these solution techniques to the transport equation.
Furthermore, some or all of the methodology explored in this paper may be
later extended to DGFEM as well \cite{zingan_2013}.

One of the main objectives of this paper is to present a method that precludes
the formation of spurious oscillations and the negativities that result from
these oscillations; these issues have been a long-standing issue in the
numerical solution of the transport equation \cite{lanthrop}.
Not only are these negativities physically inaccurate, but they can cause
simulations to terminate prematurely. Many attempts to remedy this
issue rely on ad-hoc fix-ups, such as the set-to-zero fix-up for the
classic diamond difference scheme \cite{lewis}. Recent work by Hamilton
introduced a similar fix-up for the linear discontinuous finite element
method (LDFEM) that conserves local balance and preserves third-order accuracy.
Walters and Wareing developed characteristic methods \cite{walters_NC}, but
Wareing later notes that these characteristic methods are difficult to
implement and offers a nonlinear positive spatial differencing scheme
known as the exponential discontinuous scheme \cite{wareing}.
Maginot has recently developed a consistent set-to-zero (CSZ) LDFEM
method \cite{maginot}, as well as a non-negative method for bilinear
discontinuous FEM \cite{maginot_mc2015}.

Traditional approaches to remedy the spurious oscillation issue included
the flux-corrected transport (FCT) algorithm, introduced in 1973 for finite
difference discretizations
by Boris and Book \cite{borisbook}, which has since been applied to the finite
element method \cite{kuzmin_FCT}. The idea of FCT is to blend a low-order scheme
having desirable properties with a scheme of a higher order of accuracy.

Recent work by Guermond and Popov addresses the issue of spurious oscillations
for general conservation laws by using artificial dissipation based on
local entropy production, a method known as entropy viscosity \cite{guermond_ev}.
The idea of entropy viscosity is to enforce an entropy inequality on the weak solution,
and thus filter out weak solutions containing spurious oscillations. However,
entropy viscosity solutions may still contain spurious
oscillations, albeit smaller in magnitude, and consequently negativities
are not precluded. To circumvent this deficiency, Guermond proposed using
the entropy viscosity method in conjunction with the FCT
algorithm \cite{guermond_secondorder}; the high-order scheme component in FCT,
traditionally the unmodified Galerkin scheme, is replaced with the entropy
viscosity scheme.
For the low-order
scheme, Guermond also introduced
a discrete maximum principle (DMP) preserving (and positivity-preserving)
scheme for scalar
conservation laws \cite{guermond_firstorder}.

This paper presents an FCT scheme that is largely rooted in the work by Guermond
and Popov, but is extended to allow application to the transport equation,
which does not fit the prototype of a conservation law but is instead a
balance law, which includes sinks and sources, namely the reaction term
$\totalxsec\aflux$ and the source term $\Qtot$. The presence
of these terms is also a novelty in the context of the FCT algorithm.
In addition, much of the present work on FCT has been for fully explicit time
discretizations, although there has been some work on implicit time discretizations
as well. Because speeds in radiation transport (such as the speed of light)
are so large, implicit and steady-state time discretization are important
considerations, given the CFL time step size restriction for fully explicit
methods. Thus this paper also considers implicit and steady-state FCT, which
has been implemented before \cite{implicit_FCT}.

This paper is organized as follows. Section \ref{sec:preliminaries} gives
some preliminaries such as the problem formulation and discretization.
Recall that the FCT algorithm uses a low-order scheme and a high-order scheme.
Section \ref{sec:low} presents the low-order scheme, Section \ref{sec:high}
presents the high-order scheme (which is based on entropy viscosity),
and Section \ref{sec:fct} presents the FCT scheme that combines the two. Then, Section
\ref{sec:results} presents results for a number of test problems, and
Section \ref{sec:conclusions} gives conclusions.
