% !TEX root = ../paper.tex

This section describes the entropy viscosity method applied to the scalar
PDE given by Equation \eqref{eq:scalar_model}. Recall that the entropy
viscosity method is to be used as the high-order scheme in the FCT algorithm,
instead of the standard Galerkin method, as has been done previously
\cite{kuzmin_FCT}.
Usage of the entropy viscosity method in the FCT algorithm ensures convergence
to the entropy solution \cite{guermond_secondorder}.

The entropy viscosity method has been applied to a number of PDEs
such as general scalar conservation laws of the form
\begin{equation}\label{eq:scalar_conservation_law}
  \ppt{u} + \nabla\cdot\mathbf{f}(u) = 0
\end{equation}
and the inviscid Euler equations \cite{guermond_ev}. The scalar model studied
in this paper does not fit into the category of scalar conservation laws;
instead, it is more accurately described as a \emph{balance} law, due to
the addition of the reaction term $\sigma u$ and source term $q$. Application
of entropy viscosity to this equation is novel in this work.

Since the weak form of the problem does not have a unique solution, one
must supply additional conditions called \emph{admissibility} conditions or
\emph{entropy} conditions to filter out spurious weak solutions, leaving
only the physical weak solution, often called the entropy solution.
A number of entropy conditions are valid, but usually the most convenient
entropy condition for use in numerical methods takes the form of an
\emph{entropy inequality}, such as the following, which is valid for the
general scalar conservation law given by Equation \eqref{eq:scalar_conservation_law}:
\begin{equation}
  \ppt{\eta(u)} + \nabla\cdot\mathbf{\Psi}(u) \leq 0 \eqc
\end{equation}
which holds for any convex entropy function $\eta(u)$ and associated entropy
flux $\mathbf{\Psi}(u) \equiv \int \eta'(u)\mathbf{f}'(u)du$.
If one can show that this inequality holds for an arbitrary
convex entropy function, then one proves it holds for all convex entropy
functions  \cite{leveque2002}\cite{guermond_ev}.
For the scalar PDE considered in this paper, the entropy inequality becomes
the following:
\begin{equation}
  \ppt{\eta(u)} + \nabla\cdot\mathbf{\Psi}(u) + \eta'(u)\sigma u - \eta'(u)q
    \leq 0 \eqp
\end{equation}
One can verify this inequality by multiplying the governing PDE by $\eta'(u)$
and applying reverse chain rule.

The entropy viscosity method enforces the entropy inequality by measuring
local entropy production and dissipating accordingly. In practice, one
defines the entropy residual:
\begin{equation}
  \mathcal{R}(u) \equiv \ppt{\eta(u)} + \nabla\cdot\mathbf{\Psi}(u)
    + \eta'(u)\sigma u - \eta'(u)q \eqc
\end{equation}
which can be viewed as the amount of violation of the entropy inequality.
The entropy viscosity for an element $K$ is then defined to be proportional
to this violation, for example:
\begin{equation}
  \nu^\eta_K = \frac{c_\mathcal{R}\|\mathcal{R}(u_h)\|_{L^\infty(K)}}
    {\hat{\eta}_K}
    \eqc
\end{equation}
where $\hat{\eta}_K$ is a normalization constant with the units of entropy,
$c_\mathcal{R}$ is a proportionality constant that can be modulated for
each problem, and $\|\mathcal{R}(u_h)\|_{L^\infty(K)}$ is the maximum of the
entropy residual on element $K$, which can be approximated as the maximum over
the quadrature points of element $K$. In addition to the entropy residual, it can
also be beneficial to measure the jump in the gradient of the entropy flux
across interfaces.
Note that given the definition of the entropy flux, the gradient of the entropy
flux is $\nabla\mathbf{\Psi}(u)=\nabla\eta(u)\mathbf{f}'(u)$. Then let
$\mathcal{J}_F$ denote the jump of the normal component of the entropy flux
gradient across face $F$:
\begin{equation}
  \mathcal{J}_F \equiv |\mathbf{f}'(u)\cdot\mathbf{n}_F|
    [\![\nabla\eta(u)\cdot\mathbf{n}_F]\!] \eqc
\end{equation}
where the double square brackets denote a jump quantity. Then define the
maximum jump on a cell:
\begin{equation}
  \mathcal{J}_K \equiv \max\limits_{F\in\mathcal{F}(K)} |\mathcal{J}_F| \eqp
\end{equation}
Finally, putting everything together, one can define the entropy viscosity
for a cell $K$ to be
\begin{equation}
  \nu^\eta_K = \frac{c_\mathcal{R}\|\mathcal{R}(u_h)\|_{L^\infty(K)}
    + c_\mathcal{J}\mathcal{J}_K}{\hat{\eta}_K}
    \eqp
\end{equation}
However, it is known that the low-order viscosity for an element, as computed
in Section \ref{sec:low}, gives enough local artificial diffusion for
regularization; any excess of this viscosity would be excessive and further
degrade spatial accuracy. Thus, the low-order viscosity for an element is
imposed as the upper bound for the high-order viscosity:
\begin{equation}
  \nu^H_K \equiv \min(\nu^L_K, \nu^\eta_K) \eqp
\end{equation}
One should see that in smooth regions, this high-order viscosity should be
relatively small, and in regions of strong gradients or discontinuities,
the entropy viscosity should be relatively large, ideally approximately
as large as the low-order viscosity. Note that the latter condition can
be achieved in part by tuning the parameters
$c_\mathcal{R}$ and $c_\mathcal{J}$ for the particular problem.

Putting everything together, the high-order system for steady-state,
explicit Euler, and theta time discretizations are respectively expressed
as follows:
\begin{equation}\label{eq:high_ss}
  \A^H \U = \ssrhs \eqc
\end{equation}
\begin{equation}\label{eq:high_fe}
  \M^C\frac{\U^{n+1} - \U^n}{\dt} + \A^{H,n}\U^n = \ssrhs^n \eqc
\end{equation}
\begin{equation}\label{eq:high_theta}
  \M^C\frac{\U^{n+1} - \U^n}{\dt} + \theta\A^{H,n+1}\U^{n+1}
    + (1-\theta)\A^{H,n}\U^n
    = \ssrhs^\theta \eqc
\end{equation}
where the high-order steady-state system matrix is defined as
$\A^H\equiv\A + \D^H$, and the high-order diffusion matrix is defined similarly
to the low-order case but using the high-order viscosity instead of the low-order
viscosity:
\begin{equation}\label{eq:high_order_diffusion_matrix}
  D_{i,j}^H \equiv
    \sum\limits_{K\subset S_{i,j}}\nu^H_K
    d_K(\test_j,\test_i) \eqp
\end{equation}
Note that unlike the low-order scheme, the high-order scheme does not lump the
mass matrix.

\begin{rmk}
Note that due to the nonlinearity of the entropy viscosity, the entropy viscosity
scheme is nonlinear for implicit and steady-state temporal discretizations, and
thus some nonlinear solution technique must be utilized. For the results in
this paper, a simple fixed-point iteration scheme is used. An alternative
such as Newton's method is likely to be advantageous in terms of the number
of nonlinear iterations; however, fixed-point is used here for comparison
with the nonlinear FCT scheme to be described in Section \ref{sec:fct}.
\end{rmk}
