% !TEX root = ../FCT_radiation_paper.tex

This is immediate for the steady-state case due to the assumption that the
source $q$ is non-negative. The following theorem gives that the system
right-hand-side vector for the theta system is non-negative. This theorem
extends to explicit Euler discretization since explicit Euler is a special case
of the Theta discretization.
%-------------------------------------------------------------------------------
\begin{thm}{Non-Negativity of the Theta Low-Order System Right-Hand-Side:}
  If the old solution $\U^n$ is non-negative and
  the time step size $\dt$ satisfies
\begin{equation}\label{eq:theta_cfl}
   \dt \leq \frac{M^L_{i,i}}{(1-\theta)A_{i,i}^L}
    \eqc\quad\forall i \eqc
\end{equation}
  then the new solution $\U^{L,n+1}$ of the Theta low-order
  system given by Equation \eqref{eq:low_theta} is non-negative, i.e.,
  $U^{L,n+1}_i \geq 0$, $\forall i$.
\end{thm}

\begin{prf}
The right-hand-side vector $\mathbf{y}$ of this system has the entries
\[
  y_i = \dt b^\theta_i + \pr{M^L_{i,i} - (1-\theta)\dt A^L_{i,i}} U^n_i
      - (1-\theta)\dt\sumjnoti A^L_{i,j} U^n_j
  \eqp
\]
% It now just remains to prove that these entries are non-negative.
As stated previously, the source function $q$ is non-negative and
thus $b^\theta_i \ge 0 $. Due to the time step size assumption
given by Equation \eqref{eq:theta_cfl},
\[
  M^L_{i,i} - (1-\theta)\dt A^L_{i,i} \geq 0 \eqc
\]
and because the off-diagonal terms of $\A^L$ are non-positive, the off-diagonal
sum term is non-negative. Thus $y_i$ is a sum of non-negative
terms, and the theorem is proven.
\end{prf}
%-------------------------------------------------------------------------------
