% !TEX root = ../paper.tex

\subsubsection{Limiting Coefficients}\label{sec:limiter}

The results in this paper use the classic multi-dimensional limiter introduced by Zalesak
\cite{zalesak}. Recall that a limiter takes as input only the antidiffusion bounds
$\bar{p}_i^\pm$ and the antidiffusive fluxes $P_{i,j}$ and outputs the
limiting coefficients $L_{i,j}$. \tcr{that recall sentence is a repeat. remove?}
The Zalesak limiter gives the following
definition of the limiting coefficients:
\begin{subequations}
\begin{equation}\label{eq:flux_sums}
   p_i^+ \equiv \sumj\max(0,P_{i,j}) \eqc\qquad
   p_i^- \equiv \sumj\min(0,P_{i,j}) \eqc
\end{equation}
\begin{equation}\label{eq:single_node_limiting_coefficients}
   L_i^\pm \equiv\left\{
      \begin{array}{l l}
         1 & p_i^\pm = 0\\
         \min\left(1,\frac{\bar{p}_i^\pm}
           {p_i^\pm}\right) & p_i^\pm
           \ne 0
      \end{array}
      \right. \eqc
\end{equation}
\begin{equation}\label{eq:limiting_coefficients}
   L_{i,j} \equiv\left\{
      \begin{array}{l l}
         \min(L_i^+,L_j^-)
           & P_{i,j} \geq 0\\
         \min(L_i^-,L_j^+)
           & P_{i,j} < 0
      \end{array}
      \right. \eqp
\end{equation}
\end{subequations}
The objective of a limiter is to maximize the amount of antidiffusion that
can be accepted without violating the imposed solution constraints. Zalesak's
limiter is one commonly used attempt at this objective due to its relatively
simple form; however, there are possible alternatives to this limiter that
could accept more antidiffusion without violating the imposed constraints
\tcr{you should cite a couple of examples here. one that comes to mind is the
PhD student from JLG from about the time you graduated}.
Finally, one could pass antidiffusive fluxes through a given limiter multiple
times, using the remainder antidiffusive flux as the input in each pass,
to increase the total antidiffusion accepted. \tcr{another reference here. doesn't kuzmin do this as well?}
