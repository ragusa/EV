% !TEX root = ../FCT_radiation_paper.tex

\subsubsection{Limiting Coefficients}\label{sec:limiter}

The results in this paper use the classic multi-dimensional limiter introduced by Zalesak
\cite{zalesak}:
\begin{subequations}
\begin{equation}\label{eq:flux_sums}
   p_i^+ \equiv \sumj\max(0,P_{i,j}) \eqc\qquad
   p_i^- \equiv \sumj\min(0,P_{i,j}) \eqc
\end{equation}
\begin{equation}\label{eq:single_node_limiting_coefficients}
   L_i^\pm \equiv\left\{
      \begin{array}{l l}
         1 & p_i^\pm = 0\\
         \min\left(1,\frac{\bar{p}_i^\pm}
           {p_i^\pm}\right) & p_i^\pm
           \ne 0
      \end{array}
      \right. \eqc
\end{equation}
\begin{equation}\label{eq:limiting_coefficients}
   L_{i,j} \equiv\left\{
      \begin{array}{l l}
         \min(L_i^+,L_j^-)
           & P_{i,j} \geq 0\\
         \min(L_i^-,L_j^+)
           & P_{i,j} < 0
      \end{array}
      \right. \eqp
\end{equation}
\end{subequations}
The objective of a limiter is to maximize the amount of antidiffusion that
can be accepted without violating the imposed solution constraints. Zalesak's
limiter is one commonly used attempt at this objective due to its relatively
simple form.
\begin{rmk}
Note that it is possible to devise limiters accepting more
antidiffusion than Zalesak's limiter, but one must sacrifice the simple,
closed form of Zalesak's limiter and adapt a sequential algorithm
for computing the antidiffusion coefficients. This approach has not been
found in literature, most likely because the sequential aspect of the algorithm
makes it node-order-dependent, which reduces reproducibility between
implementations.
\end{rmk}
Finally, one could pass antidiffusive fluxes through a given limiter multiple
times, using the remainder antidiffusive flux as the input in each pass,
to increase the total antidiffusion accepted \cite{kuzmin_FCT,schar}; however,
the results presented in this paper were all produced using the traditional
single-pass approach through the Zalesak limiter.
