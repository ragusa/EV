% !TEX root = ../paper.tex

The role of a low-order scheme in the context of the FCT algorithm is to
provide a fail-safe solution, which has desirable properties such as
positivity-preservation and lack of spurious oscillations. These properties
come at the cost of excessive artificial diffusion and thus a lesser degree
of accuracy. However, the idea of the FCT algorithm is to undo some of the over-dissipation 
of the low-order scheme as much as possible without violating some physically-motivated solution bounds.

Here positivity-preservation and monotonicity are achieved by 
requiring that that the matrix of the low-order system satisfies the M-matrix property.
%satisfying
% the M-matrix property of the matrix inverted in the low-order system.
M-matrices are a subset of inverse-positive matrices and have the monotone
property. For instance, consider the linear system $\A\x = \ssrhs$;
If $\A$ is an M-matrix, then the following property is verified:
\begin{equation}
\text{If }  \ssrhs \geq 0, \text{  then  }  \x \geq 0 \eqp
\end{equation}
Hence, to prove positivity-preservation, given that the linear system matrix is an
M-matrix, is achieved by proving positivity of the right-hand-side vector $\ssrhs$.
This monotonicity property of the linear system matrix is also responsible for
the lack of spurious oscillations.

In this section, a first-order viscosity method introduced by Guermond
\cite{guermond_firstorder} will be adapted to the transport equation given by
Equation \eqref{eq:scalar_model}. This method uses an element-wise artificial
viscosity definition in conjunction with a graph-theoretic local viscous
bilinear form that makes the method valid for arbitrary element shapes and
dimensions. These definitions will be shown to ensure that the system matrix
is a non-singular M-matrix.

The graph-theoretic local viscous bilinear form has the following definition.
%-------------------------------------------------------------------------------
\begin{defn}[Local Viscous Bilinear Form]
   The local viscous bilinear form for element $K$ is defined as follows:
   \begin{equation}\label{eq:bilinearform}
     d_K(\test_j,\test_i) \equiv \left\{\begin{array}{l l}
       -\frac{1}{n_K - 1}V_K & i\ne j\eqc \quad i,j\in \indices_K\eqc \\
       V_K                   & i = j \eqc \quad i,j\in \indices_K\eqc \\
       0                     & \mbox{otherwise}\eqc
     \end{array}\right.
   \end{equation}
   where $V_K$ is the volume of cell $K$, $\indices_K$ is the set of degree
   of freedom indices such that the corresponding test function has support
   on cell $K$, and $n_K$ is the number of indices in that set.
\end{defn}
%-------------------------------------------------------------------------------
This bilinear form bears resemblance to a standard Laplacian bilinear form:
the diagonal entries are positive, the off-diagonal entries are negative, and
the row sums are zero. These facts will be invoked in the proof of the M-matrix
conditions later in this section.

The element-wise low-order viscosity definition from \cite{guermond_firstorder} is
adapted to account for the reaction term in the transport equation, 
Equation \eqref{eq:scalar_model}, but otherwise remains unchanged.
%-------------------------------------------------------------------------------
\begin{defn}[Low-Order Viscosity]
  The low-order viscosity for cell $K$ is defined as follows:
  \begin{equation}
    \nu^L_K \equiv \max\limits_{i\ne j\in\indices_K}
      \frac{\max(0,A_{i,j})}
      {-\sum\limits_{T\in \mathcal{K}(S_{i,j})}d_T(\test_j,\test_i)}
      \eqc
  \end{equation}
  where $A_{i,j}$ is the $i,j$ entry of matrix $\A$ 
  given by Equation \eqref{eq:Aij}, $\indices_K$ is the set of degree of freedom
  indices corresponding to basis functions that have support on cell $K$
  (this is illustrated in Figure \ref{fig:cell_indices} -- the indicated
  nodes have degree of freedom indices belonging to $\indices_K$), and
  $\mathcal{K}(S_{i,j})$ is the set of cell indices for which the cell
  domain and the shared support $S_{i,j}$ overlap.
\end{defn}
%-------------------------------------------------------------------------------
%-------------------------------------------------------------------------------
\begin{figure}[ht]
   \centering
     \begin{tikzpicture}[
  scale=1]

\def\pointsize{2pt}

\coordinate (p1) at (1,1);
\coordinate (p2) at (2.1,0.4);
\coordinate (p3) at (3.3,0.4);
\coordinate (p4) at (4.3,0.9);
\coordinate (p5) at (5.2,1.3);
\coordinate (p6) at (5.8,2.1);
\coordinate (p7) at (5.8,3.5);
\coordinate (p8) at (5.2,4);
\coordinate (p9) at (4.5,4.8);
\coordinate (p10) at (3.6,5);
\coordinate (p11) at (2.7,4.5);
\coordinate (p12) at (1.6,4.6);
\coordinate (p13) at (0.8,3.5);
\coordinate (p14) at (1.2,2.5);
\coordinate (p15) at (2,1.5);
\coordinate (p16) at (3.3,1.2);
\coordinate (p17) at (4,1.9);
\coordinate (p18) at (5,2.2);
\coordinate (p19) at (5,3.3);
\coordinate (p20) at (4.2,3.8);
\coordinate (p21) at (3.3,3.6);
\coordinate (p22) at (2,3.7);
\coordinate (i) at (3,2.5);
\coordinate (j) at (4,3);

%\fill (p1) circle (\pointsize);
%\fill (p2) circle (\pointsize);
%\fill (p3) circle (\pointsize);
%\fill (p4) circle (\pointsize);
%\fill (p5) circle (\pointsize);
%\fill (p6) circle (\pointsize);
%\fill (p7) circle (\pointsize);
%\fill (p8) circle (\pointsize);
%\fill (p9) circle (\pointsize);
%\fill (p10) circle (\pointsize);
%\fill (p11) circle (\pointsize);
%\fill (p12) circle (\pointsize);
%\fill (p13) circle (\pointsize);
%\fill (p14) circle (\pointsize);
%\fill (p15) circle (\pointsize);
%\fill (p16) circle (\pointsize);
%\fill (p17) circle (\pointsize);
%\fill (p18) circle (\pointsize);
%\fill (p19) circle (\pointsize);
%\fill (p20) circle (\pointsize);
%\fill (p21) circle (\pointsize);
%\fill (p22) circle (\pointsize);
%\fill (i) circle (\pointsize);
%\fill (j) circle (\pointsize);

\draw (p1) -- (p2);
\draw (p1) -- (p14);
\draw (p1) -- (p15);
\draw (p2) -- (p3);
\draw (p2) -- (p15);
\draw (p2) -- (p16);
\draw (p3) -- (p4);
\draw (p3) -- (p16);
\draw (p4) -- (p5);
\draw (p4) -- (p16);
\draw (p4) -- (p17);
\draw (p4) -- (p18);
\draw (p5) -- (p6);
\draw (p5) -- (p18);
\draw (p6) -- (p7);
\draw (p6) -- (p18);
\draw (p6) -- (p19);
\draw (p7) -- (p8);
\draw (p7) -- (p19);
\draw (p8) -- (p9);
\draw (p8) -- (p19);
\draw (p8) -- (p20);
\draw (p9) -- (p10);
\draw (p9) -- (p20);
\draw (p10) -- (p11);
\draw (p10) -- (p20);
\draw (p10) -- (p21);
\draw (p11) -- (p12);
\draw (p11) -- (p21);
\draw (p11) -- (p22);
\draw (p12) -- (p13);
\draw (p12) -- (p22);
\draw (p13) -- (p14);
\draw (p13) -- (p22);
\draw (p14) -- (p15);
\draw (p14) -- (p22);
\draw (p15) -- (i);
\draw (p15) -- (p16);
\draw (p15) -- (p22);
\draw (p16) -- (i);
\draw (p16) -- (p17);
\draw (p17) -- (i);
\draw (p17) -- (j);
\draw (p17) -- (p18);
\draw (p18) -- (j);
\draw (p18) -- (p19);
\draw (p19) -- (j);
\draw (p19) -- (p20);
\draw (p20) -- (j);
\draw (p20) -- (p21);
\draw (p21) -- (i);
\draw (p21) -- (j);
\draw (p21) -- (p22);
\draw (p22) -- (i);
\draw (i) -- (j);



\draw[draw=none, fill=red, fill opacity=0.5] (p15)--(p16)--(p17)--(j)
  --(p21)--(p22)--cycle;
\draw[draw=none, fill=blue, fill opacity=0.5] (p17)--(p18)--(p19)--(p20)
  --(p21)--(i)--cycle;
\draw[draw=none, fill=yellow, fill opacity=0.5] (p16)--(i)--(j)--(p18)
  --(p4)--cycle;

\node[black] at ($0.333*(i) + 0.333*(j) + 0.333*(p17)$) {\Large $K$};

\fill[black] (i) circle (\pointsize);
\fill[black] (j) circle (\pointsize);
\fill[black] (p17) circle (\pointsize);

\end{tikzpicture}

      \caption{Illustration of Cell Degree of Freedom Indices $\indices_K$}
   \label{fig:cell_indices}
\end{figure}
%-------------------------------------------------------------------------------
This viscosity definition is designed to give the minimum amount of artificial
diffusion without violating the M-matrix conditions.

Now that the low-order artificial diffusion operator (bilinear form + viscosity definitions) 
has been provided, we describe the low-order system.
Consider a modification of the Galerkin scheme given in Equation \eqref{eq:galerkin_semidiscrete}
which lumps the mass matrix ($\M^C \rightarrow \M^L$) and adds an artificial
diffusion operator $\D^L$, hereafter called the low-order diffusion matrix:
\begin{equation}
  \M^L\ddt{\U^L} + (\A + \D^L)\U(t) = \ssrhs(t) \eqc
\end{equation}
where $\U^L(t)$ denotes the discrete low-order solution values.
Defining the low-order steady-state system matrix $\A^L\equiv\A + \D^L$,
the low-order system for the steady-state system, explicit Euler system,
and Theta system, respectively, are
\begin{subequations}
Steady-state:
\begin{equation}\label{eq:low_ss}
  \A^L \U = \ssrhs \eqc
\end{equation}
Explicit Euler:
\begin{equation}\label{eq:low_fe}
%  \M^L\frac{\U^{n+1} - \U^n}{\dt} + \A^L\U^n = \ssrhs^n \eqc
   \M^L \U^{n+1} = \M^L \U^n + \dt \pr{ \ssrhs^n - \A^L\U^n } \eqc
\end{equation}
Theta scheme:
\begin{equation}\label{eq:low_theta}
  \pr{ \M^L +\theta \dt \A^L} U^{n+1}
    = \M^L \U^n  + \dt( \ssrhs^\theta  - \A^L (1-\theta)\U^n )\eqc
%  \M^L\frac{\U^{n+1} - \U^n}{\dt} + \A^L\pr{\theta\U^{n+1} + (1-\theta)\U^n}
%    = \ssrhs^\theta \eqc
\end{equation}
\end{subequations}
where $\ssrhs^\theta \equiv (1-\theta)\ssrhs^n + \theta\ssrhs^{n+1}$.
The low-order
diffusion matrix is assembled element-wise using the local viscous bilinear
form and low-order viscosity definitions:
\begin{equation}\label{eq:low_order_diffusion_matrix}
  D_{i,j}^L \equiv
    \sum\limits_{K\in \mathcal{K}(S_{i,j})}\nu^L_K
    d_K(\test_j,\test_i) \eqp
\end{equation}

Now the low-order scheme has been fully described, some statements will be
made on its properties. Firstly the M-matrix property will be shown for the low-order
matrix $\A^L$.
%-------------------------------------------------------------------------------
\begin{thm}[M-matrix property]
  The low-order steady-state system matrix $\A^L$ is a non-singular M-matrix.
\end{thm}
In this section, it will be shown that the low-order steady-state system matrix
defined in Equation \eqref{eq:low_order_ss_matrix} is an M-matrix, which allows
a discrete maximum principle for the low-order solution to be proven in Section
\ref{sec:DMP}.
%--------------------------------------------------------------------------------
\begin{lemma}[lem:offdiagonalnegative]{Non-Positivity of Off-Diagonal Elements}
   The off-diagonal elements of the linear system matrix are non-positive:
   \[
     \ssmatrixletter^{L,\timeindex}\ij\le 0, \quad j\ne i \eqp
   \]
\end{lemma}

\begin{proof}
This proof begins by bounding the term $\diffusionmatrixletter\ij^{L,\timeindex}$:
\begin{eqnarray*}
   \diffusionmatrixletter\ij^{L,\timeindex}=
     \sumKSij\mkern-20mu\lowordercellviscosity[\timeindex]
   \localviscbilinearform{\cell}{j}{i}
   & = & \sumKSij\max\limits_{k\ne \ell\in \indicescell}
     \pr{\frac{\max(0,\ssmatrixletter_{k,\ell}^\timeindex)}
       {\mkern10mu-\mkern-20mu\sum\limits_{T:\celldomain[T]\subset\support_{k,\ell}}
       \mkern-20mu\localviscbilinearform{T}{\ell}{k}}}
     \localviscbilinearform{\cell}{j}{i} \eqp
\end{eqnarray*}
Since $\localviscbilinearform{\cell}{j}{i} < 0$ for $j\ne i$ and $y_i \leq
\max_j y_j$,
\begin{eqnarray*}
   \diffusionmatrixletter\ij^{L,\timeindex} & \le &
     \sumKSij \frac{\max(0,\ssmatrixletter\ij^\timeindex)}
   {\mkern10mu-\mkern-20mu\sumKSij[T]\mkern-20mu\localviscbilinearform{T}{j}{i}}
   \localviscbilinearform{\cell}{j}{i} \eqc\\
   &  =  & -\max(0,\ssmatrixletter\ij^\timeindex)
     \frac{\sumKSij\mkern-20mu\localviscbilinearform{\cell}{j}{i}}
     {\sumKSij[T]\mkern-20mu\localviscbilinearform{T}{j}{i}} \eqc\\
   &  =  & -\max(0,\ssmatrixletter\ij^\timeindex) \eqc\\
   & \le & -\ssmatrixletter\ij^\timeindex \eqp
\end{eqnarray*}
Applying this inequality to Equation \eqref{eq:low_order_ss_matrix} gives
\begin{eqnarray*}
  \ssmatrixletter^{L,\timeindex}\ij &  =  &
    \ssmatrixletter\ij^\timeindex + \diffusionmatrixletter\ij^{L,\timeindex}
    \eqc\\
  \ssmatrixletter^{L,\timeindex}\ij & \le &
    \ssmatrixletter\ij^\timeindex - \ssmatrixletter\ij^\timeindex
    \eqc\\
  \ssmatrixletter^{L,\timeindex}\ij & \le & 0 \eqp \qed
\end{eqnarray*}
\end{proof}
%--------------------------------------------------------------------------------
\begin{lemma}[lem:diagonalpositive]{Non-Negativity of Diagonal Elements}
   The diagonal elements  of the linear system matrix are non-negative:
   \[
     \ssmatrixletter^{L,\timeindex}_{i,i}\ge 0 \eqp
   \]
\end{lemma}

\begin{proof}
The diagonal elements of the low-order system matrix are
\[
  \ssmatrixletter^{L,\timeindex}_{i,i} =
    \intSi\mathbf{\consfluxletter}'(\approximatescalarsolution^\timeindex)\cdot
    \nabla\testfunction_i(\x)\testfunction_i(\x)d\volume
  + \intSi\sigma(\x)\testfunction_i^2(\x)d\volume
  + \sumKSi\mkern-15mu\lowordercellviscosity[\timeindex]
    \localviscbilinearform{\cell}{i}{i}
  \eqp
\]
To prove that $\ssmatrixletter^{L,\timeindex}_{i,i}$ is non-negative, it is sufficient to
prove that each term in the above expression is non-negative. The
non-negativity of the interaction term and viscous term are obvious
($\reactioncoef \ge 0, \, \lowordercellviscosity[\timeindex]\ge 0, \,
\localviscbilinearform{\cell}{i}{i}>0$), but the non-negativity of the divergence
term is not necessarily obvious. On the interior of the domain, the divergence
term gives zero contribution because the divergence integral may be transformed
into a surface integral
$\intSi\mathbf{\consfluxletter}'(\approximatescalarsolution^\timeindex)
\cdot\normalvector\frac{\testfunction_i^2}{2} d\area$ via the
divergence theorem; one can then recognize that the basis function
$\testfunction_i$ evaluates to zero on the boundary of its support
$\support_i$. On the outflow boundary of the domain, the term
$\mathbf{\consfluxletter}'(\approximatescalarsolution^\timeindex)
\cdot\normalvector \frac{\testfunction_i^2}{2}$ is positive because
$\mathbf{\consfluxletter}'(\approximatescalarsolution^\timeindex)
\cdot\normalvector > 0$ for an outflow boundary. This quantity is of
course negative for the inflow boundary, so one must consider the boundary
conditions applied for incoming boundary nodes to determine if this condition
is true and a discrete maximum principle applies. For instance, if a Dirichlet boundary condition is
applied, then a discrete maximum principle does not apply.
strongly imposed on the incoming boundary, so for degrees of freedom $i$ on the
incoming boundary, $\ssmatrixletter^{L,\timeindex}_{i,i}$ will be set equal to some positive
value such as 1 with a corresponding incoming value accounted for in the right
hand side $\ssrhs$ of the linear system.\qed
\end{proof}
%--------------------------------------------------------------------------------
\begin{lemma}{Non-Negativity of Row Sums}
   The sum of all elements in a row $i$ is non-negative:
   \[
     \sumj \ssmatrixletter^{L,\timeindex}\ij \ge 0 \eqp
   \]
\end{lemma}

\begin{proof}
Using the fact that $\sumj\testfunction_j(\x)=1$ and
$\sumj \localviscbilinearform{\cell}{j}{i}=0$,
\begin{eqnarray*}
   \sumj \ssmatrixletter^{L,\timeindex}\ij & = & \sumj \intSij
      \left(\mathbf{\consfluxletter}'(\approximatescalarsolution^\timeindex)
        \cdot\nabla\testfunction_j(\x) +
      \reactioncoef(\x)\testfunction_j(\x)\right)\testfunction_i(\x) d\volume +
      \sumj\sumKSij\mkern-20mu\lowordercellviscosity[\timeindex]\localviscbilinearform{\cell}{j}{i}
      \eqc\\
   & = & \intSi\left(
      \mathbf{\consfluxletter}'(\approximatescalarsolution^\timeindex)\cdot
      \nabla\sumj\testfunction_j(\x) +
      \reactioncoef(\x)\sumj\testfunction_j(\x)\right)
      \testfunction_i(\x) d\volume \eqc\\
   \label{eq:rowsum} & = & \intSi\reactioncoef(\x)\testfunction_i(\x) d\volume
     \eqc\\
   &\ge& 0 \eqp \qed
\end{eqnarray*}
\end{proof}
%--------------------------------------------------------------------------------
\begin{lemma}[lem:diagonallydominant]{Diagonal Dominance}
   $\loworderssmatrix[\timeindex]$ is strictly diagonally dominant:
   \[
     \left|\ssmatrixletter^{L,\timeindex}_{i,i}\right|
     \ge \sumjnoti \left|\ssmatrixletter^{L,\timeindex}\ij\right| \eqp
   \]
\end{lemma}
\begin{proof}
Using the inequalities $\sumj \ssmatrixletter^{L,\timeindex}\ij \ge 0$ and
$\ssmatrixletter^{L,\timeindex}\ij\le 0, j\ne i$, it is proven that
$\loworderssmatrix[\timeindex]$ is strictly diagonally dominant:
\begin{eqnarray*}
  \sumj     \ssmatrixletter^{L,\timeindex}\ij       & \ge & 0 \eqc\\
  \sumjnoti \ssmatrixletter^{L,\timeindex}\ij
    + \ssmatrixletter^{L,\timeindex}_{i,i} & \ge & 0 \eqc\\
  \left|\ssmatrixletter^{L,\timeindex}_{i,i}\right| & \ge &
    \sumjnoti -\ssmatrixletter^{L,\timeindex}\ij
    \eqc\\
  \left|\ssmatrixletter^{L,\timeindex}_{i,i}\right| & \ge
    & \sumjnoti \left|\ssmatrixletter^{L,\timeindex}\ij\right| \eqp \qed
\end{eqnarray*}
\end{proof}
%--------------------------------------------------------------------------------
\begin{lemma}{M-Matrix}
  $\loworderssmatrix[\timeindex]$ is an M-Matrix.
\end{lemma}
\begin{proof}
To prove that a matrix is an M-Matrix, it is sufficient to prove that
the following 3 statements are true:
\begin{enumerate}
\item $\ssmatrixletter^{L,\timeindex}\ij\le 0, j\ne i$,
\item $\ssmatrixletter^{L,\timeindex}_{i,i}\ge 0$,
\item $\left|\ssmatrixletter^{L,\timeindex}_{i,i}\right|
      \ge \sumjnoti \left|\ssmatrixletter^{L,\timeindex}\ij\right|$.
\end{enumerate}
These conditions are proven by Lemmas \ref{lem:offdiagonalnegative},
\ref{lem:diagonalpositive}, and \ref{lem:diagonallydominant}, respectively.\qed
\end{proof}
%--------------------------------------------------------------------------------

%-------------------------------------------------------------------------------
Thus far, we have been proven that the system matrix for the low-order steady-state
system is an M-matrix, and it remains to demonstrate the same for each of the
transient systems. For the explicit Euler/SSPRK systems, the system matrix
is just the lumped mass matrix $\M^L$, which is easily shown to be an M-matrix
since it is a positive, diagonal matrix. For the $\theta$ temporal
discretization, the system matrix is a linear combination of the lumped mass
matrix and the low-order steady-state system matrix; this linear combination
is also an M-matrix since it is a combination of two M-matrices with non-negative
combination coefficients.

To complete the proof of positivity preservation for the low-order scheme, 
we need to show that
the system right-hand-side vectors for each temporal discretization are
non-negative.
%-------------------------------------------------------------------------------
In this section, it will be shown that the low-order scheme for each temporal
discretization preserves non-negativity of the solution, given that a
CFL-like time step size condition is satisfied. This section
builds upon the results of Section \ref{sec:m_matrix}, which proved that the
low-order system matrix $\loworderssmatrix[n]$ is an M-matrix, which to
recall Equation \eqref{eq:m_matrix}, has the property
\[
  \mathbf{A}\mathbf{x} \geq \mathbf{0} \Rightarrow \mathbf{x} \geq \mathbf{0} \eqc
\]
and thus for a linear system $\mathbf{A}\mathbf{x} = \mathbf{b}$,
proof of non-negativity of the right-hand-side vector proves non-negativity
of the solution $\mathbf{x}$. For each temporal discretization, it will be
shown that the system matrix inverted for the corresponding low-order system
is also an M-matrix and that the right-hand-side vector for each system is
non-negative. Thus positivity-preservation of the solution will be proven.
%--------------------------------------------------------------------------------
\begin{theorem}{Non-Negativity of the Steady-State Low-Order Solution}
  The solution of the steady-state low-order system given by Equation
  \eqref{eq:low_ss} is non-negative:
  \[
    \solutionletter^L_i \geq 0 \eqc \quad \forall i\eqp
  \]
\end{theorem}

\begin{proof}
By Theorem \ref{thm:m_matrix}, the system matrix $\loworderssmatrix$ is an
M-matrix, and by assumption in Section \ref{sec:scalar}, the source $\scalarsource$
is non-negative, and thus the steady-state right-hand-side vector entries
$\ssrhsletter_i$ are non-negative. Invoking the M-matrix property
concludes the proof.\qed
\end{proof}
%--------------------------------------------------------------------------------
\begin{theorem}{Non-Negativity Preservation of the Explicit Euler Low-Order Solution}
  If the old solution $\solutionvector^n$ is non-negative and
  the time step size $\dt$ satisfies
\begin{equation}\label{eq:explicit_cfl}
  \timestepsize \leq \frac{\massmatrixletter_{i,i}^{L}}
    {\ssmatrixletter_{i,i}^{L,\timeindex}}
  \eqc\quad\forall i \eqc
\end{equation}
  then the new solution $\solutionvector^{L,n+1}$ of the explicit Euler low-order
  system given by Equation \eqref{eq:low_explicit_euler} is non-negative:
  \[
    \solutionletter^{L,n+1}_i \geq 0 \eqc \quad \forall i\eqp
  \]
\end{theorem}

\begin{proof}
Rearranging Equation \eqref{eq:low_explicit_euler},
\[
  \lumpedmassmatrix\solutionvector^{L,n+1}
    = \dt\ssrhs^n
      + \lumpedmassmatrix\solutionvector^{n}
      + \dt\loworderssmatrix[n]\solutionvector^{n}
  \eqp
\]
Thus the system matrix to invert is the lumped mass matrix, which is
an M-matrix since it is diagonal and positive. The right-hand-side
vector $\mathbf{y}$ of this system has the entries
\[
  y_i
    = \dt\ssrhsletter^n_i
      + \pr{\lumpedmassentry - \dt\ssmatrixletter^{L,n}_{i,i}}
        \solutionletter^n_i
      - \dt\sumjnoti\ssmatrixletter^{L,n}\ij\solutionletter^n_j
  \eqp
\]
It now just remains to prove that these entries are non-negative.
As stated previously, the source function $\scalarsource$ is assumed
to be non-negative and thus the steady-state right-hand-side
vector is non-negative. Due to the time step size assumption
given by Equation \eqref{eq:explicit_cfl} and Lemma \ref{lem:diagonalpositive},
\[
  \lumpedmassentry - \dt\ssmatrixletter^{L,n}_{i,i} \geq 0 \eqc
\]
and by Lemma \ref{lem:offdiagonalnegative}, the off-diagonal
sum term is also non-negative. Thus $y_i$ is a sum of non-negative
terms. Invoking the M-matrix property concludes the proof.\qed
\end{proof}
%--------------------------------------------------------------------------------
\begin{theorem}{Non-Negativity Preservation of the Theta Low-Order Solution}
  If the old solution $\solutionvector^n$ is non-negative and
  the time step size $\dt$ satisfies
\begin{equation}\label{eq:theta_cfl}
   \timestepsize \leq \frac{\massmatrixletter^L_{i,i}}{(1-\theta)
     \ssmatrixletter_{i,i}^{L,\timeindex}}
   \eqc\quad\forall i \eqc
\end{equation}
  then the new solution $\solutionvector^{L,n+1}$ of the Theta low-order
  system given by Equation \eqref{eq:low_theta} is non-negative:
  \[
    \solutionletter^{L,n+1}_i \geq 0 \eqc \quad \forall i\eqp
  \]
\end{theorem}

\begin{proof}
Rearranging Equation \eqref{eq:low_theta},
\[
  \pr{\lumpedmassmatrix+\theta\dt\loworderssmatrix[n+1]}\solutionvector^{L,n+1}
    = \dt\pr{(1-\theta)\ssrhs^n + \theta\ssrhs^{n+1}}
      + \lumpedmassmatrix\solutionvector^{n}
      - (1-\theta)\dt\loworderssmatrix[n]\solutionvector^n
  \eqp
\]
Thus the system matrix to invert is
$\lumpedmassmatrix+\dt\theta\loworderssmatrix[n+1]$, which is
an M-matrix since it is a linear combination of two M-matrices.
The right-hand-side vector $\mathbf{y}$ of this system has the entries
\[
  y_i
    = \dt\pr{(1-\theta)\ssrhsletter^n_i + \theta\ssrhsletter^{n+1}_i}
      + \pr{\lumpedmassentry - (1-\theta)\dt\ssmatrixletter^{L,n}_{i,i}}
        \solutionletter^n_i
      - (1-\theta)\dt\sumjnoti\ssmatrixletter^{L,n}\ij\solutionletter^n_j
  \eqp
\]
It now just remains to prove that these entries are non-negative.
As stated previously, the source function $\scalarsource$ is assumed
to be non-negative and thus the steady-state right-hand-side
vector is non-negative. Due to the time step size assumption
given by Equation \eqref{eq:theta_cfl} and Lemma \ref{lem:diagonalpositive},
\[
  \lumpedmassentry - (1-\theta)\dt\ssmatrixletter^{L,n}_{i,i} \geq 0 \eqc
\]
and by Lemma \ref{lem:offdiagonalnegative}, the off-diagonal
sum term is also non-negative. Thus $y_i$ is a sum of non-negative
terms. Invoking the M-matrix property concludes the proof.\qed
\end{proof}
%--------------------------------------------------------------------------------

%-------------------------------------------------------------------------------

It can also be shown that the described low-order scheme satisfies a local
discrete maximum principle, which is easily shown given the M-matrix property.
One may decide to use these bounds as the imposed bounds in the FCT
algorithm; however, this approach has been found to yield less accurate solutions
than the approach to be outlined in Section \ref{sec:fct} and is thus not discussed here 
for brevity.
