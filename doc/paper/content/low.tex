% !TEX root = ../paper.tex

\begin{itemize}
  \item M-matrix
  \item positivity preservation
  \item local DMP
\end{itemize}

The role of a low-order scheme in the context of the FCT algorithm is to
provide a fail-safe solution, which has desirable properties such as
positivity-preservation and lack of spurious oscillations. These properties
come at the cost of excessive artificial diffusion and thus a lesser degree
of accuracy. However, the idea of the FCT algorithm is to undo this disadvantage
as much as possible without violating some physically-motivated solution bounds.

Here positivity-preservation and monotonicity are achieved by satisfying
the M-matrix property of the matrix inverted in the low-order system.
M-matrices are a subset of inverse-positive matrices and have the monotone
property; for the linear system $\A\x = \ssrhs$, an M-matrix $\A$ has the property
\begin{equation}
  \ssrhs \geq 0 \quad \Rightarrow \quad \x \geq 0 \eqp
\end{equation}
Thus proving positivity-preservation, given the linear system matrix is an
M-matrix, is achieved by proving positivity of the right-hand-side vector $\ssrhs$.
This monotonicity property of the linear system matrix is also responsible for
the lack of spurious oscillations.

In this section, the first-order viscosity method introduced by Guermond
\cite{guermond_firstorder} will be adapted to the model equation given by
Equation \eqref{eq:scalar_model}. This method uses an element-wise artificial
viscosity definition in conjunction with a graph-theoretic local viscous
bilinear form that makes the method valid for arbitrary element shapes and
dimensions. These definitions will be shown to ensure that the system matrix
is a non-singular M-matrix.

The graph-theoretic local viscous bilinear form has the following definition.
%-------------------------------------------------------------------------------
\begin{defn}[Local Viscous Bilinear Form]
   The local viscous bilinear form for element $K$ is defined as follows:
   \begin{equation}\label{eq:bilinearform}
     d_K(\test_j,\test_i) \equiv \left\{\begin{array}{l l}
       -\frac{1}{n_K - 1}V_K & i\ne j\eqc \quad i,j\in \indices_K\eqc \\
       V_K                   & i = j \eqc \quad i,j\in \indices_K\eqc \\
       0                     & \mbox{otherwise}\eqc
     \end{array}\right.
   \end{equation}
   where $V_K$ is the volume of cell $K$, $\indices_K$ is the set of degree
   of freedom indices such that the corresponding test function has support
   on cell $K$, and $n_K$ is the number of indices in that set.
\end{defn}
%-------------------------------------------------------------------------------
This bilinear form bears resemblance to a standard Laplacian bilinear form:
the diagonal entries are positive, the off-diagonal entries are negative, and
the row sums are zero. These facts will be invoked in the proof of the M-matrix
conditions later in this section.

The element-wise viscosity definition from \cite{guermond_firstorder} is
adapted to account for the reaction term in Equation \eqref{eq:scalar_model}
but otherwise remains unchanged.
%-------------------------------------------------------------------------------
\begin{defn}[Low-Order Viscosity]
  The low-order viscosity for cell $K$ is defined as follows:
  \begin{equation}
    \nu^L_K \equiv \max\limits_{i\ne j\in\indices_K}
      \frac{\max(0,A_{i,j})}
      {-\sum\limits_{T\subset S_{i,j}}d_T(\test_j,\test_i)}
      \eqc
  \end{equation}
  where $A_{i,j}$ is the $i,j$th entry of the Galerkin steady-state matrix
  given by Equation \eqref{eq:Aij}.
\end{defn}
%-------------------------------------------------------------------------------
This viscosity definition is designed to give the minimum amount of artificial
diffusion without violating the M-matrix conditions.

Now the low-order system will be described.
Consider a modification of the Galerkin scheme given in Equation \eqref{eq:galerkin_semidiscrete}
which lumps the mass matrix ($\M^C \rightarrow \M^L$) and adds an artificial
diffusion operator $\D^L$, called the low-order diffusion matrix:
\begin{equation}
  \M^L\ddt{\U^L} + (\A + \D^L)\U(t) = \ssrhs(t) \eqc
\end{equation}
where $\U^L(t)$ denotes the discrete low-order solution values. The low-order
diffusion matrix is assembled element-wise using the local viscous bilinear
form and low-order viscosity definitions:
\begin{equation}\label{eq:low_order_diffusion_matrix}
  D_{i,j}^L \equiv
    \sum\limits_{K\subset S_{i,j}}\nu^L_K
    d_K(\test_j,\test_i) \eqp
\end{equation}

Now the low-order scheme has been fully described, and some statements will be
made on its properties. Firstly the M-matrix property will be claimed for
the matrix $\A^L\equiv\A + \D^L$, which will hereafter be referred to as the
low-order steady-state system matrix.
%-------------------------------------------------------------------------------
\begin{thm}[M-matrix property]
  The low-order steady-state system matrix $\A^L$ is a non-singular M-matrix.
\end{thm}
In this section, it will be shown that the low-order steady-state system matrix
defined in Equation \eqref{eq:low_order_ss_matrix} is an M-matrix, which allows
a discrete maximum principle for the low-order solution to be proven in Section
\ref{sec:DMP}.
%--------------------------------------------------------------------------------
\begin{lemma}[lem:offdiagonalnegative]{Non-Positivity of Off-Diagonal Elements}
   The off-diagonal elements of the linear system matrix are non-positive:
   \[
     \ssmatrixletter^L\ij\le 0, \quad j\ne i \eqp
   \]
\end{lemma}

\begin{proof}
This proof begins by bounding the term $\diffusionmatrixletter\ij^L$:
\begin{eqnarray*}
   \diffusionmatrixletter\ij^L=\sumKSij\lowordercellviscosity
   \localviscbilinearform{\cell}{j}{i}
   & = & \sumKSij\max\limits_{k\ne \ell\in \indicescell}
     \pr{\frac{\max(0,\ssmatrixletter_{k,\ell})}
       {-\sum\limits_{T\subset \support_{k,\ell}}
       \localviscbilinearform{T}{\ell}{k}}}
     \localviscbilinearform{\cell}{j}{i} \eqp
\end{eqnarray*}
Since $\localviscbilinearform{\cell}{j}{i} < 0$ for $j\ne i$ and $y_i \leq
\max_j y_j$,
\begin{eqnarray*}
   \diffusionmatrixletter\ij & \le & \sumKSij \frac{\max(0,\ssmatrixletter\ij)}
   {-\sum\limits_{T\subset\support\ij} \localviscbilinearform{T}{j}{i}}
   \localviscbilinearform{\cell}{j}{i} \eqc\\
   &  =  & -\max(0,\ssmatrixletter\ij)
     \frac{\sumKSij\localviscbilinearform{\cell}{j}{i}}
     {\sum\limits_{T\subset\support\ij} \localviscbilinearform{T}{j}{i}} \eqc\\
   &  =  & -\max(0,\ssmatrixletter\ij) \eqc\\
   & \le & -\ssmatrixletter\ij \eqp
\end{eqnarray*}
Applying this inequality to Equation \eqref{eq:low_order_ss_matrix} gives
\begin{eqnarray*}
  \ssmatrixletter^L\ij &  =  & \ssmatrixletter\ij + \diffusionmatrixletter\ij^L
    \eqc\\
  \ssmatrixletter^L\ij & \le & \ssmatrixletter\ij - \ssmatrixletter\ij
    \eqc\\
  \ssmatrixletter^L\ij & \le & 0 \eqp \qed
\end{eqnarray*}
\end{proof}
%--------------------------------------------------------------------------------
\begin{lemma}[lem:diagonalpositive]{Non-Negativity of Diagonal Elements}
   The diagonal elements  of the linear system matrix are non-negative:
   \[
     \ssmatrixletter^L_{i,i}\ge 0 \eqp
   \]
\end{lemma}

\begin{proof}
The diagonal elements of the low-order system matrix are
\[
  \ssmatrixletter^L_{i,i} = \intSi\nabla\cdot
  \frac{\velocity\testfunction_i^2(\x)}{2}d\volume
  + \intSi\sigma(\x)\testfunction_i^2(\x)d\volume
  + \sumKSi\lowordercellviscosity \localviscbilinearform{\cell}{i}{i} \eqp
\]
To prove that $\ssmatrixletter^L_{i,i}$ is non-negative, it is sufficient to
prove that each term in the above expression is non-negative. The
non-negativity of the interaction term and viscous term are obvious
($\reactioncoef \ge 0, \, \lowordercellviscosity\ge 0, \,
\localviscbilinearform{\cell}{i}{i}>0$), but the non-negativity of the divergence
term is not necessarily obvious. On the interior of the domain, the divergence
term gives zero contribution because the divergence integral may be transformed
into a surface integral
$\intSi\velocity\cdot\normalvector\frac{\testfunction_i^2}{2} d\volume$ via the
divergence theorem; one can then recognize that the basis function
$\testfunction_i$ evaluates to zero on the boundary of its support
$\support_i$. On the outflow boundary of the domain, the term
$\velocity\cdot\normalvector \frac{\testfunction_i^2}{2}$ is positive because
$\velocity\cdot\normalvector > 0$ for an outflow boundary. This quantity is of
course negative for the inflow boundary, but a Dirichlet boundary condition is
strongly imposed on the incoming boundary, so for degrees of freedom $i$ on the
incoming boundary, $\ssmatrixletter^L_{i,i}$ will be set equal to some positive
value such as 1 with a corresponding incoming value accounted for in the right
hand side $\ssrhs$ of the linear system.\qed
\end{proof}
%--------------------------------------------------------------------------------
\begin{lemma}{Non-Negativity of Row Sums}
   The sum of all elements in a row $i$ is non-negative:
   \[
     \sumj \ssmatrixletter^L\ij \ge 0 \eqp
   \]
\end{lemma}

\begin{proof}
Using the fact that $\sumj\testfunction_j(\x)=1$ and
$\sumj \localviscbilinearform{\cell}{j}{i}=0$,
\begin{eqnarray*}
   \sumj \ssmatrixletter^L\ij & = & \sumj \intSij
      \left(\velocity\cdot\nabla\testfunction_j(\x) +
      \reactioncoef(\x)\testfunction_j(\x)\right)\testfunction_i(\x) d\volume +
      \sumj\sumKSij\lowordercellviscosity \localviscbilinearform{\cell}{j}{i}
      \eqc\\
   & = & \intSi\left(\velocity\cdot
      \nabla\sumj\testfunction_j(\x) +
      \reactioncoef(\x)\sumj\testfunction_j(\x)\right)
      \testfunction_i(\x) d\volume \eqc\\
   \label{eq:rowsum} & = & \intSi\reactioncoef(\x)\testfunction_i(\x) d\volume
     \eqc\\
   &\ge& 0 \eqp \qed
\end{eqnarray*}
\end{proof}
%--------------------------------------------------------------------------------
\begin{lemma}[lem:diagonallydominant]{Diagonal Dominance}
   $\loworderssmatrix$ is strictly diagonally dominant:
   \[
     \left|\ssmatrixletter^L_{i,i}\right|
     \ge \sumjnoti \left|\ssmatrixletter^L\ij\right| \eqp
   \]
\end{lemma}
\begin{proof}
Using the inequalities $\sumj \ssmatrixletter^L\ij \ge 0$ and
$\ssmatrixletter^L\ij\le 0, j\ne i$, it is proven that $\loworderssmatrix$ is
strictly diagonally dominant:
\begin{eqnarray*}
  \sumj     \ssmatrixletter^L\ij       & \ge & 0 \eqc\\
  \sumjnoti \ssmatrixletter^L\ij + \ssmatrixletter^L_{i,i} & \ge & 0 \eqc\\
  \left|\ssmatrixletter^L_{i,i}\right| & \ge & \sumjnoti -\ssmatrixletter^L\ij
    \eqc\\
  \left|\ssmatrixletter^L_{i,i}\right| & \ge
    & \sumjnoti \left|\ssmatrixletter^L\ij\right| \eqp \qed
\end{eqnarray*}
\end{proof}
%--------------------------------------------------------------------------------
\begin{lemma}{M-Matrix}
  $\loworderssmatrix$ is an M-Matrix.
\end{lemma}
\begin{proof}
To prove that a matrix is an M-Matrix, it is sufficient to prove that
the following 3 statements are true:
\begin{enumerate}
\item $\ssmatrixletter^L\ij\le 0, j\ne i$,
\item $\ssmatrixletter^L_{i,i}\ge 0$,
\item $\left|\ssmatrixletter^L_{i,i}\right|
      \ge \sumjnoti \left|\ssmatrixletter^L\ij\right|$.
\end{enumerate}
These conditions are proven by Lemmas \ref{lem:offdiagonalnegative},
\ref{lem:diagonalpositive}, and \ref{lem:diagonallydominant}, respectively.\qed
\end{proof}
%--------------------------------------------------------------------------------

%-------------------------------------------------------------------------------
