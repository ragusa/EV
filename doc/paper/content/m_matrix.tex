% !TEX root = ../paper.tex

\begin{prf}
  There are many definitions that can be used to identify a non-singular M-matrix
  \cite{m_matrix_fiedler}; one
  definition gives that an M-matrix can be identified by verifying all of the
  following three properties:
  \begin{enumerate}
    \item positivity of diagonal entries: $A_{i,i} > 0$, $\forall i$,
    \item non-positivity of off-diagonal entries: $A_{i,j} \leq 0$,
      $\forall i$, $\forall j\ne i$, and
    \item strict diagonal dominance: $\sumj A_{i,j} > 0$, $\forall i$ \cite{m_matrix_poole}.
  \end{enumerate}
%-------------------------------------------------------------------------------
Firstly it will be shown that the off-diagonal elements of the matrix
$\A^L$ are non-positive.
The diffusion matrix entry $D^L_{i,j}$ is bounded as follows:
\begin{eqnarray*}
   D_{i,j}^L & = & \sumKSij\nu^L_K d_K(\test_j,\test_i)\\
          & = & \sumKSij\max\limits_{k\ne \ell\in \indices_K}
            \pr{\frac{\max(0,A_{k,\ell})}
              {-\sum\limits_{T\in\mathcal{K}(S_{k,\ell})}
              \mkern-20mu d_T(\test_\ell,\test_k)}} d_K(\test_j,\test_i) \eqp
\end{eqnarray*}
Recall $d_K(\test_j,\test_i) < 0$ for $j\ne i$.
For an arbitrary quantity $c_{k,\ell} \geq 0 \eqc \forall k\ne\ell\in\indices$,
the following is true for $i\ne j\in\indices$:
$\max\limits_{k\ne\ell\in\indices}c_{k,\ell} \geq c_{i,j}$, and thus for $a\leq 0$,
$a\max\limits_{k\ne\ell\in\indices}c_{k,\ell} \leq a c_{i,j}$.
Thus,
\begin{eqnarray*}
   D^L_{i,j} & \le & \sumKSij \frac{\max(0,A_{i,j})}
    {-\sumKSij[T] d_T(\test_j,\test_i)}
    d_K(\test_j,\test_i) \eqc\\
      & = & -\max(0,A_{i,j}) \frac{\sumKSij d_K(\test_j,\test_i)}
        {\sumKSij[T] d_T(\test_j,\test_i)} \eqc\\
        & = & -\max(0,A_{i,j}) \eqc\\
        & \le & -A_{i,j} \eqp
\end{eqnarray*}
% Applying this inequality to Equation \eqref{eq:low_order_ss_matrix} gives
% \begin{eqnarray*}
%   \ssmatrixletter^{L,\timeindex}\ij &  =  &
%     \ssmatrixletter\ij^\timeindex + \diffusionmatrixletter\ij^{L,\timeindex}
%     \eqc\\
%   \ssmatrixletter^{L,\timeindex}\ij & \le &
%     \ssmatrixletter\ij^\timeindex - \ssmatrixletter\ij^\timeindex
%     \eqc\\
%   \ssmatrixletter^{L,\timeindex}\ij & \le & 0 \eqp \qed
% \end{eqnarray*}
% %-------------------------------------------------------------------------------
% \begin{remark}
% If boundary conditions are weakly imposed, as discussed in Section
% \ref{sec:transport_bc}, then the steady-state system matrix is modified
% and $\tilde{\ssmatrix}\ij \geq \ssmatrix\ij$. In this case, the low-order
% viscosity is computed with the \emph{modified} steady-state matrix. Then
% Lemma \ref{lem:offdiagonalnegative} still holds.
% \end{remark}
% %--------------------------------------------------------------------------------
% \begin{lemma}[lem:rowsumspositive]{Non-Negativity of Row Sums}
%    The row-sums of the matrix $\loworderssmatrix[n]$ are non-negative:
%    \[
%      \sumj \ssmatrixletter^{L,\timeindex}\ij \ge 0
%        \eqc \quad \forall i \eqp
%    \]
% \end{lemma}
%
% \begin{proof}
% Using the fact that $\sumj\testfunction_j(\x)=1$ and
% $\sumj \localviscbilinearform{\cell}{j}{i}=0$,
% \begin{eqnarray*}
%    \sumj \ssmatrixletter^{L,\timeindex}\ij & = & \sumj \intSij
%       \left(\mathbf{\consfluxletter}'(\approximatescalarsolution^\timeindex)
%         \cdot\nabla\testfunction_j +
%       \reactioncoef\testfunction_j\right)\testfunction_i \dvolume +
%       \sumj\sumKSij\lowordercellviscosity[\timeindex]
%         \localviscbilinearform{\cell}{j}{i}
%       \eqc\\
%    & = & \intSi\left(
%       \mathbf{\consfluxletter}'(\approximatescalarsolution^\timeindex)\cdot
%       \nabla\sumj\testfunction_j(\x) +
%       \reactioncoef(\x)\sumj\testfunction_j(\x)\right)
%       \testfunction_i(\x) \dvolume \eqc\\
%    \label{eq:rowsum} & = & \intSi\reactioncoef(\x)\testfunction_i(\x) \dvolume
%      \eqc\\
%    &\ge& 0 \eqp \qed
% \end{eqnarray*}
% \end{proof}
% %--------------------------------------------------------------------------------
% \begin{remark}
% If boundary conditions are weakly imposed, as discussed in Section
% \ref{sec:transport_bc}, then the steady-state system matrix is modified
% and $\tilde{\ssmatrix}\ij \geq \ssmatrix\ij$.
% Thus
% \begin{equation}
%   \sum_j\tilde{\ssmatrix}\ij \geq \sum_j\ssmatrix\ij \geq 0 \eqc
% \end{equation}
% and Lemma \ref{lem:rowsumspositive} still holds.
% \end{remark}
% %--------------------------------------------------------------------------------
% \begin{lemma}[lem:diagonalpositive]{Non-Negativity of Diagonal Elements}
%    The diagonal elements of the matrix $\loworderssmatrix[n]$ are non-negative:
%    \[
%      \ssmatrixletter^{L,n}_{i,i} \ge 0
%        \eqc \quad \forall i\eqp
%    \]
% \end{lemma}
%
% \begin{proof}
% Using Lemma \ref{lem:rowsumspositive},
% \[
%   \sumj \ssmatrixletter^{L,n}\ij \ge 0 \eqp
% \]
% Thus,
% \[
%   \ssmatrixletter^{L,n}_{i,i} \ge -\sumjnoti \ssmatrixletter^{L,n}\ij \eqp
% \]
% From Lemma \ref{lem:offdiagonalnegative}, the off-diagonal elements are known
% to be non-positive: $\ssmatrixletter^{L,n}\ij \leq 0$. Thus,
% $-\ssmatrixletter^{L,n}\ij \geq 0$ and finally,
% \[
%   \ssmatrixletter^{L,n}_{i,i} \ge 0 \eqp \qed
% \]
% \end{proof}
% %--------------------------------------------------------------------------------
% \begin{lemma}[lem:diagonallydominant]{Diagonal Dominance}
%    The matrix $\loworderssmatrix[\timeindex]$ is strictly diagonally dominant:
%    \[
%      \left|\ssmatrixletter^{L,\timeindex}_{i,i}\right|
%      \ge \sumjnoti \left|\ssmatrixletter^{L,\timeindex}\ij\right|
%      \eqc \quad \forall i\eqp
%    \]
% \end{lemma}
%
% Using the inequalities $\sumj \ssmatrixletter^{L,\timeindex}\ij \ge 0$ and
% $\ssmatrixletter^{L,\timeindex}\ij\le 0, j\ne i$, it is proven that
% $\loworderssmatrix[\timeindex]$ is strictly diagonally dominant:
% \begin{eqnarray*}
%   \sumj     \ssmatrixletter^{L,\timeindex}\ij       & \ge & 0 \eqc\\
%   \sumjnoti \ssmatrixletter^{L,\timeindex}\ij
%     + \ssmatrixletter^{L,\timeindex}_{i,i} & \ge & 0 \eqc\\
%   \left|\ssmatrixletter^{L,\timeindex}_{i,i}\right| & \ge &
%     \sumjnoti -\ssmatrixletter^{L,\timeindex}\ij
%     \eqc\\
%   \left|\ssmatrixletter^{L,\timeindex}_{i,i}\right| & \ge
%     & \sumjnoti \left|\ssmatrixletter^{L,\timeindex}\ij\right| \eqp \qed
% \end{eqnarray*}
% %--------------------------------------------------------------------------------

\end{prf}
