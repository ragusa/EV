An FCT scheme has been proposed for the particle transport equation. It is has been applied
to the following model equation problem, an advection problem with reaction and source terms,
\begin{equation} \label{eq:ccl}
  \ppt{u} + v\di\cdot\nabla u\xt + \sigma(\x) u\xt = q\xt \eqc
\end{equation}
which corresponds to the classic Source Iteration equation for particle transport,
where extraneous, inscatter, and fission sources are collected in the right-hand-side term $q$.
%
The FCT methodology relies on a lower-order solution and a high-order solution. For the low-order
solution, we have used a first-order viscosity approach, based on the graph-theoric method of
Guermond at al. \cite{guermond_firstorder} for scalar conservation laws and extents these
previous works to situations with reaction and source terms present. The low-order scheme is shown
to be positivity-preserving through the use of the M-matrix properties. The high-order solution
is employs an entropy-based artificial stabilization. The entropy residual approach is derived
for the transport equation shown in Equation~\eqref{eq:ccl}.
Temporal discretizations include explicit and implicit schemes in time, as well as steady state,
the latter two cases making the FCT algorithm implicit as well. The standard Zalesak limiter is utilized
to limit between the low- and the high-order solutions.

The FCT scheme described in this paper is second-order accurate in space,
converges to the entropy solution, and preserves non-negativity.
Spurious oscillations are mitigated but are not guaranteed to be
eliminated, as smaller magnitude oscillations may exist within the imposed
solution bounds.

% Local solution bounds imposed in the FCT algorithm were derived using the method
% of characteristics and integral transport equation.
% Two sets of solution
% bounds were considered, one considering only values along the upstream
% line segment traversed in a time step, and the other considering a spherical
% neighborhood that encompasses this line segment. The former set of solution
% bounds is much tighter, which has the advantage that there is a smaller range
% of limiting coefficient values that can be used, but has the disadvantage that
% there is less room for antidiffusion.

The traditional FCT phenomenon known as ``stair-stepping'',
``terracing'', or ``plateauing'' is still an open issue, particularly for
fully explicit temporal discretizations; however, these effects
have been shown to diminish or disappear when using SSPRK33 as opposed
to explicit Euler. In addition, these effects are less pronounced for EV-FCT
than in the classic FEM-FCT scheme, which uses the standard Galerkin method as
the high-order method in FCT.

The explicit temporal discretizations of the described FCT scheme yield a robust algorithm; however, implicit and steady-state discretizations
are less robust, suffering from nonlinear convergence difficulties
in some problems. % Implicit schemes become divergent as the CFL number is increased. 
The main complication with implicit and steady-state
FCT schemes is that the imposed solution bounds are implicit with the solution,
and thus the imposed solution bounds change with each iteration of the
nonlinear solver.

Future work on this subject should mainly focus on the implicit/steady-state
iteration techniques because of the convergence difficulties
encountered for some problems. This work used fixed-point iteration, but
one can attempt using an alternative such as Newton's method. The main
challenge is the evolving solution bounds, which will present difficulty
to any nonlinear solution algorithm. Other work could be performed for
the FCT algorithm in general: for example, the terracing phenomenon still
deteriorates FCT solutions. %Perhaps there exists a limiting strategy that
%could preclude the formation of these spurious plateaus.
