The FCT scheme described in this paper is second-order accurate in space,
converges to the entropy solution, and preserves non-negativity.
Spurious oscillations are mitigated but are not guaranteed to be
eliminated, as smaller magnitude oscillations may exist within the imposed
solution bounds.

Local solution bounds imposed in the FCT algorithm were derived using the method
of characteristics and integral transport equation. Two sets of solution
bounds were considered, one considering only values along the upstream
line segment traversed in a time step, and the other considering a spherical
neighborhood that encompasses this line segment. The former set of solution
bounds is much tighter, which has the advantage that there is a smaller range
of limiting coefficient values that can be used, but has the disadvantage that
there is less room for antidiffusion.

The traditional FCT phenomenon known as ``stair-stepping'',
``terracing'', or ``plateauing'' is still an open issue, particularly for
fully explicit temporal discretizations; however, these effects
have been shown to diminish or disappear when using SSPRK33 as opposed
to forward Euler. In addition, these effects are less pronounced for EV-FCT
than in the classic FEM-FCT scheme using the standard Galerkin method as
the high-order method in FCT.

The explicit temporal discretizations of the described FCT scheme yield a
relatively robust algorithm; however, implicit and steady-state discretizations
are much less robust, suffering from severe nonlinear convergence difficulties
in some problems. Implicit schemes become increasingly divergent as the CFL
number is increased. The main complication with implicit and steady-state
FCT schemes is that the imposed solution bounds are implicit with the solution,
and thus the imposed solution bounds change with each iteration of the
nonlinear solver.
