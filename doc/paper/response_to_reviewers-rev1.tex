\documentstyle{letter}

\pagestyle{empty}

\begin{document}

Review \#1:

Title: ``Flux-Corrected Transport Techniques Applied to the Radiation Transport Equation Discretized with Continuous Finite Elements''\\
Authors: Joshua Hansel, Jean Ragusa \\
{\it Journal of Computational Physics}\\
\vspace{3cm}

Thank you for your thorough review of our manuscript.

The changes made with regard to your comments are summarized in the following
tables. Page, paragraph, and line numbers reference
the ``diff'' PDF document (which shows both deleted and added content between the original submission and this revised submission). 

In the tables on the next pages, the listed paragraph number does not necessarily correspond to the actual paragraph number
but rather the number of the textual ``blocks''; for example, displayed equations
separate a paragraph into multiple ``blocks''. Displayed equations and figures
list an equation/figure number in parentheses instead of a paragraph number
and line number. Negative line numbers denote counting from the bottom of the
text block up.

Note that the JCP submission system uses all of our uploaded latex files, so you may see the {\bf revised manuscript twice}: once without any markups and once showing the markups between the current version and the first submission.



\begin{tabular}{c c c c p{3in}}
\emph{Item} & \emph{Page} & \emph{Par.} & \emph{Line} & \emph{Change}\\\hline
1 & 2 & 3 & 4 & mentioned Monte Carlo techniques\\
2 & 1 & 1 & 6 & mentioned upfront that Cartesian geometries are assumed\\
  & 2 & 4 & 3 & mentioned the lack of decoupling for curvilinear geometries\\
3 & 2 & 5 & 1 & made requested change\\
4 & 3 & 2 & 12 & added Hamilton citation\\
5 & 4 & 5 & 3 & mentioned the assumption made on the source term sign, as
  well as the possible negativity with anisotropic scattering\\
6 & 7 & 4 & 1 & fixed mis-worded first sentence. Made second sentence more
  precise and supported it with reference.\\
7 & 8 & (20) & - & fixed notation in equation\\
  & 9 & (21a) & - & fixed notation in equation\\
8 & 10 & 1 & 2 & added $j\ne i$ to each line of equation\\
9 & - & - & - & The changes made in addressing comment 5 included that the source
  sign assumption was made in this work, which should address this comment as well.\\
10 & 11 & 7 & 1 & fixed typo\\
11 & 18 & 3 & 2 & fixed typo\\
12 & 19 & 3 & 3 & mentioned that multiple passes through the limiter were not used in results\\
13 & 19 & 5 & -1 & Added sentence stating that 3rd-order Gauss quadrature was used for all problems\\
14 & 24 & 1 & 6 & Added discussion about CFL decreasing total computational time
  but at the expense of FCT solution quality. Added a column showing $L^2$ error
  for each run to demonstrate decreasing FCT solution quality with increasing
  CFL number. Also the entries in the table have all changed because the end
  time for the study was changed from 1 to 1.5 (the end time was originally never mentioned).
  This change was made so that the steady-state should theoretically be reached; 1 second
  is not enough, considering the geometry, transport speed, and transport direction.
  The steady-state is desirable because the steady-state analytical solution is
  useful for evaluating solution error and drawing conclusions about FCT solution quality.\\
15 & 25 & 1 & 5 & gave boundary conditions, as well as transport direction\\
16 & 25 & 2 & 7 & mentioned $U^-$ and $U^+$\\
17 & 28 & (6) & - & added missing Galerkin sets to plot\\
18 & 29 & 2 & 9 & fixed typo\\
\end{tabular}

The following are additional changes made, addressing grammar, typos, poor
wording, missing details, etc.:

\begin{tabular}{c c c c p{3in}}
\emph{Item} & \emph{Page} & \emph{Par.} & \emph{Line} & \emph{Change}\\\hline
1 & 2 & 4 & 6 & changed ``Equations (3)'' to ``the system given by Equation (3)''\\
2 & 9  & (21) & - & Added $L$ superscript to denote low-order solution, to be consistent with later sections\\
3 & 14 & (34) & - & Added $H$ superscript to denote high-order solution, to be consistent with later sections\\
4 & 15 & 1 & 2 & Added appositive for low-order solution, as is already done for high-order solution in this sentence\\
5 & 15 & 1 & 2 & Changed period to colon as intended; Equation (36) is part of the statement\\
6 & 15 & (37c) & - & Fixed superscript on ambiguous $\mathbf{U}^{n+1}$ solution to be $H$, as originally intended\\
7 & 18 & 3 & 1 & changed unusual grammar: ``the Zalesak's limiter'' to ``the Zalesak limiter''\\
8 & 23 & 3 & 2 & gave the mesh size and end time used for the study\\
9 & 24 & 1 & 16 & removed ``easily'' from ``easily mitigated''; the word was unnecessary and expressed opinion\\
10 & 27 & 2 & 8 & added comment about $U^-$ and $U^+$, as for comment 16\\
11 & 29 & 2 & 5 & fixed typo: ``at al.'' to ``et al.''\\
12 & 29 & 2 & 6 & fixed typo: ``extents these'' to ``extends these''\\
13 & 29 & 2 & -2 & improved grammar: ``the low- and the high-order'' to ``the low- and high-order''.
  The second ``the'' interrupted the hyphenation of ``low-order''.\\
\end{tabular}

\end{document}
