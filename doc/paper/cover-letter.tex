\documentstyle[11pt]{letter}

%%%%%% Letter Size Setup %%%%%%%%%%%%%%%%%%%%%%%%%%%%%%%%%%%%%%%%%%%%%%%%%%
%        \addtolength{\textwidth}{2.5cm}     %% For longer or shorter text width
        \addtolength{\topmargin}{-3.5cm}    %% For more or less top margin
       \addtolength{\textheight}{7cm}    %% For longer or shorter textheight
%        \addtolength{\oddsidemargin}{-1.25cm} %% For odd side margin (twoside)
                                            %% or margin (oneside)

\address{Jean Ragusa\\ 
Department of Nuclear Engineering \\
Texas A\&M University\\
College Station, TX 77843-3133, USA\\
phone: (979) 862 2033\\
e-mail: jean.ragusa@tamu.edu \vspace{0.5cm}}

%%%%%% The Signature  and Date %%%%%%%%%%%%%%%%%%%%%%%%%%%%%%%%%%%%%%%%%%%%

\signature{\vspace{-1.25cm}Joshua Hansel \& Jean Ragusa}   


\begin{document}

\begin{letter}{Professor Morel\\
    Editor,\\
    Journal of Computational Physics}
\date{\today}
%%%%%% More vertical space can be added here %%%%%%%%%%%%%%%%%%%%%%%%%%%%%%
%         \vspace{3.0cm}

\opening{Dear Professor Morel,}
         \vspace{0.25cm}
%%%%%% More vertical space can be added here %%%%%%%%%%%%%%%%%%%%%%%%%%%%%%

Please find attached a copy of our manuscript titled ``Flux-Corrected Transport Techniques Applied to the Radiation Transport Equation Discretized with Continuous Finite Elements'' for submission to the {\em Journal of Computational Physics}. 

Flux-Corrected Techniques (FCT) find their origins with the fluid dynamics community where they have been devised to combat the formation of spurious oscillations in finite-difference solution techniques. Over the last decade or so, FCT techniques have been extended to continuous finite element discretizations (e.g., the extensive archival work of D. Kuzmin on FEM-FCT). However, these techniques have not been employed in the radiation transport community and this manuscript aims at bridging this gap. Our present work combines 
\begin{enumerate}
\item first-order radiation transport solved with CFEM
\item definition of low-order and high-order CFEM discretizations
\item extension of FCT bounds definitions for radiation transport
\item analysis of time-dependent (explicit and implicit) and steady-state FCT algorithms.
\end{enumerate}
%
An FCT algorithm requires a low-order (fail-safe) solution that is often too dissipative. 
In this paper, we derive a viscous regularization for the non-equilibrium Seven-Equation two-phase flow Model. This regularization ensures positivity of the entropy 
residual and uniqueness of the weak solution, is consistent with the viscous regularization for Euler equations when one phase disappears, does not depend on the spatial discretization scheme chosen, and is compatible with the generalized Harten entropies. 

We investigate the behavior of the proposed viscous regularization for two important limit-cases: a five-equation two-phase flow model (obtained via a Chapman-Enskog expansion) and in the low-Mach asymptotic regime.


Numerical results are provided for a shock tube problem and a multi-Mach two-phase mixture flow problem (low Mach for liquid, supersonic for vapor).


This work follows prior works by Guermond \& Popov who proposed a viscous regularization for single-phase equations : ``Viscous regularization of the Euler equations and entropy principles'', {\it SIAM J. Appl. Math} 74 (2) (2014) 284-305.


Thank you for considering this manuscript for publication in the {\it Journal of Scientific Computing}.

\vspace{0.25cm}


%%%%%% More vertical space can be added here %%%%%%%%%%%%%%%%%%%%%%%%%%%%%%

%%%%%%% The Closing %%%%%%%%%%%%%%%%%%%%%%%%%%%%%%%%%%%%%%%%%%%%%%%%%%%%%%%
\closing{Best regards, }

\end{letter}
\end{document}

