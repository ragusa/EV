\documentstyle[11pt]{letter}

%%%%%% Letter Size Setup %%%%%%%%%%%%%%%%%%%%%%%%%%%%%%%%%%%%%%%%%%%%%%%%%%
%        \addtolength{\textwidth}{2.5cm}     %% For longer or shorter text width
        \addtolength{\topmargin}{-3.5cm}    %% For more or less top margin
       \addtolength{\textheight}{7cm}    %% For longer or shorter textheight
%        \addtolength{\oddsidemargin}{-1.25cm} %% For odd side margin (twoside)
                                            %% or margin (oneside)

\address{Jean Ragusa\\ 
Department of Nuclear Engineering \\
Texas A\&M University\\
College Station, TX 77843-3133, USA\\
phone: (979) 862 2033\\
e-mail: jean.ragusa@tamu.edu \vspace{0.5cm}}

%%%%%% The Signature  and Date %%%%%%%%%%%%%%%%%%%%%%%%%%%%%%%%%%%%%%%%%%%%

\signature{\vspace{-1.25cm}Joshua Hansel \& Jean Ragusa}   


\begin{document}

\begin{letter}{Professor Morel\\
    Editor,\\
    Journal of Computational Physics}
\date{\today}
%%%%%% More vertical space can be added here %%%%%%%%%%%%%%%%%%%%%%%%%%%%%%
%         \vspace{3.0cm}

\opening{Dear Professor Morel,}
         \vspace{0.25cm}
%%%%%% More vertical space can be added here %%%%%%%%%%%%%%%%%%%%%%%%%%%%%%

Please find attached a copy of our manuscript titled ``Flux-Corrected Transport Techniques Applied to the Radiation Transport Equation Discretized with Continuous Finite Elements'' for submission to the {\em Journal of Computational Physics}. 

Flux-Corrected Techniques (FCT) find their origins within the fluid dynamics community where they have been devised to combat the formation of spurious oscillations in finite-difference solution techniques. Over the last decade or so, these FCT techniques have been extended to continuous finite element discretizations (e.g., the extensive archival work of D. Kuzmin on FEM-FCT). However, these techniques have not yet been employed in the radiation transport community and this manuscript aims at bridging this gap. Our present work combines 
\begin{enumerate}
\item solving the first-order radiation transport equation with CFEM;
\item defining low-order and high-order CFEM discretizations for the first-order radiation transport, as required in the FCT method;
\item devising FCT bounds definitions that satisfy a discrete-maximum principle for radiation transport;
\item analyzing time-dependent (explicit and implicit) and steady-state FCT algorithms with the proposed method.
\end{enumerate}
%
Rather than employing a pure Galerkin discretization as the high-order scheme, we employ an entropy-viscosity (EV) based scheme as our high-order discretization and compare both approaches,
Galerkin and EV, in the manuscript.

Numerous publications pertaining to FCT techniques have appeared in JCP and we feel this is
the proper venue for this manuscript.

This work follows prior works by Kuzmin (e.g., 
``Explicit and implicit FEM-FCT algorithms with flux linearization'',
Journal of Computational Physics, Volume 228, 2009); Guermond \& Popov 
(e.g., ``Entropy viscosity method for nonlinear conservation laws'', 
Journal of Computational Physics, Volume 230, 2011).


Thank you for considering this manuscript for publication in the {\it Journal of Computational Physics}.

\vspace{0.25cm}


%%%%%% More vertical space can be added here %%%%%%%%%%%%%%%%%%%%%%%%%%%%%%

%%%%%%% The Closing %%%%%%%%%%%%%%%%%%%%%%%%%%%%%%%%%%%%%%%%%%%%%%%%%%%%%%%
\closing{Best regards, }

\end{letter}
\end{document}

