%BeginFileInfo
%%Publisher=ESCH
%%Project=YJCPH
%%Manuscript=YJCPH7667
%%MS position=
%%Stage=203
%%TID=janina
%%Pages=17
%%Format=2010
%%Distribution=vtex
%%Destination=DVI
%%PDF type=
%%DVI.Maker=luadvi
%%PS.Maker=luaps
%%Spelled=Dictionary: American, Computer: 1GSRED583, 2017.10.25 15:33
%%History1=2017.10.24 13:03
%EndFileInfo
%
%Spelling_date
% Opcijos: [secthm,seceqn,secfloat,xxtheorem,
% noappeqn,wwwdraft]
%%
\documentclass[xchauthor,chkrefs,fixeqskip,GCNS,amsmath,amsthm]{yjcphg}
\usepackage{tabcols} %%% [debug]
\usepackage[v5.5.0]{jadtd}
\usepackage[hidebg]{tmultirow}
%\usepackage{}
\PROOF
%\preCRC
%\psdraft
\aid{7667}
\PII{S0021-9991(17)30782-9}% Updated by PTS2LaTeX.exe, 24.10.2017 13:03
\doi{10.1016/j.jcp.2017.10.029}% Updated by PTS2LaTeX.exe, 24.10.2017
%13:03
\docsubty{FLA}
\volume{00}
%\issue{00}
%\supplement{}
\pubyear{2017}
%\pagenumbering{roman}%Roman,alph,Alph
\firstpage{1}
\lastpage{17}
\TID{RT}
%
%\doctopic[code]{??}
%
\crossmark{0}% Updated by PTS2LaTeX.exe, 24.10.2017 13:03
%
% APIBREZIMAI:
%
\startlocaldefs
%

\newtheorem{thm}{Theorem}
\newtheorem{lem}[thm]{Lemma}

\theoremstyle{remark}
\newtheorem{defn}{Definition}
\newtheorem{rmk}{Remark}


\newcommand{\di}{\bm{\Omega}}
%

\begin{nonSGML}
 \newcommand{\Lbrack}{[\![}
  \newcommand{\Rbrack}{]\!]}
  \end{nonSGML}
\endlocaldefs
%
\begin{copyrightinfo}[type=standard]% Updated by PTS2LaTeX.exe,
%24.10.2017 13:03
\cpcopyrightNotice{\textCopyright\ 2017 Elsevier Inc. All rights reserved.}% Updated by PTS2LaTeX.exe, 26.10.2017 08:30
%Updated by PTS2LaTeX.exe, 24.10.2017 13:03
%\cplicenseLine{}
%\oauserLicense{}
\cpcopyrightYear{2017}% Updated by PTS2LaTeX.exe, 24.10.2017 13:03
\cpcopyrightHolderDisplayName{Elsevier Inc.}% Updated by PTS2LaTeX.exe, 26.10.2017 08:30
\cpxmlCopyrightType{limited-transfer}% Updated by PTS2LaTeX.exe, 26.10.2017 08:30
%13:03
\copyrt{2017}
\CopyrightStatus{001}
\end{copyrightinfo}
%
\begin{document}
%
\begin{frontmatter}
%% REFERSTO
%\dochead{}
\title{Flux-corrected transport techniques applied to the
radiation transport equation discretized with continuous finite elements}
% \title{}
%\artthanks[]{}
%\begin{aug}
%\runauthor{}
%\author[A]{\inits{}\fnm{} \snm{}}
%\corthanks[cor]{Corresponding author.}
%\address[A]{}
%\end{aug}
%
\begin{aug}
\author{\inits{J.E.}\fnm{Joshua E.}~\snm{Hansel}\ead
{joshua.hansel89@gmail.com}},
\author{\inits{J.C.}\fnm{Jean C.}~\snm{Ragusa}\ead
{jean.ragusa@tamu.edu}\orcid{0000-0001-9870-6447}}
%
\address{\orgname{Department of Nuclear Engineering},
\orgname{Texas A\&M University}, \city{College Station}, \state{TX}
\postcode{77840}, \cny{USA}}
\end{aug}
%
% HISTORY:
\received{\sday{6} \smonth{6} \syear{2017}}% Updated by
%PTS2LaTeX.exe, 24.10.2017 13:03
\revised{\sday{25} \smonth{9} \syear{2017}}% Updated by
%PTS2LaTeX.exe, 24.10.2017 13:03
\accepted{\sday{19} \smonth{10} \syear{2017}}% Updated by
%PTS2LaTeX.exe, 24.10.2017 13:03
%\pubonline{\sday{} \smonth{} \syear{}}
%
%\dedicated{}
%
%\begin{abstract} %%% Butinas FLA, REV tipams
%\end{abstract} %%% Nebutinas SCO tipui
%
%\begin{abstract}[class=graphical] %%% Jei yra
%\begin{figure} \includegraphics{<aid>fab} \end{figure} \abstext{}
%\end{abstract}
%
%\begin{abstract}[class=author-highlights,title=Highlights] %%% Jei yra
%\begin{itemize} %\item
%\end{itemize}
%\end{abstract}
%
\begin{abstract}
The Flux-Corrected Transport (FCT) algorithm is applied to the unsteady
and steady-state particle transport equation. The proposed FCT method
employs the following: (1) a low-order, positivity-preserving scheme,
based on the application of M-matrix properties, (2)~a~high-order scheme
based on the entropy viscosity method introduced by Guermond
\cite{guermond_ev}, and (3) local, discrete solution bounds derived from
the integral transport equation. The resulting scheme is second-order
accurate in space, enforces an entropy inequality, mitigates the
formation of spurious oscillations, and guarantees the absence of
negativities. Space discretization is achieved using continuous finite
elements. Time discretizations for unsteady problems include theta
schemes such as explicit and implicit Euler, and strong-stability
preserving Runge--Kutta (SSPRK) methods. The developed FCT scheme is
shown to be robust with explicit time discretizations but may require
damping in the nonlinear iterations for steady-state and implicit time
discretizations.
\end{abstract}
\begin{abstract}[class=author-highlights,title=Highlights]
%
\begin{itemize}
\item
 Devised an FCT technique for the first-order form of the radiation transport equation.
\item Local solution bounds derived using integral transport formalism.
\item High-order solution can use an entropy-viscosity based dissipation.
\item Implicit FCT for steady-state and time-implicit discretizations.
\end{itemize}
\end{abstract}
%
% Opcijos: [class=KWD]
%\begin{keyword} %%% Visada
%\kwd{???}%\kwd{}\kwd{}
%\end{keyword}
%
\begin{keyword}
\kwd{Entropy viscosity}
\kwd{FCT}
\kwd{Particle transport equation}
\end{keyword}
%
\end{frontmatter}
%
%spell_from *************** Text entry area ******************%

%%% PAPILDOMI REIKALAVIMAI:
%%% CME straipsniams:
%%% Jei straipsnis priskirtas sekcijai CME,
%%% o jo PIT yra FLA arba SCO, tuomet yra privaloma Introduction
%sekcija.
%%% Sekcijos pavadinimas turi buti "Introduction". Sekcijos turi buti
%numeruotos.
%%% Figure caption: Figure should be abbreviated as Fig. for all
%%% active figure citations and figure labels.
%%% Figuru dalys turi buti zymimos mazosiomis raidemis skliausteliuose
%%% (pvz. Fig. 2(a)) ir pacioje figuroje, ir nuorodose tekste.

%s1 #&#
\section{Introduction}%
\label{sec:introduction}

The radiation transport equation, or linear Boltzmann equation,
describes the transport of particles interacting with a background
medium \cite{glasstone}. Some of its applications include the
modeling of nuclear reactors, radiation therapy, astrophysical
applications, radiation shielding, and high energy density physics
\cite{glasstone,radiotherapy,astrophysics_textbook,lewis,laser_plasmas}.
This paper focuses on solution techniques applicable to the first-order
form of the transport equation in Cartesian geometries, recalled below
in Equation~\reftext{\eqref{eq:transport_scalar}}. The transport equation is a
particle balance statement in a six-dimensional phase-space volume where
$\mathbf{x}$ denotes the particle's position, $\di$ its direction of
flight, and $E$ its energy:
%
%e1 #&#
\begin{equation}
\label{eq:transport_scalar}
\frac{1}{v(E)}\frac{\partial\psi}{\partial t} + \di\cdot\nabla
\psi(\mathbf{x},\di,E,t)+ \Sigma_{\text{t}}(\mathbf{x},E,t)\psi(
\mathbf{x},\di,E,t)= Q_{\text{tot}}(\mathbf{x},\di,E,t)\,.
\end{equation}
%
$Q_{\text{tot}}(\mathbf{x},\di,E,t)$ denotes the total particle
source gains in an infinitesimal phase-space volume due to particle
scattering, extraneous source of particles (if any), and fission sources
(in the case of neutron transport in multiplying media):
%
%e2 #&#
\begin{equation}
Q_{\text{tot}}(\mathbf{x},\di,E,t)\equiv Q_{\text{sca}}(
\mathbf{x},\di,E,t)+ Q_{\text{ext}}(\mathbf{x},\di,E,t)+ Q_{
\text{fis}}(\mathbf{x},\di,E,t)\,.
\end{equation}
%
The source terms $Q_{\text{sca}}$ and $Q_{\text{fis}}$ linearly
depend on the solution variable, the angular flux, denoted by
$\psi$. Only simple configurations are amenable to an analytical
solution of Equation~\reftext{\eqref{eq:transport_scalar}}. In most cases of
relevance, the transport equation must be solved numerically; transport
calculations fall under two main categories: stochastic calculations and
deterministic calculations. The former category is referred to as Monte
Carlo and relies on sampling large numbers of particle histories using
random number generators \cite{glasstone}, and the latter involves
discretization of the phase-space and the use of iterative techniques.
This work applies to the latter category. One common angular
discretization is the discrete-ordinate or $S_{N}$ method
\cite{glasstone,lewis,duderstadt}; it is a collocation method in angle
whereby the transport equation is solved only along discrete directions
$\di_{d}$ ($1 \le d \le n_{\di}$, with $n_{\di}$ the total number of
discrete directions). One of the main advantages of the $S_{N}$
technique is that it enables an iterative approach, called source
iteration in the transport literature
\cite{glasstone,lewis,duderstadt}, to resolve both the particle's
streaming and interaction processes and the scattering events as
follows:
%
%e3 #&#
\begin{equation}
\label{eq:SI}
\frac{1}{v}\frac{\partial\psi_{d}^{(\ell)}}{\partial t}
+ \di_{d}
\cdot\nabla\psi_{d}^{(\ell)}
+ \Sigma_{\text{t}}\psi_{d}^{(
\ell)} = Q_{\text{tot},d}^{(\ell-1)} \quad\forall d \in[1,n_{\di}]
\,,
\end{equation}
%
where $\ell$ is the iteration index and $\psi_{d}^{(\ell)}(
\mathbf{x},E,t) = \psi^{(\ell)}(\mathbf{x},\di_{d},E,t)$. Hence, a
system of $n_{\di}$ decoupled equations are to be solved at a given
source iteration index~$\ell$. For curvilinear geometries, an angular
derivative term is present at iteration~$\ell$, and thus the equations
are not decoupled; in this case, the scalar FCT methodology discussed
in this work requires amendment. This allows solution techniques for
scalar conservation laws to be leveraged in solving the system given by
Equation~\reftext{\eqref{eq:SI}}. For brevity, the discrete-ordinate subscript
$d$ will be omitted hereafter and our model transport equation will
consist in one of the $n_{\di}$ transport equations for a given fixed
source (right-hand side).

A common spatial discretization method for the $S_{N}$ equations has
been a Discontinuous Galerkin finite element method (DGFEM) with
upwinding \cite{Lesaint1974,Reed_Hill_1973}. Here, however, a
Continuous Galerkin finite element method (CGFEM) is applied. Some
recent work by Guermond and Popov \cite{guermond_ev} on solution
techniques for conservation laws with CGFEM addresses some of the main
disadvantages of CGFEM versus DGFEM, including the formation of spurious
oscillations. The purpose of the present paper is to demonstrate a proof
of concept for the application of such solution techniques to the
transport equation. Furthermore, some or all of the methodology explored
in this paper can be later extended to DGFEM as well; see, for instance,
Zingan et al. \cite{zingan_2013} where the techniques proposed by
Guermond and Popov \cite{guermond_ev} have been ported to DGFEM
schemes for Euler equations.

One of the main objectives of this paper is to present a method that
precludes the formation of spurious oscillations and the negativities
that result from these oscillations; these issues have been a
long-standing issue in the numerical solution of the transport equation
\cite{lathrop}. Not only are these negativities physically
incorrect (a particle's distribution density must be non-negative), but
they can cause simulations to terminate prematurely, for example in
radiative transfer where the radiation field is nonlinearly coupled to
a material energy equation. Many attempts to remedy the negativities in
transport solutions rely on ad-hoc fix-ups, such as the set-to-zero
fix-up for the classic diamond difference scheme \cite{lewis}.
Recent work by Hamilton introduced a similar fix-up for the linear
discontinuous finite element method (LDFEM) that conserves local balance
and preserves third-order accuracy \cite{hamilton}. Walters and
Wareing developed characteristic methods \cite{walters_NC}, but
Wareing later notes that these characteristic methods are difficult to
implement and offers a nonlinear positive spatial differencing scheme
known as the exponential discontinuous scheme \cite{wareing}.
Maginot has recently developed a consistent set-to-zero (CSZ) LDFEM
method \cite{maginot}, as well as a non-negative method for
bilinear discontinuous FEM
\cite{maginot_mc2015,maginot_2017}.

In fluid dynamics, traditional approaches to remedy the issue of
spurious oscillations include the flux-corrected transport (FCT)
algorithm, introduced in 1973 as the SHASTA algorithm for finite
difference discretizations by Boris and Book \cite{borisbook},
where it was applied to linear discontinuities and gas dynamic shock
waves. To the best of our knowledge, these FCT techniques have not been
applied to the particle transport equation. The main idea of the FCT
algorithm is to blend a low-order scheme having desirable properties
with a high-order scheme which may lack these properties. Zalesak
improved methodology of the algorithm and introduced a fully
multi-dimensional limiter \cite{zalesak}. Parrott and Christie
extended the algorithm to the finite element method on unstructured
grids \cite{parrott}, thus beginning the FEM-FCT methodology.
L\"{o}hner et al. applied FEM-FCT to the Euler and Navier--Stokes
equations and began using FCT with complex geometries
\cite{lohner}. Kuzmin and M\"{o}ller introduced an algebraic FCT
approach for scalar conservation laws \cite{kuzmin_FCT} and later
introduced a general-purpose FCT scheme, which is designed to be
applicable to both steady-state and transient problems
\cite{kuzmin_general}. In these FEM-FCT works and others
\cite{moller_2008,kuzmin_failsafe,kuzmin_closepacking}, the high-order
scheme used in the FCT algorithm was the Galerkin finite element method,
but this work uses the entropy viscosity method developed by Guermond
and others \cite{guermond_ev}.

Recent work by Guermond and Popov addresses the issue of spurious
oscillations for general conservation laws by using artificial
dissipation based on local entropy production, a method known as entropy
viscosity \cite{guermond_ev}. The idea of entropy viscosity is to
enforce an entropy inequality on the weak solution, and thus filter out
weak solutions containing spurious oscillations. However, entropy
viscosity solutions may still contain spurious oscillations, albeit
smaller in magnitude, and consequently negativities are not precluded.
To circumvent this deficiency, Guermond proposed using the entropy
viscosity method in conjunction with the FCT algorithm
\cite{guermond_secondorder}; the high-order scheme component in FCT,
traditionally the unmodified Galerkin scheme, is replaced with the
entropy viscosity scheme. For the low-order scheme, Guermond also
introduced a discrete maximum principle (DMP) preserving (and
positivity-preserving) scheme for scalar conservation laws
\cite{guermond_firstorder}.

The algorithm described in this paper takes a similar approach to the
algorithm described in the work by Guermond and Popov for scalar
conservation laws, but is extended to allow application to the transport
equation, which does not fit the precise definition of a conservation
law but is instead a balance law since it includes sinks and sources,
namely the reaction term $\Sigma_{\text{t}}\psi$ and the source term
$Q_{\text{tot}}$. The presence of these terms is also a novelty in the
context of the FCT algorithm. In addition, much of the work on FCT to
date has been for fully explicit time discretizations. Because speeds
in radiation transport (such as the speed of light in the case of
photons) are so large, implicit and steady-state time discretization are
important considerations, given the CFL time step size restriction for
fully explicit methods. Thus this paper also considers implicit and
steady-state FCT.

This paper is organized as follows. Section~\ref{sec:preliminaries}
gives some preliminaries such as the problem formulation and
discretization. Recall that the FCT algorithm uses a low-order scheme
and a high-order scheme. Section~\ref{sec:low} presents the low-order
scheme, Section~\ref{sec:high} presents the high-order scheme (which is
based on entropy viscosity), and Section~\ref{sec:fct} presents the FCT
scheme that combines the two. Section~\ref{sec:results} presents results
for a number of test problems, and Section~\ref{sec:conclusions} gives
conclusions.

%s2 #&#
\section{Preliminaries}%
\label{sec:preliminaries}
For the remainder of this paper, the scalar transport model given by
Equation~\reftext{\eqref{eq:transport_scalar}} will be generalized to a scalar
balance equation having reaction terms and source terms, with the
following notation:
%
%e4 #&#
\begin{equation}
\label{eq:scalar_model}
\frac{\partial u}{\partial t} + v\di\cdot\nabla u(\mathbf{x},t)+
\sigma(\mathbf{x}) u(\mathbf{x},t)= q(\mathbf{x},t)\,,
\end{equation}
%
where $u$ is the balanced quantity, $v$ is the transport speed,
$\di$ is a constant, uniform unit direction vector, $\sigma$ is the
reaction coefficient, and $q$ is the source function, possibly including
contributions from an external source, inscattering, and fission. These
contributions are all physically non-negative, but it should be noted
that in practical deterministic simulations, if the scattering source
is not isotropic, the scattering term may be negative due to its
approximation as a truncated Legendre polynomial expansion. However,
this work makes the assumption that the source is non-negative:
$q \ge0$; the proof of non-negativity of the solution relies on this
assumption. With anisotropic sources, extra precautions may need to be
taken, but this topic is not explored in this preliminary work on the
subject.

The problem formulation is completed by supplying initial conditions on
the problem domain $\mathcal{D}$ (for transient problems):
%
%e5 #&#
\begin{equation}
u(\mathbf{x},0) = u^{0}(\mathbf{x}) \quad\mathbf{x}\in\mathcal
{D}\,,
\end{equation}
%
as well as boundary conditions, which will be assumed to be incoming
flux boundary conditions:
%
%e6 #&#
\begin{equation}
u(\mathbf{x},t)= u^{\text{inc}}(\mathbf{x},t)\quad\mathbf{x}\in
\partial\mathcal{D}^{-} \,,
\end{equation}
%
where $u^{\text{inc}}(\mathbf{x},t)$ is the incoming boundary data
function, and $\partial\mathcal{D}^{-}$ is the incoming portion of the
domain boundary:
%
%e7 #&#
\begin{equation}
\partial\mathcal{D}^{-} \equiv\{ \mathbf{x}\in\partial\mathcal{D}:
\mathbf{n}(\mathbf{x})\cdot\di\leq0 \} \,,
\end{equation}
%
where $\mathbf{n}(\mathbf{x})$ is the outward-pointing normal vector on
the domain boundary at point $\mathbf{x}$.

Application of the standard Galerkin method with piecewise linear basis
functions gives the following semi-discrete system:
%
%e8 #&#
\begin{subequations}
\label{eq:galerkin_semidiscrete}
%
%e8.a #&#
\begin{equation}
\mathbf{M}^{C}\frac{d\mathbf{U}}{dt} + \mathbf{A}\mathbf{U}(t) =
\mathbf{b}(t) \,,
\end{equation}
%
where the consistent (i.e., not lumped) mass matrix is given by
%
%e8.b #&#
\begin{equation}
M^{C}_{i,j} \equiv\int\limits_{S_{i,j}}\varphi_{i}(\mathbf
{x})\varphi
_{j}(\mathbf{x}) dV \,,
\end{equation}
%
the (steady-\xch{state}{tstate}) transport matrix is
%
%e8.c #&#
\begin{equation}
\label{eq:Aij}
A_{i,j} \equiv\int\limits_{S_{i,j}}\left(
v\di\cdot\nabla\varphi
_{j}(\mathbf{x}) +
\sigma(\mathbf{x})\varphi_{j}(\mathbf{x})\right)
\varphi_{i}(\mathbf{x}) dV \,,
\end{equation}
%
and the right-hand-side is
%
%e8.d #&#
\begin{equation}
b_{i}(t) \equiv\int\limits_{S_{i}}q(\mathbf{x})\varphi_{i}(
\mathbf{x}) dV \,.
\end{equation}
%
\end{subequations}
%
The components of the solution vector $\mathbf{U}(t)$ are denoted by
$U_{j}(t)$ and represent the degrees of freedom of the approximate
solution $u_{h}$:
%
%e9 #&#
\begin{equation}
u_{h}(\mathbf{x},t)= \sum\limits_{j}U_{j}(t) \varphi_{j}(\mathbf{x})
\,,
\end{equation}
%
where $\varphi_{j}(\mathbf{x})$ is a finite element test function.
$S_{i}$ is the support of basis function $i$ and $S_{i,j}$ is the shared
support of basis functions $i$ and $j$.

A number of temporal discretizations are considered in this paper. Fully
explicit temporal discretizations considered include forward Euler:
%
%e10 #&#
\begin{equation}
\mathbf{M}^{C}\frac{\mathbf{U}^{n+1}-\mathbf{U}^{n}}{\Delta t} +
\mathbf{A}\mathbf{U}^{n} = \mathbf{b}^{n} \,,
\end{equation}
%
as well as Strong Stability Preserving Runge Kutta (SSPRK) methods that
can be expressed in the following form:
%
%e11 #&#
\begin{subequations}
\label{eq:ssprk}
%
%e11.a #&#
%e11.b #&#
%e11.c #&#
\begin{align}
& \hat{\mathbf{U}}^{0} = \mathbf{U}^{n} \,,
\eqncr
& \hat{\mathbf{U}}^{i} = \gamma_{i} \mathbf{U}^{n} + \zeta_{i}
\left[
\hat{\mathbf{U}}^{i-1}
+ \Delta t\mathbf{G}(t^{n}+c_{i}\Delta t,
\hat{\mathbf{U}}^{i-1})\right]
\,,\quad i = 1,\ldots,s
\,,
\eqncr
& \mathbf{U}^{n+1} = \hat{\mathbf{U}}^{s} \,.
\end{align}
%
\end{subequations}
%
where $s$ is the number of stages, $\gamma_{i}$, $\zeta_{i}$, and
$c_{i}$ are coefficients that correspond to the particular SSPRK method,
and $\mathbf{G}$ represents the right-hand-side function of an ODE
%
%e12 #&#
\begin{equation}
\frac{d\mathbf{U}}{dt} = \mathbf{G}(t,\mathbf{U}(t)) \,,
\end{equation}
%
which in this case is the following:
%
%e13 #&#
\begin{equation}
\mathbf{G}(t,\mathbf{U}(t)) = (\mathbf{M}^{C})^{-1}
\left( \mathbf{b}(t)
- \mathbf{A}\mathbf{U}(t)\right) \,.
\end{equation}
%
SSPRK methods are a subclass of Runge Kutta methods that offer
high-order accuracy while preserving stability
\cite{gottlieb,macdonald}. The form given in Equation \reftext{\eqref{eq:ssprk}} makes it clear that these SSPRK methods can be
expressed as a linear combination of steps resembling forward Euler
steps, with the only difference being that the explicit time dependence
of the source is not necessarily on the old time $t^{n}$ but instead is
on a stage time $t^{n} + c_{i}\Delta t$. An example is the 3-stage,
3rd-order accurate SSPRK method has the following coefficients:
%
%e14 #&#
\begin{equation}
\gamma= \left[
%
\begin{array}{c}
0
\\
\frac{3}{4}
\cr \noalign{\vspace{2pt}}
\frac{1}{3}
\end{array}
%
\right]
\,,\quad\zeta= \left[
%
\begin{array}{c}
1
\\
\frac{1}{4}
\cr \noalign{\vspace{2pt}}
\frac{2}{3}
\end{array}
%
\right]
\,,\quad c = \left[
%
\begin{array}{c}
0
\\
1
\\
\frac{1}{2}
\end{array}
%
\right] \,.
\end{equation}

This work also considers the Theta-family of temporal discretizations:
%
%e15 #&#
\begin{equation}
\mathbf{M}^{C}\frac{\mathbf{U}^{n+1}-\mathbf{U}^{n}}{\Delta t} +
\mathbf{A}((1-\theta)\mathbf{U}^{n} + \theta\mathbf{U}^{n+1})
= (1-
\theta)\mathbf{b}^{n} + \theta\mathbf{b}^{n+1} \,,
\end{equation}
%
where $0\leq\theta\leq1$ is the implicitness parameter. For example,
$\theta$ values of $0$, $\frac{1}{2}$, and $1$ correspond to forward
Euler, Crank--\xch{Nicholson}{Nicohlson}, and backward Euler discretizations, respectively.

Finally, in the case of a steady-state solve, we have the following
system of equations:
%
%e16 #&#
\begin{equation}
\mathbf{A}\mathbf{U}= \mathbf{b}\,.
\end{equation}

%s3 #&#
\section{FCT methodology applied to particle transport}%
\label{sec:methodology}

Recall that the FCT algorithm is built from a low-order scheme and a
high-order scheme. Section~\ref{sec:low} describes the low-order scheme,
and Section~\ref{sec:high} describes the high-order scheme. Section~\ref{sec:fct} describes the FCT scheme combined from these components.

%s3.1 #&#
\subsection{Low-order scheme}%
\label{sec:low}

The role of a low-order scheme in the context of the FCT algorithm is
to provide a fail-safe solution that has desirable properties such as
positivity-preservation and lack of spurious oscillations. These
properties come at the cost of excessive artificial diffusion and thus
a lesser degree of accuracy. However, the idea of the FCT algorithm is
to undo some of the over-dissipation of the low-order scheme as much as
possible without violating some physically-motivated solution bounds.

Here positivity-preservation and monotonicity are achieved by requiring
\xch{that}{that that} the matrix of the low-order system satisfies the M-matrix
property. M-matrices are a subset of inverse-positive matrices and have
the monotone property. For instance, consider the linear system
$\mathbf{A}\mathbf{x}= \mathbf{b}$; If $\mathbf{A}$ is an M-matrix, then
the following property is verified:
%
%e17 #&#
\begin{equation}
\text{If } \mathbf{b}\geq0, \text{ then } \mathbf{x}\geq0 \,.
\end{equation}
%
Hence, given that the linear system matrix is an M-matrix,
positivity-preservation is proven by proving positivity of the
right-hand-side vector $\mathbf{b}$. This monotonicity property of the
linear system matrix is also responsible for the satisfaction of a
discrete maximum principle \cite{guermond_firstorder}.

In this section, a first-order viscosity method introduced by Guermond
\cite{guermond_firstorder} will be adapted to the transport
equation given by Equation \reftext{\eqref{eq:scalar_model}}. This method uses an
element-wise artificial viscosity definition in conjunction with a
graph-theoretic local viscous bilinear form that makes the method valid
for arbitrary element shapes and dimensions. These definitions will be
shown to ensure that the system matrix is a non-singular M-matrix.

The graph-theoretic local viscous bilinear form has the following
definition.

%d1 #&#
\begin{defn}[Local viscous bilinear form] The local viscous bilinear form for
element $K$ is defined as follows:
%
%e18 #&#
\begin{equation}
\label{eq:bilinearform}
d_{K}(\varphi_{j},\varphi_{i}) \equiv\left\{
%
\begin{array}{l@{\quad} l}
-\frac{1}{n_{K} - 1}V_{K} & i\ne j\,,\quad i,j\in\mathcal{I}_{K}\,,
\\
V_{K} & i = j \,,\quad i,j\in\mathcal{I}_{K}\,,
\\
0 & \mbox{otherwise}\,,
\end{array}
%
\right.
\end{equation}
%
where $V_{K}$ is the volume of cell $K$, $\mathcal{I}_{K}$ is the set
of degree of freedom indices such that the corresponding test function
has support on cell $K$, and $n_{K}$ is the number of indices in that
set.
\end{defn}

This bilinear form bears resemblance to a standard Laplacian bilinear
form: the diagonal entries are positive, the off-diagonal entries are
negative, and the row sums are zero. These facts will be invoked in the
proof of the M-matrix conditions later in this section.

The element-wise low-order viscosity definition from
\cite{guermond_firstorder} is adapted to account for the reaction term
in the transport equation, Equation \reftext{\eqref{eq:scalar_model}}, but
otherwise remains unchanged.

%d2 #&#
\begin{defn}[Low-order viscosity] The low-order viscosity for cell $K$ is defined
as follows:
%
%e19 #&#
\begin{equation}
\nu^{L}_{K} \equiv\max\limits_{i\ne j\in\mathcal{I}_{K}}
\frac{
\max(0,A_{i,j})}{-\sum\limits_{T\in\mathcal{K}(S_{i,j})}d_{T}(
\varphi_{j},\varphi_{i})}
\,,
\end{equation}
%
where $A_{i,j}$ is the $i,j$ entry of matrix $\mathbf{A}$ given by
Equation \reftext{\eqref{eq:Aij}}, $\mathcal{I}_{K}$ is the set of degree of
freedom indices corresponding to basis functions that have support on
cell $K$ (this is illustrated in \reftext{Fig.~\ref{fig:cell_indices}} -- the
indicated nodes have degree of freedom indices belonging to
$\mathcal{I}_{K}$), and $\mathcal{K}(S_{i,j})$ is the set of cell
indices for which the cell domain and the shared support $S_{i,j}$
overlap.
\end{defn}

This viscosity definition is designed to give the minimum amount of
artificial diffusion without violating the M-matrix conditions.

%f1 #&#
\begin{figure}
\includegraphics{7667f01}
\caption{Illustration of cell degree of freedom indices $\mathcal{I}_{K}$.}
\label{fig:cell_indices}
\end{figure}

Now that the low-order artificial diffusion operator (bilinear form $+$
viscosity definitions) has been provided, we describe the low-order
system. Consider a modification of the Galerkin scheme given in Equation \reftext{\eqref{eq:galerkin_semidiscrete}} which lumps the mass matrix ($
\mathbf{M}^{C} \rightarrow\mathbf{M}^{L}$) and adds an artificial
diffusion operator $\mathbf{D}^{L}$, hereafter called the low-order
diffusion matrix:
%
%e20 #&#
\begin{equation}
\mathbf{M}^{L}\frac{d\mathbf{U}^{L}}{dt} + (\mathbf{A}+ \mathbf{D}
^{L})\mathbf{U}^{L}(t) = \mathbf{b}(t) \,,
\end{equation}
%
where $\mathbf{U}^{L}(t)$ denotes the discrete low-order solution
values. Defining the low-order steady-state system matrix $\mathbf{A}
^{L}\equiv\mathbf{A}+ \mathbf{D}^{L}$, the low-order system for the
steady-state system, explicit Euler system, and Theta system,
respectively, are:\\
%
%e21 #&#
\begin{subequations}
Steady-state:
%
%e21.a #&#
\begin{equation}
\label{eq:low_ss}
\mathbf{A}^{L} \mathbf{U}^{L} = \mathbf{b}\,\xch{.}{,}
\end{equation}
%
Explicit Euler:
%
%e21.b #&#
\begin{equation}
\label{eq:low_fe}
\mathbf{M}^{L} \mathbf{U}^{L,n+1} = \mathbf{M}^{L} \mathbf{U}^{n} +
\Delta t\left( \mathbf{b}^{n} - \mathbf{A}^{L}\mathbf{U}^{n} \right)
\,\xch{.}{,}
\end{equation}
%
Theta scheme:
%
%e21.c #&#
\begin{equation}
\label{eq:low_theta}
\left( \mathbf{M}^{L} +\theta\Delta t\mathbf{A}^{L}\right)
\mathbf{U}^{L,n+1}
= \mathbf{M}^{L} \mathbf{U}^{n} + \Delta t(
\mathbf{b}^{\theta}- \mathbf{A}^{L} (1-\theta)\mathbf{U}^{n} )\,,
\end{equation}
%
\end{subequations}
%
where $\mathbf{b}^{\theta}\equiv(1-\theta)\mathbf{b}^{n} + \theta
\mathbf{b}^{n+1}$. The low-order diffusion matrix is assembled
element-wise using the local viscous bilinear form and low-order
viscosity definitions:
%
%e22 #&#
\begin{equation}
\label{eq:low_order_diffusion_matrix}
D_{i,j}^{L} \equiv\sum\limits_{K\in\mathcal{K}(S_{i,j})}\nu^{L}_{K}
d_{K}(\varphi_{j},\varphi_{i}) \,.
\end{equation}

Now the low-order scheme has been fully described, some statements will
be made on its properties. Firstly the M-matrix property will be shown
for the low-order matrix $\mathbf{A}^{L}$.

%t1 #&#
\begin{thm}[M-matrix property]
The low-order steady-state system matrix $\mathbf{A}^{L}$ is a
non-singular M-matrix.
\end{thm}

\begin{proof}
There are many definitions that can be used to identify a non-singular
M-matrix; one definition gives that an M-matrix can be identified by
verifying both of the following properties \cite{plemmons}:
%
\begin{enumerate}
%
\item
strict positivity of diagonal entries: $A_{i,i} > 0$, $\forall i$ and
%
\item
non-positivity of off-diagonal entries: $A_{i,j} \leq0$, $\forall i$,
$\forall j\ne i$.
\end{enumerate}
%
First, we show that the off-diagonal elements of the matrix
$\mathbf{A}^{L}$ are non-positive. The diffusion matrix entry
$D^{L}_{i,j}$ is bounded as follows:
%
\begin{eqnarray*}
D_{i,j}^{L}
& = & \sum\limits_{K\in\mathcal{K}(S_{i,j})}\nu^{L}_{K} d
_{K}(\varphi_{j},\varphi_{i})
\\
& = & \sum\limits_{K\in\mathcal{K}(S_{i,j})}\max
\limits_{k\ne\ell\in\mathcal{I}_{K}}
\left( \frac{\max(0,A_{k,
\ell})}{-\sum\limits_{T\in\mathcal{K}(S_{k,\ell})}
\mkern-20mu d
_{T}(\varphi_{\ell},\varphi_{k})}\right) d_{K}(\varphi_{j},\varphi
_{i}) \,.
\end{eqnarray*}
%
For an arbitrary quantity $c_{k,\ell} \geq0 \,,\forall k\ne\ell
\in\mathcal{I}$, the following is true for $i\ne j\in\mathcal{I}$:
$\max\limits_{k\ne\ell\in\mathcal{I}}c_{k,\ell} \geq c_{i,j}$, and
thus for $a\leq0$, $a\max\limits_{k\ne\ell\in\mathcal{I}}c_{k,
\ell} \leq a c_{i,j}$. Recall that $d_{K}(\varphi_{j},\varphi_{i}) <
0$ for $j\ne i$. Thus, we have:
%
\begin{eqnarray*}
D^{L}_{i,j}
& \le& \sum\limits_{K\in\mathcal{K}(S_{i,j})}\frac{
\max(0,A_{i,j})}{-\sum\limits_{T\in\mathcal{K}(S_{i,j})} d_{T}(
\varphi_{j},\varphi_{i})}
d_{K}(\varphi_{j},\varphi_{i}) \,,
\qquad
j\ne i\,,
\\
& = & -\max(0,A_{i,j}) \frac{\sum\limits_{K\in\mathcal{K}(S_{i,j})}d
_{K}(\varphi_{j},\varphi_{i})}{\sum\limits_{T\in\mathcal{K}(S_{i,j})}
d_{T}(\varphi_{j},\varphi_{i})} \,,
\qquad
j\ne i\,,
\\
& = & -\max(0,A_{i,j}) \,,
\qquad
j\ne i\,,
\\
& \le& -A_{i,j}\,,
\qquad
j\ne i\,.
\end{eqnarray*}
%
Then applying this relation to the definition of the low-order steady
state matrix gives
%
\begin{equation*}
A^{L}_{i,j} = A_{i,j} + D_{i,j}^{L} \le0 \,.
\end{equation*}
%
Next it will be shown that the row sums are non-negative. Using the fact
that $\sum\limits_{j}\varphi_{j}(\mathbf{x})=1$ and $\sum\limits_{j}d
_{K}(\varphi_{j},\varphi_{i})=0$,
%
\begin{eqnarray*}
\sum\limits_{j}A^{L}_{i,j}
& = & \sum\limits_{j}\int\limits_{S_{i,j}}
\left( \mathbf{f}'(u_{h})
\cdot\nabla\varphi_{j} +
\sigma\varphi
_{j}\right) \varphi_{i} dV +
\sum\limits_{j}\sum
\limits_{K\in\mathcal{K}(S_{i,j})}\nu^{L}
d_{K}(\varphi_{j},\varphi
_{i})
\,,
\\
& = & \int\limits_{S_{i}}\left(
\mathbf{f}'(u_{h})\cdot\nabla
\sum\limits_{j}\varphi_{j}(\mathbf{x}) +
\sigma(\mathbf{x})\sum
\limits_{j}\varphi_{j}(\mathbf{x})\right)
\varphi_{i}(\mathbf{x}) dV
\,,
\\
\label{eq:rowsum}
& = & \int\limits_{S_{i}}\sigma(\mathbf{x})\varphi_{i}(\mathbf
{x}) dV
\,,
\\
& \ge& 0 \,.
\end{eqnarray*}

%r1 #&#
\begin{rmk}
If incoming flux boundary conditions are weakly imposed, then the
steady-state system matrix is modified: $\mathbf{A}\rightarrow
\tilde{\mathbf{A}}$, and the low-order viscosity then uses the
\emph{modified} steady-state matrix $\tilde{\mathbf{A}}$. The
non-positivity property of the off-diagonal elements still holds. The
non-negativity property of the row sums also holds, owing to the
relation $\tilde{A}_{i,j} \geq A_{i,j}$.
\end{rmk}

If the support $S_{i}$ is not entirely vacuum ($\sigma(\mathbf{x})
\ge0$ with $\sigma(\mathbf{x}) > 0$ for some $\mathbf{x}$), then the
row sum is \emph{strictly} positive. Proof of strict positivity of the
diagonal elements directly follows from proof of non-positivity of the
off-diagonal elements and strict positivity of the row sums. Thus both
conditions for the non-singular M-matrix property have been met.
\end{proof}

Thus far, we have proven that the system matrix for the low-order
steady-state system is an M-matrix, and it remains to demonstrate the
same for each of the transient systems. For the explicit Euler/SSPRK
systems, the system matrix is just the lumped mass matrix $\mathbf{M}
^{L}$, which is easily shown to be an M-matrix since it is a positive,
diagonal matrix. For the $\theta$ temporal discretization, the system
matrix is a linear combination of the lumped mass matrix and the
low-order steady-state system matrix; this linear combination is also
an M-matrix since it is a combination of two M-matrices with
non-negative combination coefficients.

To complete the proof of positivity preservation for the low-order
scheme, we need to show that the system right-hand-side vectors for each
temporal discretization are non-negative.

This is immediate for the steady-state case due to the assumption that
the source $q$ is non-negative. The following theorem gives that the
system right-hand-side vector for the theta system is non-negative. This
theorem extends to explicit Euler discretization since explicit Euler
is a special case of the Theta discretization.

%t2 #&#
\begin{thm}[Non-negativity of the theta low-order system right-hand-side] If the
old solution $\mathbf{U}^{n}$ is non-negative and the time step size
$\Delta t$ satisfies
%
%e23 #&#
\begin{equation}
\label{eq:theta_cfl}
\Delta t\leq\frac{M^{L}_{i,i}}{(1-\theta)A_{i,i}^{L}}
\,,\quad
\forall i \,,
\end{equation}
%
then the new solution $\mathbf{U}^{L,n+1}$ of the Theta low-order system
given by Equation \reftext{\eqref{eq:low_theta}} is non-negative, i.e.,
$U^{L,n+1}_{i} \geq0$, $\forall i$.
\end{thm}

\begin{proof}
The right-hand-side vector $\mathbf{y}$ of this system has the entries
%
\begin{equation*}
y_{i} = \Delta tb^{\theta}_{i} + \left( M^{L}_{i,i} - (1-\theta)
\Delta tA^{L}_{i,i}\right) U^{n}_{i}
- (1-\theta)\Delta t\sum
\limits_{j\ne i}A^{L}_{i,j} U^{n}_{j}
\,.
\end{equation*}
%
As stated previously, the source function $q$ is non-negative and thus
$b^{\theta}_{i} \ge0 $. Due to the time step size assumption given by
Equation \reftext{\eqref{eq:theta_cfl}},
%
\begin{equation*}
M^{L}_{i,i} - (1-\theta)\Delta tA^{L}_{i,i} \geq0 \,,
\end{equation*}
%
and because the off-diagonal terms of $\mathbf{A}^{L}$ are non-positive,
the off-diagonal sum term is non-negative. Thus $y_{i}$ is a sum of
non-negative terms, and the theorem is proven.
\end{proof}

It can also be shown that the described low-order scheme satisfies a
local discrete maximum principle, which is easily shown given the
M-matrix property. One may decide to use these bounds as the imposed
bounds in the FCT algorithm; however, this approach has been found to
yield less accurate solutions than the approach to be outlined in
Section~\ref{sec:fct} and is thus not discussed here for brevity.

%s3.2 #&#
\subsection{High-order scheme}%
\label{sec:high}

This section describes the entropy viscosity method applied to the
scalar conservation law given by Equation \reftext{\eqref{eq:scalar_model}}.
Recall that the entropy viscosity method is to be used as the high-order
scheme in the FCT algorithm, instead of the standard Galerkin method as
has been used previously in FCT-FEM schemes; for Galerkin FCT-FEM
examples, see, for instance,
\cite{kuzmin_FCT,moller_2008,lohner,kuzmin_failsafe,kuzmin_closepacking}.
Usage of the entropy viscosity method in the FCT algorithm ensures
convergence to the entropy solution \cite{guermond_secondorder}.

The entropy viscosity method has been applied to a number of PDEs such
as general scalar conservation laws of the form
%
%e24 #&#
\begin{equation}
\label{eq:scalar_conservation_law}
\frac{\partial u}{\partial t} + \nabla\cdot\mathbf{f}(u) = 0 \,,
\end{equation}
%
the inviscid Euler equations \cite{guermond_ev,marco_low_mach},
and the two-phase seven-equation fluid model \cite{marco_SEM}. The
scalar model studied in this paper does not fit into the general form
given by Equation~\reftext{\eqref{eq:scalar_conservation_law}}, due to the
addition of the reaction term $\sigma u$ and source term $q$.
Application of entropy viscosity method to the transport equation model
is novel and it is described below.

Since the weak form of the problem does not have a unique solution, one
must supply additional conditions called \emph{admissibility} conditions
or \emph{entropy} conditions to filter out spurious weak solutions,
leaving only the physical weak solution, often called the entropy
solution. A number of entropy conditions are valid, but usually the most
convenient entropy condition for use in numerical methods takes the form
of an \emph{entropy inequality}, such as the following, which is valid
for the general scalar conservation law given by Equation \reftext{\eqref{eq:scalar_conservation_law}}:
%
%e25 #&#
\begin{equation}
\frac{\partial\eta(u)}{\partial t} + \nabla\cdot\bm{\Psi}(u)
\leq0 \,,
\end{equation}
%
which holds for any convex entropy function $\eta(u)$ and associated
entropy flux $\bm{\Psi}(u) \equiv\int\eta'(u)\mathbf{f}'(u)du$.
If one can show that this inequality holds for an arbitrary convex
entropy function, then one proves it holds for all convex entropy
functions \cite{leveque2002,guermond_ev}. For the scalar PDE
considered in this paper, the entropy inequality becomes the following:
%
%e26 #&#
\begin{equation}
\frac{\partial\eta(u)}{\partial t} + \nabla\cdot\bm{\Psi}(u)
+ \eta'(u)\sigma u - \eta'(u)q
\leq0 \,.
\end{equation}
%
One can verify this inequality by multiplying the governing PDE by
$\eta'(u)$ and applying reverse chain rule.

The entropy viscosity method enforces the entropy inequality by
measuring local entropy production and dissipating accordingly. In
practice, one defines the entropy residual:
%
%e27 #&#
\begin{equation}
\mathcal{R}(u) \equiv\frac{\partial\eta(u)}{\partial t} + \nabla
\cdot\bm{\Psi}(u)
+ \eta'(u)\sigma u - \eta'(u)q \,,
\end{equation}
%
which can be viewed as the amount of violation of the entropy
inequality. The entropy viscosity for an element $K$ is then defined to
be proportional to this violation, for example:
%
%e28 #&#
\begin{equation}
\nu^{\eta}_{K} = \frac{c_{\mathcal{R}}\|\mathcal{R}(u_{h})\|_{L^{
\infty}(K)}}{\hat{\eta}_{K}}
\,,
\end{equation}
%
where $\hat{\eta}_{K}$ is a normalization constant with the units of
entropy, $c_{\mathcal{R}}$ is a proportionality constant that can be
modulated for each problem, and
$\|\mathcal{R}(u_{h})\|_{L^{\infty}(K)}$ is the maximum of the entropy
residual on element $K$, which can be approximated as the maximum over
the quadrature points of element $K$. Designing a universally
appropriate normalization constant $\hat{\eta}_{K}$ remains a challenge
for the entropy viscosity method (see \cite{marco_low_mach} for
an alternate normalization for low-Mach flows). The objective of this
normalization coefficient is to prevent the user from needing to make
significant adjustments to the tuning parameter $c_{\mathcal{R}}$ for
different problems. A definition that produces reasonable results for
a large number of problems is the following:
%
%e29 #&#
\begin{equation}
\hat{\eta}_{K} \equiv\left\| \eta-\bar{\eta}\right\|
_{L^{\infty}(
\mathcal{D})} \,,
\end{equation}
%
where $\bar{\eta}$ is the average entropy over the entire computational
domain.

In addition to the entropy residual, it can also be beneficial to
measure the jump in the gradient of the entropy flux across cell
interfaces. Note that given the definition of the entropy flux, the
gradient of the entropy flux is $\nabla\bm{\Psi}(u)=\nabla
\eta(u)\mathbf{f}'(u)$. Then let $\mathcal{J}_{F}$ denote the jump of
the normal component of the entropy flux gradient across face $F$:
%
%e30 #&#
\begin{equation}
\mathcal{J}_{F} \equiv|\mathbf{f}'(u)\cdot\mathbf{n}_{F}|
\Lbrack
\nabla\eta(u)\cdot\mathbf{n}_{F}\Rbrack \,,
\end{equation}
%
where the double square brackets denote a jump quantity. Then we define
the maximum jump on a cell:
%
%e31 #&#
\begin{equation}
\mathcal{J}_{K} \equiv\max\limits_{F\in\mathcal{F}(K)} |\mathcal{J}
_{F}| \,.
\end{equation}
%
Finally, putting everything together, one can define the entropy
viscosity for a cell $K$ to be
%
%e32 #&#
\begin{equation}
\nu^{\eta}_{K} = \frac{c_{\mathcal{R}}\|\mathcal{R}(u_{h})\|_{L^{
\infty}(K)}
+ c_{\mathcal{J}}\mathcal{J}_{K}}{\hat{\eta}_{K}}
\,.
\end{equation}
%
However, it is known that the low-order viscosity for an element, as
computed in Section~\ref{sec:low}, gives enough local artificial
diffusion for regularization; any amount of viscosity larger than this
low-order viscosity would be excessive. Thus, the low-order viscosity
for an element is imposed as the upper bound for the high-order
viscosity:
%
%e33 #&#
\begin{equation}
\nu^{H}_{K} \equiv\min(\nu^{L}_{K}, \nu^{\eta}_{K}) \,.
\end{equation}
%
One can note that, in smooth regions, this high-order viscosity will be
small, and, in regions of strong gradients or discontinuities, the
entropy viscosity can be relatively large.

Finally, the high-order system of equations for the steady-state system, explicit Euler system, and Theta system, respectively, are as follows:\\
%
%e34 #&#
\begin{subequations}
Steady-state:
%
%e34.a #&#
\begin{equation}
\label{eq:high_ss}
\mathbf{A}^{H} \mathbf{U}^{H} = \mathbf{b}\,,
\end{equation}
%
Explicit Euler:
%
%e34.b #&#
\begin{equation}
\label{eq:high_fe}
\mathbf{M}^{C}\frac{\mathbf{U}^{H,n+1} - \mathbf{U}^{n}}{\Delta t} +
\mathbf{A}^{H,n}\mathbf{U}^{n} = \mathbf{b}^{n} \,,
\end{equation}
%
Theta scheme:
%
%e34.c #&#
\begin{equation}
\label{eq:high_theta}
\mathbf{M}^{C}\frac{\mathbf{U}^{H,n+1} - \mathbf{U}^{n}}{\Delta t} +
\theta\mathbf{A}^{H,n+1}\mathbf{U}^{H,n+1}
+ (1-\theta)\mathbf{A}
^{H,n}\mathbf{U}^{n}
= \mathbf{b}^{\theta}\,,
\end{equation}
%
\end{subequations}
%
where the high-order steady-state system matrix is defined as
$\mathbf{A}^{H}\equiv\mathbf{A}+ \mathbf{D}^{H}$, and the high-order
diffusion matrix $\mathbf{D}^{H}$ is defined similarly to the low-order
case but using the high-order viscosity instead of the low-order
viscosity:
%
%e35 #&#
\begin{equation}
\label{eq:high_order_diffusion_matrix}
D_{i,j}^{H} \equiv\sum\limits_{K\in\mathcal{K}(S_{i,j})}\nu^{H}_{K}
d_{K}(\varphi_{j},\varphi_{i}) \,.
\end{equation}
%
Note that unlike the low-order scheme, the high-order scheme does not
lump the mass matrix.

%r2 #&#
\begin{rmk}
Note that due to the nonlinearity of the entropy viscosity, the entropy
viscosity scheme is nonlinear for implicit and steady-state temporal
discretizations, and thus some nonlinear solution technique must be
utilized. For the results in this paper, a simple fixed-point iteration
scheme is used. An alternative such as Newton's method is likely to be
advantageous in terms of the number of nonlinear iterations; however,
fixed-point is used here for comparison with the nonlinear FCT scheme
to be described in Section~\ref{sec:fct}.
\end{rmk}

%s3.3 #&#
\subsection{FCT scheme}%
\label{sec:fct}

%s3.3.1 #&#
\subsubsection{The FCT system}

The entropy viscosity method described in Section~\ref{sec:high}
enforces the entropy condition and thus produces numerical
approximations that converge to the entropy solution. However, numerical
solutions may still contain spurious oscillations and negativities,
although these effects are smaller in magnitude than for the
corresponding Galerkin solution. In this paper, the flux-corrected
transport (FCT) algorithm is used to further mitigate the formation of
spurious oscillations and to guarantee the absence of negativities.

The first ingredient of the FCT algorithm is the definition of the
antidiffusive fluxes. To arrive at this definition, the low-order
systems, given by Equations \reftext{\eqref{eq:low_ss}}, \reftext{\eqref{eq:low_fe}}, and \reftext{\eqref{eq:low_theta}} for each temporal discretization, are augmented
with the addition of the \emph{antidiffusion source} $\mathbf{p}$, which
now, instead of producing the low-order solution $\mathbf{U}^{L}$,
produces the high-order solution $\mathbf{U}^{H}$:
%
%e36 #&#
\begin{subequations}
%
%e36.a #&#
%e36.b #&#
%e36.c #&#
\begin{eqnarray}[ll]
\label{eq:antidiffusionsource_ss}
\mathbf{A}^{L} \mathbf{U}^{H} = \mathbf{b}+ \mathbf{p}\,,\eqncr
\label{eq:antidiffusionsource_fe}
\mathbf{M}^{L}\frac{\mathbf{U}^{H} - \mathbf{U}^{n}}{\Delta t} +
\mathbf{A}^{L}\mathbf{U}^{n} = \mathbf{b}^{n} + \mathbf{p}^{n} \,
,\eqncr
\label{eq:antidiffusionsource_theta}
\mathbf{M}^{L}\frac{\mathbf{U}^{H} - \mathbf{U}^{n}}{\Delta t} +
\mathbf{A}^{L}\left( \theta\mathbf{U}^{H} +
(1-\theta)\mathbf{U}^{n}\right) = \mathbf{b}^{\theta}+ \mathbf{p}
^{\theta}\,.
\end{eqnarray}
%
\end{subequations}
%
Then the corresponding high-order systems, given by Equations \reftext{\eqref{eq:high_ss}}, \reftext{\eqref{eq:high_fe}}, \reftext{\eqref{eq:high_theta}} are
subtracted from these equations to give the following definitions for
$\mathbf{p}$:
%
%e37 #&#
\begin{subequations}
%
%e37.a #&#
%e37.b #&#
%e37.c #&#
\begin{eqnarray}[ll]
\label{eq:antidiffusionsourcei_ss}
\mathbf{p}\equiv\left( \mathbf{D}^{L} - \mathbf{D}^{H}\right)
\mathbf{U}^{H} \,,\eqncr
\label{eq:antidiffusionsourcei_fe}
\mathbf{p}^{n} \equiv-\left( \mathbf{M}^{C} - \mathbf{M}^{L}\right
) \frac{
\mathbf{U}^{H} - \mathbf{U}^{n}}{\Delta t} + \left( \mathbf{D}^{L} -
\mathbf{D}^{H}\right) \mathbf{U}^{n} \,,\eqncr
\label{eq:antidiffusionsourcei_theta}
\mathbf{p}^{\theta}\equiv-\left( \mathbf{M}^{C} - \mathbf
{M}^{L}\right) \frac{
\mathbf{U}^{H} - \mathbf{U}^{n}}{\Delta t}
+ (1-\theta)\left( \mathbf{D}
^{L} - \mathbf{D}^{H,n}\right) \mathbf{U}^{n}
+ \theta\left( \mathbf{D}^{L} - \mathbf{D}^{H,n+1}\right) \mathbf{U}
^{H} \,\xch{.}{,}
\end{eqnarray}
%
\end{subequations}
%
The next step is to decompose each antidiffusive source $p_{i}$ into a
sum of antidiffusive fluxes: $p_{i} = \sum_{j} P_{i,j}$. Because the
matrices $\mathbf{M}^{C}-\mathbf{M}^{L}$ and $\mathbf{D}^{L}-
\mathbf{D}^{H}$ are symmetric and feature row sums of zero, the
following are valid antidiffusive flux decompositions:
%
%e38 #&#
\begin{subequations}
%
%e38.a #&#
%e38.b #&#
%e38.c #&#
\begin{eqnarray}[ll]
P_{i,j} = \left( D^{L}_{i,j} - D^{H}_{i,j}\right) \left( U^{H}_{j} - U
^{H}_{i}\right) \,,\eqncr
P^{n}_{i,j} = -M^{C}_{i,j}\left( \frac{U^{H}_{j} - U^{n}_{j}}{\Delta t}
- \frac{U^{H}_{i} - U^{n}_{i}}{\Delta t}\right)
+ \left( D^{L}_{i,j} -
D^{H,n}_{i,j}\right) \left( U^{n}_{j} - U^{n}_{i}\right) \,,\eqncr
P^{\theta}_{i,j} = -M^{C}_{i,j}\left( \frac{U^{H}_{j} - U^{n}_{j}}{
\Delta t} - \frac{U^{H}_{i} - U^{n}_{i}}{\Delta t}\right)
+ (1-\theta
)\left( D^{L}_{i,j} - D^{H,n}_{i,j}\right) \left( U^{n}_{j} -
U^{n}_{i}\right)
+ \theta\left( D^{L}_{i,j} - D^{H,n+1}_{i,j}\right) \left(
U^{H}_{j} -
U^{H}_{i}\right) \,.\nonumber\\
\end{eqnarray}
%
\end{subequations}
%
Note that this decomposition yields equal and opposite antidiffusive
flux pairs since the antidiffusion matrix $\mathbf{P}$ is skew
symmetric: $P_{j,i}=-P_{i,j}$ (and likewise for $P_{j,i}^{n}$ and
$P_{j,i}^{\theta}$). Up until this point, no changes have been made to
the high-order scheme: solving Equations \reftext{\eqref{eq:antidiffusionsource_ss}} through \reftext{\eqref{eq:antidiffusionsource_theta}} still produces the high-order
solution. FCT is applied to these equations by applying limiting
coefficients $L_{i,j}$ to each antidiffusive flux $P_{i,j}$. Thus the
FCT systems are the following:
%
%e39 #&#
\begin{subequations}
%
%e39.a #&#
%e39.b #&#
%e39.c #&#
\begin{eqnarray}[ll]
\label{eq:fct_ss}
\mathbf{A}^{L} \mathbf{U}^{H} = \mathbf{b}+ \hat{\mathbf{p}} \,
,\eqncr
\label{eq:fct_fe}
\mathbf{M}^{L}\frac{\mathbf{U}^{H} - \mathbf{U}^{n}}{\Delta t} +
\mathbf{A}^{L}\mathbf{U}^{n} = \mathbf{b}^{n} + \hat{\mathbf{p}}^{n}
\,,\eqncr
\label{eq:fct_theta}
\mathbf{M}^{L}\frac{\mathbf{U}^{H} - \mathbf{U}^{n}}{\Delta t} +
\mathbf{A}^{L}\left( \theta\mathbf{U}^{H} +
(1-\theta)\mathbf{U}^{n}\right) = \mathbf{b}^{\theta}+
\hat{\mathbf{p}}^{\theta}\,,
\end{eqnarray}
%
\end{subequations}
%
where the hat denotes limitation: $\hat{p}_{i}\equiv\sum_{j} L_{i,j}P
_{i,j}$. The limiting coefficients range between zero and one,
representing full limitation and no limitation, respectively. For
example, setting all limiting coefficients to zero would result in the
low-order solution, and setting all to one would result in the
high-order solution. The actual values of the limiting coefficients are
determined by the limiter, which operates on the following goal:
maximize the limiting coefficients such that the imposed solution bounds
are not violated.

As will be discussed in Section~\ref{sec:solution_bounds}, the solution
bounds for implicit FCT and steady-state FCT are implicit, and thus the
systems given by Equations \reftext{\eqref{eq:fct_theta}} and \reftext{\eqref{eq:fct_ss}}
are nonlinear, since the limiting coefficients contained in
$\hat{\mathbf{p}}$ are nonlinear. In this paper, a fixed-point iteration
scheme is used to resolve the nonlinearities. For any nonlinear
iteration scheme, the imposed solution bounds must be computed using the
previous solution iterate:
%
%e40 #&#
\begin{equation}
U_{i}^{-,(\ell)} \leq U_{i}^{(\ell+1)} \leq U_{i}^{+,(\ell)} \,.
\end{equation}
%
Though the solution bounds are lagged, the antidiffusion bounds
$\hat{p}_{i}^{\pm}$ still contains terms at iteration $\ell+1$; these
terms must be lagged as well. As a consequence, the solution bounds for
implicit/steady-state FCT schemes are only satisfied upon nonlinear
convergence, not at each iteration.

%s3.3.2 #&#
\subsubsection{Solution bounds}%
\label{sec:solution_bounds}

The integral form of the transport equation can be derived using the
method of characteristics. Consider a frame of reference moving with the
radiation field so that position is a function of time, resulting in a
family of characteristic curves (since the transport equation is linear,
these curves are straight lines) $\mathbf{x}(t)$ that solve the
following ODE:
%
%e41 #&#
\begin{equation}
\frac{d\mathbf{x}}{dt}=v\di\,,\quad\mathbf{x}(0)=\mathbf{x}_{0}\,.
\end{equation}
%
Then taking the time derivative of $u(\mathbf{x}(t),t)$ gives
%
%e42 #&#
%e43 #&#
\begin{align}
\frac{du}{dt}
& = \frac{\partial u}{\partial t} + \nabla\cdot(u(
\mathbf{x}(t),t))\frac{d\mathbf{x}}{dt}
\eqncr
& = \frac{\partial u}{\partial t} + \nabla\cdot(u(\mathbf
{x}(t),t))\frac{d
\mathbf{x}}{dt}
\end{align}
%
Finally, combining this with Equation \reftext{\eqref{eq:scalar_model}} and
solving the resulting ODE gives the integral transport equation
\cite{glasstone}:
%
%e44 #&#
\begin{eqnarray}
\label{eq:integral_transport}
u(\mathbf{x},t) = u_{0}(\mathbf{x}- v t\di) e^{-\int\limits_{0}^{t}
\sigma(\mathbf{x}- v(t -t')\di)v dt'}
+ \int\limits_{0}^{t} q(\mathbf{x}- v(t -t')\di,t') e^{-\int
\limits
_{t'}^{t}
\sigma(\mathbf{x}- v(t -{t''})\di)v d{t''}} v dt' \,.
\end{eqnarray}
%
If the time step size $\Delta t$ satisfies the condition
%
%e45 #&#
\begin{equation}
\label{eq:cfl_analytic_dmp}
v\Delta t\leq h_{\min} \,,\quad h_{\min} \equiv\min\limits_{K} h
_{K} \,,
\end{equation}
%
where $h_{K}$ is the diameter of cell $K$, then the following discrete
solution bounds apply:
%
%e46 #&#
\begin{subequations}
\label{eq:solution_bounds}
%
%e46.a #&#
\begin{equation}
U^{-}_{i} \le U_{i}^{n+1} \le U^{+}_{i} \,,
\end{equation}
%
where
%
%e46.b #&#
\begin{equation}
U^{-}_{i}
\equiv\left\{
%
\begin{array}{l@{\quad} l}
U_{\min,i}^{n} e^{-v\Delta t\sigma_{\max,i}}
+ \frac{q_{\min,i}}{
\sigma_{\max,i}}
(1 - e^{-v\Delta t\sigma_{\max,i}}) \,,&
\sigma_{\max,i} \ne0
\\
U_{\min,i}^{n}
+ v\Delta tq_{\min,i} \,,& \sigma_{\max,i} = 0\,,
\end{array}
%
\right.
\end{equation}
%
and
%
%e46.c #&#
\begin{equation}
U^{+}_{i}
\equiv\left\{
%
\begin{array}{l@{\quad} l}
U_{\max,i}^{n} e^{-v\Delta t\sigma_{\min,i}}
+ \frac{q_{\max,i}}{
\sigma_{\min,i}}
(1 - e^{-v\Delta t\sigma_{\min,i}}) \,,&
\sigma_{\min,i} \ne0
\\
U_{\max,i}^{n}
+ v\Delta tq_{\max,i} \,,& \sigma_{\min,i} = 0\,.
\end{array}
%
\right.
\end{equation}
%
The other quantities used in the above expressions are:
%
%e46.d #&#
%e46.e #&#
%e46.f #&#
\begin{eqnarray}[ll]
U_{\max,i}^{n} \equiv\max\limits_{j\in\mathcal{I}(S_{i})}U_{j}^{n}
\,,\quad U_{\min,i}^{n} \equiv\min\limits_{j\in\mathcal{I}(S_{i})}U
_{j}^{n} \,,\eqncr
\sigma_{\max,i} \equiv\max\limits_{\mathbf{x}\in S_{i}}\sigma(
\mathbf{x}) \,,\quad\sigma_{\min,i} \equiv\min
\limits_{\mathbf{x}\in S_{i}}\sigma(\mathbf{x}) \,,\eqncr
q_{\max,i} \equiv\max\limits_{\mathbf{x}\in S_{i}}q(\mathbf{x}) \,,
\quad q_{\min,i} \equiv\min\limits_{\mathbf{x}\in S_{i}}q(
\mathbf{x}) \,.
\end{eqnarray}
%
\end{subequations}
%
Note the time step size condition given by Equation \reftext{\eqref{eq:cfl_analytic_dmp}} implies that when using CFL numbers
greater than
1 with implicit time discretizations, these bounds no longer apply. Similar
bounds can be derived for $v\Delta t> h_{min}$; however, these bounds
for a
node $i$ will no longer only depend on the solution values of the immediate
neighbors of $i$; instead, a larger neighborhood must be used in the bounds,
making the local solution bounds wider and thus less restrictive and arguably
less useful in the FCT algorithm. This represents a significant disadvantage
for implicit FCT, not only because the converged FCT solution could contain
more undesirable features but also because the wider bounds typically result
in a greater number of nonlinear iterations because of the increased freedom
in the limiting coefficients.

Steady-state FCT solution bounds can be inferred from Equation \reftext{\eqref{eq:solution_bounds}} by making the substitution $v\Delta t
\rightarrow s$, where $0\leq s \leq h_{min}$. This restriction of
$s$ similarly ensures that only the nearest neighbors of $i$ are needed
for the solution bounds of~$i$. Steady-state FCT unfortunately suffers
many of the same drawbacks as implicit FCT because like implicit FCT,
its solution bounds are implicit and thus change with each iteration.

%s3.3.3 #&#
\subsubsection{Antidiffusion bounds}

Bounds imposed on a solution value $U_i$, such as the bounds described in
Section~\ref{sec:solution_bounds}, directly translate into bounds on the
limited antidiffusion source $\hat{p}_{i}$. These antidiffusion bounds
$\hat{p}_{i}^{\pm}$ for steady-state, explicit Euler, and Theta
discretization are respectively derived by solving Equations \reftext{\eqref{eq:fct_ss}}, \reftext{\eqref{eq:fct_fe}}, and \reftext{\eqref{eq:fct_theta}} for
$\hat{p}_{i}$ and manipulating the inequality $U^{-}_{i}\leq U_{i}
\leq U^{+}_{i}$. This yields:
%
%e47 #&#
\begin{subequations}
%
%e47.a #&#
%e47.b #&#
%e47.c #&#
\begin{eqnarray}[ll]
\label{eq:antidiffusion_bounds_ss}
\hat{p}_{i}^{\pm}\equiv A_{i,i}^{L} U_{i}^{\pm}+ \sum\limits_{j
\ne i}A_{i,j}^{L} U_{j} - b_{i} \,,\eqncr
\label{eq:antidiffusion_bounds_fe}
\hat{p}_{i}^{\pm}\equiv M^{L}_{i,i}
\frac{U_{i}^{\pm}- U_{i}^{n}}{
\Delta t}
+ \sum\limits_{j}A_{i,j}^{L} U_{j}^{n}
- b_{i}^{n} \,,\eqncr
\label{eq:antidiffusion_bounds_theta}
\hat{p}_{i}^{\pm}\equiv\left( \frac{M^{L}_{i,i}}{\Delta t}+\theta A
_{i,i}^{L}\right)
U_{i}^{\pm}+ \left( (1-\theta) A_{i,i}^{L}-\frac{M
^{L}_{i,i}}{\Delta t}\right)
U_{i}^{n}
+\sum\limits_{j\ne i}A_{i,j}
^{L} U_{j}^{\theta}-b_{i}^{\theta}\,.
\end{eqnarray}
%
\end{subequations}
%
We note that, if the limiting coefficients $L_{i,j}$ are selected such
that $\hat{p}_{i}^{-}\leq\hat{p}_{i}\leq\hat{p}_{i}^{+}$, then the
solution bounds are satisfied: $U^{-}_{i}\leq U_{i}\leq U^{+}_{i}$.

Limiters such as the Zalesak limiter described in Section~\ref{sec:limiter}
are algebraic operators, taking as input the antidiffusion bounds
$\hat{p}_{i}^{\pm}$ and the antidiffusive fluxes $P_{i,j}$ and
returning as
output the limiting coefficients $L_{i,j}$. It is important to note
that most
limiters, including the limiter described in this paper, assume the
following: $\hat{p}_{i} ^{-}\leq0$, $\hat{p}_{i}^{+}\geq0$; the reasoning
for this assumption is as follows. Recall that FCT starts from the low-order
scheme, which is equivalent to the solution with $\hat{p}_{i}=0$. The limiter
should start from this point so that there is a fail-safe solution for the
FCT \xch{algorithm}{algoritm}: the low-order solution. Otherwise, there is no guarantee that
any combination of values of limiting coefficients will achieve the desired
condition $\hat{p}_{i}^{-}\leq\hat{p}_{i}\leq\hat{p}_{i} ^{+}$. If
$\hat{p}_{i}^{-} > 0$ or $\hat{p}_{i}^{+} < 0$, then the starting
state, the
low-order solution, with $\hat{p}_{i}=0$ is an invalid solution of the FCT
algorithm. Some solution bounds automatically satisfy $\hat
{p}_{i}^{-}\leq0$
and $\hat{p}_{i}^{+} \geq0$, but in general these conditions must be
enforced. In this paper, the solution bounds are possibly widened by directly
enforcing these assumptions:
%
%e48 #&#
%e49 #&#
\begin{eqnarray}[ll]
\hat{p}_{i}^{-} \gets\min(0,\hat{p}_{i}^{-}) \,,\eqncr
\hat{p}_{i}^{+} \gets\max(0,\hat{p}_{i}^{+}) \,.
\end{eqnarray}
%
We have noted that omitting this step can lead to poor results. Without this
step, the assumptions of the limiter are violated, and thus limiting
coefficients that do not satisfy the imposed solution bounds may be
generated.

%s3.3.4 #&#
\subsubsection{Limiting coefficients}%
\label{sec:limiter}
The results in this paper use the classic multi-dimensional limiter
introduced by Zalesak \cite{zalesak}:
%
%e50 #&#
\begin{subequations}
%
%e50.a #&#
%e50.b #&#
%e50.c #&#
\begin{eqnarray}[ll]
\label{eq:flux_sums}
p_{i}^{+} \equiv\sum\limits_{j}\max(0,P_{i,j}) \,,
\qquad
p_{i}^{-} \equiv\sum\limits_{j}\min(0,P_{i,j}) \,,\eqncr
\label{eq:single_node_limiting_coefficients}
L_{i}^{\pm}\equiv\left\{
%
\begin{array}{l@{\quad} l}
1 & p_{i}^{\pm}= 0
\\
\min\left( 1,\frac{\bar{p}_{i}^{\pm}}{p_{i}^{\pm}}\right) & p_{i}
^{\pm}\ne0\,,
\end{array}
%
\right. \eqncr
\label{eq:limiting_coefficients}
L_{i,j} \equiv\left\{
%
\begin{array}{l@{\quad} l}
\min(L_{i}^{+},L_{j}^{-})
& P_{i,j} \geq0
\cr \noalign{\vspace{2pt}}
\min(L_{i}^{-},L_{j}^{+})
& P_{i,j} < 0\,.
\end{array}
%
\right.
\end{eqnarray}
%
\end{subequations}
%
The objective of a limiter is to maximize the amount of antidiffusion
that can be accepted without violating the imposed solution constraints.
Zalesak's limiter is one commonly used attempt at this objective due to
its relatively simple form.

%r3 #&#
\begin{rmk}
Note that it is possible to devise limiters accepting more antidiffusion
than Zalesak's limiter, but one must sacrifice the simple, closed form
of Zalesak's limiter and adapt a sequential algorithm for computing the
antidiffusion coefficients. This approach has not been found in
literature, most likely because the sequential aspect of the algorithm
makes it node-order-dependent, which reduces reproducibility between
implementations.
\end{rmk}

Finally, one could pass antidiffusive fluxes through a given limiter
multiple times, using the remainder antidiffusive flux as the input in
each pass, to increase the total antidiffusion accepted
\cite{kuzmin_FCT,schar}; however, the results presented in this paper
were all produced using the traditional single-pass approach through the
Zalesak limiter.

%s4 #&#
\section{Results}%
\label{sec:results}
This section presents results for a number of test problems, which
compare the solutions obtained using:
%
\begin{itemize}
%
\item
the standard Galerkin FEM, \xch{labeled}{labelled} as ``Galerkin'' in the plots,
%
\item
the low-order method, \xch{labeled}{labelled} in plots as ``Low'',
%
\item
the entropy viscosity method, \xch{labeled}{labelled} in plots as ``EV'',
%
\item
the standard Galerkin FEM with FCT, \xch{labeled}{labelled} in plots as
``Galerkin-FCT'', and
%
\item
the entropy viscosity method with FCT, \xch{labeled}{labelled} in plots as ``EV-FCT''.
\end{itemize}
%
All problems assume a speed of $v=1$ (the speed effectively just changes
the units of $\Delta t$) and an entropy function of $\eta(u)=
\frac{1}{2}u^{2}$. Unless otherwise specified, the transport direction
is in the positive $x$ direction: $\bm{\Omega}=\mathbf{e}_{x}$, and
the entropy viscosity tuning parameters of $c_{\mathcal{R}}$ and
$c_{\mathcal{J}}$ are set to 0.1. A third-order Gauss quadrature is used
for spatial integration in all test cases.

%s4.1 #&#
\subsection{Spatial convergence tests}

This 1-D, steady-state test problem uses the Method of Manufactured
Solutions (MMS) with a solution of $u(x)=\sin(\pi x)$ on the domain
$x\in(0,1)$. Zero Dirichlet boundary conditions are imposed on both
boundaries. With $\sigma(x)=1$, the MMS source becomes $q(x)=\pi
\cos(\pi x) + \sin(\pi x)$. The number of cells in the study starts
at 8 for the coarsest mesh, and cells are refined by a factor of 2 in
each cycle, ending with 256 cells.

\reftext{Fig.~\ref{fig:mms_sinx_ss}} shows the $L^{2}$ norm errors for this
convergence study and indicates first-order spatial convergence for the
low-order method and second-order spatial convergence for the entropy
viscosity (EV) method and EV-FCT method, as expected.

%f2 #&#
\begin{figure}
\includegraphics{7667f02}
\caption{Spatial convergence for MMS problem.}
\label{fig:mms_sinx_ss}
\end{figure}

%f3 #&#
\begin{figure}
\includegraphics{7667f03}
\caption{Comparison of solutions for the glance-in-void test problem using
explicit Euler time discretization\xch{. (For interpretation of the colors in this figure, the
reader is referred to the web version of this article.)}{.}}
\label{fig:glance_in_void_fe}
\end{figure}


%s4.2 #&#
\subsection{Glancing beam in a void}

This 2-D test problem is on the unit square: $\mathbf{x}\in(0,1)^{2}$
and simulates a beam incident on the bottom boundary of a void region
($\sigma(\mathbf{x})= 0$, $q(\mathbf{x})=0$) at a shallow angle of
$21.94^{\circ}$ with respect to the $x$-axis ($\Omega_{x}=\cos(21.94^{
\circ})$, $\Omega_{y}=\sin(21.94^{\circ})$). The exact solution of
this problem contains a discontinuity along the line $y = \frac{\Omega
_{y}}{\Omega_{x}}x$, which presents opportunity for the formation of
spurious oscillations. This is run as a pseudo-transient problem with
zero initial conditions until steady-state is reached. A Dirichlet
boundary condition is imposed on the incoming sides of the problem, with
a value of 1 on the bottom boundary and a value of 0 on the left
boundary.

This problem is run with Explicit Euler time discretization and a CFL
number of 0.5 on a $64\times64$ mesh. \reftext{Fig.~\ref{fig:glance_in_void_fe}} compares the numerical solutions for this
problem obtained with the low-order, EV, Galerkin-FCT, and EV-FCT
schemes. The Galerkin scheme (without FCT) produced spurious
oscillations without bound, so those results are omitted here. The same
color scale is used for each image: the dark red shown in the low-order
and the two FCT sub-plots corresponds to the incoming value of 1, and
the blue in these same sub-plots corresponds to zero. The darker blues
and reds shown in the EV sub-plot indicate undershoots and overshoots,
respectively. Both FCT solutions keep the solution within the imposed
physical bounds, but one can see from the Galerkin-FCT results that some
``terracing'' effects are present; this behavior is a well-known
artifact of traditional FCT schemes \cite{kuzmin_FCT}. In this
case, and in all observed cases of the terracing phenomenon, the EV-FCT
scheme shows a reduction of this effect: the addition of the
entropy-based artificial viscosity decreases the magnitude of the
spurious oscillations in the high-order scheme and thus lessens the
burden on the limiter.


%s4.3 #&#
\subsection{Obstruction test}%
\label{sec:obstruction}
This is a 2-D, two-region problem on the unit square $(0,1)^{2}$ with
a beam incident on the left and bottom boundaries at an angle of
$45^{\circ}$ with the x-axis. The center region $(\frac{1}{3},
\frac{2}{3})^{2}$ is an absorber region with $\sigma(\mathbf{x})=10$
and $q(\mathbf{x})=0$, and the surrounding region is a void ($\sigma
(\mathbf{x})=0$, $q(\mathbf{x})=0$).

This problem was run with Implicit Euler with a CFL of 1 to steady-state
on a $32\times32$ mesh. The results are shown in \reftext{Fig.~\ref{fig:obstruction_be}}. The low-order solution is especially diffusive
and shows a fanning of the solution after it passes the corners of the
obstruction, which is not present in any of the high-order schemes. The
EV solution contains oscillations, although they are much less
significant than in the Galerkin solution. Both FCT schemes show a lack
of these oscillations, but the Galerkin-FCT solutions shows a terracing
effect. The EV-FCT solution also has this effect but to a much smaller
degree.

%f4 #&#
\begin{figure}
\includegraphics{7667f04}
\caption{Comparison of solutions for the obstruction test problem using implicit
Euler time discretization.}
\label{fig:obstruction_be}
\end{figure}

%t1 #&#
\begin{table}[b]
\tablewidth=8cm
\caption{Nonlinear iterations vs. CFL number for the obstruction test problems.}
\label{tab:obstruction_iterations}
\thead
%
\begin{tabular*}{\tablewidth}{*{7}{l}}
\hline
\tmultirow{2}{*}{CFL} & \tmultirow{2}{*}{Relax} & \multicolumn{2}{l}{{EV}} & \multicolumn{2}{l}{{FCT}} &
\tmultirow{2}{*}{$L^{2}$ {Err.}}\\
\ccline{3-4,5-6}
& & {Total} & {Avg.} & {Total} & {Avg.} &\\
\hline
\endthead
0.1 & -- & 6204 & 8.43 & 5223 & 7.23 & $5.084\times10^{-2}$\\
0.5 & -- & 1386 & 9.36 & 2239 & 15.13 & $5.079\times10^{-2}$\\
1.0 & -- & 791 & 10.69 & 1588 & 21.46 & $5.111\times10^{-2}$\\
5.0 & -- & 265 & 17.67 & 1780 & 118.67 & $5.980\times10^{-2}$\\
10.0 & -- & 150 & 18.75 & 1298 & 162.25 & $9.854\times10^{-2}$\\
20.0 & -- & -- & -- & \multicolumn{2}{c}{(failure)} & --\\
20.0 & 0.9 & 66 & 16.50 & 935 & 233.50 & $1.295\times10^{-1}$\\
\hline
\end{tabular*}
%
\end{table}

\reftext{Table~\ref{tab:obstruction_iterations}} shows the results for a
parametric study on the number of nonlinear iterations required for EV
and EV-FCT for various CFL numbers on a $16\times16$ mesh, to an end
time of $t=1.5$. Recall that in the FCT algorithm, one first computes
the high-order solution (here, EV), which is necessary for computation
of the antidiffusive fluxes. The number of nonlinear iterations for this
solve is given in the column labeled as ``EV''. The FCT limiting
procedure is also nonlinear; the number of nonlinear iterations for this
solve is given in the ``FCT'' column. While these results indicate that
increasing the CFL number decreases the total computational work in
reaching the end of the transient, it should be noted that the quality
of the FCT solution deteriorates significantly for large CFL numbers;
see the ``$L^{2}$ Err.'' column. As discussed previously, the solution
bounds are implicit and thus change with each iteration. This challenge
is compounded in the case of large time step sizes; as time step size
increases, the solution bounds widen, and successive FCT solution
iterates differ more than for smaller time step sizes. This trend is
suggested in the rate of increase in the number of FCT iterations per
time step in \reftext{Table~\ref{tab:obstruction_iterations}}, which is
significantly larger than the rate of increase of EV iterations per time
step. However, this issue can be mitigated using a relaxation factor on
the iterative solution updates; for example, for the failing case of CFL
number equal to 20, convergence can be achieved with a relaxation factor
of 0.9, giving average iterations per time step of 14.33 and 221.67 for
EV and FCT, respectively.


%s4.4 #&#
\subsection{Two-region interface}

This 1-D test problem simulates the interface between two regions with
varying cross section and source values on the domain $(0,1)$. The left
half of the domain has values $\sigma(\mathbf{x})=10$ and $q(
\mathbf{x})=10$, for which the transport solution will reach a
saturation value of $\frac{q}{\sigma}=1$, while the right half has
values of $\sigma(\mathbf{x})=40$ and $q(\mathbf{x})=20$, giving it a
saturation value of $\frac{q}{\sigma}=0.5$. The transport direction is
$\di=\mathbf{e}_{x}$, with zero incident flux on the left boundary.

This problem was run using SSPRK33 time discretization with a CFL of 1
to steady-state with 32 cells. \reftext{Fig.~\ref{fig:interface}} shows the
results for this test problem. The low-order solution suffers from the
significant artificial diffusion, while the Galerkin solution suffers
from significant spurious oscillations. The EV scheme eliminates some
of the first oscillations, but a number of oscillations of a similar
magnitude to those produced by the Galerkin scheme still exists to the
left of the interface. The sets ``$U^{-}$'' and ``$U^{+}$'' correspond
to the minimum and maximum solution bounds, respectively, of each FCT
scheme (the differences between the bounds of the two FCT schemes are
insignificant here). Both FCT schemes effectively eliminate spurious
oscillations without approaching the level of unnecessary artificial
diffusion achieved by the low-order scheme. For this test problem, the
Galerkin-FCT solution is slightly superior to the EV-FCT solution and
the ``terracing'' phenomenon of FCT is not present.

%f5 #&#
\begin{figure}
\includegraphics{7667f05}
\caption{Comparison of solutions for the two-region interface test problem using
SSPRK33 time discretization.}
\label{fig:interface}
\end{figure}

%s4.5 #&#
\subsection{Source in a void test}

This 1-D test problem has two regions, the left being a void but with
a source: $\sigma(\mathbf{x})=0$ and $q(\mathbf{x})=1$, and the right
being an absorber without a source: $\sigma(\mathbf{x})=10$ and
$q(\mathbf{x})=0$. This does not represent \xch{necessarily}{necessairly} a non-physical
problem: due to the energy dependence of the particle distribution, it
is possible to have a strong inscattering source term in an energy
bandwidth where the interaction coefficient is small; here we emphasize
the weak interaction probability by zero-ing out the reaction term. A
zero Dirichlet boundary condition is imposed on the incoming (left)
boundary, and this problem is run as a steady-state problem on the
domain $(0,1)$ with 32 cells.

For this problem, the entropy residual and jump coefficients
$c_{\mathcal{R}}$ and $c_{\mathcal{J}}$ are set to 0.5; the default
value of 0.1 was found to be too small for this problem.
\reftext{Fig.~\ref{fig:source_in_void}} shows the results for this problem. The
Galerkin solution has ``kinks'' along the void (left) region, but shows
point-wise matching of the exact solution in the absorber (right)
region. The EV solution, however, eliminates the terracing effect in the
void region but suffers from an inflated peak at the interface, due to
entropy production at the interface. The EV-FCT solution bounds
generated for this problem are denoted as ``$U^{-}$, EV-FCT'' and ``$U
^{+}$, EV-FCT''. For this test problem, the Galerkin-FCT scheme gives
superior results to that of the EV-FCT scheme. However, both FCT schemes
suffer from the familiar FCT phenomenon known as ``peak-clipping'',
whereby the ``mass'' that should be present in the peak has been
redistributed by the FCT limiter, flattening the peak.

%f6 #&#
\begin{figure}
\includegraphics{7667f06}
\caption{Comparison of solutions for the source-in-void test problem using
steady-state time discretization.}
\label{fig:source_in_void}
\end{figure}

%s5 #&#
\section{Conclusions}%
\label{sec:conclusions}
An FCT scheme has been proposed for the particle transport equation. It
is has been applied to the following model equation problem, an
advection problem with reaction and source terms,
%
%e51 #&#
\begin{equation}
\label{eq:ccl}
\frac{\partial u}{\partial t} + v\di\cdot\nabla u(\mathbf{x},t)+
\sigma(\mathbf{x}) u(\mathbf{x},t)= q(\mathbf{x},t)\,,
\end{equation}
%
which corresponds to the classic Source Iteration equation for particle
transport, where extraneous, inscatter, and fission sources are
collected in the right-hand-side term $q$. The FCT methodology relies
on a lower-order solution and a high-order solution. For the low-order
solution, we have used a first-order viscosity approach, based on the
graph-\xch{theoretic}{theoric} method of Guermond et al.
\cite{guermond_firstorder} for scalar conservation laws and extends
these previous works to situations with reaction and source terms
present. The low-order scheme is shown to be positivity-preserving
through the use of the M-matrix properties. The high-order solution
employs an entropy-based artificial stabilization. The entropy residual
approach is derived for the transport equation shown in
Equation~\reftext{\eqref{eq:ccl}}. Temporal discretizations include explicit and
implicit schemes in time, as well as steady state, the latter two cases
making the FCT algorithm implicit as well. The standard Zalesak limiter
is utilized to limit between the low- and high-order solutions.

The FCT scheme described in this paper is second-order accurate in
space, converges to the entropy solution, and preserves non-negativity.
Spurious oscillations are mitigated but are not guaranteed to be
eliminated, as smaller magnitude oscillations may exist within the
imposed solution bounds.

The traditional FCT phenomenon known as ``stair-stepping'',
``terracing'', or ``plateauing'' is still an open issue, particularly
for fully explicit temporal discretizations; however, these effects have
been shown to diminish or disappear when using SSPRK33 as opposed to
explicit Euler. In addition, these effects are less pronounced for
EV-FCT than in the classic FEM-FCT scheme, which uses the standard
Galerkin method as the high-order method in FCT.

The explicit temporal discretizations of the described FCT scheme yield
a robust algorithm; however, implicit and steady-state discretizations
are less robust, suffering from nonlinear convergence difficulties in
some problems. The main complication with implicit and steady-state FCT
schemes is that the imposed solution bounds are implicit with the
solution, and thus the imposed solution bounds change with each
iteration of the nonlinear solver.

Future work on this subject should mainly focus on the
implicit/steady-state iteration techniques because of the convergence
difficulties encountered for some problems. This work used fixed-point
iteration, but one can attempt using an alternative such as Newton's
method. The main challenge is the evolving solution bounds, which will
present difficulty to any nonlinear solution algorithm. Other work could
be performed for the FCT algorithm in general: for example, the
terracing phenomenon still deteriorates FCT solutions.

\begin{acks}
This research was carried out under the auspices of the \gsponsor[id=GS3,sponsor-id=100000015,country=United States]{\xch{U.S. Department of Energy}{US Department of Energy}} (Grant number \gnumber[refid=GS3]{DE-NE-0000112}) and the Idaho National
Laboratory, a contractor of the \gsponsor[id=GS4]{U.S. Government} under contract No.
\gnumber[refid=GS4]{DEAC07-05ID14517}.
\end{acks}

%\begin{appm}
%\def\the...{}
%\reset{}{}
%\appendix{}
%\appendix*{}
%\end{appm}%
%spell_to ********** End of text entry *****************
%
\begin{backmatter}%
%
% structpyb loaded by ritac, 2017-10-24 13:32:56
\begin{thebibliography}{}

%b1 ###
%b1 #&#
\bibitem{guermond_ev}
\begin{bsubitem}
\begin{bcontribution}%[language=fr]%de,it,pl,ru
\bauthor{\fnm{J.-L.} \snm{Guermond}}
\bauthor{\fnm{R.} \snm{Pasquetti}}
\bauthor{\fnm{B.} \snm{Popov}}
\btitle{Entropy viscosity method for nonlinear conservation laws}
\end{bcontribution}
\begin{bhost}
\begin{bissue}
\bseries{\btitle{J. Comput. Phys.} \bvolumeno{230}}
\bdate{2011}
\end{bissue}
\bpages{\bfirstpage{4248}\blastpage{4267}}
\end{bhost}
\end{bsubitem}
%
\OrigBibText
J.-L. Guermond, R.~Pasquetti, B.~Popov, Entropy viscosity method for
nonlinear conservation laws, Journal of Computational Physics 230 (2011)
4248--4267.
\endOrigBibText
\bptok{structpyb}%
\endbibitem

%b2 ###
%b2 #&#
\bibitem{glasstone}
\begin{bsubitem}
\begin{bcontribution}%[language=fr]%de,it,pl,ru
\bauthor{\fnm{G.I.} \snm{Bell}}
\bauthor{\fnm{S.} \snm{Glasstone}}
\btitle{Nuclear Reactor Theory}
\end{bcontribution}
\begin{bhost}
\begin{bbook}
\bdate{1970}
\bpublisher{\bname{Litton Educational Publishing, Inc.}}
\end{bbook}
\end{bhost}
\end{bsubitem}
%
\OrigBibText
G.~I. Bell, S.~Glasstone, Nuclear Reactor Theory, Litton Educational
Publishing, Inc., 1970.
\endOrigBibText
\bptok{structpyb}%
\endbibitem

%b3 ###
%b3 #&#
\bibitem{radiotherapy}
\begin{bsubitem}
\begin{bcontribution}%[language=fr]%de,it,pl,ru
\bauthor{\fnm{M.} \snm{Sch\"{a}fer}}
\bauthor{\fnm{M.} \snm{Frank}}
\bauthor{\fnm{M.} \snm{Herty}}
\btitle{Optimal treatment planning in radiotherapy based on Boltzmann transport calculations}
\end{bcontribution}
\begin{bhost}
\begin{bissue}
\bseries{\btitle{Math. Models Methods Appl. Sci.} \bvolumeno{18}}
\bissueno{4}
\bdate{2008}
\end{bissue}
\bpages{\bfirstpage{573}\blastpage{592}}
\bdoi[https://doi.org/10.1142/S0218202508002784]{10.1142/S0218202508002784}
\end{bhost}
\end{bsubitem}
%
\OrigBibText
M.~Sch\"afer, M.~Frank, M.~Herty, Optimal treatment planning in
radiotherapy based on {B}oltzmann transport calculations, Mathematical
Models and Methods in Applied Sciences 18~(4) (2008) 573--592.
\newblock
doi:10.1142/S0218202508002784.
\endOrigBibText
\bptok{structpyb}%
\endbibitem

%b4 ###
%b4 #&#
\bibitem{astrophysics_textbook}
\begin{bsubitem}
\begin{bcontribution}%[language=fr]%de,it,pl,ru
\bauthor{\fnm{P.} \snm{Bodenheimer}} \betal
\btitle{Numerical Methods in Astrophysics: An Introduction}
\end{bcontribution}
\begin{bhost}
\begin{bbook}
\bdate{2006}
\bpublisher{\bname{CRC Press}}
\end{bbook}
\end{bhost}
\end{bsubitem}
%
\OrigBibText
P.~Bodenheimer, et~al., Numerical Methods in Astrophysics: An
Introduction, CRC Press, 2006.
\endOrigBibText
\bptok{structpyb}%
\endbibitem

%b5 ###
%b5 #&#
\bibitem{lewis}
\begin{bsubitem}
\begin{bcontribution}%[language=fr]%de,it,pl,ru
\bauthor{\fnm{E.E.} \snm{Lewis}}
\bauthor{\fnm{W.F.} \snm{Miller}}
\btitle{Computational Methods of Neutron Transport}
\end{bcontribution}
\begin{bhost}
\begin{bbook}
\bdate{1993}
\bpublisher{\bname{American Nuclear Society}\blocation{La Grange Park, IL}}
\end{bbook}
\end{bhost}
\end{bsubitem}
%
\OrigBibText
E.~E. Lewis, W.~F. Miller, Computational Methods of Neutron Transport,
American Nuclear Society, La Grange Park, IL, 1993.
\endOrigBibText
\bptok{structpyb}%
\endbibitem

%b6 ###
%b6 #&#
\bibitem{laser_plasmas}
\begin{bsubitem}
\begin{bcontribution}%[language=fr]%de,it,pl,ru
\bauthor{\fnm{K.} \snm{Eidmann}}
\btitle{Radiation transport and atomic physics modeling in high-energy-density
laser-produced plasmas}
\end{bcontribution}
\begin{bhost}
\begin{bissue}
\bseries{\btitle{Laser Part. Beams} \bvolumeno{12}}
\bissueno{2}
\bdate{1994}
\end{bissue}
\bpages{\bfirstpage{223}\blastpage{244}}
\end{bhost}
\end{bsubitem}
%
\OrigBibText
K.~Eidmann, Radiation transport and atomic physics modeling in
high-energy-density laser-produced plasmas, Laser and Particle Beams
12~(2) (1994) 223--244.
\endOrigBibText
\bptok{structpyb}%
\endbibitem

%b7 ###
%b7 #&#
\bibitem{duderstadt}
\begin{bsubitem}
\begin{bcontribution}%[language=fr]%de,it,pl,ru
\bauthor{\fnm{J.J.} \snm{Duderstadt}}
\bauthor{\fnm{W.R.} \snm{Martin}}
\btitle{Transport Theory}
\end{bcontribution}
\begin{bhost}
\begin{bbook}
\bdate{1979}
\bpublisher{\bname{John Wiley \& Sons}}
\end{bbook}
\end{bhost}
\end{bsubitem}
%
\OrigBibText
J.~J. Duderstadt, W.~R. Martin, Transport Theory, John Wiley \& Sons,
1979.
\endOrigBibText
\bptok{structpyb}%
\endbibitem

%b8 ###
%b8 #&#
\bibitem{Lesaint1974}
\begin{bsubitem}
\begin{bcontribution}%[language=fr]%de,it,pl,ru
\bauthor{\fnm{P.} \snm{Lesaint}}
\bauthor{\fnm{P.A.} \snm{Raviart}}
\btitle{On a finite element method for solving the neutron transport equation}
\end{bcontribution}
\begin{bhost}
\begin{bissue}
\bseries{\btitle{Publ. Math. Inf. Rennes} \bvolumeno{4}}
\bdate{1974}
\end{bissue}
\bpages{\bfirstpage{1}\blastpage{40}}
\end{bhost}
\begin{bhost}
\begin{behost}
\binterref[locator-type=url]{http://eudml.\wwwbreak org/doc/273730}
\end{behost}
\end{bhost}
\end{bsubitem}
%
\OrigBibText
P.~Lesaint, P.~A. Raviart, On a finite
element method for solving the neutron transport equation, Publications
math\'ematiques et informatique de Rennes S4 (1974) 1--40.
http://eudml.org/doc/273730
\endOrigBibText
\bptok{structpyb}%
\endbibitem

%b9 ###
%b9 #&#
\bibitem{Reed_Hill_1973}
\begin{bsubitem}
\begin{bcontribution}%[language=fr]%de,it,pl,ru
\bauthor{\fnm{W.} \snm{Reed}}
\bauthor{\fnm{T.} \snm{Hill}}
\btitle{Triangular Mesh Methods for the Neutron Transport Equation}
\end{bcontribution}
\bcomment{Tech. Rep. LA-UR-73-479}\prnsep{,\ }
\begin{bhost}
\begin{bbook}[class=report]
\bdate{1973}
\bpublisher{\bname{Los Alamos Scientific Laboratory}}
\end{bbook}
\end{bhost}
\end{bsubitem}
%
\OrigBibText
W.~Reed, T.~Hill, Triangular mesh methods for the neutron transport
equation, Tech. Rep. LA-UR-73-479, Los Alamos Scientific Laboratory
(1973).
\endOrigBibText
\bptok{structpyb}%
\endbibitem

%b10 ###
%b10 #&#
\bibitem{zingan_2013}
\begin{bsubitem}
\begin{bcontribution}%[language=fr]%de,it,pl,ru
\bauthor{\fnm{V.} \snm{Zingan}}
\bauthor{\fnm{J.-L.} \snm{Guermond}}
\bauthor{\fnm{J.} \snm{Morel}}
\bauthor{\fnm{B.} \snm{Popov}}
\btitle{Implementation of the entropy viscosity method with the discontinuous Galerkin method}
\end{bcontribution}
\begin{bhost}
\begin{bissue}
\bseries{\btitle{Comput. Methods Appl. Mech. Eng.} \bvolumeno{253}}
\bdate{2013}
\end{bissue}
\bpages{\bfirstpage{479}\blastpage{490}}
\bdoi[https://doi.org/10.1016/j.cma.2012.08.018]{10.1016/j.cma.2012.08.018}
\end{bhost}
\end{bsubitem}
%
\OrigBibText
V.~{Zingan}, J.-L. Guermond, J.~{Morel}, B.~{Popov}, {Implementation
of the entropy viscosity method with the discontinuous Galerkin method},
Computer Methods in Applied Mechanics and Engineering 253 (2013)
479--490.
\newblock doi:10.1016/j.cma.2012.08.018.
\endOrigBibText
\bptok{structpyb}%
\endbibitem

%b11 ###
%b11 #&#
\bibitem{lathrop}
\begin{bsubitem}
\begin{bcontribution}%[language=fr]%de,it,pl,ru
\bauthor{\fnm{K.D.} \snm{Lathrop}}
\btitle{Spatial differencing of the transport equation: positivity vs. accuracy}
\end{bcontribution}
\begin{bhost}
\begin{bissue}
\bseries{\btitle{J. Comput. Phys.} \bvolumeno{4}}
\bdate{1969}
\end{bissue}
\bpages{\bfirstpage{475}\blastpage{498}}
\end{bhost}
\end{bsubitem}
%
\OrigBibText
K.~D. Lathrop, Spatial differencing of the transport equation:
Positivity vs. accuracy, Journal of Computational Physics 4 (1969)
475--498.
\endOrigBibText
\bptok{structpyb}%
\endbibitem

%b12 ###
%b12 #&#
\bibitem{hamilton}
\begin{bsubitem}
\begin{bcontribution}%[language=fr]%de,it,pl,ru
\bauthor{\fnm{S.} \snm{Hamilton}}
\bauthor{\fnm{M.} \snm{Benzi}}
\btitle{Negative flux fixups in discontinuous finite element sn transport}
\end{bcontribution}
\begin{bhost}
\begin{beditedbook}
\btitle{International Conference on Mathematics, Computational Methods \& Reactor Physics}
\bdate{2009}
\end{beditedbook}
\end{bhost}
\end{bsubitem}
%
\OrigBibText
S.~Hamilton, M.~Benzi, Negative flux fixups in discontinuous finite
element sn transport, in: International Conference on Mathematics,
Computational Methods \& Reactor Physics, 2009.
\endOrigBibText
\bptok{structpyb}%
\endbibitem

%b13 ###
%b13 #&#
\bibitem{walters_NC}
\begin{bsubitem}
\begin{bcontribution}%[language=fr]%de,it,pl,ru
\bauthor{\fnm{W.F.} \snm{Walters}}
\bauthor{\fnm{T.A.} \snm{Wareing}}
\btitle{An accurate, strictly-positive, nonlinear characteristic scheme
for the discrete ordinates equations}
\end{bcontribution}
\begin{bhost}
\begin{bissue}
\bseries{\btitle{Transp. Theory Stat. Phys.} \bvolumeno{25}}
\bissueno{2}
\bdate{1996}
\end{bissue}
\bpages{\bfirstpage{197}\blastpage{215}}
\end{bhost}
\end{bsubitem}
%
\OrigBibText
W.~F. Walters, T.~A. Wareing, An accurate, strictly-positive, nonlinear
characteristic scheme for the discrete ordinates equations, Transport
Theory and Statistical Physics 25~(2) (1996) 197--215.
\endOrigBibText
\bptok{structpyb}%
\endbibitem

%b14 ###
%b14 #&#
\bibitem{wareing}
\begin{bsubitem}
\begin{bcontribution}%[language=fr]%de,it,pl,ru
\bauthor{\fnm{T.A.} \snm{Wareing}}
\btitle{An exponential discontinuous scheme for discrete-ordinate calculations
in cartesian geometries}
\end{bcontribution}
\begin{bhost}
\begin{beditedbook}
\btitle{Joint International Conference on Mathematical Methods and Supercomputing
in Nuclear Applications}
\bconference{Saratoga Springs, NY}
\bdate{1997}
\end{beditedbook}
\end{bhost}
\end{bsubitem}
%
\OrigBibText
T.~A. Wareing, An exponential discontinuous scheme for discrete-ordinate
calculations in cartesian geometries, in: Joint International Conference
on Mathematical Methods and Supercomputing in Nuclear Applications,
Saratoga Springs, NY, 1997.
\endOrigBibText
\bptok{structpyb}%
\endbibitem

%b15 ###
%b15 #&#
\bibitem{maginot}
\begin{bsubitem}
\begin{bcontribution}%[language=fr]%de,it,pl,ru
\bauthor{\fnm{P.} \snm{Maginot}}
\btitle{A Nonlinear Positive Extension of the Linear Discontinuous Spatial
Discretization of the Transport Equation}
\end{bcontribution}
\bcomment{Master's Thesis}\prnsep{,\ }
\begin{bhost}
\begin{bbook}[class=report]
\bdate{December 2010}
\bpublisher{\bname{Texas A\&M University}}
\end{bbook}
\end{bhost}
\end{bsubitem}
%
\OrigBibText
P.~Maginot, A nonlinear positive extension of the linear discontinuous
spatial discretization of the transport equation, Master's thesis, Texas
A\&M University (December 2010).
\endOrigBibText
\bptok{structpyb}%
\endbibitem

%b16 ###
%b16 #&#
\bibitem{maginot_mc2015}
\begin{bsubitem}
\begin{bcontribution}%[language=fr]%de,it,pl,ru
\bauthor{\fnm{P.} \snm{Maginot}}
\btitle{A non-negative, non-linear Petrov--Galerkin method for bilinear
discontinuous differencing of the Sn equations}
\end{bcontribution}
\begin{bhost}
\begin{beditedbook}
\btitle{Joint International Conference on Mathematics and Computation,
Supercomputing in Nuclear Applications, and the Monte Carlo Method}
\bconference{M\&C 2015, Nashville, TN}
\bdate{2015}
\end{beditedbook}
\end{bhost}
\end{bsubitem}
%
\OrigBibText
P.~Maginot, A non-negative, non-linear {P}etrov-{G}alerkin method for
bilinear discontinuous differencing of the {S}n equations, in: Joint
International Conference on Mathematics and Computation, Supercomputing
in Nuclear Applications, and the Monte Carlo Method (M\&C 2015),
Nashville, TN, 2015.
\endOrigBibText
\bptok{structpyb}%
\endbibitem

%b17 ###
%b17 #&#
\bibitem{maginot_2017}
\begin{bsubitem}
\begin{bcontribution}%[language=fr]%de,it,pl,ru
\bauthor{\fnm{P.G.} \snm{Maginot}}
\bauthor{\fnm{J.C.} \snm{Ragusa}}
\bauthor{\fnm{J.E.} \snm{Morel}}
\btitle{Nonnegative methods for bilinear discontinuous differencing of
the sn equations on quadrilaterals}
\end{bcontribution}
\begin{bhost}
\begin{bissue}
\bseries{\btitle{Nucl. Sci. Eng.} \bvolumeno{185}}
\bissueno{1}
\bdate{2017}
\end{bissue}
\bpages{\bfirstpage{53}\blastpage{69}}
\end{bhost}
\end{bsubitem}
%
\OrigBibText
P.~G. Maginot, J.~C. Ragusa, J.~E. Morel, Nonnegative methods for
bilinear discontinuous differencing of the sn equations on
quadrilaterals, Nuclear Science and Engineering 185~(1) (2017) 53--69.
\endOrigBibText
\bptok{structpyb}%
\endbibitem

%b18 ###
%b18 #&#
\bibitem{borisbook}
\begin{bsubitem}
\begin{bcontribution}%[language=fr]%de,it,pl,ru
\bauthor{\fnm{J.P.} \snm{Boris}}
\bauthor{\fnm{D.L.} \snm{Book}}
\btitle{Flux-corrected transport, I: SHASTA, a fluid transport algorithm that works}
\end{bcontribution}
\begin{bhost}
\begin{bissue}
\bseries{\btitle{J. Comput. Phys.} \bvolumeno{11}}
\bdate{1973}
\end{bissue}
\bpages{\bfirstpage{38}\blastpage{69}}
\end{bhost}
\end{bsubitem}
%
\OrigBibText
J.~P. Boris, D.~L. Book, Flux-corrected transport i. SHASTA, a fluid
transport algorithm that works, Journal of Computational Physics 11
(1973) 38--69.
\endOrigBibText
\bptok{structpyb}%
\endbibitem

%b19 ###
%b19 #&#
\bibitem{zalesak}
\begin{bsubitem}
\begin{bcontribution}%[language=fr]%de,it,pl,ru
\bauthor{\fnm{S.T.} \snm{Zalesak}}
\btitle{Fully multidimensional flux-corrected transport algorithms for fluids}
\end{bcontribution}
\begin{bhost}
\begin{bissue}
\bseries{\btitle{J. Comput. Phys.} \bvolumeno{31}}
\bdate{1979}
\end{bissue}
\bpages{\bfirstpage{335}\blastpage{362}}
\end{bhost}
\end{bsubitem}
%
\OrigBibText
S.~T. Zalesak, Fully multidimensional flux-corrected transport
algorithms for fluids, Journal of Computational Physics 31 (1979)
335--362.
\endOrigBibText
\bptok{structpyb}%
\endbibitem

%b20 ###
%b20 #&#
\bibitem{parrott}
\begin{bsubitem}
\begin{bcontribution}%[language=fr]%de,it,pl,ru
\bauthor{\fnm{A.K.} \snm{Parrott}}
\bauthor{\fnm{M.A.} \snm{Christie}}
\btitle{FCT applied to the 2-d finite element solution of tracer transport
by single phase flow in a porous medium}
\end{bcontribution}
\begin{bhost}
\begin{beditedbook}
\btitle{Proceedings on the ICFD Conference on Numerical Methods in Fluid Dynamics}
\bdate{1986}
\bpublisher{\bname{Oxford University Press}}
\end{beditedbook}
\bpages{\bfirstpage{609}}
\end{bhost}
\end{bsubitem}
%
\OrigBibText
A.~K. Parrott, M.~A. Christie, Fct applied to the 2-d finite element
solution of tracer transport by single phase flow in a porous medium,
in: Proceedings on the ICFD Conference on Numerical Methods in Fluid
Dynamics, Oxford University Press, 1986, p. 609.
\endOrigBibText
\bptok{structpyb}%
\endbibitem

%b21 ###
%b21 #&#
\bibitem{lohner}
\begin{bsubitem}
\begin{bcontribution}%[language=fr]%de,it,pl,ru
\bauthor{\fnm{R.} \snm{L\"{o}hner}}
\bauthor{\fnm{K.} \snm{Morgan}}
\bauthor{\fnm{J.} \snm{Peraire}}
\bauthor{\fnm{M.} \snm{Vahdati}}
\btitle{Finite element flux-corrected transport (FEM-{FCT}) for the Euler
and Navier--Stokes equations}
\end{bcontribution}
\begin{bhost}
\begin{bissue}
\bseries{\btitle{Int. J. Numer. Methods Fluids} \bvolumeno{7}}
\bdate{1987}
\end{bissue}
\bpages{\bfirstpage{1093}\blastpage{1109}}
\end{bhost}
\end{bsubitem}
%
\OrigBibText
R.~L\"ohner, K.~Morgan, J.~Peraire, M.~Vahdati, Finite element
flux-corrected transport (FEM-{FCT}) for the {E}uler and
{N}avier-{S}tokes equations, International Journal for Numerical Methods
in Fluids 7 (1987) 1093--1109.
\endOrigBibText
\bptok{structpyb}%
\endbibitem

%b22 ###
%b22 #&#
\bibitem{kuzmin_FCT}
\begin{bsubitem}
\begin{bcontribution}%[language=fr]%de,it,pl,ru
\bauthor{\fnm{D.} \snm{Kuzmin}}
\bauthor{\fnm{R.} \snm{L\"{o}hner}}
\bauthor{\fnm{S.} \snm{Turek}}
\btitle{Flux-Corrected Transport}
\end{bcontribution}
\begin{bhost}
\begin{bbook}
\bedition{1st edition}
\bdate{2005}
\bpublisher{\bname{Springer-Verlag}\blocation{Berlin, Heidelberg, Germany}}
\end{bbook}
\end{bhost}
\end{bsubitem}
%
\OrigBibText
D.~Kuzmin, R.~L\"ohner, S.~Turek, Flux-Corrected Transport, 1st
Edition, Springer-Verlag Berlin Heidelberg, Germany, 2005.
\endOrigBibText
\bptok{structpyb}%
\endbibitem

%b23 ###
%b23 #&#
\bibitem{kuzmin_general}
\begin{bsubitem}
\begin{bcontribution}%[language=fr]%de,it,pl,ru
\bauthor{\fnm{D.} \snm{Kuzmin}}
\btitle{On the design of general-purpose flux limiters for finite element
schemes, I: scalar convection}
\end{bcontribution}
\begin{bhost}
\begin{bissue}
\bseries{\btitle{J. Comput. Phys.} \bvolumeno{219}}
\bdate{2006}
\end{bissue}
\bpages{\bfirstpage{513}\blastpage{531}}
\end{bhost}
\end{bsubitem}
%
\OrigBibText
D.~Kuzmin, On the design of general-purpose flux limiters for finite
element schemes. I. scalar convection, Journal of Computational
Physics 219 (2006) 513--531.
\endOrigBibText
\bptok{structpyb}%
\endbibitem

%b24 ###
%b24 #&#
\bibitem{moller_2008}
\begin{bsubitem}
\begin{bcontribution}%[language=fr]%de,it,pl,ru
\bauthor{\fnm{M.} \snm{M\"{o}ller}}
\bauthor{\fnm{D.} \snm{Kuzmin}}
\bauthor{\fnm{D.} \snm{Kourounis}}
\btitle{Implicit {FEM-FCT} algorithms and discrete Newton methods for
transient convection problems}
\end{bcontribution}
\begin{bhost}
\begin{bissue}
\bseries{\btitle{Int. J. Numer. Methods Fluids} \bvolumeno{57}}
\bdate{2008}
\end{bissue}
\bpages{\bfirstpage{761}\blastpage{792}}
\bdoi[https://doi.org/10.1002/fld.1654]{10.1002/fld.1654}
\end{bhost}
\end{bsubitem}
%
\OrigBibText
M.~M\"oller, D.~Kuzmin, D.~Kourounis, Implicit {FEM-FCT} algorithms
and discrete {N}ewton methods for transient convection problems,
International Journal for Numerical Methods in Fluids 57 (2008)
761--792.
\newblock doi:10.1002/fld.1654.
\endOrigBibText
\bptok{structpyb}%
\endbibitem

%b25 ###
%b25 #&#
\bibitem{kuzmin_failsafe}
\begin{bsubitem}
\begin{bcontribution}%[language=fr]%de,it,pl,ru
\bauthor{\fnm{D.} \snm{Kuzmin}}
\bauthor{\fnm{M.} \snm{M\"{o}ller}}
\bauthor{\fnm{J.N.} \snm{Shadid}}
\bauthor{\fnm{M.} \snm{Shashkov}}
\btitle{Failsafe flux limiting and constrained data projections for equations of gas dynamics}
\end{bcontribution}
\begin{bhost}
\begin{bissue}
\bseries{\btitle{J. Comput. Phys.}}
\bdate{2010}
\end{bissue}
\bpages{\bfirstpage{761}\blastpage{792}}
\bdoi[https://doi.org/10.1016/j.jcp.2010.08.009]{10.1016/j.jcp.2010.08.009}
\end{bhost}
\end{bsubitem}
%
\OrigBibText
D.~Kuzmin, M.~M\"oller, J.~N. Shadid, M.~Shashkov, Failsafe flux
limiting and constrained data projections for equations of gas dynamics,
Journal of Computational
Physicsdoi:10.1016/j.jcp.2010.08.009.
\endOrigBibText
\bptok{structpyb}%
\endbibitem

%b26 ###
%b26 #&#
\bibitem{kuzmin_closepacking}
\begin{bsubitem}
\begin{bcontribution}%[language=fr]%de,it,pl,ru
\bauthor{\fnm{D.} \snm{Kuzmin}}
\bauthor{\fnm{Y.} \snm{Gorb}}
\btitle{A flux-corrected transport algorithm for handling the close-packing
limit in dense suspensions}
\end{bcontribution}
\begin{bhost}
\begin{bissue}
\bseries{\btitle{J. Comput. Appl. Math.} \bvolumeno{236}}
\bdate{2012}
\end{bissue}
\bpages{\bfirstpage{4944}\blastpage{4951}}
\bdoi[https://doi.org/10.1016/j.cam.2011.10.019]{10.1016/j.cam.2011.10.019}
\end{bhost}
\end{bsubitem}
%
\OrigBibText
D.~Kuzmin, Y.~Gorb, A flux-corrected transport algorithm for handling
the close-packing limit in dense suspensions, Journal of Computational
and Applied Mathematics 236 (2012) 4944--4951.
\newblock doi:10.1016/j.cam.2011.10.019.
\endOrigBibText
\bptok{structpyb}%
\endbibitem

%b27 ###
%b27 #&#
\bibitem{guermond_secondorder}
\begin{bsubitem}
\begin{bcontribution}%[language=fr]%de,it,pl,ru
\bauthor{\fnm{J.-L.} \snm{Guermond}}
\bauthor{\fnm{M.} \snm{Nazarov}}
\bauthor{\fnm{B.} \snm{Popov}}
\bauthor{\fnm{Y.} \snm{Yang}}
\btitle{A second-order maximum principle preserving Lagrange finite
element technique for nonlinear scalar conservation equations}
\end{bcontribution}
\begin{bhost}
\begin{bissue}
\bseries{\btitle{SIAM J. Numer. Anal.} \bvolumeno{52}}
\bdate{2014}
\end{bissue}
\bpages{\bfirstpage{2163}\blastpage{2182}}
\end{bhost}
\end{bsubitem}
%
\OrigBibText
J.-L. Guermond, M.~Nazarov, B.~Popov, Y.~Yang, A second-order maximum
principle preserving {L}agrange finite element technique for nonlinear
scalar conservation equations, SIAM Journal on Numerical Analysis 52
(2014) 2163--2182.
\endOrigBibText
\bptok{structpyb}%
\endbibitem

%b28 ###
%b28 #&#
\bibitem{guermond_firstorder}
\begin{bsubitem}
\begin{bcontribution}%[language=fr]%de,it,pl,ru
\bauthor{\fnm{J.-L.} \snm{Guermond}}
\bauthor{\fnm{M.} \snm{Nazarov}}
\btitle{A maximum-principle preserving {$C^0$} finite element method
for scalar conservation equations}
\end{bcontribution}
\begin{bhost}
\begin{bissue}
\bseries{\btitle{Comput. Methods Appl. Mech. Eng.} \bvolumeno{272}}
\bdate{2014}
\end{bissue}
\bpages{\bfirstpage{198}\blastpage{213}}
\end{bhost}
\end{bsubitem}
%
\OrigBibText
J.-L. Guermond, M.~Nazarov, A maximum-principle preserving {$C^{0}$}
finite element method for scalar conservation equations, Computational
Methods in Applied Mechanics and Engineering 272 (2014) 198--213.
\endOrigBibText
\bptok{structpyb}%
\endbibitem

%b29 ###
%b29 #&#
\bibitem{gottlieb}
\begin{bsubitem}
\begin{bcontribution}%[language=fr]%de,it,pl,ru
\bauthor{\fnm{S.} \snm{Gottlieb}}
\btitle{On high order strong stability preserving Runge--Kutta and
multi step time discretizations}
\end{bcontribution}
\begin{bhost}
\begin{bissue}
\bseries{\btitle{J. Sci. Comput.} \bvolumeno{25}}
\bissueno{1}
\bdate{2005}
\end{bissue}
\end{bhost}
\end{bsubitem}
%
\OrigBibText
S.~Gottlieb, On high order strong stability preserving {R}unge-{K}utta
and multi step time discretizations, Journal of Scientific Computing
25~(1).
\endOrigBibText
\bptok{structpyb}%
\endbibitem

%b30 ###
%b30 #&#
\bibitem{macdonald}
\begin{bsubitem}
\begin{bcontribution}%[language=fr]%de,it,pl,ru
\bauthor{\fnm{C.B.} \snm{Macdonald}}
\btitle{Constructing High-Order Runge--Kutta Methods with Embedded
Strong-Stability-Preserving Pairs}
\end{bcontribution}
\bcomment{Master's Thesis}\prnsep{,\ }
\begin{bhost}
\begin{bbook}[class=report]
\bdate{August 2003}
\bpublisher{\bname{Acadia University}}
\end{bbook}
\end{bhost}
\end{bsubitem}
%
\OrigBibText
C.~B. Macdonald, Constructing high-order {R}unge-{K}utta methods with
embedded strong-stability-preserving pairs, Master's thesis, Acadia
University (August 2003).
\endOrigBibText
\bptok{structpyb}%
\endbibitem

%b31 ###
%b31 #&#
\bibitem{plemmons}
\begin{bsubitem}
\begin{bcontribution}%[language=fr]%de,it,pl,ru
\bauthor{\fnm{R.J.} \snm{Plemmins}}
\btitle{M-matrix characterizations, I: nonsingular m-matrices}
\end{bcontribution}
\begin{bhost}
\begin{bissue}
\bseries{\btitle{Linear Algebra Appl.} \bvolumeno{18}}
\bissueno{2}
\bdate{1977}
\end{bissue}
\bpages{\bfirstpage{175}\blastpage{188}}
\bdoi[https://doi.org/10.1016/0024-3795\wwwbreak (77)90073-8]{10.1016/0024-3795(77)90073-8}
\end{bhost}
\end{bsubitem}
%
\OrigBibText
R.~J. Plemmins, M-matrix characterizations I. -- nonsingular
m-matrices, Linear Algebra and its Applications 18~(2) (1977) 175--188.
\newblock doi:10.1016/0024-3795(77)90073-8.
\endOrigBibText
\bptok{structpyb}%
\endbibitem

%b32 ###
%b32 #&#
\bibitem{marco_low_mach}
\begin{bsubitem}
\begin{bcontribution}%[language=fr]%de,it,pl,ru
\bauthor{\fnm{M.-O.} \snm{Delchini}}
\btitle{Entropy-based viscous regularization for the multi-dimensional Euler
equations in low-Mach and transonic flows}
\end{bcontribution}
\begin{bhost}
\begin{bissue}
\bseries{\btitle{Comput. Fluids} \bvolumeno{118}}
\bdate{2015}
\end{bissue}
\bpages{\bfirstpage{225}\blastpage{244}}
\end{bhost}
\end{bsubitem}
%
\OrigBibText
M.-O. Delchini, Entropy-based viscous regularization for the multi-d
{E}uler equations in low-mach and transonic flows, Computers and Fluids
118~(225--244).
\endOrigBibText
\bptok{structpyb}%
\endbibitem

%b33 ###
%b33 #&#
\bibitem{marco_SEM}
\begin{bsubitem}
\begin{bcontribution}%[language=fr]%de,it,pl,ru
\bauthor{\fnm{M.-O.} \snm{Delchini}}
\bauthor{\fnm{J.C.} \snm{Ragusa}}
\bauthor{\fnm{R.A.} \snm{Berry}}
\btitle{Viscous regularization for the non-equilibrium seven-equation two-phase flow model}
\end{bcontribution}
\begin{bhost}
\begin{bissue}
\bseries{\btitle{J. Sci. Comput.} \bvolumeno{69}}
\bissueno{2}
\bdate{2016}
\end{bissue}
\bpages{\bfirstpage{764}\blastpage{804}}
\bdoi[https://doi.org/10.1007/s10915-016-0217-6]{10.1007/s10915-016-0217-6}
\end{bhost}
\end{bsubitem}
%
\OrigBibText
M.-O. Delchini, J.~C. Ragusa, R.~A. Berry, Viscous regularization for
the non-equilibrium seven-equation two-phase flow model, Journal of
Scientific
Computing, doi:10.1007/s10915-016-0217-6.
http://dx.doi.org/10.1007/s10915-016-0217-6
\endOrigBibText
\bptok{structpyb}%
\endbibitem

%b34 ###
%b34 #&#
\bibitem{leveque2002}
\begin{bsubitem}
\begin{bcontribution}%[language=fr]%de,it,pl,ru
\bauthor{\fnm{R.J.} \snm{LeVeque}}
\btitle{Finite Volume Methods for Hyperbolic Problems}
\end{bcontribution}
\begin{bhost}
\begin{bbook}
\bdate{2002}
\bpublisher{\bname{Cambridge University Press}}
\end{bbook}
\end{bhost}
\end{bsubitem}
%
\OrigBibText
R.~J. LeVeque, Finite Volume Methods for Hyperbolic Problems, Cambridge
University Press, 2002.
\endOrigBibText
\bptok{structpyb}%
\endbibitem

%b35 ###
%b35 #&#
\bibitem{schar}
\begin{bsubitem}
\begin{bcontribution}%[language=fr]%de,it,pl,ru
\bauthor{\fnm{C.} \snm{Sch\"{a}r}}
\bauthor{\fnm{P.K.} \snm{Smolarkiewicz}}
\btitle{A synchronous and iterative flux-correction formalism for coupled transport equations}
\end{bcontribution}
\begin{bhost}
\begin{bissue}
\bseries{\btitle{J. Comput. Phys.} \bvolumeno{128}}
\bissueno{1}
\bdate{1996}
\end{bissue}
\bpages{\bfirstpage{101}\blastpage{120}}
\bdoi[https://doi.org/10.1006/jcph.1996.0198]{10.1006/jcph.1996.0198}
\end{bhost}
\end{bsubitem}
%
\OrigBibText
C.~Sch\"ar, P.~K. Smolarkiewicz, A synchronous and iterative
flux-correction formalism for coupled transport equations, Journal of
Computational Physics 128~(1) (1996) 101--120.
\newblock doi:10.1006/jcph.1996.0198.
\endOrigBibText
\bptok{structpyb}%
\endbibitem

\end{thebibliography}

%
\end{backmatter}
%
\end{document}
%
