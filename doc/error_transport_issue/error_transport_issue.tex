\documentclass{article}
\author{Joshua Hansel}
\date{March 4, 2015}
\title{Observation of Super-asymptotic Convergence for Hyperbolic Problems}
\begin{document}

\maketitle

It was observed in the course of convergence studies that in some circumstances,
a convergence rate was greater than expected. More specifically, a spatially
first-order scheme was showing second-order convergence in space. This
article demonstrates how this can occur.

In short, this is believed to be an issue unique to hyperbolic PDEs that
occurrs when the end time is not large enough.
This issue will be demonstrated using the simplest hyperbolic PDE, the
linear advection equation, but the analysis easily extends to more complicated
hyperbolic PDEs. The low-order scheme is assumed to employ a first-order
artificial diffusion term $h\frac{\partial^2 u_h}{\partial x^2}$.
Consider the PDEs for the exact solution $u$ and the approximate solution
$u_h$, where $h$ is the mesh size:

\begin{equation}
  \frac{\partial u}{\partial t} + \frac{\partial u}{\partial x} = q(x,t)
\end{equation}

\begin{equation}
  \frac{\partial u_h}{\partial t} + \frac{\partial u_h}{\partial x} = h\frac{\partial^2 u_h}{\partial x^2} + q(x,t)
\end{equation}

\noindent Taking a difference of these equations gives a PDE for the error:

\begin{equation}\label{error}
  \frac{\partial e}{\partial t} + \frac{\partial e}{\partial x} = -h\frac{\partial^2 u_h}{\partial x^2},
\end{equation}

\noindent where $e=u-u_h$. Thus the error equation is a hyperbolic equation of the same form as
the original equation, but with the source term being the first-order artificial diffusion error.
This equation will now be solved using the method of characteristics.
Regarding $x$ as a function of time,

\[
  x\rightarrow x(t)=x_0 + t, \qquad u(x,t)\rightarrow u(x(t),t),
\]

\noindent Equation \ref{error} may be rewritten as

\begin{equation}
  \frac{de}{dt} = -h\frac{\partial^2 u_h}{\partial x^2},
\end{equation}

\noindent Integrating from $t'=0$ to $t'=t$ gives

\begin{equation}
  e = e_0 -h\int\limits_0^t\frac{\partial^2 u_h}{\partial x^2}dt',
\end{equation}

\noindent where $e_0$ is the error at time 0, assumed to be only error due to interpolation
of the initial conditions. With $p$-degree finite elements, the interpolation error
is $O(h^{p+1})$. Here, linear finite elements are used, so $e_0=O(h^2)$. 
Expanding the diffusion term in a Taylor series
about $t'=0$ gives

\[
  \frac{\partial^2 u_h}{\partial x^2} = \left.\frac{\partial^2 u_h}{\partial x^2}\right|_0
    + O(t')
\]

\noindent Thus the error becomes

\[
  e = O(h^2) -h t\left.\frac{\partial^2 u_h}{\partial x^2}\right|_0 + O(h t^2).
\]

\noindent If $t<h$, this becomes

\begin{eqnarray*}
  e & = & O(h^2) + O(h^2) + O(h^3)\\
    & = & O(h^2),
\end{eqnarray*}

\noindent instead of the case $t>h$:

\begin{eqnarray*}
  e & = & O(h^2) + O(h)\\
    & = & O(h).
\end{eqnarray*}

\end{document}
