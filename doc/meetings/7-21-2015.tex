\section*{7-21-2015}

I met with Dr. Ragusa, and John Peterson and David Andrs of the MOOSE team
to discuss the possibility of implementing our FCT scheme in MOOSE. John
outlined the main challenges of implementation to be the following:
\begin{itemize}
  \item MOOSE currently can store only \emph{one} matrix, whereas the FCT
     algorithm, as it is currently formulated, requires the storage of
     several matrices.
  \item MOOSE currently can only do one solve instead of the two solves
     per time step required by FCT (the first for the high-order solution
     and the second for the FCT solution).
\end{itemize}
It was suggested that perhaps the FCT algorithm may be formulated to
allow matrix entries to be used on-the-fly and thus not stored. The
possibility of creating ``auxiliary'' matrices to be used was expressed.
Other concerns that were discussed include the end project goal:
\begin{itemize}
  \item What is the goal to be completed by the end of my anticipated
     graduation/disseration (May 2016)?
  \item Is the end goal to be able to apply FCT to general conservation
     law \emph{systems} or just general \emph{scalar} conservation laws?
  \item Is the end goal to have these abilities in MOOSE or to have
     this capability somewhere (such as in \texttt{deal.II})?
\end{itemize}
It was suggested that if the end goal is to apply FCT to \emph{systems}
that we should develop the theory for systems first and perhaps even
write the systems code in \texttt{deal.II} since MOOSE currently
lacks the ability to use FCT in its current formulation. Also it was
suggested that it would be advantageous to write this \texttt{deal.II}
code in parallel to expose any parallelization challenges.
