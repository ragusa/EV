Marco and I met to discuss my questions asked by email, namely why the
solution given by the method of characteristics at the boundary, which
are at the new time value, $\timevalue^{\timeindex+1}$, are of interest,
since we are using explicit Euler, which only is concerned with values
at $\timevalue^{\timeindex}$. The answer is that Marco had intended that
I strongly impose the values given by the characteristics, instead of
using them to compute the boundary fluxes, as I had been doing previously.

The other question I had was how to implement wall boundary conditions,
namely the condition $\velocity\cdot\normalvector=0$ in 2-D; my confusion
was that setting the normal component of the velocity only fixed one of
two degrees of freedom for the velocity, so I didn't know what the other
component was. The answer was that it didn't matter; the boundary fluxes
only require knowledge of $\velocity\cdot\normalvector$. Therefore the
wall condition cancels the continuity boundary fluxes and just leaves
a ``pressure'' term in the momentum boundary fluxes.

We also ran the code with entropy viscosity for the perturbed lake-at-rest
problem and couldn't find parameters that allowed a steady-state
to be achieved. It was suggested that I might try the following:
\begin{enumerate}
  \item Smoothing the perturbation profile instead of having a step function, and
  \item Smoothing the viscosity variations between cells, e.g., take the viscosity
    to be the maximum of it and its neighbors. This prevents oscillations in the
    viscosity profile.
\end{enumerate}

