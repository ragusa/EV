The main goal of this research is to develop numerical methods for hyperbolic
PDEs that address the development of spurious oscillations and negativities
for physically positive solution quantities. The equations and models under
consideration are the scalar transport PDE model given in Section \ref{sec:scalar_model}
and the shallow water equations (SWE), given in Section \ref{sec:sw_model}.
The methodology developed for these equations is intended to be largely
extendible to other hyperbolic systems. The following list discusses specific
primary research objectives that derive from the main research goal.

\begin{itemize}
  \item \textbf{Demonstrate 2nd-order spatial convergence rates for smooth problems.}
    For smooth problems, CFEM with piecewise linear basis functions should
    have 2nd-order convergence rates for both $L^1$ and $L^2$ norms.
  \item \textbf{Demonstrate theoretical convergence rates for non-smooth problems.}
    For non-smooth problems, CFEM with piecewise linear basis functions should
    have convergence rates of $\frac{3}{4}$ and $\frac{3}{8}$ for $L^1$ and $L^2$
    norms, respectively.
  \item \textbf{Prove and demonstrate stability.}
    Stability should be able to be proven theoretically and demonstrated with
    numerical illustrations. Stability in the $L^\infty$ norm can be easily
    be theoretically proved for the FCT scheme recursive application of the
    imposed solution bounds.
  \item \textbf{Prove and demonstrate positivity.}
    Positivity for physically positive solution quantities should be proved
    theoretically and demonstrated with numerical illustrations. For the FCT
    scheme, proving positivity theoretically amounts to proving that the
    imposed lower bound on the solution is non-negative.
  \item \textbf{Demonstrate advantages of entropy-based FCT over Galerkin FCT.}
    Traditionally the high-order method used in the FCT algorithm is simply the
    Galerkin spatial discretization, not the entropy-based method discussed in
    Section \ref{sec:methodology}. It is desirable to compare the two approaches
    and demonstrate the advantages of entropy-based FCT over Galerkin FCT, as
    well as give justification with some heuristics.
  \item \textbf{Prove and demonstrate satisfaction of a discrete maximum
    principle (DMP) for the scalar case.}
    For the scalar case, it is possible to theoretically prove that the FCT
    scheme satisfies a DMP. This is a desired goal, as well as demonstration
    with numerical illustrations.
  \item \textbf{Demonstrate a reduction in spurious oscillations.}
    The FCT algorithm to date cannot guarantee the absence of spurious
    oscillations and artifacts; however, it should be proven that the incidence
    of these spurious artifacts is reduced from the entropy-based method
    without FCT.
\end{itemize}
  
The following list gives secondary objectives of the research, which are
not priorities of the research but would be advantageous to complete
if the primary objectives are completed with sufficient remaining time:

\begin{itemize}
  \item \textbf{Perform timing studies.}
    This research contains a number of schemes of varying complexity, and
    it advantageous to compare the relative computational costs in
    comparing merits of each scheme. The entropy-based scheme and FCT
    algorithm aim to provide numerous advantages but each comes with
    an added cost.
  \item \textbf{Develop a parallel implementation.}
    The research is not aimed at parallel computing, and a serial
    implementation is sufficient for meeting primary objectives;
    however, a parallel implementation would allow larger problems
    to be run. Fine spatial refinement studies in multiple dimensions
    are particularly expensive and often are computationally intractable
    with a serial implementation.
  \item \textbf{Run large benchmark problems.}
    It is desirable to compare how the methods developed in this research
    compare to existing methods in literature for large benchmark
    problems. One example is the 2-D partial dam-break problem for the SWE.
\end{itemize}

