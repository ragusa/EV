The main goal of this research is to develop numerical methods for hyperbolic
PDEs that address the development of spurious oscillations and negativities for
physically positive solution quantities. The equations and models under
consideration are the scalar transport PDE model given in Section
\ref{sec:scalar_model} and the shallow water equations (SWE) with flat bottom
topography ($\nabla\bathymetry=\mathbf{0}$), given in Section
\ref{sec:sw_model}.  The methodology developed for these equations is intended
to be largely extendible to other hyperbolic systems. The following list
discusses specific research objectives that derive from the main research goal.

\begin{itemize}
  \item \textbf{Develop low-order scheme.}
    For the scalar case, this first-order-accurate scheme should satisfy a discrete maximum principle
    as described in Section \ref{sec:low_order_scheme}. For the systems case,
    this scheme should have an invariant domain. The domain-invariant scheme
    is developed for general conservation law systems; however, the scheme
    requires an estimation of maximum wave speeds, which entails an
    analysis of the characteristics for the specific hyperbolic system. This
    research performs this analysis for the shallow water equations with flat
    bottom topography. The monotonicity and
    positivity-preservation of these schemes should be demonstrated with
    numerical examples.

    For the \emph{scalar} case, the low-order scheme should be
    developed for the following time discretizations:
    \begin{itemize}
      \item steady-state,
      \item implicit $\theta$ methods: implicit Euler and Crank-Nicolson,
      \item explicit methods of the SSPRK family.
    \end{itemize}
    For the \emph{systems} case, the low-order scheme should be
    developed for the following time discretizations:
    \begin{itemize}
      \item explicit methods of the SSPRK family.
    \end{itemize}
  \item \textbf{Develop high-order entropy-based scheme.}
    The entropy-based scheme should be second-order-accurate and should
    demonstrate stability with numerical examples. A reduction in the occurrence
    and magnitude of spurious oscillations should be observed. The scalar
    case will use an arbitrary convex entropy function such as
    $\entropy(\scalarsolution)=\half\scalarsolution^2$, while for the shallow
    water equations, the entropy function will be the total energy.

    For the \emph{scalar} case, the high-order scheme should be
    developed for the following time discretizations:
    \begin{itemize}
      \item steady-state,
      \item implicit $\theta$ methods: implicit Euler and Crank-Nicolson,
      \item explicit methods of the SSPRK family.
    \end{itemize}
    For the \emph{systems} case, the high-order scheme should be
    developed for the following time discretizations:
    \begin{itemize}
      \item explicit methods of the SSPRK family.
    \end{itemize}
  \item \textbf{Develop Galerkin FCT scheme.}
    The basic FEM-FCT scheme, as described by Kuzmin in \cite{kuzmin_FCT}, should
    be implemented. This scheme uses the standard second-order-accurate Galerkin
    scheme as the high-order scheme in the algorithm. Positivity-preservation and
    a reduction of spurious oscillations should be demonstrated.

    For the \emph{scalar} case, the Galerkin FCT scheme should be
    developed for the following time discretizations:
    \begin{itemize}
      \item steady-state,
      \item implicit $\theta$ methods: implicit Euler and Crank-Nicolson,
      \item explicit methods of the SSPRK family.
    \end{itemize}
    For the \emph{systems} case, the Galerkin FCT scheme should be
    developed for the following time discretizations:
    \begin{itemize}
      \item explicit methods of the SSPRK family.
    \end{itemize}
  \item \textbf{Develop entropy-based FCT scheme.}
    The basic FEM-FCT scheme should be modified to replace the Galerkin scheme
    with the high-order entropy-based scheme in the FCT algorithm. This should
    demonstrate positivity-preservation and a further reduction in spurious
    oscillations from the entropy-based method alone.

    For the \emph{scalar} case, the entropy-based FCT scheme should be
    developed for the following time discretizations:
    \begin{itemize}
      \item steady-state,
      \item implicit $\theta$ methods: implicit Euler and Crank-Nicolson,
      \item explicit methods of the SSPRK family.
    \end{itemize}
    For the \emph{systems} case, the entropy-based FCT scheme should be
    developed for the following time discretizations:
    \begin{itemize}
      \item explicit methods of the SSPRK family.
    \end{itemize}
  \item \textbf{Compare schemes.}
    Schemes developed in this research should be compared in a number of
    test problems. In addition to comparing accuracy and convergence rates,
    it is advantageous to determine rough computational cost estimates
    of each method, since the developed methods have the cost of additional
    complexity.
\end{itemize}

