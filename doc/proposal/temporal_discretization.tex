This research mainly considers explicit time discretizations but also considers
the steady-state case and implicit
temporal discretizations such as $\theta$ methods (which include backward
Euler and Crank-Nicolson, for example). The steady-state and implicit time
discretizations are complicated by the fact that the
imposed bounds are implicit, so nonlinear solves are required, unlike
for explicit time discretizations.

The remainder of this proposal assumes that forward Euler (FE) is used for the
temporal discretization. The methodology presented in this research is best
understood using FE; however, a class of explicit temporal discretizations
called Strong-Stability-Preserving Runge-Kutta (SSPRK) methods are expressible
as a combination of FE steps, so the methods given in this research are
applicable to higher-order temporal discretizations.

The FE discretization for a general system $\mathbf{M}d\solutionvector/dt=
\mathbf{r}(\solutionvector(t),t)$ is the following:
\begin{equation}
  \mathbf{M}\frac{\solutionvector^{n+1}-\solutionvector^n}{\dt} =
    \mathbf{r}(\solutionvector^n,t^n) \eqc
\end{equation}
where $\dt$ is the time step size.
