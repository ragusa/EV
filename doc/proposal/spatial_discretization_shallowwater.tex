The FEM basis functions for each solution component are chosen to be identical,
so one may take the viewpoint that degrees of freedom are vector-valued
and that the test functions are scalar:
\begin{equation}
  \approximatevectorsolution\xt = \sumj \solutionvector_j(\timevalue)
  \testfunction_j(\x) \eqc
\end{equation}
where $\solutionvector_j(\timevalue)$ is a vector of the degrees of freedom of all
solution components at a node $j$:
$\solutionvector_j(\timevalue)=[\height_j(\timevalue),
\heightmomentum_j(\timevalue)]\transpose$. Note that in a practical
implementation, the basis functions would be viewed as vector-valued, with
degrees of freedom being scalar.

As opposed to the scalar conservation law case, the vector case takes a
group finite element approach: the conservation law flux is interpolated
using the flux evaluated at nodes:
\begin{equation}
  \consfluxvector(\vectorsolution\xt) \rightarrow
  \Pi\consfluxvector(\vectorsolution\xt) 
    \equiv \sumj\testfunction_j(\x)\consfluxvector
      (\vectorsolution(\x_j,\timevalue))
  \eqc
\end{equation}
where hereafter the nodal flux values used as interpolation values,
$\consfluxvector(\vectorsolution(\x_j,\timevalue))$,
will be denoted as $\consfluxinterpolant_j(\timevalue)$. This is done
as a step for proving the domain-invariance of the low-order scheme,
which is introduced in Section \ref{sec:low_order_scheme_system}.

Rearranging Equation \eqref{eq:shallow_water_equations},
substituting the approximate FEM
solution and conservation law flux,
testing with a test function $\testfunction_i$,
and integrating by parts gives
\begin{equation}
  \sumj\consistentmassentry
    \ddt{\solutionvector_j}
    + \sum_j\gradiententry\cdot\consfluxinterpolant_j(t)
    = \ssrhs_i(t) \eqc
\end{equation}
where
\begin{equation}
  \gradiententry \equiv
    \intSij\testfunction_i(\x)
      \nabla\testfunction_j(\x) d\volume
  \eqp
\end{equation}
For the 2-D shallow water equations,
\begin{equation}
  \consfluxinterpolant_j(\timevalue) = \sq{\begin{array}{c}
    \approximate{\heightmomentum}_j \\
    \frac{\approximate{\heightmomentumletter}_{x,j}}{\approximate{\height}_j}
      \approximate{\heightmomentum}_j
      + \frac{1}{2}\gravity\approximate{\height}_j^2\unitvector{x} \\
    \frac{\approximate{\heightmomentumletter}_{y,j}}{\approximate{\height}_j}
      \approximate{\heightmomentum}_j
      + \frac{1}{2}\gravity\approximate{\height}_j^2\unitvector{y}
  \end{array}} \eqc \qquad
  \ssrhs_i(t) = \sq{\begin{array}{c}
    0\\
    - \int_{\support_i}\testfunction_i(\x)
      \gravity\approximate{\height}\xt\partial_x\bathymetry\dvolume \\
    - \int_{\support_i}\testfunction_i(\x)
      \gravity\approximate{\height}\xt\partial_y\bathymetry\dvolume
  \end{array}} \eqp
\end{equation}
