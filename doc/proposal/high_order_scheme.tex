As stated in the introduction, the FCT algorithm requires the use of a
high-order scheme. Traditionally, the standard Galerkin scheme given
by Equation \eqref{eq:semidiscrete} has been used in this role\cite{kuzmin_FCT};
however, this research considers an alternative high-order scheme
that uses entropy-based artificial dissipation, introduced by Guermond
in 2011\cite{guermond_ev}.

Conservation law systems usually have at
least one entropy function that satisfies an auxiliary entropy inequality,
\begin{equation}\label{eq:entropy_inequality}
  \partial_t\entropy(\scalarsolution)
    + \divergence\entropyflux(\scalarsolution) \leq 0 \eqc
\end{equation}
and scalar conservation laws have many entropy/entropy flux pairs;
the Cauchy problem, which is restricted to a bounded domain,
has a unique entropy solution that satisfies an entropy inequality
for any convex entropy function $\entropy(\scalarsolution)$ and associated
entropy flux $\entropyflux(\scalarsolution)\equiv
\int\entropy'(\scalarsolution)\mathbf{\consfluxletter}'(\scalarsolution)
  d\scalarsolution$
\cite{guermond_ev}\cite{bardos1979}\cite{kruzkov1970}.
Entropy functions satisfy a conservation equality in smooth regions and
satisfy an entropy inequality in shocks\cite{guermond_ev}. The
idea of the entropy viscosity method introduced in \cite{guermond_ev}
is to add dissipation in proportion to the entropy production
$\entropyresidual(\scalarsolution)\equiv
  \partial_t\entropy(\scalarsolution)
    + \divergence\entropyflux(\scalarsolution)$,
since the lack of equality in Equation \eqref{eq:entropy_inequality} indicates
a shock is present. This artificial dissipation then lowers the entropy
in this region and in this way enforces the entropy inequality.

As Guermond states, for convex scalar conservation law fluxes $\consfluxscalar$
in 1-D, a single entropy/entropy flux pair is enough to select the
unique entropy solution\cite{guermond_ev}\cite{lellis2004}\cite{panov1994}.
Therefore for scalar conservation laws it suffices to choose any
convex function for the definition of $\entropy(\scalarsolution)$,
such as $\entropy(\scalarsolution) = \half\scalarsolution^2$.

In practice, the entropy viscosity is set to be proportional to a linear
combination of the local entropy residual $R_K(u) =
\left\|R(u)\right\|_{L^\infty(K)}$ and entropy jumps $J_F(u)$ across the faces:
\begin{equation}
  \nu^{\eta,n}_K = \frac{c_R R_K(\approximatescalarsolution^n)
    + c_J\max\limits_{F\in\partial K}J_F(\approximatescalarsolution^n)}
    {\|\eta(\approximatescalarsolution^n)-\bar{\eta}(\approximatescalarsolution^n)\|_{L^\infty(\mathcal{D})}} \eqc
\end{equation}
where the denominator is just a normalization constant. The viscosity used
in the high-order scheme uses the entropy viscosity with an upper bound
of the low-order viscosity for a given cell, since a viscosity above
this would be in excess:
\begin{equation}
  \nu^{H,n}_K = \min(\nu^{L}_K,\nu^{\eta,n}_K) \eqp
\end{equation}
After computation of the high-order cell viscosities, the high-order diffusion
matrix is assembled just like the low-order diffusion matrix but replacing
the low-order viscosity with the high-order viscosity:
\begin{equation}
  D\ij^{H,n} = \sum\limits_{K\subset S\ij}\nu_K^{H,n}
    b_K(\testfunction_j,\testfunction_i) \eqp
\end{equation}
Putting everything together, the discrete high-order system for the scalar
case is the following:
\begin{equation}
  \consistentmassmatrix\frac{\solutionvector^{H,n+1}
    -\solutionvector^n}{\dt} + (\ssmatrix + \diffusionmatrix^{H,n})
    \solutionvector^n = \ssrhs^n \eqc
\end{equation}
and one can note that no mass lumping is performed in this case.
