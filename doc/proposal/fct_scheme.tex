\begin{itemize}
   \item Recall that FCT defines antidiffusive correction fluxes
      from a low-order, monotone scheme to a high-order scheme. Calling
      these fluxes $\correctionfluxvector$, this gives
      \begin{equation}
        \lumpedmassmatrix\frac{\solutionvector^{H,n+1}-\solutionvector^n}{\dt}
          + (\ssmatrix+\diffusionmatrix^L)\solutionvector^n = \ssrhs^n
          + \correctionfluxvector \eqp
      \end{equation}
   \item Subtracting the high-order scheme equation from this gives the
      definition of $\correctionfluxvector$:
      \begin{equation}
        \correctionfluxvector \equiv
          -(\consistentmassmatrix-\lumpedmassmatrix)
          \frac{\solutionvector^{H,n+1}-\solutionvector^n}{\dt}
          + (\diffusionmatrix^L-\diffusionmatrix^{H,n})\solutionvector^n \eqp
      \end{equation}
   \item Decomposing $\correctionfluxvector$ into internodal fluxes
      $\correctionfluxij$ such that $\sum_j \correctionfluxij =
      \correctionfluxletter_i$, where $\Delta_{j,i}[\mathbf{y}]$
      denotes $y_j - y_i$:
   \begin{equation}
     \correctionfluxij = -M\ij^C\Delta_{j,i}\left[
       \frac{\solutionvector^{H,n+1}-\solutionvector^n}{\Delta t}\right]
       + (D\ij^L-D\ij^{H,n})\Delta_{j,i}[\solutionvector^n] \eqp
   \end{equation}
   \item Recall that the objective of FCT is to limit these antidiffusive
      fluxes to enforce some physical bounds.
   \item The chosen bounds take the form of the DMP satisfied by the
      low-order scheme:
      \begin{equation}
         W_i^-\leq
         U_i^{n+1}\leq
         W_i^+\qquad\forall i \eqp
      \end{equation}
   \item This is achieved by applying a limiting coefficient $L\ij$ to each
      internodal flux $\correctionfluxij$:
      \begin{equation}
        \lumpedmassmatrix\frac{\solutionvector^{n+1}-\solutionvector^n}{\dt}
          + \ssmatrix^L\solutionvector^n = \ssrhs
          + \limitermatrix\cdot\correctionfluxmatrix \eqp
      \end{equation}
   \item Each limiting coefficient is between zero and unity: $0\leq L\ij\leq 1$.
   \begin{itemize}
      \item If all $L\ij$ are zero, then the low-order scheme is produced.
      \item If all $L\ij$ are one, then the high-order scheme is produced.
   \end{itemize}
   \item The enforced bounds can be rearranged to bound the limited flux sums:
      \begin{equation}
         Q^-_i \leq \sum\limits_j L\ij \correctionfluxij \leq Q^+_i \eqc
      \end{equation}
      where $Q_i^\pm$ has the following definition:
      \begin{equation}
         Q_i^\pm \equiv M_{i,i}^L\frac{W_i^\pm-U_i^n}{\Delta t}
         + \sum\limits_j A_{i,j}^L U_j^n - b_i \eqp
      \end{equation}
   \item The classic Zalesak limiting strategy starts by separating the
      negative and positive fluxes:
      \begin{equation}
         Q^-_i \leq \sum\limits_{j:\correctionfluxij<0} L\ij \correctionfluxij +
            \sum\limits_{j:\correctionfluxij>0} L\ij \correctionfluxij\leq Q^+_i \eqp
      \end{equation}
      The positive fluxes risk violating $Q_i^+$, and the negative fluxes risk
      violating $Q_i^-$.
   \item Zalesak's limiting coefficients assume that
      all positive fluxes into a node $i$ have the same limiting coefficient
      $L^+_i$ and similarly, negative fluxes have the same limiting coefficient
      $L^-_i$:
      \begin{equation}
        Q^-_i \leq L^-_i \correctionfluxletter^-_i
          + L^+_i \correctionfluxletter^+_i \leq Q^+_i \eqp
      \end{equation}
      where
      \begin{equation}
        \correctionfluxletter_i^- \equiv \sum\limits_{j:\correctionfluxij<0}
          \correctionfluxij \eqc \qquad
        \correctionfluxletter_i^+ \equiv \sum\limits_{j:\correctionfluxij>0}
          \correctionfluxij \eqp
      \end{equation}
   \item As a conservative bound for $L^+_i$, contributions from negative fluxes
      are ignored (pretending $L_i^-=0$), giving
      $L^+_i \leq \frac{Q_i^+}{\correctionfluxletter_i^+}$
      and similarly for $L^-_i$ and the lower bound.
   \item Then, recalling that limiting coefficients are not greater than unity:
      \begin{equation}
         L_i^\pm \equiv\left\{
            \begin{array}{l l}
               1 & \correctionfluxletter_i^\pm = 0\\
               \min\left(1,\frac{Q_i^\pm}{\correctionfluxletter_i^\pm}\right) &
                 \correctionfluxletter_i^\pm \ne 0
            \end{array}
            \right. \eqp
      \end{equation}
   \item However, to limit fluxes conservatively, limited correction fluxes must
      be equal and opposite:
      \begin{equation}
        L\ij \correctionfluxij = -L_{j,i}
          \MakeUppercase{\correctionfluxletter}_{j,i} \eqp
      \end{equation}
      Since $\correctionfluxij$ happens to be skew symmetric
      ($\MakeUppercase{\correctionfluxletter}_{j,i}=-\correctionfluxij$) due to the
      chosen flux decomposition, the limiting coefficients must be symmetric:
      $L_{j,i} = L\ij$.
   \item Thus when deciding the limiting coefficient $L\ij$ for a flux $\correctionfluxij$, 
      one must not only consider the bounds for $i$ but also the bounds for $j$.
      Specifically, a positive flux $\correctionfluxij$ risks violating $Q_i^+$ and $Q_j^-$.
      Putting everything together,
      \begin{equation}
         L\ij \equiv\left\{
            \begin{array}{l l}
               \min(L_i^+,L_j^-) & \correctionfluxij \geq 0\\
               \min(L_i^-,L_j^+) & \correctionfluxij < 0
            \end{array}
            \right. \eqp
      \end{equation}
\end{itemize}
