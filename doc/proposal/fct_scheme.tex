Recall that FCT defines antidiffusive correction fluxes from a low-order,
monotone scheme to a high-order scheme. Calling these fluxes
$\correctionfluxvector$, this gives
\begin{equation}
  \lumpedmassmatrix\frac{\solutionvector^{H,n+1}-\solutionvector^n}{\dt}
    + (\ssmatrix+\diffusionmatrix^L)\solutionvector^n = \ssrhs^n
    + \correctionfluxvector \eqp
\end{equation}
Subtracting the high-order scheme equation from this gives the
definition of $\correctionfluxvector$:
\begin{equation}
  \correctionfluxvector \equiv
    -(\consistentmassmatrix-\lumpedmassmatrix)
    \frac{\solutionvector^{H,n+1}-\solutionvector^n}{\dt}
    + (\diffusionmatrix^L-\diffusionmatrix^{H,n})\solutionvector^n \eqp
\end{equation}
Now it is necessary to decompose these fluxes into internodal fluxes
$\correctionfluxij$ such that $\sum_j\correctionfluxij=\correctionfluxletter_i$:
\begin{equation}
  \correctionfluxij = -M\ij^C\pr{
    \frac{\MakeUppercase{\solutionletter}^{H,n+1}_j
      -\MakeUppercase{\solutionletter}^n_j}{\Delta t}
    -\frac{\MakeUppercase{\solutionletter}^{H,n+1}_i
      -\MakeUppercase{\solutionletter}^n_i}{\Delta t}
  }
  + (D\ij^L-D\ij^{H,n})(\MakeUppercase{\solutionletter}^n_j
    -\MakeUppercase{\solutionletter}^n_i) \eqp
\end{equation}
Recall that the objective of FCT is to limit these antidiffusive
fluxes to enforce some physical bounds. For the scalar case, one can use
the discrete maximum principle bounds given by Equation \eqref{eq:dmp}.
The limitation is achieved by applying a limiting coefficient $L\ij$ to each
internodal flux $\correctionfluxij$:
\begin{equation}
  \lumpedmassmatrix\frac{\solutionvector^{n+1}-\solutionvector^n}{\dt}
    + \ssmatrix^L\solutionvector^n = \ssrhs
    + \bar{\correctionfluxvector} \eqc
\end{equation}
where $\bar{\correctionfluxvector}$ denotes the limited antidiffusion vector:
$\bar{\correctionfluxletter}_i\equiv\sum_j\limiterletter\ij\correctionfluxij$.
Each limiting coefficient is between zero and unity: $0\leq L\ij\leq 1$.
If for example, all $L\ij$ are zero, then the low-order scheme is reproduced;
conversely, if all $L\ij$ are one, then the high-order scheme is reproduced.

Now the definition for the limiting coefficients is given. Firstly,
the enforced bounds on the solution, given by Equation \eqref{eq:dmp}
are rearranged to give bounds on the limited flux sums:
\begin{subequations}
\begin{equation}
  Q^-_i \leq \sum\limits_j L\ij \correctionfluxij \leq Q^+_i \eqc
\end{equation}
\begin{equation}
  Q_i^\pm \equiv M_{i,i}^L\frac{W_i^\pm-U_i^n}{\Delta t}
    + \sum\limits_j A_{i,j}^L U_j^n - b_i \eqp
\end{equation}
\end{subequations}
The classic Zalesak limiter has the following definition\cite{zalesak}:
\begin{equation}
  \correctionfluxletter_i^- \equiv \sum\limits_{j:\correctionfluxij<0}
    \correctionfluxij \eqc \qquad
  \correctionfluxletter_i^+ \equiv \sum\limits_{j:\correctionfluxij>0}
    \correctionfluxij \eqp
\end{equation}
\begin{equation}
  L_i^\pm \equiv\left\{
    \begin{array}{l l}
      1 & \correctionfluxletter_i^\pm = 0\\
      \min\left(1,\frac{Q_i^\pm}{\correctionfluxletter_i^\pm}\right) &
      \correctionfluxletter_i^\pm \ne 0
    \end{array}
  \right. \eqp
\end{equation}
\begin{equation}
  L\ij \equiv\left\{
    \begin{array}{l l}
      \min(L_i^+,L_j^-) & \correctionfluxij \geq 0\\
      \min(L_i^-,L_j^+) & \correctionfluxij < 0
    \end{array}
  \right. \eqp
\end{equation}
Since the internodal antidiffusive fluxes are skew-symmetric, i.e.,
$\correctionfluxij=-\correctionfluxji$, and the limiting coefficients
are symmetric, i.e., $L_{i,j}=L_{j,i}$, the limited antidiffusion added
to the low-order scheme is conservative because
$\sum_i\sum_j L_{i,j}\correctionfluxij=0$.

