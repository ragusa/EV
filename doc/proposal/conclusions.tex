Preliminary results show that the developed entropy-based FCT scheme
is effective at preventing the formation of spurious oscillations and
negativities; however, even with discrete LED criterion satisfied, it is
possible to generate spurious plateaus. While the monotonicity property is
not guaranteed, positivity preservation is guaranteed.
For scalar conservation laws,
the theoretical understanding
is greater, and a discrete maximum principle can be proved and enforced
on the FCT solution; this has been demonstrated for the scalar radiation
transport model. Results indicate that Galerkin FCT may show superior
results to entropy-based FCT in some cases; however, in general it is
necessary to use the entropy-based scheme to ensure convergence to the
entropy solution.
Some convergence studies have been performed for the scalar case and
demonstrate the expected spatial convergence rates.

Goals that remain include improvement of the FCT scheme for systems;
the limitation procedure for systems is complicated by the fact that
limitation of one set of variables (such as the conservative variables)
does not necessarily imply the same constraints are imposed on
auxiliary quantities. Research remains to consider what limitation
procedure(s) will give satisfactory results for the shallow water
equations.
