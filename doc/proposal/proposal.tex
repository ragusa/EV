\documentclass[12pt]{article}
\usepackage[letterpaper]{geometry}
\geometry{verbose,tmargin=1in,bmargin=1in,lmargin=1in,rmargin=1in}
\usepackage{amsmath}
\usepackage{amssymb}
\usepackage{booktabs} % for different rule commands such as \toprule
\usepackage{graphicx} % for including graphics
\usepackage{subcaption}
\usepackage[hidelinks]{hyperref}

% create path commands
\newcommand{\refdir}{../dissertation}
\newcommand{\contentdir}{../dissertation/content}

% input user-defined commands
% required packages
\usepackage{xcolor}
\usepackage{stmaryrd} % jump brackets: \llbracket, \rrbracket

% create a provideenvironment command
\makeatletter
\def\provideenvironment{\@star@or@long\provide@environment}
\def\provide@environment#1{%
  \@ifundefined{#1}%
    {\def\reserved@a{\newenvironment{#1}}}%
    {\def\reserved@a{\renewenvironment{dummy@environ}}}%
  \reserved@a
}
\def\dummy@environ{}
\makeatother

% general
\newcommand{\x}{\mathbf{x}}
\newcommand{\qpoint}{\x_q}
\newcommand{\timevalue}{t}
\newcommand{\timestepsize}{\Delta\timevalue}
\newcommand{\dt}{\timestepsize}
\newcommand{\timeindex}{n}
\newcommand{\speed}{v}
\newcommand{\velocity}{\mathbf{\speed}}
\newcommand{\velocityx}{u}
\newcommand{\normalvectorletter}{n}
\newcommand{\normalvector}{\mathbf{\normalvectorletter}}
\newcommand{\normalx}{\normalvectorletter_x}
\newcommand{\normaly}{\normalvectorletter_y}
\newcommand{\ndimensions}{N_\text{dim}}
\newcommand{\ncomponents}{N_\text{comp}}
\newcommand{\ndofs}{N_\text{dof}}
\newcommand{\nnodes}{N_\text{node}}
\newcommand{\dofindex}{j}
\newcommand{\nodeindex}{k}
\newcommand{\componentindex}{m}
\newcommand{\transpose}{^{\text{T}}}

% schemes
\newcommand{\low}{L}
\newcommand{\high}{H}

% solution
\newcommand{\scalarsolution}{u}
\newcommand{\vectorsolution}{\mathbf{\scalarsolution}}
\newcommand{\approximate}[1]{\tilde{#1}}
\newcommand{\approximatescalarsolution}{\approximate{\scalarsolution}}
\newcommand{\approximatevectorsolution}{\approximate{\vectorsolution}}
\newcommand{\solutionletter}{U}
\newcommand{\solutionvector}{\mathbf{\solutionletter}}
\newcommand{\U}{\solutionvector}
\newcommand{\lowordersolution}[1][]{
  \ifthenelse{\equal{#1}{}}{\solutionvector^L}{\solutionvector^{L,#1}}}
\newcommand{\highordersolution}[1][]{
  \ifthenelse{\equal{#1}{}}{\solutionvector^H}{\solutionvector^{H,#1}}}

% sets
\newcommand{\faces}{\mathcal{F}}
\newcommand{\quadraturepoints}{\mathcal{Q}}

% domain and FEM
\newcommand{\domain}{\mathcal{D}}
\newcommand{\celldomain}[1][\cell]{\domain_#1}
\newcommand{\facedomain}{\domain}
\newcommand{\domainboundary}{\partial\domain}
\newcommand{\incomingdomainboundary}{\domainboundary^{\text{inc}}}
\newcommand{\cellindex}{K}
\newcommand{\cell}{K}
\newcommand{\celldiameter}{\Delta x}
\newcommand{\maxcelldiameter}{\Delta x_{\text{max}}}
\newcommand{\volume}{V}
\newcommand{\dvolume}{\,d\volume}
\newcommand{\area}{A}
\newcommand{\darea}{\,d\area}
\newcommand{\testfunction}{\varphi}
\newcommand{\vectortestfunctionscalar}{\Phi}
\newcommand{\vectortestfunction}{\mathbf{\vectortestfunctionscalar}}
\newcommand{\support}{S}
\newcommand{\maxdof}{N}
\newcommand{\interpolant}{\Pi}

% local viscous bilinear form
\newcommand{\localvisc}{b}
\newcommand{\localviscbilinearform}[3]{\localvisc_#1(\testfunction_#2, \testfunction_#3)}
\newcommand{\cellvolume}{|\celldomain|}
\newcommand{\cardinality}[1][]{\ifthenelse{\equal{#1}{}}{n_\cell}{n_#1}}
\newcommand{\cardsystem}{\bar{n}}
\newcommand{\indices}{\mathcal{I}}
\newcommand{\indicesnode}{\indices^{\text{node}}_\cell}
\newcommand{\indicescell}[1][]{\ifthenelse{\equal{#1}{}}{\indices_{\cell}}
  {\indices_{#1}}}

% entropy viscosity
\newcommand{\entropy}{\eta}
\newcommand{\entropyflux}{\mathbf{\consfluxletter}^\eta}
\newcommand{\entropyjump}{\mathcal{J}}
\newcommand{\entropyresidual}{\mathcal{R}}
\newcommand{\entropyresidualcoef}{c_\entropyresidual}
\newcommand{\entropyjumpcoef}{c_\entropyjump}
\newcommand{\entropynormalization}{\hat{\entropy}}

% conservation law
\newcommand{\consfluxletter}{f}
\newcommand{\consflux}{\mathbf{\consfluxletter}}
\newcommand{\consfluxsystem}{\mathbf{\MakeUppercase{\consfluxletter}}}
\newcommand{\consfluxscalar}[1][\scalarsolution]{\mathbf{\consfluxletter}(#1)}
\newcommand{\consfluxvector}{\mathbf{\MakeUppercase{\consfluxletter}}}
\newcommand{\consfluxinterpolant}{\mathrm{F}}
\newcommand{\conssource}{\mathbf{s}}

% viscosity
\newcommand{\viscosity}{\nu}
\newcommand{\cellviscosity}{\viscosity_\cellindex}
\newcommand{\lowordercellviscosity}[1][]{
  \ifthenelse{\equal{#1}{}}{\cellviscosity^L}
  {\cellviscosity^{L,#1}}}
\newcommand{\highordercellviscosity}[1][]{
  \ifthenelse{\equal{#1}{}}{\cellviscosity^H}
  {\cellviscosity^{H,#1}}}
\newcommand{\entropycellviscosity}[1][]{
  \ifthenelse{\equal{#1}{}}{\cellviscosity^\entropy}
  {\cellviscosity^{\entropy,#1}}}

% viscous fluxes
\newcommand{\viscstring}{\text{visc}}
\newcommand{\viscflux}[1]{\mathbf{\consfluxletter}^{\viscstring,#1}}
\newcommand{\viscconsfluxvector}
  {\mathbf{\MakeUppercase{\consfluxletter}}^\viscstring
  (\vectorsolution,\viscosity)}

% mass matrix
\newcommand{\massmatrixletter}{M}
\newcommand{\massmatrix}{\mathbf{\massmatrixletter}}
\newcommand{\M}{\massmatrix}
\newcommand{\consistentmassmatrix}{\massmatrix^C}
\newcommand{\consistentmassentry}{\massmatrixletter^C_{i,j}}
\newcommand{\lumpedmassmatrix}{\massmatrix^L}
\newcommand{\lumpedmassentry}{\massmatrixletter^L_{i,i}}

% gradient matrix (for conservation law systems)
\newcommand{\gradientmatrixletter}{c}
\newcommand{\gradientmatrix}{\mathbf{\MakeUppercase{\gradientmatrixletter}}}
\newcommand{\gradiententry}{\mathbf{\gradientmatrixletter}\ij}

% steady-state system matrix and rhs
\newcommand{\ssmatrixletter}{A}
\newcommand{\ssmatrix}[1][]{
  \ifthenelse{\equal{#1}{}}
  {\mathbf{\ssmatrixletter}}
  {\mathbf{\ssmatrixletter}^#1}}
\newcommand{\A}{\ssmatrix}
\newcommand{\loworderssmatrix}[1][]{
  \ifthenelse{\equal{#1}{}}
  {\ssmatrix^L}
  {\ssmatrix^{L,#1}}}
\newcommand{\highorderssmatrix}[1][]{
  \ifthenelse{\equal{#1}{}}
  {\ssmatrix^H}
  {\ssmatrix^{H,#1}}}
\newcommand{\ssrhsletter}{b}
\newcommand{\ssrhs}[1][]{
  \ifthenelse{\equal{#1}{}}
  {\mathbf{\ssrhsletter}}
  {\mathbf{\ssrhsletter}^#1}}
\renewcommand{\b}{\ssrhs}
\newcommand{\ssresletter}{r}
\newcommand{\ssres}{\mathbf{\ssresletter}}
\renewcommand{\r}{\ssres}
\newcommand{\B}{\mathbf{B}}
\newcommand{\s}{\mathbf{s}}

% diffusion matrix
\newcommand{\diffusionmatrixletter}{D}
\newcommand{\diffusionmatrix}[1][]{
  \ifthenelse{\equal{#1}{}}
  {\mathbf{\diffusionmatrixletter}}
  {\mathbf{\diffusionmatrixletter}^#1}}
\newcommand{\D}{\diffusionmatrix}
\newcommand{\loworderdiffusionmatrix}[1][]{
  \ifthenelse{\equal{#1}{}}
  {\diffusionmatrix^L}
  {\diffusionmatrix^{L,#1}}}
\newcommand{\highorderdiffusionmatrix}[1][]{
  \ifthenelse{\equal{#1}{}}
  {\diffusionmatrix^H}
  {\diffusionmatrix^{H,#1}}}

% Runge-Kutta
\newcommand{\RKstagesolution}{\hat{\mathbf{\solutionletter}}}
\newcommand{\RKintermediatesolution}{\tilde{\mathbf{\solutionletter}}}
\newcommand{\RKoldsolutioncoef}{\alpha}
\newcommand{\RKstagesolutioncoef}{\beta}
\newcommand{\RKtimecoef}{c}
\newcommand{\RKstagetime}{\hat{\timevalue}}
\newcommand{\RKnstages}{s}

% FCT
\newcommand{\DMPbound}{W}
\newcommand{\DMPboundsi}{\DMPbound^\pm_i}
\newcommand{\limitedfluxbound}{Q}
\newcommand{\limitedfluxboundsi}{\limitedfluxbound^\pm_i}
\newcommand{\limiterletter}{L}
\newcommand{\limitermatrix}{\mathbf{\limiterletter}}
\newcommand{\correctionfluxletter}{p}
\newcommand{\correctionfluxvector}{\mathbf{\correctionfluxletter}}
\newcommand{\correctionfluxentry}{\MakeUppercase{\correctionfluxletter}}
\newcommand{\correctionfluxij}{\correctionfluxentry_{i,j}}
\newcommand{\correctionfluxji}{\correctionfluxentry_{j,i}}
\newcommand{\correctionfluxmatrix}{\mathbf{\MakeUppercase{\correctionfluxletter}}}
\newcommand{\correctionfluxsumsi}{\MakeUppercase{\correctionfluxletter}^\pm_i}
\newcommand{\limitedfluxsum}{\limitermatrix\cdot\correctionfluxmatrix}
\newcommand{\limitedfluxsumi}{\sumj\limiterletter\ij
  \MakeUppercase{\correctionfluxletter}\ij}
\newcommand{\F}{\correctionfluxmatrix}
\newcommand{\LF}{\limitermatrix\cdot\correctionfluxmatrix}

% radiation transport
\newcommand{\angularflux}{\psi}
\newcommand{\scalarflux}{\phi}
\newcommand{\speedoflight}{c}
\newcommand{\totalcrosssection}{\Sigma_t}
\newcommand{\reactioncoef}{\sigma}
\newcommand{\directionvector}{\mathbf{\Omega}}
\newcommand{\scalarsource}{q}
\newcommand{\radiationsource}{Q}

% Euler equations
\newcommand{\density}{\rho}
\newcommand{\totalenergy}{E}
\newcommand{\momentum}{\mathbf{m}}
\newcommand{\pressure}{p}
\newcommand{\gasconstant}{\gamma}
\newcommand{\identity}{\mathbf{I}}

% shallow water equations
\newcommand{\height}{h}
\newcommand{\heightmomentumletter}{q}
\newcommand{\heightmomentum}{\mathbf{\heightmomentumletter}}
\newcommand{\heightmomentumx}{\heightmomentumletter_x}
\newcommand{\heightmomentumy}{\heightmomentumletter_y}
\newcommand{\heightmomentumd}{\heightmomentumletter_d}
\newcommand{\dischargex}{\heightmomentumletter}
\newcommand{\bathymetry}{b}
\newcommand{\waterlevel}{w}
\newcommand{\gravity}{g}
\newcommand{\speedofsound}{a}
\newcommand{\froude}{\mathrm{Fr}}

% Riemann solvers
\newcommand{\shockspeed}{S}
\newcommand{\eigenvalue}{\lambda}
\newcommand{\eigenvaluematrix}{\mathbf{\Lambda}}
\newcommand{\eigenvector}{\mathbf{k}}
\newcommand{\eigenvectormatrix}{\mathbf{K}}
\newcommand{\jacobianx}{\mathbf{A}}
\newcommand{\characteristicsolution}{\mathbf{w}}
\newcommand{\wavespeed}{\eigenvalue}
\newcommand{\maxwavespeed}[1][]{
  \ifthenelse{\equal{#1}{}}{\wavespeed^{\text{max}}}{\wavespeed^{\text{max},#1}}}
\newcommand{\wavestrength}{\mathcal{W}}

%==============================================================================
% colors
\colorlet{lightBlue}{blue!20!white}
\colorlet{lightGreen}{green!20!white}

% indexing
\renewcommand{\ij}{_{i,j}}
\newcommand{\ji}{_{j,i}}
\newcommand{\kl}{_{k,\ell}}
\newcommand{\lk}{_{\ell,k}}
\newcommand{\nodei}{_{\nodeindex(i)}}
\newcommand{\nodej}{_{\nodeindex(j)}}
\newcommand{\nodeij}{_{\nodeindex(i),\nodeindex(j)}}
\newcommand{\nodeji}{_{\nodeindex(j),\nodeindex(i)}}
\newcommand{\nodequantity}[1]{\underline{#1}}

% sums and integrals
\newcommand{\sumj}{\sum\limits_j}
\newcommand{\sumjnoti}{\sum\limits_{j\ne i}}
\newcommand{\sumKSi}{\sum\limits_{\cell:\celldomain\subset\support_i}}
\newcommand{\sumKSij}[1][\cell]
  {\sum\limits_{#1:\celldomain[#1]\subset\support_{i,j}}}
\newcommand{\sumallcells}{\sum\limits_{\cell}}
\newcommand{\intdomain}[1]{\int\limits_\domain #1 \,\dvolume}
\newcommand{\intboundary}[1]{\int\limits_{\domainboundary} #1 \,d\area}
\newcommand{\intSi}{\int\limits_{\support_i}}
\newcommand{\intSij}{\int\limits_{\support_{i,j}}}

% math
\newcommand{\ltwonorm}[1]{\left\|#1\right\|_{L^2}} % L-2 norm

% BC
\newcommand{\interior}{_{\text{in}}}
\newcommand{\BC}{_{\text{BC}}}

% common fractions
\newcommand{\half}{\frac{1}{2}}
\newcommand{\fourth}{\frac{1}{4}}

% derivatives
\newcommand{\dd}[2]{\frac{d #1}{d #2}}               % ordinary derivative
\newcommand{\pd}[2]{\frac{\partial #1}{\partial #2}} % partial derivative
\newcommand{\ppt}[1]{\pd{#1}{t}}                     % partial d/dt
\newcommand{\ddt}[1]{\frac{d#1}{dt}}                 % ordinary d/dt

% typesetting
\newcommand{\pr}[1]{\left(#1\right)} % parentheses
\newcommand{\sq}[1]{\left[#1\right]} % square brackets
\newcommand{\jumpbrackets}[1]{\left\llbracket#1\right\rrbracket} % jump brackets
\newcommand{\tab}{\hspace*{0.5cm}}   % tab for verbatim evironments
\newcommand{\eqp}{\,.} % equation period
\newcommand{\eqc}{\,,} % equation comma

% miscellaneous
\newcommand{\xt}{\pr{\x,\timevalue}}
\newcommand{\divergence}{\nabla\cdot}
\newcommand{\unitvector}[1]{\hat{\mathbf{e}}_{#1}}

% command to highlight term in equation
\newcommand{\highlightblue}[1]{
  \colorbox{lightBlue}{$\displaystyle#1$}}
\newcommand{\highlightgreen}[1]{
  \colorbox{lightGreen}{$\displaystyle#1$}}

% QED symbol command
\providecommand{\qed}{\nobreak \ifvmode \relax \else
  \ifdim\lastskip<1.5em \hskip-\lastskip
  \hskip1.5em plus0em minus0.5em \fi \nobreak
  \vrule height0.75em width0.5em depth0.25em\fi}

% math environments
\provideenvironment{proof}[1][Proof]{\begin{trivlist}
\item[\hskip \labelsep {\bfseries #1}]}{\end{trivlist}}
\provideenvironment{example}[1][Example]{\begin{trivlist}
\item[\hskip \labelsep {\bfseries #1}]}{\end{trivlist}}
\newenvironment{remark}[1][Remark]{\begin{trivlist}
\item[\hskip \labelsep {\bfseries #1}]}{\end{trivlist}}

% table environment
% #1 = caption
% #2 = label
% #3 = table format (columns)
% #4 = header row
\newenvironment{mytable}[4]
  {\begin{table}[htb]\caption{#1\label{tab:#2}}\begin{center}
    \begin{tabular}
    {#3}\hline #4\\\hline}
  {\hline\end{tabular}\end{center}\end{table}}

% references commands
%\newcommand{\refsec}[1]{, \S#1}
\newcommand{\refsec}[1]{}

% algorithm shortcuts
\newcommand{\objective}{\phi}
\newcommand{\hmin}{\height_{\text{min}}}
\newcommand{\hmax}{\height_{\text{max}}}
\newcommand{\hlow}{\check{\height}}
\newcommand{\hhigh}{\hat{\height}}
\newcommand{\hrarefaction}{\tilde{\height}_*}
\newcommand{\tol}{\epsilon}
\newcommand{\minwavespeed}{\wavespeed_{\text{min}}}
\newcommand{\lowwavespeedone}{\check{\wavespeed}_1}
\newcommand{\highwavespeedone}{\hat{\wavespeed}_1}
\newcommand{\lowwavespeedtwo}{\check{\wavespeed}_2}
\newcommand{\highwavespeedtwo}{\hat{\wavespeed}_2}
\newcommand{\hinterplow}{\height_d}
\newcommand{\hinterphigh}{\height_u}

% checkboxes
\usepackage{amssymb}
\usepackage{xcolor}
\definecolor{myorangeheavy}{RGB}{255,150,0}
\newcommand{\checked}{
  \makebox[0pt][l]{$\square$}\raisebox{.15ex}
  {\hspace{0.1em}\textcolor{myorangeheavy}{$\checkmark$}}}
\newcommand{\unchecked}{
  \makebox[0pt][l]{$\square$}\hspace{0.9em}}

% highlighting
\newcommand{\hlorange}[1]{\textcolor{myorangeheavy}{#1}}

% invariant domains
\newcommand{\invariantset}{A}
\newcommand{\admissibleset}{\mathcal{A}}
\newcommand{\discreteprocess}{S}
\newcommand{\convexcoefficient}{a}
\newcommand{\convexelement}{\mathbf{b}}

% spaces
\newcommand{\realspace}[1][]{
  \ifthenelse{\equal{#1}{}}{\mathbb{R}}{\mathbb{R}^{#1}}}


%==============================================================================
% Theorem environments for dissertation
%==============================================================================
\usepackage{ifthen}

\newtheorem{mytheorem}{Theorem}[section]
\newtheorem{mylemma}{Lemma}[section]
\newtheorem{mycorollary}{Corollary}[section]
\newtheorem{mydefinition}{Definition}[section]
\newtheorem{myproposition}{Proposition}[section]

\newenvironment{theorem}[2]
   {\ifthenelse{\equal{#1}{}}{\begin{mytheorem}}{\begin{mytheorem}[#1]}
   \ifthenelse{\equal{#2}{}}{}{\label{#2}}}
   {\end{mytheorem}}
\newenvironment{lemma}[2]
   {\ifthenelse{\equal{#1}{}}{\begin{mylemma}}{\begin{mylemma}[#1]}
   \ifthenelse{\equal{#2}{}}{}{\label{#2}}}
   {\end{mylemma}}
\newenvironment{corollary}[2]
   {\ifthenelse{\equal{#1}{}}{\begin{mycorollary}}{\begin{mycorollary}[#1]}
   \ifthenelse{\equal{#2}{}}{}{\label{#2}}}
   {\end{mycorollary}}
\newenvironment{definition}[2]
   {\ifthenelse{\equal{#1}{}}{\begin{mydefinition}}{\begin{mydefinition}[#1]}
   \ifthenelse{\equal{#2}{}}{}{\label{#2}}}
   {\end{mydefinition}}
\newenvironment{proposition}[2]
   {\ifthenelse{\equal{#1}{}}{\begin{myproposition}}{\begin{myproposition}[#1]}
   \ifthenelse{\equal{#2}{}}{}{\label{#2}}}
   {\end{myproposition}}



%###############################################################################
% Document
%###############################################################################
\begin{document}

\begin{center}
  {\large
    Application of the Entropy Viscosity Method and the Flux-Corrected Transport
    Algorithm to Scalar Transport Equations and the Shallow Water Equations
  }

  {\scriptsize
    Dissertation Proposal
  }

  \vspace{1em}

  Joshua E. Hansel
\end{center}

%###############################################################################
\section{Introduction}
%###############################################################################
The solution of conservation law equations such as the neutron transport
equation presents a number of unique challenges; in the vicinity of strong
gradients and discontinuities, numerical solutions are prone to spurious
oscillations that may generate unphysical values. For example, physically
non-negative quantities such as scalar flux or angular flux may have negative
numerical solution values if care is not taken in the numerical scheme.
These negativities are not only undesirable because they are physically
incorrect, but also because often numerical algorithms completely break
down, causing simulations to abort, or worse, altering results without
discovery that unphysical values were encountered. The consequences of
these results may lead to poor design choices, which can pose serious safety
risks when a design is implemented.

These issues of spurious oscillations and negativities are a well-known
phenomenon in the simulation of hyperbolic partial differential equations,
for example, linear advection, Burger's equation, the inviscid Euler equations
of gas dynamics, and the shallow water (or Saint-Venant) equations.
These PDEs result from taking the differential form of the corresponding integral
conservation law equations; however, the differential forms of these equations
are not physically correct - they do not hold in the differential case
because they break down in the presence of a discontinuity.
Moreover, some physics is omitted in these models, such as the
lack of an entropy condition.
The mathematical formulations for these problems
do not always guarantee a unique solution; this is a manifestation of the
neglect of physics in the underlying PDE model.

Attempts to address these issues have taken a number
of different approaches for different discretization schemes.


The organization of this proposal is as follows.
Section \ref{sec:problem_formulation} gives and discusses the physical models,
problem formulation, and discretization.
Section \ref{sec:methodology} discusses the methodology used, including
entropy-based artificial dissipation and the flux-corrected transport
algorithm.
Section \ref{sec:results} gives a sample of preliminary results.
Section \ref{sec:goals} discusses the goals of this research.
Section \ref{sec:conclusions} gives a summary and conclusions of this proposal.

%###############################################################################
\section{Problem Formulation\label{sec:problem_formulation}}
%###############################################################################
%===============================================================================
\subsection{Physical Models}
%===============================================================================
This section introduces the PDEs under consideration. Section
\ref{sec:scalar_model} introduces the scalar hyperbolic PDE model, and
Section \ref{sec:sw_model} introduces the shallow water equations.
%-------------------------------------------------------------------------------
\subsubsection{Scalar Transport\label{sec:scalar_model}}
%-------------------------------------------------------------------------------
The primary application of this research is radiation transport;
however, most of the analysis performed is valid
for any conservation law of the following form:
\begin{equation}\label{eq:scalar_transport}
   \ppt{\scalarsolution} + \divergence\consfluxscalar
   + \reactioncoef\xt \scalarsolution\xt = \scalarsource\xt \eqc
\end{equation}
where $\scalarsolution\xt$ is a general scalar conserved quantity at position
$\x$ and time $\timevalue$, $\consfluxscalar$ is a general flux
function,
$\reactioncoef\xt$ is a reaction term, and $\scalarsource\xt$ is a source
term. This notation will be used throughout this document to keep
the analysis as general as possible; radiation transport notation
will only be adopted when assumptions are needed.

A Scalar radiation transport equation,
\begin{equation}\label{eq:rad_transport}
  \frac{1}{\speed}\ppt{\angularflux} + \directionvector\cdot\nabla\angularflux\xt
  + \totalcrosssection(\x)\angularflux\xt = \radiationsource\xt \eqc
\end{equation}
fits the conservation law model of Equation \eqref{eq:scalar_transport} by
making the following substitutions:
\[
  \consfluxscalar\rightarrow\velocity\scalarsolution
  \eqc \quad
  \velocity\rightarrow\speed\directionvector
  \eqc \quad
  \scalarsolution\rightarrow\angularflux
  \eqc \quad
  \reactioncoef\rightarrow\speed\totalcrosssection
  \eqc \quad
  \scalarsource\rightarrow\speed\radiationsource
  \eqc
\]
where $\angularflux\xt$ is the angular flux in direction $\directionvector$,
$\speed$ is the transport speed, $\totalcrosssection(\x)$
is the macroscopic total cross-section, and $\radiationsource\xt$ is the
total source (extraneous plus scattering).

To complete the problem formulation, boundary
conditions must be provided, as well as initial conditions if the
problem is transient:
\begin{equation}
   \scalarsolution(\x,0) = \scalarsolution^0(\x)
   \quad \forall \x\in\domain \eqp
\end{equation}
Boundary conditions will depend on the chosen conservation law and
the particular problem. 
For linear transport a well-posed problem can be completed with an incoming flux
boundary condition:
\begin{equation}
   \scalarsolution\xt = \scalarsolution^{inc}\xt \quad \forall \x
   \in \domainboundary^-,
     \quad \domainboundary^- = \{\x\in\domainboundary:
     \velocity\cdot\normalvector(\x)<0\} \eqp
\end{equation}
For nonlinear conservation laws, care must be taken to ensure that the
boundary conditions used result in a well-posed problem.

%-------------------------------------------------------------------------------
\subsubsection{The Shallow Water Equations\label{sec:sw_model}}
%-------------------------------------------------------------------------------
The shallow water equations, also known as the Saint-Venant equations, are an
approximation of conservation of mass and momentum equations applied to free
surface flows, which assume the fluid to be incompressible, non-viscous, and
non-heat-conducting\cite{toro2009}. The shallow water equations are derived
by making the additional
approximation that the vertical component of acceleration can be neglected due to
horizontal length scales being much greater than the depth length
scale and then depth-integrating the conservation equations:
\cite{toro2009}\cite{leveque2002}\cite{fjordholm2011}:
\begin{equation}\label{eq:shallow_water_equations}
\begin{gathered}
  \ppt{\vectorsolution} + \nabla\cdot\consfluxvector(\vectorsolution)
  = \conssource(\vectorsolution) \eqc
\\
  \vectorsolution
    = \left[\begin{array}{c}
        \height\\
        \heightmomentumx\\
        \heightmomentumy
      \end{array}\right]
  \eqc\quad
  \consfluxvector(\vectorsolution)
  = \left[\begin{array}{c c}
      \heightmomentumx & \heightmomentumy\\
      \frac{\heightmomentumx^2}{\height} + \half\gravity\height^2
        & \frac{\heightmomentumx\heightmomentumy}{\height}\\
      \frac{\heightmomentumx\heightmomentumy}{\height}
        & \frac{\heightmomentumy^2}{\height} + \half\gravity\height^2\\
    \end{array}\right]
  \eqc\quad
  \conssource(\vectorsolution)
  = \left[\begin{array}{c}
      0\\
     -\gravity\height\pd{\bathymetry}{x}\\
     -\gravity\height\pd{\bathymetry}{y}\\
    \end{array}\right]
  \eqc
\end{gathered}
\end{equation}
written more concisely as
\[
  \vectorsolution
    = \left[\begin{array}{c}\height\\\heightmomentum\end{array}\right]
  \eqc\quad
  \consfluxvector(\vectorsolution)
  = \left[\begin{array}{c}\heightmomentum\\
      \frac{\heightmomentum\otimes\heightmomentum}{\height}
      + \half\gravity\height^2\identity
    \end{array}\right]
  \eqc\quad
  \conssource(\vectorsolution)
  = \left[\begin{array}{c}0\\-\gravity\height\nabla\bathymetry\end{array}
    \right] \eqc
\]
where $\height$ is the height of the water, which plays the role of density
in the continuity equation, $\heightmomentum=\height\velocity$ is sometimes
referred to as \emph{discharge} and plays the role of momentum (hereafter,
$\heightmomentum$ will usually just be referred to as ``momentum''),
$\velocity$ is velocity, $\gravity$
is acceleration due to gravity, and $\bathymetry$ is the topography of the
bottom terrain of the fluid body, hereafter referred to as the \emph{bathymetry}
function.
Note that the shallow water equations are only valid in 1-D or 2-D, not 3-D,
since they are depth-integrated equations.

The shallow water equations (SWE) are a popular model for flows in lakes, rivers,
irrigation channels, and ocean shores, and thus are of great interest
in hydrology, oceanography, and climate modeling\cite{bernetti2008}
\cite{fjordholm2011}.

Initial conditions are included if the problem is transient:
\begin{equation}
   \vectorsolution(\x,0) = \vectorsolution^0(\x)
   \quad \forall \x\in\domain \eqp
\end{equation}
To complete the problem formulation, boundary
conditions must be provided, some examples being
Dirichlet boundary conditions, open boundary conditions,
wall boundary conditions,
etc. One must be careful with specifying boundary conditions to have
a well-posed problem for hyperbolic systems.
In general a characteristic analysis is required; there is a large body of research
addressing this area alone. For simplicity, problems in this work are
chosen such that initial data never reaches the boundary
or boundary conditions are implemented as natural conditions
rather than using the method of characteristics.



%===============================================================================
\subsection{Discretization}
%===============================================================================
This section gives the spatial and temporal discretizations. Section
\ref{sec:spatial_discretization_scalar} gives the spatial discretization
for the scalar model given in Section \ref{sec:scalar_model}, and
Section \ref{sec:spatial_discretization_sw} gives the spatial discretization
for the shallow water equations, introduced in Section \ref{sec:sw_model}.
In both cases, the continuous Galerkin finite element method (FEM) is employed
with basis functions from a piecewise linear polynomial space.
Section \ref{sec:temporal_discretization} gives the temporal discretizations
used.

%-------------------------------------------------------------------------------
\subsubsection{Spatial Discretization of the Scalar Hyperbolic PDE Model
\label{sec:spatial_discretization_scalar}}
%-------------------------------------------------------------------------------
The continuous Galerkin (CG) finite element method (FEM) is used for spatial
discretization.  The numerical solution is thus approximated using an
expansion of basis functions $\testfunction_j(\x)$:
\begin{equation}
  \approximatescalarsolution\xt = \sumj \solutionletter_j(\timevalue)
  \testfunction_j(\x) \eqc
\end{equation}
where the coefficients $\solutionletter_j(\timevalue)$ are the basis function
expansion coefficients at time $\timevalue$. Substituting the approximate
solution into Equation \eqref{eq:scalar_transport} and testing with basis
function $\testfunction_i(\x)$ gives
\begin{equation}
   \intSi\ppt{\approximatescalarsolution}\testfunction_i(\x) \dvolume
      + \intSi\left(\divergence\consfluxscalar[\approximatescalarsolution]
      + \reactioncoef(\x)\approximatescalarsolution\xt\right)
      \testfunction_i(\x) \dvolume
      = \intSi \scalarsource\xt \testfunction_i(\x) \dvolume \eqc
\end{equation}
where $\support_i$ is the support of $\testfunction_i(\x)$. If the flux
function $\consfluxscalar$ is linear with respect to $\scalarsolution$, i.e.,
$\consfluxscalar = \velocity\scalarsolution$ for some uniform velocity field
$\velocity$, then the system to be solved is linear:
\begin{equation}\label{eq:semidiscrete}
  \consistentmassmatrix\ddt{\solutionvector}+\ssmatrix\solutionvector(t)
  = \ssrhs(\timevalue) \eqc
\end{equation}
with the elements of $\ssmatrix$ being the following:
\begin{equation}\label{eq:Aij}
  \ssmatrixletter\ij \equiv \intSij\left(
  \velocity\cdot\nabla\testfunction_j(\x) +
  \reactioncoef(\x)\testfunction_j(\x)\right)\testfunction_i(\x) \dvolume \eqc
\end{equation}
where $\support\ij$ is the dual support of $\testfunction_i(\x)$ and
$\testfunction_j(\x)$.
If the flux function $\consfluxscalar$ is nonlinear, then the system is
nonlinear, but it may be expressed in a quasilinear form:
\begin{equation}\label{eq:semi_quasilinear}
   \consistentmassmatrix\ddt{\solutionvector}
   + \ssmatrix(\approximatescalarsolution)\solutionvector(\timevalue)
   = \ssrhs(\timevalue) \eqc
\end{equation}
where $\approximatescalarsolution\xt$ is the numerical solution, and the
quasilinear matrix (i.e., the Jacobian matrix) entries are
\begin{equation}\label{eq:Aij_nonlinear}
  \ssmatrixletter\ij(\approximatescalarsolution) \equiv \intSij\left(
  \mathbf{\consfluxletter}'(\approximatescalarsolution)\cdot\nabla
  \testfunction_j(\x) +
  \reactioncoef(\x)\testfunction_j(\x)\right)
  \testfunction_i(\x) \dvolume \eqp
\end{equation}
The elements of $\ssrhs(\timevalue)$ are
\begin{equation}
  \ssrhsletter_i(\timevalue) \equiv \intSi \scalarsource\xt\testfunction_i(\x)
  \dvolume \eqp
\end{equation}
$\consistentmassmatrix$ is the consistent mass matrix, which has the entries
\begin{equation}\label{eq:massmatrix}
  \massmatrixletter^C\ij \equiv \intSij
  \testfunction_j(\x)\testfunction_i(\x) \dvolume \eqp
\end{equation}
Similarly, for the linear steady-state case, the linear system is
\begin{equation}
  \ssmatrix\solutionvector = \ssrhs \eqc
\end{equation}
or for the nonlinear case,
\begin{equation}
  \ssmatrix(\approximatescalarsolution)\solutionvector = \ssrhs \eqp
\end{equation}


%-------------------------------------------------------------------------------
\subsubsection{Spatial Discretization of the Shallow Water Equations
\label{sec:spatial_discretization_sw}}
%-------------------------------------------------------------------------------
The FEM basis functions for each solution component are chosen to be identical,
so one may take the viewpoint that degrees of freedom are vector-valued
and that the test functions are scalar:
\begin{equation}
  \approximatevectorsolution\xt = \sumj \solutionvector_j(\timevalue)
  \testfunction_j(\x) \eqc
\end{equation}
where $\solutionvector_j(\timevalue)$ is a vector of the degrees of freedom of all
solution components at a node $j$:
$\solutionvector_j(\timevalue)=[\height_j(\timevalue),
\heightmomentum_j(\timevalue)]\transpose$. Note that in a practical
implementation, the basis functions would be viewed as vector-valued, with
degrees of freedom being scalar.

As opposed to the scalar conservation law case, the vector case takes a
group finite element approach: the conservation law flux is interpolated
using the flux evaluated at nodes:
\begin{equation}
  \consfluxvector(\vectorsolution\xt) \rightarrow
  \Pi\consfluxvector(\vectorsolution\xt) 
    \equiv \sumj\testfunction_j(\x)\consfluxvector
      (\vectorsolution(\x_j,\timevalue))
  \eqc
\end{equation}
where hereafter the nodal flux values used as interpolation values,
$\consfluxvector(\vectorsolution(\x_j,\timevalue))$,
will be denoted as $\consfluxinterpolant_j(\timevalue)$. This is done
as a step for proving the domain-invariance of the low-order scheme,
which is introduced in Section \ref{sec:low_order_scheme_system}.

Rearranging Equation \eqref{eq:shallow_water_equations},
substituting the approximate FEM
solution and conservation law flux,
testing with a test function $\testfunction_i$,
and integrating by parts gives
\begin{equation}
  \sumj\consistentmassentry
    \ddt{\solutionvector_j}
    + \sum_j\gradiententry\cdot\consfluxinterpolant_j(t)
    = \ssrhs_i(t) \eqc
\end{equation}
where
\begin{equation}
  \gradiententry \equiv
    \intSij\testfunction_i(\x)
      \nabla\testfunction_j(\x) d\volume
  \eqp
\end{equation}
For the 2-D shallow water equations,
\begin{equation}
  \consfluxinterpolant_j(\timevalue) = \sq{\begin{array}{c}
    \approximate{\heightmomentum}_j \\
    \frac{\approximate{\heightmomentumletter}_{x,j}}{\approximate{\height}_j}
      \approximate{\heightmomentum}_j
      + \frac{1}{2}\gravity\approximate{\height}_j^2\unitvector{x} \\
    \frac{\approximate{\heightmomentumletter}_{y,j}}{\approximate{\height}_j}
      \approximate{\heightmomentum}_j
      + \frac{1}{2}\gravity\approximate{\height}_j^2\unitvector{y}
  \end{array}} \eqc \qquad
  \ssrhs_i(t) = \sq{\begin{array}{c}
    0\\
    - \int_{\support_i}\testfunction_i(\x)
      \gravity\approximate{\height}\xt\partial_x\bathymetry\dvolume \\
    - \int_{\support_i}\testfunction_i(\x)
      \gravity\approximate{\height}\xt\partial_y\bathymetry\dvolume
  \end{array}} \eqp
\end{equation}


%-------------------------------------------------------------------------------
\subsubsection{Temporal discretization
\label{sec:temporal_discretization}}
%-------------------------------------------------------------------------------
The remainder of this proposal assumes that forward Euler (FE) is used for the
temporal discretization. The methodology presented in this research is best
understood using FE; however, a class of explicit temporal discretizations
called Strong-Stability-Preserving Runge-Kutta (SSPRK) methods are expressible
as a combination of FE steps, so the methods given in this research are
applicable to higher-order temporal discretizations.
Implicit temporal discretizations such as $\theta$ methods like backward
Euler and Crank-Nicolson are also an option; however, in these cases, the
imposed bounds are implicit as well, so nonlinear solves are required.

The FE discretization for a general system $\mathbf{M}d\solutionvector/dt=
\mathbf{r}(\solutionvector(t),t)$ is the following:
\begin{equation}
  \mathbf{M}\frac{\solutionvector^{n+1}-\solutionvector^n}{\dt} =
    \mathbf{r}(\solutionvector^n,t^n) \eqc
\end{equation}
where $\dt$ is the time step size.


%###############################################################################
\section{Methodology\label{sec:methodology}}
%###############################################################################
This section gives the methodology to be used in this research. As stated in
the introduction, both scalar and systems of hyperbolic PDEs are to be
considered. However, for brevity, this section gives only the scalar
methodology, since the methodology is very similar for systems. As stated in the
introduction, the FCT algorithm combines a low-order scheme and a high-order
scheme.  Thus this section is organized as follows.  Section
\ref{sec:low_order_scheme} gives the low-order scheme, Section
\ref{sec:high_order_scheme} gives the high-order scheme, and Section
\ref{sec:fct_scheme} gives the FCT scheme that blends the two schemes via the
FCT algorithm.
%===============================================================================
\subsection{Low-Order Scheme\label{sec:low_order_scheme}}
%===============================================================================
A \underline{\bf monotonicity-preserving, positivity-preserving low-order} scheme
is defined by lumping the mass matrix and adding a low-order diffusion
operator:

\begin{equation}\label{eq:loworderscheme}
   \mathbf{M}^L\frac{\mathbf{U}^{L,n+1}-\mathbf{U}^n}{\Delta t}
      +\left(\mathbf{A}+\tcr{\mathbf{D}^L}\right)\mathbf{U}^n = \mathbf{b},
\end{equation}

where the diffusion matrix $\tcr{\mathbf{D}^L}$ entries are computed using a local low-order
viscosity and viscous bilinear form:

\begin{equation}\label{eq:loworderD}
   \tcr{D^L_{i,j}} = \sum\limits_{K\subset S_{i,j}}\nu_K^L b_K(\varphi_j,\varphi_i).
\end{equation}

The local viscous bilinear form for an element $K$ takes a graph-theoretic
approach introduced by Guermond~\cite{guermond_firstorder}:

\begin{equation}\label{eq:bilinearform}
      b_K(\varphi_j, \varphi_i) \equiv \left\{\begin{array}{l l}
         -\frac{1}{n_K - 1}V_K & i\ne j, \quad i,j\in \mathcal{I}(K),\\
         V_K                   & i = j,  \quad i,j\in \mathcal{I}(K),\\
         0                     & i\notin\mathcal{I}(K) \quad | \quad j\notin\mathcal{I}(K),
      \end{array}\right.
\end{equation}

where $V_K$ is the volume of cell $K$,
$\mathcal{I}(K)\equiv \{j\in\{1,\ldots,N\}: |S_j\cap K|\ne 0\}$
is the set of indices corresponding to degrees of freedom in
the support of cell $K$, and $n_K \equiv \mbox{card}(\mathcal{I}(K))$.
The local low-order viscosity is defined as the following:

\begin{equation}
   \nu_K^L \equiv \max\limits_{i\ne j\in \mathcal{I}(K)}\frac{\max(0,A_{i,j})}
      {-\sum\limits_{T\subset S_{i,j}} b_T(\varphi_j, \varphi_i)},
\end{equation}

If the CFL condition $\Delta t \leq \frac{M_{i,i}^L}{A_{i,i}^L}$
is satisfied for all $i$, then the explicit
low-order scheme given in Eqn. \ref{eq:loworderscheme} \underline{\bf satisfies the following
discrete maximum principle}:

\begin{equation}\label{eq:dmp}
   U_{\min,i}^n\left(1-\frac{\Delta t}{M_{i,i}^L}
      \sum\limits_j A^L_{i,j}\right)
      + \frac{\Delta t}{M_{i,i}^L}b_i\leq
   U_i^{L,n+1}\leq
   U_{\max,i}^n\left(1-\frac{\Delta t}{M_{i,i}^L}
      \sum\limits_j A^L_{i,j}\right)
      + \frac{\Delta t}{M_{i,i}^L}b_i\quad\forall i,
\end{equation}

where $U_{\min,i}^n = \min\limits_{j\in \mathcal{I}(S_i)}U_j^n$,
$U_{\max,i}^n = \max\limits_{j\in \mathcal{I}(S_i)}U_j^n$
and $\mathcal{I}(S_i)$ is the set of indices of degrees of freedom in the
support of degree of freedom $i$.


%===============================================================================
\subsection{High-Order Scheme\label{sec:high_order_scheme}}
%===============================================================================
\section{High-Order Schemes}
%================================================================================
\subsection{High-Order Semidiscrete Scheme}
%================================================================================
In general, a high-order scheme is of the form
\begin{equation}\label{high_semidiscrete}
   \M^C\frac{d\U^{H}}{dt} + \A^H(t)\U^H(t) = \b(t) \eqc
\end{equation}
or in the steady-state case,
\begin{equation}\label{high_ss}
   \A^H\U^H = \b \eqc
\end{equation}
where the high-order steady-state system matrix $\A^H(t)$ is
defined as the sum of the inviscid steady-state matrix $\A$
and a high-order artificial diffusion matrix $\D^H(t)$:
\begin{equation}
   \A^H(t) = \A + \D^H(t) \eqp
\end{equation}
If the Galerkin scheme given in Section \ref{galerkindef} is to be
used as the high order scheme, then $\D^H(t)$ is zero.
%--------------------------------------------------------------------------------
\subsection{High-Order Explicit Euler Scheme}
%--------------------------------------------------------------------------------
The high-order explicit Euler scheme is the following:
\begin{equation}\label{high_FE}
   \M^C\frac{\U^{H}-\U^n}{\dt^{n+1}} + \A^{H,n}\U^n = \b^n \eqc
\end{equation}
where the high-order steady-state system matrix $\A^{H,n}$ is
defined as the sum of the inviscid steady-state matrix $\A$
and a high-order artificial diffusion matrix $\D^{H,n}$:
\begin{equation}
   \A^{H,n} = \A + \D^{H,n} \eqp
\end{equation}
If the Galerkin scheme given in Section \ref{galerkindef} is to be
used as the high order scheme, then $\D^{H,n}$ is zero.
%--------------------------------------------------------------------------------
\subsection{High-Order Theta Scheme}
%--------------------------------------------------------------------------------
The high-order $\theta$ scheme is the following:
\begin{equation}\label{high_theta}
  \M^C\frac{\U^H-\U^n}{\dt}
  + (1-\theta)\A^{H,n}\U^n + \theta\A^{H,n+1}\U^H
  = (1-\theta)\b^n + \theta\b^{n+1}.
\end{equation}
where the high-order steady-state system matrix $\A^{H,n}$ is
defined as the sum of the inviscid steady-state matrix $\A$
and a high-order artificial diffusion matrix $\D^{H,n}$:
\begin{equation}
   \A^{H,n} = \A + \D^{H,n} \eqp
\end{equation}
If the Galerkin scheme given in Section \ref{galerkindef} is to be
used as the high order scheme, then $\D^{H,n}$ is zero.
%================================================================================
\subsection{Graph-Theoretic High-Order Scheme}\label{gthighorder}
%================================================================================
To construct a high-order scheme, the concept of entropy viscosity is used in
conjunction with the bilinear form introduced in Equation \eqref{bilinearform}.
The high-order viscosity $\nu^{H,n}$ is computed as the minimum of the low-order
viscosity $\nu^{L}_K$ and the entropy viscosity $\nu^{E,n}_K$:
\begin{equation}
   \nu^{H,n}_K = \min(\nu^{L}_K,\nu^{E,n}_K),
\end{equation}
where the entropy viscosity is defined as
\begin{equation}
   \nu^{E,n}_K = \frac{c_E R_K^n(u_h^n,u_h^{n-1})
      + c_J\max\limits_{F\in\partial K}J_F(u_h^n)}
      {\|E(u_h^n)-\bar{E}(u_h^n)\|_{L^\infty(\Omega)}}.
\end{equation}
The entropy is defined to be some convex function of $u$ such as
$E(u)=\frac{1}{2}u^2$. The entropy residual $R_K^n(u_h^n,u_h^{n-1})$ is the
following:
\begin{equation}
    R_K^n(u_h^n,u_h^{n-1}) = \left\|\frac{1}{c}\frac{E(u_h^n)-E(u_h^{n-1})}{\Delta t^n}
      + E'(u_h^n)\left[\mathbf{\Omega}\cdot\nabla u_h^n
      + \sigma u_h^n
      - q\right]\right\|_{L^\infty(K)},
\end{equation}
where the $L^\infty(K)$ norm is approximated as the maximum of the norm operand evaluated
at each quadrature point on $K$.
The entropy jumps are also computed on each face $F$ on the boundary of $K$:
\begin{equation}
   J_F(u_h^n) = \|\mathbf{\Omega}\cdot
      \mathbf{n}_F[\![\partial_n E(u_h^n)]\!]\|_{L^\infty(F)},
\end{equation}
where $\mathbf{n}_F$ is the outward unit vector for face $F$ and
the $L^\infty(F)$ norm is approximated as the maximum of the norm operand evaluated
at each quadrature point on $F$. The term $[\![\partial_n E(u_h^n)]\!]$ is computed as
\begin{eqnarray}
   [\![\partial_n E(u_h^n)]\!] & = & [\![\nabla E(u_h^n)\cdot\mathbf{n}_F]\!]\\
                        & = & [\![u_h^n\nabla u_h^n\cdot\mathbf{n}_F]\!]\\
                        & = & (u_h^n|_K\nabla u_h^n|_K - u_h^n|_{K'}
                           \nabla u_h^n|_{K'})\cdot\mathbf{n}_F
\end{eqnarray}
where $\cdot|_K$ denotes the computation of $\cdot$ from $K$, and $\cdot|_{K'}$
denotes the computation of $\cdot$ from the neighbor $K'$ sharing the face $F$.

The high-order counterpart of the low-order artificial diffusion matrix defined
in Equation \eqref{loworderdiffusionGT} uses the high-order viscosity $\nu_K^{H,n}$
instead of the low-order viscosity:
\begin{equation}
   D^{H,n}_{i,j} = \sum\limits_{K\subset S_{i,j}}\nu_K^{H,n} b_K(\varphi_j,\varphi_i).
\end{equation}
Similarly to the low-order scheme, a high-order steady-state system matrix is
defined to be the sum of the Galerkin steady-state system matrix $\A$ and the
high-order artificial diffusion matrix $\D^{H,n}$:
\begin{equation}
   \A^{H,n} = \A + \D^{H,n},
\end{equation}
and the high order scheme is the following:
\begin{equation}\label{gthighorderscheme}
   \M_C\frac{\U^{H,n+1}-\U^n}{\Delta t}
      +\A^{H,n}\U^n = \b,
\end{equation}
where $\U^{H,n+1}$ is the high-order solution at time $t^{n+1}$.


%===============================================================================
\subsection{FCT Scheme\label{sec:fct_scheme}}
%===============================================================================
Recall that FCT defines antidiffusive correction fluxes from a low-order,
monotone scheme to a high-order scheme. Calling these fluxes
$\correctionfluxvector$, this gives
\begin{equation}
  \lumpedmassmatrix\frac{\solutionvector^{H,n+1}-\solutionvector^n}{\dt}
    + (\ssmatrix+\diffusionmatrix^L)\solutionvector^n = \ssrhs^n
    + \correctionfluxvector \eqp
\end{equation}
Subtracting the high-order scheme equation from this gives the
definition of $\correctionfluxvector$:
\begin{equation}
  \correctionfluxvector \equiv
    -(\consistentmassmatrix-\lumpedmassmatrix)
    \frac{\solutionvector^{H,n+1}-\solutionvector^n}{\dt}
    + (\diffusionmatrix^L-\diffusionmatrix^{H,n})\solutionvector^n \eqp
\end{equation}
Now it is necessary to decompose these fluxes into internodal fluxes
$\correctionfluxij$ such that $\sum_j\correctionfluxij=\correctionfluxletter_i$:
\begin{equation}
  \correctionfluxij = -M\ij^C\pr{
    \frac{\MakeUppercase{\solutionletter}^{H,n+1}_j
      -\MakeUppercase{\solutionletter}^n_j}{\Delta t}
    -\frac{\MakeUppercase{\solutionletter}^{H,n+1}_i
      -\MakeUppercase{\solutionletter}^n_i}{\Delta t}
  }
  + (D\ij^L-D\ij^{H,n})(\MakeUppercase{\solutionletter}^n_j
    -\MakeUppercase{\solutionletter}^n_i) \eqp
\end{equation}
Recall that the objective of FCT is to limit these antidiffusive
fluxes to enforce some physical bounds. For the scalar case, one can use
the discrete maximum principle bounds given by Equation \eqref{eq:dmp}.
The limitation is achieved by applying a limiting coefficient $L\ij$ to each
internodal flux $\correctionfluxij$:
\begin{equation}
  \lumpedmassmatrix\frac{\solutionvector^{n+1}-\solutionvector^n}{\dt}
    + \ssmatrix^L\solutionvector^n = \ssrhs
    + \bar{\correctionfluxvector} \eqc
\end{equation}
where $\bar{\correctionfluxvector}$ denotes the limited antidiffusion vector:
$\bar{\correctionfluxletter}_i\equiv\sum_j\limiterletter\ij\correctionfluxij$.
Each limiting coefficient is between zero and unity: $0\leq L\ij\leq 1$.
If for example, all $L\ij$ are zero, then the low-order scheme is reproduced;
conversely, if all $L\ij$ are one, then the high-order scheme is reproduced.

Now the definition for the limiting coefficients is given. Firstly,
the enforced bounds on the solution, given by Equation \eqref{eq:dmp}
are rearranged to give bounds on the limited flux sums:
\begin{subequations}
\begin{equation}
  Q^-_i \leq \sum\limits_j L\ij \correctionfluxij \leq Q^+_i \eqc
\end{equation}
\begin{equation}
  Q_i^\pm \equiv M_{i,i}^L\frac{W_i^\pm-U_i^n}{\Delta t}
    + \sum\limits_j A_{i,j}^L U_j^n - b_i \eqp
\end{equation}
\end{subequations}
The classic Zalesak limiter has the following definition\cite{zalesak}:
\begin{equation}
  \correctionfluxletter_i^- \equiv \sum\limits_{j:\correctionfluxij<0}
    \correctionfluxij \eqc \qquad
  \correctionfluxletter_i^+ \equiv \sum\limits_{j:\correctionfluxij>0}
    \correctionfluxij \eqp
\end{equation}
\begin{equation}
  L_i^\pm \equiv\left\{
    \begin{array}{l l}
      1 & \correctionfluxletter_i^\pm = 0\\
      \min\left(1,\frac{Q_i^\pm}{\correctionfluxletter_i^\pm}\right) &
      \correctionfluxletter_i^\pm \ne 0
    \end{array}
  \right. \eqp
\end{equation}
\begin{equation}
  L\ij \equiv\left\{
    \begin{array}{l l}
      \min(L_i^+,L_j^-) & \correctionfluxij \geq 0\\
      \min(L_i^-,L_j^+) & \correctionfluxij < 0
    \end{array}
  \right. \eqp
\end{equation}
Since the internodal antidiffusive fluxes are skew-symmetric, i.e.,
$\correctionfluxij=-\correctionfluxji$, and the limiting coefficients
are symmetric, i.e., $L_{i,j}=L_{j,i}$, the limited antidiffusion added
to the low-order scheme is conservative because
$\sum_i\sum_j L_{i,j}\correctionfluxij=0$.



%###############################################################################
\section{Preliminary Results\label{sec:results}}
%###############################################################################
% !TEX root = ../paper.tex

\begin{itemize}
  \item smooth MMS problem - second-order convergence (based on length, could be a summary at the beginning of the results section + appendix)
  \item 1-D, 2-D void-to-absorber (normally-incident)
  \item 2-D obstruction test problem
  \item 1-D source-in-void test problem
  \item 1-D interface or three-region test problem
\end{itemize}

This section presents results for a number of test problems, which compare
solutions obtained using:
\begin{itemize}
  \item the standard Galerkin FEM, titled in plots as ``Galerkin'',
  \item the low-order method, titled in plots as ``Low'',
  \item the entropy viscosity method, titled in plots as ``EV'',
  \item the standard Galerkin FEM with FCT, titled in plots as ``Galerkin-FCT'', and
  \item the entropy viscosity method with FCT, titled in plots as ``EV-FCT''.
\end{itemize}

%===============================================================================
\subsection{Spatial Convergence}
%===============================================================================

\begin{figure}[htb]
   \centering
      \includegraphics[width=\textwidth]
        {images/convergence_sinx.pdf}
      \caption{Spatial Convergence for MMS Problem}
   \label{fig:mms_sinx_ss}
\end{figure}
\clearpage
%===============================================================================
\subsection{Glancing Beam in a Void}
%===============================================================================
\begin{figure}[ht]
   \centering
   \begin{subfigure}{0.45\textwidth}
      \includegraphics[width=\textwidth]
        {images/glance_Low.png}
      \caption{Low-Order}
   \end{subfigure}
   \begin{subfigure}{0.45\textwidth}
      \includegraphics[width=\textwidth]
        {images/glance_EV.png}
      \caption{EV}
   \end{subfigure}
   \begin{subfigure}{0.45\textwidth}
      \includegraphics[width=\textwidth]
        {images/glance_GalFCT.png}
      \caption{Galerkin-FCT}
   \end{subfigure}
   \begin{subfigure}{0.45\textwidth}
      \includegraphics[width=\textwidth]
        {images/glance_EVFCT.png}
      \caption{EV-FCT}
   \end{subfigure}
   \caption{Comparison of Solutions for the Glance-in-Void Test
     Problem Using Explicit Euler Time Discretization}
   \label{fig:glance_in_void_fe}
\end{figure}
\clearpage
%===============================================================================
\subsection{Obstruction}
%===============================================================================
\begin{figure}[ht]
   \centering
   \begin{subfigure}{0.3\textwidth}
      \includegraphics[width=\textwidth]
        {images/obstruction_low.png}
      \caption{Low-Order}
   \end{subfigure}
   \begin{subfigure}{0.3\textwidth}
      \includegraphics[width=\textwidth]
        {images/obstruction_Gal.png}
      \caption{Galerkin}
   \end{subfigure}
   \begin{subfigure}{0.3\textwidth}
      \includegraphics[width=\textwidth]
        {images/obstruction_EV.png}
      \caption{EV}
   \end{subfigure}
   \begin{subfigure}{0.3\textwidth}
      \includegraphics[width=\textwidth]
        {images/obstruction_GalFCT.png}
      \caption{Galerkin-FCT}
   \end{subfigure}
   \begin{subfigure}{0.3\textwidth}
      \includegraphics[width=\textwidth]
        {images/obstruction_EVFCT.png}
      \caption{EV-FCT}
   \end{subfigure}
   \caption{Comparison of Solutions for the Obstruction Test
     Problem Using Implicit Euler Time Discretization}
   \label{fig:obstruction_be}
\end{figure}
\clearpage
%===============================================================================
\subsection{Two-Region Interface}
%===============================================================================
\begin{figure}[htb]
   \centering
      \includegraphics[width=\textwidth]
        {images/solution_interface.pdf}
      \caption{Comparison of Solutions for the Two-Region Interface Test
       Problem Using SSPRK33 Time Discretization}
   \label{fig:interface}
\end{figure}
\clearpage
%===============================================================================
\subsection{Source in a Void}
%===============================================================================
%solution_source_in_void.pdf
\begin{figure}[htb]
   \centering
      \includegraphics[width=\textwidth]
        {images/solution_source_in_void.pdf}
      \caption{Comparison of Solutions for the Source-in-Void Test
       Problem Using Steady-State Time Discretization}
   \label{fig:source_}
\end{figure}
\clearpage


%###############################################################################
\section{Research Goals\label{sec:goals}}
%###############################################################################
The main goal of this research is to develop numerical methods for hyperbolic
PDEs that address the development of spurious oscillations and negativities for
physically positive solution quantities. The equations and models under
consideration are the scalar transport PDE model given in Section
\ref{sec:scalar_model} and the shallow water equations (SWE) with flat bottom
topography ($\nabla\bathymetry=\mathbf{0}$), given in Section
\ref{sec:sw_model}.  The methodology developed for these equations is intended
to be largely extendible to other hyperbolic systems. The following list
discusses specific research objectives that derive from the main research goal.

\begin{itemize}
  \item \textbf{Develop low-order scheme.}
    For the scalar case, this first-order-accurate scheme should satisfy a discrete maximum principle
    as described in Section \ref{sec:low_order_scheme}. For the systems case,
    this scheme should have an invariant domain. The domain-invariant scheme
    is developed for general conservation law systems; however, the scheme
    requires an estimation of maximum wave speeds, which entails an
    analysis of the characteristics for the specific hyperbolic system. This
    research performs this analysis for the shallow water equations with flat
    bottom topography. The monotonicity and
    positivity-preservation of these schemes should be demonstrated with numerical examples.
  \item \textbf{Develop high-order entropy-based scheme.}
    The entropy-based scheme should be second-order-accurate and should
    demonstrate stability with numerical examples. A reduction in the occurrence
    and magnitude of spurious oscillations should be observed. The scalar
    case will use an arbitrary convex entropy function such as
    $\entropy(\scalarsolution)=\half\scalarsolution^2$, while for the shallow
    water equations, the entropy function will be the total energy.
  \item \textbf{Develop Galerkin FCT scheme.}
    The basic FEM-FCT scheme, as described by Kuzmin in \cite{kuzmin_FCT}, should
    be implemented. This scheme uses the standard second-order-accurate Galerkin
    scheme as the high-order scheme in the algorithm. Positivity-preservation and
    a reduction of spurious oscillations should be demonstrated.
  \item \textbf{Develop entropy-based FCT scheme.}
    The basic FEM-FCT scheme should be modified to replace the Galerkin scheme
    with the high-order entropy-based scheme in the FCT algorithm. This should
    demonstrate positivity-preservation and a further reduction in spurious
    oscillations from the entropy-based method alone.
  \item \textbf{Compare schemes.}
    Schemes developed in this research should be compared in a number of
    test problems. In addition to comparing accuracy and convergence rates,
    it is advantageous to determine rough computational cost estimates
    of each method, since the developed methods have the cost of additional
    complexity.
\end{itemize}



%###############################################################################
\section{Conclusions\label{sec:conclusions}}
%###############################################################################
In conclusion, the FCT scheme presented guarantees
a non-negative solution that satisfies a discrete maximum principle.
While this FCT scheme does not guarantee monotonicity, it has been
found to be successful in many simple test problems. The underlying
high-order scheme based on entropy viscosity has been found to
produce a higher quality FCT solution than using an inviscid
high-order scheme. A number of challenges remain - for example,
FCT transients can give rise to spurious plateaus and can have non-monotone
solutions within the bounds of the imposed discrete maximum principle.
These unphysical effects arise due to unphysical oscillations in the
high-order solution; improving the high-order solution improves
the quality of the FCT scheme.

Future work will extend this scheme to implicit time
discretizations since the CFL condition required by explicit time
discretizations can be very restrictive, particularly for radiation transport simulations.
In addition, many problems of interest involve steady-state solutions
of the transport equation, so a steady-state FCT scheme will also be
a subject of future work.



%###############################################################################
\bibliographystyle{plain}
\bibliography{\refdir/references}
%###############################################################################

\end{document}
