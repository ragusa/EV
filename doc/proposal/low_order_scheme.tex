As discussed in the introduction, the FCT algorithm employs a monotone,
low-order scheme, although one does not actually compute the low-order
solution in the algorithm. If one were to apply full limitation to
the antidiffusive fluxes in the algorithm, one would obtain the low-order
solution. In this way, the low-order scheme serves as a fail-safe in the FCT
algorithm; if the antidiffusive fluxes in question would violate the imposed
monotonicity bounds, then they are limited partially or in full. In this section,
the low-order scheme for the scalar case is given.

In the general case of conservation law systems, the desired approach to
achieving monotonicity of the low-order scheme is to prove an \emph{invariant
domain} for the scheme, which in essence implies that the numerical
solution cannot leave some domain determined by initial data. For the
scalar case, the invariant domain amounts to a \emph{discrete maximum principle},
which is described later in this section.

Monotonicity of the scalar low-order
scheme can be proven by proving that the system matrix of the scheme in
the form $\mathbf{A}\solutionvector^{n+1}=\mathbf{b}$ is an M-matrix, i.e.,
that it is an \emph{inverse-positive} matrix, also known as a \emph{monotone}
matrix. The consequence of this property
is the following: if $\mathbf{A}$ is an M-matrix and $\mathbf{b}$ is
non-negative, then $\solutionvector^{n+1}$ is non-negative. Thus proving
$\mathbf{A}$ is an M-matrix and $\mathbf{b}\geq 0$ proves the
positivity-preservation property. It turns out that the M-matrix
property of the system matrix property can be proven by
lumping the mass matrix:
$\consistentmassmatrix\rightarrow\lumpedmassmatrix$,
and adding a low-order diffusion term
$\loworderdiffusionmatrix$ (definition to be given shortly).
Thus the low-order scheme is the following:
\begin{equation}
  \lumpedmassmatrix\frac{\solutionvector^{L,n+1}-\solutionvector^n}{\dt}
    + (\ssmatrix + \diffusionmatrix^L)\solutionvector^n = \ssrhs^n \eqp
\end{equation}

The low-order diffusion matrix $\diffusionmatrix^L$ is designed to
guarantee $D^L\ij \leq -A\ij, \quad j\ne i$
This in turn is used to prove the M-matrix property of the system matrix.
The definition used to give this property is now given.
The diffusion matrix is assembled elementwise,
where $K$ denotes an element, using a local bilinear form $b_K$ and a
local low-order artificial viscosity $\nu_K^L$:
\begin{equation}
  D\ij^L = \sumKSij \nu_K^L b_K(\testfunction_j,\testfunction_i) \eqc
\end{equation}
and the local bilinear form is defined as follows, where $|K|$ denotes
the volume of element $K$, $\mathcal{I}(K)$ is the set of indices
corresponding to degrees of freedom with nonempty support on $K$, and
$n_K$ is the cardinality of this set.
\begin{equation}
  b_K(\testfunction_j, \testfunction_i) \equiv \left\{\begin{array}{l l}
    -\frac{1}{n_K - 1}\cellvolume & i\ne j, \quad i,j\in \mathcal{I}(K)\\
    \cellvolume                   & i = j,  \quad i,j\in \mathcal{I}(K)\\
    0                & i\notin\mathcal{I}(K)\,|\, j\notin\mathcal{I}(K)
  \end{array}\right. \eqp
\end{equation}
Some useful properties of this bilinear form are
$\sum\limits_j b_K(\testfunction_j, \testfunction_i) = 0$ and
$b_K(\testfunction_i, \testfunction_i) > 0$.
Finally, the low-order viscosity is defined as
\begin{equation}
  \lowordercellviscosity[\timeindex] \equiv \max\limits_{i\ne j\in\indicescell}
  \frac{\max(0,\ssmatrixletter\ij^\timeindex)}
  {-\mkern-20mu\sumKSij[T]\mkern-20mu\localviscbilinearform{T}{j}{i}}
  \eqp
\end{equation}

In addition to guaranteeing monotonicity and positivity, the low-order
viscous terms guarantee a discrete maximum principle (DMP), which amounts
to locally bounding each solution degree of freedom from above and
below. Not only is this a useful property in its own right, but it
also turns out to be a useful tool in the FCT algorithm: it is
used as the ``physically-motivated'' bounds to impose on the FCT
solution. The DMP that is proved for the FE low-order scheme is
the following, where $U^n_{\max,i} = \max\limits_{j\in\mathcal{I}(S_i)}U^n_j$,
$U^n_{\min,i} = \min\limits_{j\in\mathcal{I}(S_i)}U^n_j$:
\begin{subequations}\label{eq:dmp}
\begin{equation}
  W_i^-\leq
  U_i^{L,n+1}\leq
  W_i^+\qquad\forall i \eqc
\end{equation}
\begin{equation}
  W_i^\pm \equiv U_{\substack{\max\\\min},i}^n\left(
  1-\frac{\dt}{M_{i,i}^L}
  \sum\limits_j A\ij^L\right)
  + \frac{\Delta t}{M_{i,i}^L}b_i^n \eqp
\end{equation}
\end{subequations}
Note that when there is no reaction term or source term, these bounds
are readily identified as discrete local extremum diminishing (LED) bounds:
\begin{equation}
  U^n_{\min,i}\leq
  U_i^{L,n+1}\leq
  U^n_{\max,i}\qquad\forall i \eqp
\end{equation}
These LED bounds imply that if a degree of freedom $i$ is a local
minimum / maximum, then it cannot shrink / grow. For the systems
case, as opposed to this scalar case, there is no DMP available,
and thus this LED criterion is used, although the basis for this
is largely empirical rather than given by proof.

