With linear piecewise FEM, the solution is approximated as a linear combination of
the piecewise linear basis functions $\testfunction_j(\x)$:
\begin{equation}
  \approximatescalarsolution\xt = \sumj \solutionletter_j(\timevalue)
  \testfunction_j(\x) \eqc
\end{equation}
where the coefficients $\solutionletter_j(\timevalue)$ are the basis function
expansion coefficients at time $\timevalue$. Substituting the approximate
solution into Equation \eqref{eq:scalar_transport}
(with $\consfluxscalar\equiv\velocity\scalarsolution$) and testing with basis
function $\testfunction_i(\x)$ gives
\begin{equation}
   \intSi\ppt{\approximatescalarsolution}\testfunction_i(\x) d\volume
      + \intSi\left(\velocity\cdot\nabla\approximatescalarsolution\xt
      + \reactioncoef(\x)\approximatescalarsolution\xt\right)
      \testfunction_i(\x) d\volume
      = \intSi \scalarsource\xt \testfunction_i(\x) d\volume \eqc
\end{equation}
where $\support_i$ is the support of $\testfunction_i(\x)$.
With the assumption of the linear flux function, the system to be solved
is linear:
\begin{equation}\label{eq:semidiscrete}
  \consistentmassmatrix\ddt{\solutionvector}+\ssmatrix\solutionvector(t)
  = \ssrhs(\timevalue) \eqc
\end{equation}
where $\consistentmassmatrix$ is the consistent mass matrix:
\begin{equation}\label{eq:massmatrix}
  \massmatrixletter^C\ij \equiv \intSij
  \testfunction_j(\x)\testfunction_i(\x) d\volume \eqc
\end{equation}
$\ssrhs(\timevalue)$ is the source vector:
\begin{equation}
  \ssrhsletter_i(\timevalue) \equiv \intSi \scalarsource\xt\testfunction_i(\x)
  d\volume \eqc
\end{equation}
and $\ssmatrix$ is the steady-state matrix:
\begin{equation}\label{eq:Aij}
  \ssmatrixletter\ij \equiv \intSij\left(
  \velocity\cdot\nabla\testfunction_j(\x) +
  \reactioncoef(\x)\testfunction_j(\x)\right)\testfunction_i(\x) d\volume \eqc
\end{equation}
where $\support\ij$ is the dual support of $\testfunction_i(\x)$ and
$\testfunction_j(\x)$.
If the flux function $\consfluxscalar$ were nonlinear, then the system would
be nonlinear, but it would be expressible in a quasilinear form:
$\ssmatrix\rightarrow\ssmatrix(\approximatescalarsolution)$.

