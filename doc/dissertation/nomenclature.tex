\chapter*{NOMENCLATURE}
\addcontentsline{toc}{chapter}{NOMENCLATURE}  

Table \ref{tab:abbreviations} lists the abbreviations used in this dissertation,
and Table \ref{tab:symbols} lists the symbols used in this dissertation.
In general, any symbol in non-boldface font is a scalar (zero-order tensor),
any lowercase symbol in a boldface font is a vector (first-order tensor),
and any uppercase symbol in a boldface font is a matrix (second-order tensor).
One exception to this rule is the solution vector, $\solutionvector$, which
is uppercase despite being a vector, not a matrix. This is to distinguish
it from the symbol for a vector-valued solution function,
$\vectorsolution$, since ``u'' is traditionally used in both circumstances.

Individual matrix entries are denoted by the non-boldface equivalent of
the matrix symbol, with two subscripted indices, and individual vector entries
are denoted by the non-boldface equivalent, with one subscripted index.
Symbols used for degree of freedom indices include $i$, $j$, $k$, and $\ell$,
and cell indices are usually denoted by $\cell$.
Temporal indices
use $n$ as a superscript, and iteration indices use $(\ell)$
as a superscript. Other symbols appearing in subscripts or superscripts
are typically descriptors, e.g., to distinguish a function for different
components of a vector-valued system, or to distinguish a low-order order
vs. a high-order quantity.

Sums often give only the summation index, with the lower/upper bounds or the
set being understood from context. For example, if the set of all degrees of freedom indices
is $\mathcal{I}(\domain)=\{1,\ldots,\maxdof\}$, where $\maxdof$ is the maximum
degree of freedom index, then the sum notation
$\sum\limits_{j\in\mathcal{I}(\domain)}$ or $\sum\limits_{j=1}^N$ is shortened
to $\sumj$.

The notation $\mathbf{a}\otimes\mathbf{b}$ denotes a tensor product, otherwise
known as an outer product, of column vectors $\mathbf{a}$ and $\mathbf{b}$,
i.e., $\mathbf{a}\otimes\mathbf{b}\equiv\mathbf{a}\mathbf{b}^T$. Thus,
$(\mathbf{a}\otimes\mathbf{b})\ij = a_i b_j$.

Functions of time often omit the time argument and instead take the time index
as a superscript, e.g., $f(\timevalue^\timeindex)$ is abbreviated as
$f^\timeindex$. Similarly, functions of a time-dependent solution or other
time-dependent arguments are sometimes abbreviated by dropping the dependence
on the argument and adopting the time index as a superscript, e.g.,
$f(\approximatescalarsolution^\timeindex)$ becomes $f^\timeindex$.


\begin{mytable}{Abbreviations Used in Dissertation}{abbreviations}{l l}
{\emph{Abbreviation} & \emph{Description}}
BE    & backward (implicit) Euler\\
CFEM  & continuous finite element method\\
CN    & Crank-Nicolson\\
DI    & domain-invariant\\
DMP   & discrete maximum principle\\
EV    & entropy viscosity\\
FCT   & flux-corrected transport\\
FE    & forward (explicit) Euler\\
FEM   & finite element method\\
FV    & finite volume\\
IVP   & initial value problem\\
LED   & local extremum diminishing\\
ODE   & ordinary differential equation\\
PDE   & partial differential equation\\
RK    & Runge-Kutta\\
SS    & steady-state\\
SSPRK & strong stability-preserving Runge-Kutta\\
SSPRK33 & 3-stage, 3-order-accurate SSPRK scheme, also known as the Shu-Osher method\\
\end{mytable}

\begin{center}
\begin{longtable}{l p{4.8in}}
\caption{Symbols Used in Dissertation\label{tab:symbols}}\\
\hline
\emph{Symbol} & \emph{Description}\\
\hline
\endfirsthead
\multicolumn{2}{c}{Table \thetable\ -- Continued}\vspace{3ex}\\
\hline
\emph{Symbol} & \emph{Description}\\
\hline
\endhead
\hline
\endfoot
\hline
\endlastfoot
%==============================================================================
% Latin alphabet: A-Z
%==============================================================================
% A-B
%------------------------------------------------------------------------------
$\speedofsound$    & speed of sound\\
$\ssmatrix$        & steady-state matrix\\
$\highorderssmatrix$ & high-order steady-state matrix\\
$\loworderssmatrix$ & low-order steady-state matrix\\
$\bathymetry$      & bathymetry (bottom topography) function for shallow
                     water equations\\
$\ssrhs$           & steady-state right hand side vector\\
$\localviscbilinearform{\cellindex}{i}{j}$ &
                     viscous bilinear form for cell $\cellindex$ and test
                     functions $\testfunction_i$ and $\testfunction_j$\\

% C
%------------------------------------------------------------------------------
$\speedoflight$    & speed of light\\
$\RKtimecoef_i$    & time step size fraction for stage $i$ of a
                     Runge-Kutta time step\\
$\entropyjumpcoef$ & coefficient for the entropy jump term in the definition
                     of entropy viscosity\\
$\entropyresidualcoef$ & coefficient for the entropy residual term in the
                         definition of entropy viscosity\\

% D
%------------------------------------------------------------------------------
$\highorderdiffusionmatrix$ & high-order diffusion matrix\\
$\loworderdiffusionmatrix$ & low-order diffusion matrix\\
$\domain$          & problem domain\\
$\domainboundary$  & problem domain boundary\\
$\cellvolume$      & volume of cell $\cellindex$\\

% E-K
%------------------------------------------------------------------------------
$\unitvector{d}$   & unit vector for dimension $d$\\
$\totalenergy$     & total energy density\\
$\consfluxscalar$  & conservation law flux function for scalar case\\
$\consfluxvector(\vectorsolution)$  & conservation law flux function for vector case\\
$\gravity$         & acceleration due to gravity\\
$\height$          & height\\
$\indicescell$     & set of degree of freedom indices with support on
                     cell $\cellindex$\\
$\entropyjump_F$ & entropy jump for face $F$\\
$\cellindex$       & cell index\\

% L-M
%------------------------------------------------------------------------------
$\limitermatrix$   & matrix of limiting coefficients\\
$\limiterletter_{i,j}$ & limiting coefficient for degrees of freedom $i$
                         and $j$\\
$\limiterletter^\pm_i$ & Zalesak intermediate limiting coefficient bounds for
                         degree of freedom $i$\\
$\momentum$        & momentum\\
$\consistentmassmatrix$ & consistent mass matrix\\
$\lumpedmassmatrix$ & lumped mass matrix\\

% N-P
%------------------------------------------------------------------------------
$\timeindex$       & time index\\
$\normalvector$    & normal vector to a surface\\
$\cardinality$     & cardinality of cell $\cellindex$\\
$\pressure$        & pressure\\
$\correctionfluxvector$ & FCT correction flux vector\\
$\correctionfluxmatrix$ & matrix of FCT internodal correction flux entries\\
$\correctionfluxij$ & FCT internodal correction flux for degrees of freedom
                      $i$ and $j$\\
$\correctionfluxsumsi$ & sum of positive ($+$) and negative ($-$) FCT internodal
                         correction fluxes for degree of freedom $i$\\

% Q-T
%------------------------------------------------------------------------------
$\scalarsource$    & scalar source function\\
$\heightmomentum$  & ``momentum'' for shallow water equations,
                     $\height\velocity$\\
$\limitedfluxboundsi$ & upper and lower bounds for limited flux sum for degree
                        of freedom $i$\\
$\ssres$           & steady-state residual vector\\
$\entropyresidual_\cellindex$ & entropy residual for cell $\cellindex$\\
$\RKnstages$       & number of stages for a Runge-Kutta scheme\\
$\support_i$       & support of FEM test/basis function $i$\\
$\support\ij$      & shared support of FEM test/basis functions $i$ and $j$\\
$\conssource(\vectorsolution)$ & conservation law source function\\
$\timevalue$       & time\\

% U
%------------------------------------------------------------------------------
$\RKstagetime_i$   & stage $i$ time of a Runge-Kutta time step\\
$\scalarsolution$  & solution variable for a scalar conservation law\\
$\vectorsolution$  & vector of solution variables for a conservation law
                     system\\
$\approximatescalarsolution$ & numerical approximation to $\scalarsolution$\\
$\solutionvector$  & solution vector\\
$\RKstagesolution^i$ & stage solution vector for stage $i$ of a Runge-Kutta time
                       step\\
$\RKintermediatesolution^i$ & intermediate solution vector for stage $i$ of a
                              Runge-Kutta time step\\
$\highordersolution$ & high-order solution vector\\
$\lowordersolution$ & low-order solution vector\\

% V-Z
%------------------------------------------------------------------------------
$\speed$           & speed\\
$\velocity$        & velocity\\
$\volume$          & volume\\
$\solutionbound_i^\pm$ & upper and lower solution bounds for degree of freedom $i$\\
$\DMPbounds_i$     & upper and lower DMP bounds for degree of freedom $i$\\
$\x$               & spatial position\\
%==============================================================================
% Greek alphabet
%==============================================================================
% alpha - mu
%------------------------------------------------------------------------------
$\RKoldsolutioncoef_i$ & coefficient for the old solution vector in stage
                         $i$ of a Runge-Kutta time step\\
$\RKstagesolutioncoef_i$ & coefficient for the stage solution vector in stage
                           $i$ of a Runge-Kutta time step\\
$\timestepsize$    & time step size\\
$\entropy$         & entropy function\\
$\theta$           & parameter for $\theta$ time discretization scheme\\

% nu - omega
%------------------------------------------------------------------------------
$\lowordercellviscosity$  & low-order viscosity for cell $\cellindex$\\
$\highordercellviscosity$ & high-order viscosity for cell $\cellindex$\\
$\entropycellviscosity$   & entropy viscosity for cell $\cellindex$\\
$\density$         & density\\
$\reactioncoef$    & reaction coefficient\\
$\totalcrosssection$ & total macroscopic cross-section\\
$\testfunction$    & FEM test/basis function\\
$\angularflux$     & angular flux\\
$\directionvector$ & unit transport direction vector\\
%==============================================================================
\end{longtable}
\end{center}

\pagebreak{}
