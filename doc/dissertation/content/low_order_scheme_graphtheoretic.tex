\subsection{Graph-Theoretic Low-Order Scheme}\label{gtloworder}
%================================================================================
In this section, a graph-theoretic approach is taken to define a low-order
scheme that is maximum principle preserving. The following bilinear form
is employed in the low-order scheme:
\begin{definition}{Local Bilinear Form}{}
   The local bilinear form for cell $K$ is defined as follows:
   \begin{equation}\label{bilinearform}
      b_K(\varphi_j, \varphi_i) \equiv \left\{\begin{array}{l l}
         -\frac{1}{n_K - 1}|K| & i\ne j, \quad i,j\in \mathcal{I}(K),\\
         |K|                   & i = j,  \quad i,j\in \mathcal{I}(K),\\
         0                     & i\notin\mathcal{I}(K) \quad | \quad j\notin\mathcal{I}(K),
      \end{array}\right.
   \end{equation}
   where $\mathcal{I}(K)\equiv \{j\in\{1,\ldots,N\}: |S_j\cap K|\ne 0\}$
   is the set of indices corresponding to degrees of freedom in
   the support of cell $K$ and $n_K \equiv \mbox{card}(\mathcal{I}(K))$.
\end{definition}
%--------------------------------------------------------------------------------
The low-order viscosity is defined as follows:
%--------------------------------------------------------------------------------
\begin{definition}{Low-Order Viscosity}{}
   The low-order viscosity for the radiative transfer equation on cell $K$ is
   defined as follows:
   \begin{equation}
      \nu_K^L \equiv \max\limits_{i\ne j\in \mathcal{I}(K)}\frac{\max(0,A_{i,j})}
         {-\sum\limits_{T\subset S_{i,j}} b_T(\varphi_j, \varphi_i)},
   \end{equation}
   where $A_{i,j}$ is the $i,j$th entry of the Galerkin steady-state
   matrix given by Equation \eqref{Aij} and $S_{i,j}=S_i\cap S_j$ is the
   dual-support of test functions $\varphi_i$ and $\varphi_j$.
\end{definition}
%--------------------------------------------------------------------------------
\begin{definition}{Low-Order Artificial Diffusion Matrix}{}
   The low-order artificial diffusion matrix $\mathbf{D}^L$ is defined as
   \begin{equation}\label{loworderdiffusionGT}
      D_{i,j}^L \equiv \sum\limits_{K\subset S_{i,j}}\nu_K^L b_K(\varphi_j,\varphi_i).
   \end{equation}
\end{definition}
%--------------------------------------------------------------------------------
The low-order system matrix is then defined as the sum of the inviscid
steady-state system matrix $\mathbf{A}$ and the low-order artificial diffusion
matrix $\mathbf{D}^L$:
%--------------------------------------------------------------------------------
\begin{definition}{Low-Order Steady-State System Matrix}{}
   The low-order steady-state system matrix is
   \begin{equation}\label{loworderssmatrixGT}
      \mathbf{A}^L \equiv \mathbf{A} + \mathbf{D}^L.
   \end{equation}
\end{definition}
%--------------------------------------------------------------------------------
The low-order semidiscrete scheme is again given by Equation \eqref{semidiscretelow}.
%--------------------------------------------------------------------------------
\subsubsection{M-Matrix Property of the Steady-State System Matrix}
%================================================================================
In this section, it will be shown that the low-order steady-state
system matrix defined in Equation \eqref{loworderssmatrixGT} is an M-matrix, which
allows a discrete maximum principle for the low-order solution to be proven in
Section \ref{DMP}.
%--------------------------------------------------------------------------------
\begin{lemma}{Non-Positivity of Off-Diagonal Elements}{offdiagonalnegative_gt}
   The off-diagonal elements of the linear system matrix are non-positive:
   $A^L_{i,j}\le 0, j\ne i$.
\end{lemma}

\begin{proof}
This proof begins by bounding the term
$D_{i,j}^L=\sum\limits_{K\subset S_{i,j}}\nu_K^L b_K(\varphi_j, \varphi_i)$:
\begin{eqnarray*}
   D_{i,j}^L=\sumKSij\nu_K^L b_K(\varphi_j, \varphi_i)
   & = & \sumKSij\max\limits_{k\ne \ell\in \mathcal{I}(K)}
      \pr{\frac{\max(0,A_{k,\ell})}
		{-\sum\limits_{T\subset S_{k,\ell}} b_T(\varphi_{\ell}, \varphi_k)}}
      b_K(\varphi_j,\varphi_i)
\end{eqnarray*}
Since $b_K(\varphi_j, \varphi_i) < 0$ for $j\ne i$ and $y_i \leq \max_j y_j$,
\begin{eqnarray*}
   D\ij & \le & \sum\limits_{K\subset S_{i,j}} \frac{\max(0,A_{i,j})}
		{-\sum\limits_{T\subset S_{i,j}} b_T(\varphi_j, \varphi_i)}b_K(\varphi_j,\varphi_i)\\
   & =   & -\max(0,A_{i,j}) \frac{\sum\limits_{K\subset S_{i,j}}
      b_K(\varphi_j,\varphi_i)}
		{\sum\limits_{T\subset S_{i,j}} b_T(\varphi_j, \varphi_i)}\\   
   & = & -\max(0,A_{i,j})\\
   & \le & -A_{i,j}
\end{eqnarray*}
From Equations \eqref{loworderssmatrixGT} and \eqref{loworderdiffusionGT},
\begin{eqnarray*}
	A^L_{i,j} & = & A_{i,j} + D_{i,j}^L\\
   A^L_{i,j} & \le & A_{i,j} - A_{i,j}\\
   A^L_{i,j} & \le & 0.\qed
\end{eqnarray*}
\end{proof}
%--------------------------------------------------------------------------------
\begin{lemma}{Non-Negativity of Diagonal Elements}{diagonalpositive_gt}
   The diagonal elements  of the linear system matrix are non-negative: $A^L_{i,i}\ge 0$.
\end{lemma}

\begin{proof}
The diagonal elements of the low-order system matrix are
\[
   A^L_{i,i} = \intSi\nabla\cdot
   \frac{\v\varphi_i^2(\x)}{2} dV
      + \intSi\sigma(\x)\varphi_i^2(\x)dV
      + \sumKSi\nu_K^L b_K(\varphi_i, \varphi_i).
\]
To prove that $A^L_{i,i}$ is non-negative, it is sufficient to prove that
each term in the above expression is non-negative. The non-negativity of
the interaction term and viscous term are obvious ($\sigma \ge 0, 
\, \nu_K^L\ge 0, \, b_K(\varphi_i, \varphi_i)>0$), but
the non-negativity of the divergence term is not necessarily obvious. On the interior of
the domain, the divergence term gives zero contribution because the divergence integral may
be transformed into a surface integral $\int_{S_i}\v\cdot\mathbf{n}\frac{\varphi_i^2}{2} dV$
via the divergence theorem; one can then recognize that
the basis function $\varphi_i$ evaluates to zero on the boundary of its support $S_{i}$.
On the outflow boundary of the domain, the term $\v\cdot\mathbf{n}
\frac{\varphi_i^2}{2}$ is positive because $\v\cdot\mathbf{n} >0$
for an outflow boundary. This quantity is of course negative for the inflow boundary,
but a Dirichlet boundary condition is strongly imposed on the incoming boundary, so
for degrees of freedom $i$ on the incoming boundary, $A^L_{i,i}$ will be set equal
to some positive value such as 1 with a corresponding incoming value
accounted for in the right hand side $\b$ of the linear system.\qed
\end{proof}
%--------------------------------------------------------------------------------
\begin{lemma}{Non-Negativity of Row Sums}{}
   The sum of all elements in a row $i$ is non-negative: $\sum\limits_j A^L_{i,j} \ge 0$.
\end{lemma}

\begin{proof}
Using the fact that $\sum\limits_j\varphi_j(\x)=1$ and
$\sum\limits_j b_K(\varphi_j,\varphi_i)=0$,
\begin{eqnarray*}
   \sumj A^L_{i,j} & = & \sumj \intSij
      \left(\v\cdot\nabla\varphi_j(\x) +
      \sigma(\x)\varphi_j(\x)\right)\varphi_i(\x) dV +
      \sumj\sumKSij\nu_K^L b_K(\varphi_j, \varphi_i)\\
   & = & \intSi\left(\v\cdot
      \nabla\sumj\varphi_j(\x) +
      \sigma(\x)\sumj\varphi_j(\x)\right)
      \varphi_i(\x) dV\\
   \label{gtrowsum} & = & \intSi\sigma(\x)\varphi_i(\x) dV\\
   &\ge& 0.\qed
\end{eqnarray*}
\end{proof}
%--------------------------------------------------------------------------------
\begin{lemma}{Diagonal Dominance}{diagonallydominant_gt}
   $\mathbf{A}^L$ is strictly diagonally dominant:
   $\left|A^L_{i,i}\right| \ge \sum\limits_{j\ne i} \left|A^L_{i,j}\right|$.
\end{lemma}
\begin{proof}
Using the inequalities $\sum\limits_j A^L_{i,j} \ge 0$ and $A^L_{i,j}\le 0, j\ne i$,
it is proven that $\mathbf{A}^L$ is strictly diagonally dominant:
\begin{eqnarray*}
	\sum\limits_j A^L_{i,j} & \ge & 0\\
	\sum\limits_{j\ne i} A^L_{i,j} + A^L_{i,i} & \ge & 0\\
	\left|A^L_{i,i}\right| & \ge & \sum\limits_{j\ne i} -A^L_{i,j}\\
	\left|A^L_{i,i}\right| & \ge & \sum\limits_{j\ne i} \left|A^L_{i,j}\right|.\qed
\end{eqnarray*}
\end{proof}
%--------------------------------------------------------------------------------
\begin{lemma}{M-Matrix}{}
   $\mathbf{A}^L$ is an M-Matrix.
\end{lemma}
\begin{proof}
To prove that a matrix is an M-Matrix, it is sufficient to prove that:
\[
\left\{\begin{array}{l}
A^L_{i,j}\le 0, j\ne i\\
A^L_{i,i}\ge 0\\
\left|A^L_{i,i}\right| \ge \sum\limits_{j\ne i} \left|A^L_{i,j}\right|\\
\end{array}
\right.,
\]
which are given by Lemmas \ref{offdiagonalnegative_gt}, \ref{diagonalpositive_gt}, and
\ref{diagonallydominant_gt}, respectively.\qed
\end{proof}
%--------------------------------------------------------------------------------
