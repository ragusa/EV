%--------------------------------------------------------------------------------
\subsection{Transient FCT Schemes}
%--------------------------------------------------------------------------------
In this section, the general FCT scheme for the transient case is presented.
The flux correction vector $\f$ is defined such that
\begin{equation}\label{fdef_semidiscrete}
   \M^L\ddt{\U^{H}} + \A^L\U^H(t) = \b(t) + \f \eqp
\end{equation}
Decomposing the correction flux into $\F$ and applying a limiter $\L$ gives
\begin{equation}\label{limited_semidiscrete}
   \M^L\ddt{\U} + \A^L\U(t) = \b(t) + \LF \eqp
\end{equation}
where $(\LF)_i = L_{i,:}F_{i,:}^T = \sumj L\ij F\ij$
and $\U(t)$ is the FCT solution. The limiter $\L$ is
defined in Section \ref{L}.
Combining Equations \eqref{fdef_semidiscrete} and \eqref{high_semidiscrete}
gives the definition of $\f$:
\begin{equation}\label{f_semidiscrete}
   \f \equiv -\pr{\M^C-\M^L}\ddt{\U^H} + \pr{\D^L-\D^H(t)}\U^H(t) \eqp
\end{equation}
Since $\M^C-\M^L$ and $\D^L-\D^H(t)$ are symmetric
and feature zero row and column sums, a valid decomposition for $\f$,
called $\F$, is
\begin{equation}
   F\ij = -M^C\ij\pr{\ddt{U_j^H} - \ddt{U_i^H}}
   + \pr{D\ij^L-D\ij^H(t)}\pr{U_j^H(t) - U_i^H(t)} \eqp
\end{equation}
%--------------------------------------------------------------------------------
\subsubsection{Explicit Euler FCT Scheme}
%--------------------------------------------------------------------------------
In this section, the explicit Euler FCT scheme is presented.
The flux correction vector $\f$ is defined such that
\begin{equation}\label{fdef_FE}
   \M^L\frac{\U^{H}-\U^n}{\dt^{n+1}} + \A^L\U^{n} = \b^n + \f \eqp
\end{equation}
Decomposing the correction flux into $\F$ and applying a limiter $\L$ gives
\begin{equation}\label{limited_FE}
   \M^L\frac{\U^{n+1}-\U^n}{\dt^{n+1}} + \A^L\U^{n} = \b^n + \LF \eqc
\end{equation}
where $(\LF)_i = L_{i,:}F_{i,:}^T = \sumj L\ij F\ij$
and $\U^{n+1}$ is the FCT solution. The limiter $\L$ is
defined in Section \ref{L}.
Combining Equations \eqref{fdef_FE} and \eqref{high_FE}
gives the definition of $\f$:
\begin{equation}\label{f_FE}
   \f \equiv -\pr{\M^C-\M^L}\frac{\U^H-\U^n}{\dt^{n+1}}
   + \pr{\D^L-\D^{H,n}}\U^n \eqp
\end{equation}
Since $\M^C-\M^L$ and $\D^L-\D^{H,n}$ are symmetric
and feature zero row and column sums, a valid decomposition for $\f$,
called $\F$, is
\begin{equation}
   F\ij = -M^C\ij\pr{\frac{U_j^H - U_j^n}{\dt^{n+1}}
   - \frac{U_i^H - U_i^n}{\dt^{n+1}}}
   + \pr{D\ij^L-D\ij^{H,n}}\pr{U_j^n - U_i^n} \eqp
\end{equation}
%--------------------------------------------------------------------------------
\subsubsection{Theta FCT Scheme}
%--------------------------------------------------------------------------------
In this section, the $\theta$ FCT scheme is presented.
The flux correction vector $\f$ is defined such that
\begin{equation}\label{fdef_theta}
  \M^L\frac{\U^H-\U^n}{\dt}
  + (1-\theta)\A^L\U^n + \theta\A^L\U^H
  = (1-\theta)\b^n + \theta\b^{n+1} + \f \eqp
\end{equation}
Decomposing the correction flux into $\F$ and applying a limiter $\L$ gives
\begin{equation}\label{limited_theta}
  \M^L\frac{\U^{n+1}-\U^n}{\dt}
  + (1-\theta)\A^L\U^n + \theta\A^L\U^{n+1}
  = (1-\theta)\b^n + \theta\b^{n+1} + \LF \eqp
\end{equation}
where $(\LF)_i = L_{i,:}F_{i,:}^T = \sumj L\ij F\ij$
and $\U^{n+1}$ is the FCT solution. The limiter $\L$ is
defined in Section \ref{L}.
Combining Equations \eqref{fdef_theta} and \eqref{high_theta}
gives the definition of $\f$:
\begin{equation}\label{f_theta}
   \f \equiv -\pr{\M^C-\M^L}\frac{\U^H-\U^n}{\dt^{n+1}}
   + (1-\theta)\pr{\D^L-\D^{H,n}}\U^n 
   + \theta    \pr{\D^L-\D^{H,n+1}}\U^H \eqp
\end{equation}
Since $\M^C-\M^L$ and $\D^L-\D^{H,n}$ are symmetric
and feature zero row and column sums, a valid decomposition for $\f$,
called $\F$, is
\begin{multline}
   F\ij = -M^C\ij\pr{\frac{U_j^H - U_j^n}{\dt^{n+1}}
   - \frac{U_i^H - U_i^n}{\dt^{n+1}}}
   + (1-\theta)\pr{D\ij^L-D\ij^{H,n}}  \pr{U_j^n - U_i^n}\\
   + \theta    \pr{D\ij^L-D\ij^{H,n+1}}\pr{U_j^H - U_i^H} \eqp
\end{multline}
%--------------------------------------------------------------------------------
%\begin{proposition}{Discrete $L^\infty$ Norm Stability}{}
   %Schemes that satisfy the bounds given in Definition \ref{FCTbounds} are
   %stable in the discrete $L^\infty$ norm.
%\end{proposition}
%
%\begin{proof}
   %Let $k$ be the index corresponding to the degree of freedom for which
   %the maximum discrete value is obtained, i.e.,
   %\[
      %U_k^{n+1}\equiv\max\limits_j U_j^{n+1}=\|\U^{n+1}\|_\infty.
   %\]
   %Starting with the upper bound given by Definition \ref{FCTbounds},
   %\[
      %\|\U^{n+1}\|_\infty = U_k^{n+1} \le
      %U_{\max,k}^n e^{-c\Delta t\sigma_{\min,k}} \le
      %U_{\max,k}^n \le
      %\max\limits_j U_j^n = \|\U^{n}\|_\infty.
   %\]
   %Applying this inequality recursively shows that the solution is bounded
   %by the initial data $\U^0$:
   %\[
      %\|\U^{n+1}\|_\infty\le
      %\|\U^{n}\|_\infty\le
      %\|\U^{n-1}\|_\infty\le\ldots\le
      %\|\U^{0}\|_\infty.\qed
   %\]
%\end{proof}
%--------------------------------------------------------------------------------
