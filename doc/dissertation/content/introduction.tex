%===============================================================================
\section{Purpose}
%===============================================================================
The solution of conservation law equations such as the neutron transport
equation represents a number of unique challenges; in the vicinity of strong
gradients and discontinuities, numerical solutions are prone to spurious
oscillations that may generate unphysical values. For example, physically
non-negative quantities such as scalar flux or angular flux may have negative
numerical solution values if care is not taken in the numerical scheme.
These negativities are not only undesirable because they are physically
incorrect, but also because their presence in a simulation or analysis
can cause drastic consequences due to the assumption of non-negativity;
for instance, properties dependent on this quantity may be computed
to be highly unphysical values, which then propagate further into
a simulation. These inaccuracies can lead to poor conclusions being drawn
from a simulation and thus lead to poor decisions being made by the analysts.

These issues of spurious oscillations and negativities are a well-known
phenomenon, and attempts to address these issues have taken a number
of different approaches for different discretization schemes.
