For implicit time discretizations, such as the Theta discretization
and steady-state, the FCT algorithm is nonlinear in general because the
solution bounds are implicit in general.
However, if one chooses solution bounds that do not depend on the FCT solution,
then the resulting system of equations is only nonlinear if both the conservation law
is nonlinear and an implicit time discretization is used.

This section considers nonlinear FCT systems in which the solution
bounds depend on the FCT solution. With the presence of
conditional statements in the computation of the limiting coefficients,
the computation of the Jacobian of the nonlinear system is complicated,
and thus instead of a Newton iteration, a defect correction scheme
such as that described in Section \ref{sec:nonlinear_iteration} is
considered here.

Suppose that in a particular iteration of a nonlinear FCT scheme, one
aims to find the FCT solution iterate $\solutionvector^{(\ell+1)}$.
In the nonlinear FCT case, the solution bounds depend on the solution:
$\solutionbound_i^\pm(\solutionvector)$, so one needs to consider which
bounds are actually being imposed in that iteration. Ideally, one would
like to impose the solution bounds
$\solutionbound_i^\pm(\solutionvector^{(\ell+1)})$; however,
$\solutionvector^{(\ell+1)}$ is not available. Therefore one must employ
previous solution iterates such as $\solutionvector^{(\ell)}$.
Thus the imposed solution bounds are the following:
\begin{equation}\label{eq:iterative_solution_bounds} 
  \solutionbound_i^-(\solutionvector^{(\ell)})
  \leq \solutionletter_i^{(\ell+1)}
  \leq \solutionbound_i^+(\solutionvector^{(\ell)})
  \eqp
\end{equation}
As shown in Section \ref{sec:antidiffusion_bounds}, the antidiffusion bounds
$\antidiffusionbound_i^\pm$ are defined by the time discretization and are
functions of the solution bounds. As an example, the steady-state
antidiffusion bounds from Equation \eqref{eq:antidiffusion_bounds_steady_state}
for an iteration would be
\begin{equation}\label{eq:antidiffusion_bounds_ideal}
  \limitedfluxbound_i^\pm =
    \ssmatrixletter_{i,i}^L \solutionbound_i^\pm(\solutionvector^{(\ell)})
    + \sumjnoti\ssmatrixletter\ij^L\solutionletter_j^{(\ell+1)} - \ssrhsletter_i
  \eqp
\end{equation}
However, this reveals an issue: the antidiffusion bounds definition depends on
$\solutionvector^{(\ell+1)}$, which again is not available in a fixed-point-type
iteration scheme.
Thus one must use another definition for the antidiffusion bounds, such as the
following for steady-state:
\begin{equation}\label{eq:antidiffusion_bounds_actual}
  \limitedfluxbound_i^{\pm,(\ell)} \equiv
    \ssmatrixletter_{i,i}^L \solutionbound_i^\pm(\solutionvector^{(\ell)})
    + \sumjnoti\ssmatrixletter\ij^L\solutionletter_j^{(\ell)} - \ssrhsletter_i
  \eqp
\end{equation}
Unfortunately, the transition from Equation
\eqref{eq:antidiffusion_bounds_ideal} to \eqref{eq:antidiffusion_bounds_actual}
implies that there is no longer a guarantee that the FCT bounds at each
iteration, given by Equation \eqref{eq:iterative_solution_bounds}, are satisfied.
Effectively, the modified antidiffusion bounds given by Equation
\eqref{eq:antidiffusion_bounds_actual} correspond to different solution
bounds:
\begin{subequations}
\begin{equation}
  \tilde{\solutionbound}_i^{-,(\ell)}
    \leq \solutionletter_i^{(\ell+1)}
    \leq \tilde{\solutionbound}_i^{+,(\ell)}
  \eqc
\end{equation}
\begin{equation}
  \tilde{\solutionbound}_i^{\pm,(\ell)} \equiv
  \solutionbound_i^\pm(\solutionvector^{(\ell)})
    + \sumjnoti\ssmatrixletter\ij^L(\solutionletter_j^{(\ell)}
      - \solutionletter_j^{(\ell+1)})
  \eqp
\end{equation}
\end{subequations}
However, upon convergence, the original solution bounds are satisfied because
upon convergence, Equations \eqref{eq:antidiffusion_bounds_ideal} and
\eqref{eq:antidiffusion_bounds_actual} are equal.

