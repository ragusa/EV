\section{Analytic Discrete Maximum Principles}
In this section, analytic discrete maximum principles (DMPs) are derived
for the case of the scalar conservation law, given by Equation
\eqref{eq:cons_law} having a constant, linear flux $\mathbf{g}(u)$, i.e.,
$\mathbf{g}(u) = \v u$ with $\nabla\cdot(\v u) = \v\cdot\nabla u$,
where $\v$ is the constant velocity field. This analysis is valid for
radiation transport, where the constant velocity field is $\v=v\O$, with
$v$ being the radiation speed ($c$ for photon transport).

The analytic DMPs are derived using the method of characteristics, whereby
paths in the $x-t$ plane are found along which the governing PDE is an ODE.
This is simple for the case of constant linear transport because in this case
the characteristics are constant.

%================================================================================
\subsection{Integral Form of the Time-Dependent Radiation Transport Equation}
%================================================================================
\begin{theorem}{Integral Form of the Time-Dependent Radiation Transport Equation}{}
   An implicit solution to the initial value problem
   \begin{equation}\label{PDE}
      \frac{\partial \psi}{\partial t} + c\mathbf{\Omega}\cdot\nabla\psi(\mathbf{x},t)
      + c\sigma(\mathbf{x})\psi(\mathbf{x},t) = c q(\mathbf{x},t),
      \qquad \psi(\mathbf{x},0) = \psi_0(\mathbf{x})
   \end{equation}
   is the following:
   \begin{equation}\label{exact}
      \psi(\mathbf{x},t) = \psi_0(\mathbf{x} - c\mathbf{\Omega}t)
         e^{-c\int\limits_0^t \sigma(\mathbf{x} - c\mathbf{\Omega}(t -t'))dt'} +
         c \int\limits_0^t q(\mathbf{x} - c\mathbf{\Omega}(t -t'),t')
         e^{-c\int\limits_{t'}^t\sigma(\mathbf{x}
         - c\mathbf{\Omega}(t -\bar{t}))d\bar{t}} dt'.
   \end{equation}
\end{theorem}

\begin{proof}
   This proof will proceed by using the method of characteristics. The position
   $\mathbf{x}$ will be regarded as a function of time: $\mathbf{x}=\mathbf{x}(t)$.
   The characteristic $\mathbf{x}(t)$ is the solution of the following initial value problem:
   \[
      \frac{d\mathbf{x}}{dt} = c\mathbf{\Omega},\qquad \mathbf{x}(0) = \mathbf{x}_0,
   \]
   which is
   \[
      \mathbf{x}(t) = \mathbf{x}_0 + c\mathbf{\Omega}t.
   \]
   Taking the derivative of $\psi(\mathbf{x}(t),t)$ gives
   \begin{eqnarray*}
      \frac{d\psi}{dt} & = &\frac{\partial\psi}{\partial t} +
      \nabla\cdot\psi(\mathbf{x}(t),t) \frac{d\mathbf{x}}{dt}\\
      & = &\frac{\partial \psi}{\partial t} +
      c\mathbf{\Omega}\cdot\nabla\psi(\mathbf{x}(t),t),
   \end{eqnarray*}
   which when combined with the PDE in Equation \eqref{PDE}, gives
   \[
      \frac{d\psi}{dt} + c\sigma(\mathbf{x}(t))\psi(\mathbf{x}(t),t) = c q(\mathbf{x},t).
   \]
   This is a 1st-order linear ODE, which may be solved using an integrating factor
   \[
      \mu(t)=e^{c\int\limits_0^t\sigma(\mathbf{x}(t'))dt'}.
   \]
   Multiplying both sides by this integrating factor and using the product rule,
   \[
      \frac{d}{dt}\left[\psi(\mathbf{x}(t),t)\mu(t)\right] = c q(\mathbf{x}(t),t) \mu(t),
   \]
   and integrating from $0$ to $t$ gives
   \[
      \psi(\mathbf{x}(t),t)\mu(t)-\psi(\mathbf{x}(0),0)\mu(0) =
         \int\limits_0^t c q(\mathbf{x}(t'),t') \mu(t') dt'.
   \]
   Simplifying,
   \[
      \psi(\mathbf{x}(t),t) = \psi(\mathbf{x}(0),0)
         e^{-c\int\limits_0^t \sigma(\mathbf{x}(t'))dt'} +
         \left(c \int\limits_0^t q(\mathbf{x}(t'),t')
         e^{c\int\limits_0^{t'}\sigma(\mathbf{x}(\bar{t}))d\bar{t}} dt'\right)
         e^{-c\int\limits_0^t\sigma(\mathbf{x}(t'))dt'},
   \]
   \[
      \psi(\mathbf{x}(t),t) = \psi(\mathbf{x}(0),0)
         e^{-c\int\limits_0^t \sigma(\mathbf{x}(t'))dt'} +
         c \int\limits_0^t q(\mathbf{x}(t'),t')
         e^{-c\int\limits_{t'}^t\sigma(\mathbf{x}(\bar{t}))d\bar{t}} dt'.
   \]
   Finally, expressing $\mathbf{x}(t)$ in terms of $\mathbf{x}$, $\mathbf{\Omega}$,
   and $t$ gives
   \[
      \psi(\mathbf{x},t) = \psi_0(\mathbf{x} - c\mathbf{\Omega}t)
         e^{-c\int\limits_0^t \sigma(\mathbf{x} - c\mathbf{\Omega}(t -t'))dt'} +
         c \int\limits_0^t q(\mathbf{x} - c\mathbf{\Omega}(t -t'),t')
         e^{-c\int\limits_{t'}^t\sigma(\mathbf{x}
         - c\mathbf{\Omega}(t -\bar{t}))d\bar{t}} dt'.\qed
   \]
\end{proof}
%--------------------------------------------------------------------------------
\subsection{Integral Form of the Steady-State Radiation Transport Equation}
%================================================================================
\begin{theorem}{Integral Form of the Steady-State Radiation Transport Equation}{}
   An implicit solution to the equation
   \begin{equation}\label{PDEss}
      \mathbf{\Omega}\cdot\nabla\psi(\mathbf{x})
      + \sigma(\mathbf{x})\psi(\mathbf{x}) = q(\mathbf{x})
   \end{equation}
   is the following:
   \begin{equation}\label{exactss}
      \psi(\mathbf{x}) = \psi(\mathbf{x} - s\mathbf{\Omega})
         e^{-\int\limits_0^s \sigma(\mathbf{x} - \mathbf{\Omega}(s -s'))ds'} +
         \int\limits_0^s q(\mathbf{x} - \mathbf{\Omega}(s -s'))
         e^{-\int\limits_{s'}^s\sigma(\mathbf{x}
         - \mathbf{\Omega}(s -\bar{s}))d\bar{s}} ds'.
   \end{equation}
\end{theorem}

\begin{proof}
   This proof will proceed by using the method of characteristics. The position
   $\mathbf{x}$ will be regarded as a function of position $s$ along the line
   with direction $\mathbf{\Omega}$ that passes through $\mathbf{x}$:
   \[
      \mathbf{x}(s) = \mathbf{x}_0 + \mathbf{\Omega}s.
   \]
   Using the chain rule,
   \[
      \frac{d\psi}{ds} = \nabla\psi(\mathbf{x}(s)) \cdot \frac{d\mathbf{x}}{ds}
         = \mathbf{\Omega} \cdot \nabla\psi(\mathbf{x}(s)),
   \]
   which when combined with the PDE in Equation \eqref{PDEss}, gives
   \[
      \frac{d\psi}{ds} + \sigma(\mathbf{x}(s))\psi(\mathbf{x}(s)) = q(\mathbf{x}(s)).
   \]
   This is a 1st-order linear ODE, which may be solved using an integrating factor
   \[
      \mu(s)=e^{\int\limits_0^s\sigma(\mathbf{x}(s'))ds'}.
   \]
   Multiplying both sides by this integrating factor and using the product rule,
   \[
      \frac{d}{ds}\left[\psi(\mathbf{x}(s))\mu(s)\right] = q(\mathbf{x}(s)) \mu(s),
   \]
   and integrating from $0$ to $s$ gives
   \[
      \psi(\mathbf{x}(s))\mu(s)-\psi(\mathbf{x}(0))\mu(0) =
         \int\limits_0^s q(\mathbf{x}(s')) \mu(s') ds'.
   \]
   Simplifying,
   \[
      \psi(\mathbf{x}(s)) = \psi(\mathbf{x}(0))
         e^{-\int\limits_0^s \sigma(\mathbf{x}(s'))ds'} +
         \left(\int\limits_0^s q(\mathbf{x}(s'))
         e^{\int\limits_0^{s'}\sigma(\mathbf{x}(\bar{s}))d\bar{s}} ds'\right)
         e^{-\int\limits_0^s\sigma(\mathbf{x}(s'))ds'},
   \]
   \[
      \psi(\mathbf{x}(s)) = \psi(\mathbf{x}(0))
         e^{-\int\limits_0^s \sigma(\mathbf{x}(s'))ds'} +
         \int\limits_0^s q(\mathbf{x}(s'))
         e^{-\int\limits_{s'}^s\sigma(\mathbf{x}(\bar{s}))d\bar{s}} ds'.
   \]
   Finally, expressing $\mathbf{x}(s)$ in terms of $\mathbf{x}$, $\mathbf{\Omega}$,
   and $s$ gives
   \[
      \psi(\mathbf{x}) = \psi(\mathbf{x} - \mathbf{\Omega}s)
         e^{-\int\limits_0^s \sigma(\mathbf{x} - \mathbf{\Omega}(s -s'))ds'} +
         \int\limits_0^s q(\mathbf{x} - \mathbf{\Omega}(s -s'))
         e^{-\int\limits_{s'}^s\sigma(\mathbf{x}
         - \mathbf{\Omega}(s -\bar{s}))d\bar{s}} ds'.\qed
   \]
\end{proof}
%--------------------------------------------------------------------------------
\subsection{Continuous Maximum Principles}
%================================================================================
\begin{theorem}{Continuous Maximum Principle}{cont}
   The following continuous maximum principle is valid for the solution to the
   problem given by Equation \eqref{PDE}:
   \begin{equation}
      \psi_{\min,N}^0 e^{-c\tau\sigma_{\max,N}} + 
            \frac{q_{\min,N}}{\sigma_{\max,N}}(1 - e^{-c\sigma_{\max,N}\tau})
      \le\psi(\mathbf{x},\tau)\le
      \psi_{\max,N}^0 e^{-c\tau\sigma_{\min,N}} + 
            \frac{q_{\max,N}}{\sigma_{\min,N}}(1 - e^{-c\sigma_{\min,N}\tau}).
   \end{equation}
   where $\psi_{\max,N}^0\equiv\max\limits_{\mathbf{y}\in N(\mathbf{x})}\psi(\mathbf{y},0)$,
   $\sigma_{\max,N}\equiv\max\limits_{\mathbf{y}\in N(\mathbf{x})}\sigma(\mathbf{y})$, with
   $\psi_{\min,N}^0$ and $\sigma_{\min,N}$ defined similarly, and the neighborhood $N$ is a
   sphere centered at $\mathbf{x}$ with radius $c\tau$:
   \begin{equation}
      N(\mathbf{x})\equiv\left\{\mathbf{y}\in\mathbb{R}^d : 
         \|\mathbf{y} - \mathbf{x}\| \le c\tau\right\}.
   \end{equation}
\end{theorem}

\begin{proof}
   Rewriting Equation \eqref{exact} with $t=\tau$ gives
   \[
      \psi(\mathbf{x},\tau) = \psi_0(\mathbf{x} - c\mathbf{\Omega}\tau)
         e^{-c\int\limits_0^\tau \sigma(\mathbf{x} - c\mathbf{\Omega}(\tau -t'))dt'} +
         c \int\limits_0^\tau q(\mathbf{x} - c\mathbf{\Omega}(\tau -t'),t')
         e^{-c\int\limits_{t'}^\tau\sigma(\mathbf{x}
         - c\mathbf{\Omega}(\tau -\bar{t}))d\bar{t}} dt'.
   \]
   Let $L(\mathbf{x})$ be the line segment that spans between 
   $\mathbf{x}-c\mathbf{\Omega}\tau$ and $\mathbf{x}$:
   \[
      L(\mathbf{x})\equiv \left\{\mathbf{y}\in\mathbb{R}^d : \mathbf{y}
         = \mathbf{x}-ct\mathbf{\Omega},\qquad t\in(0,\tau) \right\}.
   \]
   One can bound the first term in the right hand side of Equation \eqref{exactss}
   as follows:
   \[
      \psi_{\min,L}^0 e^{-c\tau\sigma_{\max,L}} \le
      \psi_0(\mathbf{x} - c\mathbf{\Omega}\tau)
         e^{-c\int\limits_0^\tau \sigma(\mathbf{x} - c\mathbf{\Omega}(\tau -t'))dt'} \le
      \psi_{\max,L}^0 e^{-c\tau\sigma_{\min,L}},
   \]
   where $\psi_{\max,L}^0 \equiv\max\limits_{\mathbf{y}\in L(\mathbf{x})}\psi_0(\mathbf{y})$,
   $\sigma_{\max,L}\equiv\max\limits_{\mathbf{y}\in L(\mathbf{x})}\sigma(\mathbf{y})$,
   with $\psi_{\min,L}^0$ and $\sigma_{\min,L}$
   defined similarly.
   The source term can be bounded as follows:
   \begin{eqnarray*}
      c \int\limits_0^\tau q(\mathbf{x} - c\mathbf{\Omega}(\tau -t'),t')
         e^{-c\int\limits_{t'}^\tau\sigma(\mathbf{x}
         - c\mathbf{\Omega}(\tau -\bar{t}))d\bar{t}} dt' & \le &
         c q_{\max,L}\int\limits_0^\tau 
         e^{-c\int\limits_{t'}^\tau\sigma(\mathbf{x}
         - c\mathbf{\Omega}(\tau -\bar{t}))d\bar{t}} dt'\\
      & \le & c q_{\max,L}\int\limits_0^\tau 
         e^{-c\sigma_{\min,L}\int\limits_{t'}^\tau d\bar{t}} dt'\\
      & = & c q_{\max,L} \int\limits_0^\tau e^{-c\sigma_{\min,L}(\tau-t')} dt'\\
      & = & c q_{\max,L}e^{-c\sigma_{\min,L}\tau}
         \int\limits_0^\tau e^{c\sigma_{\min,L}t'} dt'\\
      & = & \left\{\begin{array}{l l}
            \frac{q_{\max,L}}{\sigma_{\min,L}}(1 - e^{-c\sigma_{\min,L}\tau})
               & \sigma_{\min,L} \ne 0\\
            c\tau q_{\max,L} & \sigma_{\min,L} = 0
            \end{array}\right.
   \end{eqnarray*}
   A similar analysis is performed for the lower bound. Putting everything together,
   the following is a maximum principle on $L(\mathbf{x})$:
   \[
      \psi_{\min,L}^0 e^{-c\tau\sigma_{\max,L}} + 
            \frac{q_{\min,L}}{\sigma_{\max,L}}(1 - e^{-c\sigma_{\max,L}\tau})
      \le\psi(\mathbf{x},\tau)\le
      \psi_{\max,L}^0 e^{-c\tau\sigma_{\min,L}} + 
            \frac{q_{\max,L}}{\sigma_{\min,L}}(1 - e^{-c\sigma_{\min,L}\tau}).
   \]
   Since $L(\mathbf{x})\subset N(\mathbf{x})$, the following is true:
   \[
      \psi_{\min,N}^0 e^{-c\tau\sigma_{\max,N}} + 
            \frac{q_{\min,N}}{\sigma_{\max,N}}(1 - e^{-c\sigma_{\max,N}\tau})
      \le\psi(\mathbf{x},\tau)\le
      \psi_{\max,N}^0 e^{-c\tau\sigma_{\min,N}} + 
            \frac{q_{\max,N}}{\sigma_{\min,N}}(1 - e^{-c\sigma_{\min,N}\tau}).\qed
   \]
\end{proof}
%--------------------------------------------------------------------------------
%\begin{theorem}{Upwind Continuous Maximum Principle}{}
   %The following continuous maximum principle is valid for the solution to the
   %problem given by Equation \eqref{PDE}:
   %\begin{equation}
      %\psi_{\min,N^-}^0 e^{-c\tau\sigma_{\max,N^-}} \le
      %\psi(\mathbf{x},\tau) \le
      %\psi_{\max,N^-}^0 e^{-c\tau\sigma_{\min,N^-}},
   %\end{equation}
   %where $\psi_{\max,N^-}^0\equiv\max\limits_{\mathbf{y}\in N^-(\mathbf{x})}
   %\psi(\mathbf{y},0)$, $\sigma_{\max,N^-}\equiv\max\limits_{\mathbf{y}\in N^-(\mathbf{x})}
   %\sigma(\mathbf{y})$, with $\psi_{\min,N^-}^0$ and $\sigma_{\min,N^-}$
   %defined similarly, and $N^-(\mathbf{x})$ is the upwind subset of $N(\mathbf{x})$:
   %\begin{equation}
      %N^-(\mathbf{x})\equiv\left\{\mathbf{y}\in N : (\mathbf{y} - \mathbf{x})\cdot \mathbf{\Omega} \le 0\right\}.
   %\end{equation}
%\end{theorem}
%\begin{proof}
   %The proof proceeds exactly as for Theorem \ref{cont}, except replacing $N$ with $N^-$.
%\end{proof}
