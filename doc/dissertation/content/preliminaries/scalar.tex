The primary application of this research is radiation transport;
however, most of the analysis performed is valid
for any conservation law of the following form:
\begin{equation}\label{eq:scalar_transport}
   \ppt{\scalarsolution} + \divergence\consfluxscalar
   + \reactioncoef\xt \scalarsolution\xt = \scalarsource\xt \eqc
\end{equation}
where $\scalarsolution\xt$ is a general scalar conserved quantity at position
$\x$ and time $\timevalue$, $\consfluxscalar$ is a general flux
function,
$\reactioncoef\xt$ is a reaction term, and $\scalarsource\xt$ is a source
term. This notation will be used throughout this document to keep
the analysis as general as possible; radiation transport notation
will only be adopted when assumptions are needed.

A Scalar radiation transport equation,
\begin{equation}\label{eq:rad_transport}
  \frac{1}{\speed}\ppt{\angularflux} + \directionvector\cdot\nabla\angularflux\xt
  + \totalcrosssection(\x)\angularflux\xt = \radiationsource\xt \eqc
\end{equation}
fits the conservation law model of Equation \eqref{eq:scalar_transport} by
making the following substitutions:
\[
  \consfluxscalar\rightarrow\velocity\scalarsolution
  \eqc \quad
  \velocity\rightarrow\speed\directionvector
  \eqc \quad
  \scalarsolution\rightarrow\angularflux
  \eqc \quad
  \reactioncoef\rightarrow\speed\totalcrosssection
  \eqc \quad
  \scalarsource\rightarrow\speed\radiationsource
  \eqc
\]
where $\angularflux\xt$ is the angular flux in direction $\directionvector$,
$\speed$ is the transport speed, $\totalcrosssection(\x)$
is the macroscopic total cross-section, and $\radiationsource\xt$ is the
total source (extraneous plus scattering).

To complete the problem formulation, boundary
conditions must be provided, as well as initial conditions if the
problem is transient:
\begin{equation}
   \scalarsolution(\x,0) = \scalarsolution^0(\x)
   \quad \forall \x\in\domain \eqp
\end{equation}
Boundary conditions will depend on the chosen conservation law and
the particular problem. 
For linear transport a well-posed problem can be completed with an incoming flux
boundary condition:
\begin{equation}
   \scalarsolution\xt = \scalarsolution^{inc}\xt \quad \forall \x
   \in \domainboundary^-,
     \quad \domainboundary^- = \{\x\in\domainboundary:
     \velocity\cdot\normalvector(\x)<0\} \eqp
\end{equation}
For nonlinear conservation laws, care must be taken to ensure that the
boundary conditions used result in a well-posed problem.
