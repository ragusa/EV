The main focus of this research is on radiation transport;
however, most of the analysis performed is valid
for any general conservation law of the following form:
\begin{equation}\label{eq:cons_law}
   \ppt{\scalarsolution} + \divergence\consfluxscalar
   + \reactioncoef\xt \scalarsolution\xt = \scalarsource\xt \eqc
\end{equation}
where $\scalarsolution\xt$ is a general scalar conserved quantity at position
$\x$ and time $\timevalue$, $\consfluxscalar$ is a general flux
function,
$\reactioncoef\xt$ is a reaction term, and $\scalarsource\xt$ is a source
term. This notation will be used throughout this document to keep
the analysis as general as possible; radiation transport notation
will only be adopted when assumptions are needed. This is often the case when
the assumption of a constant, linear flux function $\consfluxscalar$ is needed.
In this case, the constant velocity field is denoted by $\velocity$:
$\consfluxscalar=\velocity\scalarsolution$.

For radiation transport, the conservation law equation is the following:
\begin{equation}\label{eq:rad_transport}
  \frac{1}{\speed}\ppt{\angularflux} + \directionvector\cdot\nabla\angularflux\xt
  + \totalcrosssection(\x)\angularflux\xt = \radiationsource\xt \eqc
\end{equation}
where $\angularflux\xt$ is the angular flux in direction $\directionvector$,
$\speed$ is the transport speed, $\totalcrosssection(\x)$
is the macroscopic total cross-section, and $\radiationsource\xt$ is the
total source (extraneous plus scattering).

To complete the problem formulation, one must provide boundary
conditions and, for transient problems, intitial conditions:
\begin{equation}
   \scalarsolution\xt = \scalarsolution^0(\x)
   \quad \forall \x\in\domain \eqp
\end{equation}
Boundary conditions will depend on the chosen conservation law and
the particular problem. 
For radiation transport, and any conservation law for which the velocity field
is constant, a well-posed problem can be completed with an incoming flux
boundary condition:
\begin{equation}
   \scalarsolution\xt = \scalarsolution^{inc}\xt \quad \forall \x
   \in \domainboundary^-,
     \quad \domainboundary^- = \{\x\in\domainboundary:
     \velocity\cdot\normalvector(\x)<0\} \eqp
\end{equation}
