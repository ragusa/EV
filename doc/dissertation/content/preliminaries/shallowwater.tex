The shallow water equations, also known as the Saint-Venant equations, are an
approximation of conservation of mass and momentum equations applied to free
surface flows, which assume the fluid to be incompressible, non-viscous, and
non-heat-conducting\cite{toro2009}. The shallow water equations are derived
by making the additional
approximation that the vertical component of acceleration can be neglected due to
horizontal length scales being much greater than the depth length
scale and then depth-integrating the conservation equations:
\cite{toro2009}\cite{leveque2002}\cite{fjordholm2011}:
\begin{equation}\label{eq:shallow_water_equations}
\begin{gathered}
  \ppt{\vectorsolution} + \nabla\cdot\consfluxvector(\vectorsolution)
  = \conssource(\vectorsolution) \eqc
\\
  \vectorsolution
    = \left[\begin{array}{c}
        \height\\
        \heightmomentumx\\
        \heightmomentumy
      \end{array}\right]
  \eqc\quad
  \consfluxvector(\vectorsolution)
  = \left[\begin{array}{c c}
      \heightmomentumx & \heightmomentumy\\
      \frac{\heightmomentumx^2}{\height} + \half\gravity\height^2
        & \frac{\heightmomentumx\heightmomentumy}{\height}\\
      \frac{\heightmomentumx\heightmomentumy}{\height}
        & \frac{\heightmomentumy^2}{\height} + \half\gravity\height^2\\
    \end{array}\right]
  \eqc\quad
  \conssource(\vectorsolution)
  = \left[\begin{array}{c}
      0\\
     -\gravity\height\pd{\bathymetry}{x}\\
     -\gravity\height\pd{\bathymetry}{y}\\
    \end{array}\right]
  \eqc
\end{gathered}
\end{equation}
written more concisely as
\[
  \vectorsolution
    = \left[\begin{array}{c}\height\\\heightmomentum\end{array}\right]
  \eqc\quad
  \consfluxvector(\vectorsolution)
  = \left[\begin{array}{c}\heightmomentum\\
      \frac{\heightmomentum\otimes\heightmomentum}{\height}
      + \half\gravity\height^2\identity
    \end{array}\right]
  \eqc\quad
  \conssource(\vectorsolution)
  = \left[\begin{array}{c}0\\-\gravity\height\nabla\bathymetry\end{array}
    \right] \eqc
\]
where $\height$ is the height of the water, which plays the role of density
in the continuity equation, $\heightmomentum=\height\velocity$ is sometimes
referred to as \emph{discharge} and plays the role of momentum (hereafter,
$\heightmomentum$ will usually just be referred to as ``momentum''),
$\velocity$ is velocity, $\gravity$
is acceleration due to gravity, and $\bathymetry$ is the topography of the
bottom terrain of the fluid body, hereafter referred to as the \emph{bathymetry}
function.
Note that the shallow water equations are only valid in 1-D or 2-D, not 3-D,
since they are depth-integrated equations.

The shallow water equations (SWE) are a popular model for flows in lakes, rivers,
irrigation channels, and ocean shores, and thus are of great interest
in hydrology, oceanography, and climate modeling\cite{bernetti2008}
\cite{fjordholm2011}.

Initial conditions are included if the problem is transient:
\begin{equation}
   \vectorsolution(\x,0) = \vectorsolution^0(\x)
   \quad \forall \x\in\domain \eqp
\end{equation}
To complete the problem formulation, boundary
conditions must be provided, some examples being
Dirichlet boundary, open boundary, wall boundary, etc.
One must be careful with specifying boundary conditions to have
a well-posed problem for hyperbolic systems.
In general a characteristic analysis is required; there is a large body of research
addressing this area alone. For simplicity, problems in this work are
chosen such that initial data never reaches the boundary
or boundary conditions are implemented as natural conditions
rather than using the method of characteristics.

