The shallow water equations, also known as the Saint-Venant equations, are an
approximation of conservation of mass and momentum equations applied to free
surface flows, which assume the fluid to be incompressible, non-viscous, and
non-heat-conducting. The shallow water equations make the additional
approximation that the depth component of acceleration can be neglected due to
horizontal length scales being much greater than the depth length
scale. Depth-integrating the conservation equations gives the shallow
water equations\cite{toro2009}\cite{leveque2002}:
\begin{equation}
\begin{gathered}
  \ppt{\vectorsolution} + \nabla\cdot\consfluxvector
  = \conssource \eqc
\\
  \vectorsolution
    = \left[\begin{array}{c}
        \height\\
        \heightmomentumx\\
        \heightmomentumy
      \end{array}\right]
  \eqc\quad
  \consfluxvector
  = \left[\begin{array}{c c}
      \heightmomentumx & \heightmomentumy\\
      \frac{\heightmomentumx^2}{\height} + \half\gravity\height^2
        & \frac{\heightmomentumx\heightmomentumy}{\height}\\
      \frac{\heightmomentumx\heightmomentumy}{\height}
        & \frac{\heightmomentumy^2}{\height} + \half\gravity\height^2\\
    \end{array}\right]
  \eqc\quad
  \conssource
  = \left[\begin{array}{c}
      0\\
     -\gravity\height\pd{\bathymetry}{x}\\
     -\gravity\height\pd{\bathymetry}{y}\\
    \end{array}\right]
  \eqc
\end{gathered}
\end{equation}
written more concisely as
\[
  \vectorsolution
    = \left[\begin{array}{c}\height\\\heightmomentum\end{array}\right]
  \eqc\quad
  \consfluxvector
  = \left[\begin{array}{c}\heightmomentum\\
      \frac{\heightmomentum\otimes\heightmomentum}{\height}
      + \half\gravity\height^2\identity
    \end{array}\right]
  \eqc\quad
  \conssource
  = \left[\begin{array}{c}0\\-\gravity\height\nabla\bathymetry\end{array}
    \right] \eqc
\]
where $\height$ is the height of the water, which plays the role of density
in the continuity equation, $\heightmomentum=\height\velocity$ is sometimes
referred to as \emph{discharge} and plays the role of momentum (hereafter,
$\heightmomentum$ will usually just be referred to as ``momentum''),
$\velocity$ is velocity, $\gravity$
is acceleration due to gravity, and $\bathymetry$ is the topography of the
bottom terrain of the fluid body, hereafter referred to as the \emph{bathymetry}
function.

Note that the shallow water equations are only valid in 1-D or 2-D, not 3-D,
since they are depth-integrated equations.

Viscous regularization of the shallow water equations is achieved by adding
artificial viscous fluxes:
\begin{equation}
\begin{gathered}
  \ppt{\vectorsolution} + \nabla\cdot\consfluxvector
  + \nabla\cdot\viscconsfluxvector
  = \conssource \eqc
\\
  \viscconsfluxvector
  \equiv \left[\begin{array}{c}
    \viscflux{\height}(\vectorsolution,\viscosity)\\
    \viscflux{\heightmomentum}(\vectorsolution,\viscosity)
    \end{array}\right]
  \equiv \left[\begin{array}{c}
    \viscosity\nabla\height\\
    \viscosity\nabla\heightmomentum
    \end{array}\right] \eqc
\end{gathered}
\end{equation}
where $\viscosity$ represents the viscosity employed by the particular
scheme, either low-order viscosity or entropy viscosity, which are discussed in
Sections \ref{sec:shallowwater_low_order_viscosity} and
\ref{sec:shallowwater_entropy_viscosity}, respectively.
