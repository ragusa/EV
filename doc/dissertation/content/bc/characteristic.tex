In 1-D, the boundary integrals in Equations \eqref{eq:shallowwater_height_weak_form}
and \eqref{eq:shallowwater_momentumx_weak_form} reduce to differences between
the right and left boundaries:
\begin{subequations}
\begin{equation}\label{eq:boundary_height_1d}
  \intboundary{\testfunction_i^\height\tilde{\heightmomentum}\cdot\normalvector}
  = \pr{\height\velocityx}_R - \pr{\height\velocityx}_L \eqc 
\end{equation}
\begin{equation}\label{eq:boundary_momentumx_1d}
  \intboundary{\testfunction_i^{\heightmomentumx}
    \pr{\frac{\tilde{\heightmomentumx}}{\height}\approximate{\heightmomentum}
    + \frac{1}{2}\gravity\tilde{\height}^2\unitvector{x}}\cdot\normalvector}
  = \pr{\height\velocityx^2 + \half\gravity\height^2}_R
    -\pr{\height\velocityx^2 + \half\gravity\height^2}_L
  \eqc
\end{equation}
\end{subequations}
where $R$ and $L$ denote right and left boundaries, respectively.

In 1-D, there are 2 characteristics:
\begin{subequations}
\begin{equation}
  d\velocityx - 2d\speedofsound = 0\eqc
\end{equation}
\begin{equation}
  d\velocityx + 2d\speedofsound = 0 \eqc
\end{equation}
\end{subequations}
giving the Riemann invariants $\velocityx - 2\speedofsound$ and
$\velocityx + 2\speedofsound$,
which correspond to the eigenvalues $\lambda_1=\velocityx - \speedofsound$ and
$\lambda_2=\velocityx + \speedofsound$, respectively.
Integrating the characteristics from the boundary position $x\BC$ to an interior
position $x\interior$ gives
\begin{subequations}
\begin{equation}\label{eq:characteristic_bc_1}
  \velocityx\interior - \velocityx\BC
  = 2\pr{\speedofsound\interior - \speedofsound\BC} \eqc
\end{equation}
\begin{equation}\label{eq:characteristic_bc_2}
  \velocityx\interior - \velocityx\BC
  = 2\pr{\speedofsound\BC - \speedofsound\interior} \eqc
\end{equation}
\end{subequations}
associated with $\lambda_1$ and $\lambda_2$, respectively.
At each boundary, one
must determine whether the waves associated with each eigenvalue are coming
into the domain or going out of the domain; this determines how many external
boundary conditions must be applied at each boundary.
The Froude number $\mbox{Fr}\equiv\frac{|\velocityx|}{\speedofsound}$,
along with the sign of the velocity $\velocityx$, determines the sign
of each of the 2 eigenvalues, which is summarized in Table
\ref{tab:shallowwater_eigenvalue_signs}.

\begin{mytable}{Signs of Eigenvalues for Different Cases}
{shallowwater_eigenvalue_signs}{l c c}
{\textbf{Froude Sign} & $\velocityx<0$ & $\velocityx\geq 0$}
  $\mbox{Fr} < 1$ & $\lambda_1\leq 0$ & $\lambda_1< 0$\\
                  & $\lambda_2> 0$    & $\lambda_2\geq 0$\\\hline
  $\mbox{Fr} \geq 1$ & $\lambda_1\leq 0$ & $\lambda_1\geq 0$\\
                     & $\lambda_2\leq 0$    & $\lambda_2\geq 0$\\
\end{mytable}

If $\lambda_i\normalx \leq 0$, then an external boundary condition must be
applied for the $i$-wave; otherwise, internal information is used for that
boundary condition.

For subcritical flow, i.e., $|\velocityx|\leq\speedofsound$,
the signs of each eigenvalue are $\lambda_1\leq 0$ and $\lambda_2\geq 0$.
For supercritical flow, the signs are $\lambda_1\geq 0$
and $\lambda_2\geq 0$ for $\velocityx<0$, and for $\velocityx\geq 0$,
the signs are $\lambda_1\leq 0$ and $\lambda_2\leq 0$. Thus for supercritical
flow, inlets, i.e., boundaries for which $\velocityx\normalx<0$, require
2 external boundary conditions, whereas outlets use 2 internal boundary
conditions. Tables \ref{tab:shallowwater_open_bc} and \ref{tab:shallowwater_wall_bc}
summarize the
application of open and wall boundary conditions, respectively.

\begin{mytable}
{Summary of Open Boundary Conditions for the 1-D Shallow Water Equations}
{shallowwater_open_bc}{l l}{\textbf{Case} & \textbf{Equations}}
  Subcritical Left Boundary &
    Provide $\speedofsound\BC$ \\
  (Inlet or Outlet)         &
    $\velocityx\BC = \velocityx\interior
      + 2\pr{\speedofsound\BC - \speedofsound\interior}$ \\\hline
  Subcritical Right Boundary &
    Provide $\speedofsound\BC$ \\
  (Inlet or Outlet)          &
    $\velocityx\BC = \velocityx\interior
      + 2\pr{\speedofsound\interior - \speedofsound\BC}$ \\\hline
  Supercritical Inlet &
    Provide $\speedofsound\BC$ \\
  & Provide $\velocityx\BC$ \\\hline
  Supercritical Outlet &
    $\speedofsound\BC = \speedofsound\interior$ \\
  & $\velocityx\BC = \velocityx\interior$ \\
\end{mytable}

\begin{mytable}
{Summary of Wall Boundary Conditions for the 1-D Shallow Water Equations}
{shallowwater_wall_bc}{l l}{\textbf{Case} & \textbf{Equations}}
  Subcritical Left Boundary &
    Provide $\velocityx\BC = 0$\\
  (Inlet or Outlet)         &
    $\speedofsound\BC = \speedofsound\interior
      + \half\pr{\velocityx\BC - \velocityx\interior}$\\\hline
  Subcritical Right Boundary &
    Provide $\velocityx\BC=0$\\
  (Inlet or Outlet)          &
    $\speedofsound\BC = \speedofsound\interior
      + \half\pr{\velocityx\interior - \velocityx\BC}$ \\\hline
  Supercritical Inlet &
    Provide $\velocityx\BC$\\
  & Provide $\speedofsound\BC$\\\hline
  Supercritical Outlet &
    $\velocityx\BC = \velocityx\interior$ \\
  & $\speedofsound\BC = \speedofsound\interior$ \\
\end{mytable}
