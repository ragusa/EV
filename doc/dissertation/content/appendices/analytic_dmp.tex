%================================================================================
\section{Introduction}
%================================================================================
In this section, an analytic local maximum principle is derived for
scalar conservation laws having a constant, linear flux $\consfluxscalar$,
i.e., $\consfluxscalar = \velocity u$ with $\nabla\cdot(\velocity u) =
\velocity\cdot\nabla u$, where $\velocity$ is the constant velocity field. This
analysis is valid for radiation transport, where the constant velocity field is
$\velocity=\speed\directionvector$, with $\speed$ being the radiation speed.

The analytic DMPs are derived using the method of characteristics, whereby
paths in the $x-t$ plane are found, along which the governing PDE becomes an ODE.
This is simple for the case of constant linear transport because in this case
the characteristics are constant.

%================================================================================
\section{Integral Form of the Linear Transport Equation}
%================================================================================
\begin{theorem}{Integral Form of the Linear Transport Equation}{}
   An implicit solution to the initial value problem
   \begin{equation}\label{PDE}
     \frac{1}{\speed}\ppt{\scalarsolution}
       + \directionvector\cdot\nabla\scalarsolution(\x,t)
        + \reactioncoef(\x)\scalarsolution(\x,t)
        = \scalarsource(\x,t),
     \qquad \scalarsolution(\x,0) = \scalarsolution_0(\x)
   \end{equation}
   is the following:
   \begin{equation}\label{eq:integral_form}
      \scalarsolution(\x,t) = \scalarsolution_0(\x - \speed t\directionvector)
         e^{-\int\limits_0^t \reactioncoef(\x - \speed(t -t')\directionvector)\speed dt'} +
         \int\limits_0^t \scalarsource(\x - \speed(t -t')\directionvector,t')
           e^{-\int\limits_{t'}^t\reactioncoef(\x
             - \speed(t -\bar{t})\directionvector)\speed d\bar{t}} \speed dt' \eqp
   \end{equation}
\end{theorem}

\begin{proof}
   This proof will proceed by using the method of characteristics. The position
   $\x$ will be regarded as a function of time: $\x=\x(t)$.
   The characteristic $\x(t)$ is the solution of the following initial value problem:
   \[
      \frac{d\x}{dt} = \speed\directionvector \eqc \qquad \x(0) = \x_0 \eqc
   \]
   which is
   \[
      \x(t) = \x_0 + \speed t\directionvector \eqp
   \]
   Taking the derivative of $\scalarsolution(\x(t),t)$ gives
   \begin{eqnarray*}
      \frac{d\scalarsolution}{dt} & = &\ppt{\scalarsolution}
        + \nabla\cdot\scalarsolution(\x(t),t) \frac{d\x}{dt}\\
        & = & \ppt{\scalarsolution}
        + \speed\directionvector\cdot\nabla\scalarsolution(\x(t),t) \eqc
   \end{eqnarray*}
   which when combined with the PDE in Equation \eqref{PDE}, gives
   \begin{equation}\label{eq:pre_integrating_factor}
      \ddt{\scalarsolution} + \speed\reactioncoef(\x(t))\scalarsolution(\x(t),t)
        = \speed\scalarsource(\x(t),t) \eqp
   \end{equation}
   This is a 1st-order linear ODE, which may be solved using an integrating factor
   \[
      \mu(t)=e^{\int\limits_0^t\reactioncoef(\x(t'))\speed dt'}.
   \]
   Multiplying both sides of Equation \eqref{eq:pre_integrating_factor}
   by this integrating factor and using the product rule,
   \[
      \frac{d}{dt}\left[\scalarsolution(\x(t),t)\mu(t)\right]
        = \speed\scalarsource(\x(t),t) \mu(t) \eqc
   \]
   and integrating from $0$ to $t$ gives
   \[
      \scalarsolution(\x(t),t)\mu(t)-\scalarsolution(\x(0),0)\mu(0) =
         \int\limits_0^t \scalarsource(\x(t'),t') \mu(t') \speed dt' \eqp
   \]
   Simplifying,
   \[\begin{split}
      \scalarsolution(\x(t),t) &= \scalarsolution(\x(0),0)
         e^{-\int\limits_0^t \reactioncoef(\x(t'))\speed dt'} +
         \left(\int\limits_0^t \scalarsource(\x(t'),t')
           e^{\int\limits_0^{t'}\reactioncoef(\x(\bar{t}))\speed d\bar{t}}
           \speed dt'\right)
         e^{-\int\limits_0^t\speed\reactioncoef(\x(t'))dt'} \eqc\\
      &= \scalarsolution(\x(0),0)
         e^{-\int\limits_0^t \reactioncoef(\x(t'))\speed dt'} +
         \int\limits_0^t \scalarsource(\x(t'),t')
           e^{-\int\limits_{t'}^t\reactioncoef(\x(\bar{t}))\speed d\bar{t}}
           \speed dt' \eqp
   \end{split}\]
   Finally, expressing $\x(t)$ in terms of $\x$, $\speed$, $\directionvector$,
   and $t$ gives
   \begin{multline*}
      \scalarsolution(\x,t) = \scalarsolution_0(\x - \speed t\directionvector)
         e^{-\int\limits_0^t \reactioncoef(\x - \speed(t -t')\directionvector)
           \speed dt'}\\
         + \int\limits_0^t \scalarsource(\x - \speed(t -t')\directionvector,t')
           e^{-\int\limits_{t'}^t\reactioncoef(\x
             - \speed(t -\bar{t})\directionvector)\speed d\bar{t}}
             \speed dt'\eqp \qed
   \end{multline*}
\end{proof}
%================================================================================
\section{Local Maximum Principles\label{sec:local_max_principles}}
%================================================================================
Before giving an analytic local discrete maximum principle, a local maximum
principle applying to a general region is given by the following theorem.

\begin{theorem}[thm:analytic_max_principle]{Analytic Local Maximum Principle}
   Let $L(\x,\tau)$ be the line segment that spans between
   $\x-\speed\tau\directionvector$ and $\x$:
   \begin{equation}
      L(\x,\tau)\equiv \left\{\mathbf{y}\in\mathbb{R}^d : \mathbf{y}
         = \x-\speed t\directionvector \eqc \qquad t\in(0,\tau) \right\} \eqp
   \end{equation}
   See Figure \ref{fig:neighborhood} for an illustration.
   The following local maximum principle is valid for the solution to the
   problem given by Equation \eqref{PDE}:
   \begin{subequations}\label{eq:local_max_principle}
   \begin{equation}
      \scalarsolution_{\text{min}} \le \scalarsolution(\x,\tau)
        \le \scalarsolution_{\text{max}} \eqc
   \end{equation}
   \begin{equation}
      \scalarsolution_{\text{min}}
        \equiv \left\{\begin{array}{l l}
          %\scalarsolution_{\min,N}^0
          \scalarsolution_0(\x - \speed \tau\directionvector)
             e^{-\speed\tau\reactioncoef_{\max,L}}
            + \frac{\scalarsource_{\min,L}}{\reactioncoef_{\max,L}}
             (1 - e^{-\speed\tau\reactioncoef_{\max,L}}) \eqc
          & \reactioncoef_{\max,L} \ne 0 \\
          %\scalarsolution_{\min,L}^0
          \scalarsolution_0(\x - \speed \tau\directionvector)
            + \speed\tau\scalarsource_{\min,L} \eqc
          & \reactioncoef_{\max,L} = 0
        \end{array}\right.\eqc
   \end{equation}
   \begin{equation}
      \scalarsolution_{\text{max}}
        \equiv \left\{\begin{array}{l l}
          %\scalarsolution_{\max,N}^0
          \scalarsolution_0(\x - \speed \tau\directionvector)
            e^{-\speed\tau\reactioncoef_{\min,L}}
            + \frac{\scalarsource_{\max,L}}{\reactioncoef_{\min,L}}
            (1 - e^{-\speed\tau\reactioncoef_{\min,L}}) \eqc
          & \reactioncoef_{\min,L} \ne 0 \\
          %\scalarsolution_{\max,N}^0
          \scalarsolution_0(\x - \speed \tau\directionvector)
            + \speed\tau\scalarsource_{\max,L} \eqc
          & \reactioncoef_{\min,L} = 0
        \end{array}\right.\eqc
   \end{equation}
   \begin{equation}
     \reactioncoef_{\min,L}\equiv\min\limits_{\mathbf{y}\in L(\x,\tau)}
       \reactioncoef(\mathbf{y}) \eqc \quad
     \reactioncoef_{\max,L}\equiv\max\limits_{\mathbf{y}\in L(\x,\tau)}
       \reactioncoef(\mathbf{y}) \eqc
   \end{equation}
   \begin{equation}
     \scalarsource_{\min,L}\equiv\min\limits_{\mathbf{y}\in L(\x,\tau)}
       \scalarsource(\mathbf{y}) \eqc \quad
     \scalarsource_{\max,L}\equiv\max\limits_{\mathbf{y}\in L(\x,\tau)}
       \scalarsource(\mathbf{y}) \eqp
   \end{equation}
   \end{subequations}
\end{theorem}
%-------------------------------------------------------------------------------
\begin{figure}[htb]
   \centering
     \begin{tikzpicture}[
  scale=1]

\def\pointsize{2pt}
\def\radius{2}

\coordinate (mycenter) at (0,0);
\fill (mycenter) circle (\pointsize);
\draw[draw=red, fill=red, fill opacity=0.25] (mycenter) circle (\radius);
\draw (mycenter) node[below] {$\x$};
\draw[-latex] (mycenter) -- (-30:\radius) node[pos=0.5,sloped,above] {$\speed\tau$};
\draw[blue,line width=1.5pt] (mycenter) -- (210:\radius)
  node[pos=0.5,sloped,above] {$L(\x,\tau)$};
\draw[dashed] (mycenter) -- (30:\radius);
\node[red] at ($0.5*(0,\radius)$) {$N(\x,\tau)$};
\coordinate (dircenter) at (1.2*\radius,0.3*\radius);
\draw[|-latex,shift=(dircenter)] (0,0) -- (30:0.75)
  node[pos=0.5,sloped,above] {$\di$};

\end{tikzpicture}

      \caption{Illustration of Neighborhoods $L(\x,\tau)$ and $N(\x,\tau)$}
   \label{fig:neighborhood}
\end{figure}
%-------------------------------------------------------------------------------
\begin{proof}
   Rewriting Equation \eqref{eq:integral_form} with $t=\tau$ gives
   \begin{multline*}
      \scalarsolution(\x,\tau) = \scalarsolution_0(\x - \speed\tau\directionvector)
         e^{-\int\limits_0^\tau \reactioncoef(\x
           - \speed(\tau -t')\directionvector)\speed dt'}\\
         +
         \int\limits_0^\tau \scalarsource(\x - \speed(\tau -t')\directionvector,t')
         e^{-\int\limits_{t'}^\tau \reactioncoef(\x
         - \speed(\tau -\bar{t})\directionvector)\speed d\bar{t}}\speed dt' \eqp
   \end{multline*}
   One can bound the first term in the right-hand-side of Equation
   \eqref{eq:integral_form} by considering
   the maximum and minimum cross section on the line segment $L(\x,\tau)$
   for the lower and upper bounds, respectively:
   \[
      %\scalarsolution_{\min,L}^0
      \scalarsolution_0(\x - \speed\tau\directionvector)
        e^{-\speed\tau\reactioncoef_{\max,L}} \le
      \scalarsolution_0(\x - \speed\tau\directionvector)
        e^{-\int\limits_0^\tau \reactioncoef(\x
           - \speed(\tau -t')\directionvector)\speed dt'} \le
      %\scalarsolution_{\max,L}^0
      \scalarsolution_0(\x - \speed\tau\directionvector)
        e^{-\speed\tau\reactioncoef_{\min,L}} \eqp
   \]
   The source term can be bounded as follows:
   \begin{align*}
     \scalarsolution_\scalarsource
      & \equiv
         \int\limits_0^\tau \scalarsource(\x - \speed(\tau -t')\directionvector,t')
         e^{-\int\limits_{t'}^\tau \reactioncoef(\x
           - \speed(\tau -\bar{t})\directionvector)\speed d\bar{t}} \speed dt'\\
      & \le
         \scalarsource_{\max,L}\int\limits_0^\tau
         e^{-\int\limits_{t'}^\tau \reactioncoef(\x
           - \speed(\tau -\bar{t})\directionvector)\speed d\bar{t}}\speed dt'\\
      & \le
         \scalarsource_{\max,L}\int\limits_0^\tau
         e^{-\reactioncoef_{\min,L}\int\limits_{t'}^\tau \speed d\bar{t}}\speed dt'\\
      & =
         \scalarsource_{\max,L} \int\limits_0^\tau
         e^{-\speed(\tau-t')\reactioncoef_{\min,L}}\speed dt'\\
      & =
         \scalarsource_{\max,L}e^{-\speed\tau\reactioncoef_{\min,L}}
         \int\limits_0^\tau e^{\reactioncoef_{\min,L}\speed t'}\speed dt'\\
      & =
         \left\{\begin{array}{l l}
            \frac{\scalarsource_{\max,L}}{\reactioncoef_{\min,L}}
              (1 - e^{-\speed\tau\reactioncoef_{\min,L}}) \eqc
               & \reactioncoef_{\min,L} \ne 0\\
            \speed\tau \scalarsource_{\max,L} \eqc & \reactioncoef_{\min,L} = 0
            \end{array}\right.
   \end{align*}
   A similar analysis is performed for the lower bound.
   Putting the two components together gives the bounds given by Equation
   \eqref{eq:local_max_principle}.\qed
\end{proof}
%-------------------------------------------------------------------------------

This result gives relatively tight solution bounds; however, its use as
solution bounds for FCT may prove difficult in practice (especially for
multi-dimensional problems), as one must
compute the solution at the point $\x-\speed\tau\di$ and must be able
to evaluate the minimum and maximum of the reaction coefficients and
sources on the line segment $L(\x_i,\tau)$.
The following corollary loosens the solution bounds for use in a
more simple implementation of solution bounds for FCT. It considers
not just the upstream line segment of length $\speed\tau$, but the
sphere of radius $\speed\tau$ centered at $\x_i$.

%-------------------------------------------------------------------------------
\begin{corollary}[cly:loose_analytic_max_principle]
  {Loose Analytic Local Maximum Principle}
Let $N(\x,\tau)$ denote the sphere centered at $\x$ with radius
$\speed\tau$, as shown in Figure \ref{fig:neighborhood}:
   \begin{equation}\label{eq:neighborhood}
      N(\x,\tau)\equiv\left\{\mathbf{y}\in\mathbb{R}^d :
         \|\mathbf{y} - \x\| \le \speed\tau\right\} \eqp
   \end{equation}
The following, looser, local maximum principle is valid for the solution to the
problem given by Equation \eqref{PDE}:
\begin{subequations}\label{eq:loose_local_max_principle}
   \begin{equation}
      \scalarsolution_{\text{min}} \le \scalarsolution(\x,\tau)
        \le \scalarsolution_{\text{max}} \eqc
   \end{equation}
   \begin{equation}
      \scalarsolution_{\text{min}}
        \equiv \left\{\begin{array}{l l}
          \scalarsolution_{\min,N}^0 e^{-\speed\tau\reactioncoef_{\max,N}}
            + \frac{\scalarsource_{\min,N}}{\reactioncoef_{\max,N}}
             (1 - e^{-\speed\tau\reactioncoef_{\max,N}}) \eqc
          & \reactioncoef_{\max,N} \ne 0 \\
          \scalarsolution_{\min,N}^0
            + \speed\tau\scalarsource_{\min,N} \eqc
          & \reactioncoef_{\max,N} = 0
        \end{array}\right.\eqc
   \end{equation}
   \begin{equation}
      \scalarsolution_{\text{max}}
        \equiv \left\{\begin{array}{l l}
          \scalarsolution_{\max,N}^0 e^{-\speed\tau\reactioncoef_{\min,N}}
            + \frac{\scalarsource_{\max,N}}{\reactioncoef_{\min,N}}
            (1 - e^{-\speed\tau\reactioncoef_{\min,N}}) \eqc
          & \reactioncoef_{\min,N} \ne 0 \\
          \scalarsolution_{\max,N}^0
            + \speed\tau\scalarsource_{\max,N} \eqc
          & \reactioncoef_{\min,N} = 0
        \end{array}\right.\eqc
   \end{equation}
   \begin{equation}
     \scalarsolution_{\min,N}^0 \equiv \min\limits_{\mathbf{y}\in N(\x,\tau)}
       \scalarsolution(\mathbf{y},0) \eqc \quad
     \scalarsolution_{\max,N}^0 \equiv \max\limits_{\mathbf{y}\in N(\x,\tau)}
       \scalarsolution(\mathbf{y},0) \eqc
   \end{equation}
   \begin{equation}
     \reactioncoef_{\min,L}\equiv\min\limits_{\mathbf{y}\in L(\x,\tau)}
       \reactioncoef(\mathbf{y}) \eqc \quad
     \reactioncoef_{\max,L}\equiv\max\limits_{\mathbf{y}\in L(\x,\tau)}
       \reactioncoef(\mathbf{y}) \eqc
   \end{equation}
   \begin{equation}
     \scalarsource_{\min,L}\equiv\min\limits_{\mathbf{y}\in L(\x,\tau)}
       \scalarsource(\mathbf{y}) \eqc \quad
     \scalarsource_{\max,L}\equiv\max\limits_{\mathbf{y}\in L(\x,\tau)}
       \scalarsource(\mathbf{y}) \eqp
   \end{equation}
\end{subequations}
\end{corollary}
%-------------------------------------------------------------------------------
\begin{proof}
Because $\x-\speed\tau\di\in N(\x,\tau)$,
\[
  \scalarsolution_0(\x - \speed\tau\directionvector) \geq \scalarsolution_{\min,N}^0
  \eqc \quad
  \scalarsolution_0(\x - \speed\tau\directionvector) \leq \scalarsolution_{\max,N}^0
  \eqp
\]
Because $L(\x,\tau)\subset N(\x,\tau)$ (see Figure \ref{fig:neighborhood}),
the following is true:
\[
  \scalarsource_{\min,N} \leq \scalarsource_{\min,L}
  \eqc \quad
  \scalarsource_{\max,N} \geq \scalarsource_{\max,L}
  \eqc
\]
\[
  \reactioncoef_{\min,N} \leq \reactioncoef_{\min,L}
  \eqc \quad
  \reactioncoef_{\max,N} \geq \reactioncoef_{\max,L}
  \eqp
\]
Applying these inequalities to Equation \eqref{eq:local_max_principle}
proves Equation \eqref{eq:loose_local_max_principle}.\qed
\end{proof}
%-------------------------------------------------------------------------------

The following theorem applies Corollary \ref{cly:loose_analytic_max_principle} to derive
an analytic discrete maximum principle for radiation transport.

%-------------------------------------------------------------------------------
\begin{theorem}[thm:analytic_dmp]{Analytic Discrete Maximum Principle}
If the time step size $\dt$ satisfies the condition
\begin{equation}\label{eq:cfl_analytic_dmp}
  \speed\dt \leq \celldiameter_{\min} \eqc
    \quad \celldiameter_{\min} \equiv \min\limits_K \celldiameter_K \eqc
\end{equation}
where $\celldiameter_K$ is the diameter of cell $K$, then Theorem
\ref{thm:analytic_max_principle} gives the following analytic discrete maximum
principle.
\begin{subequations}\label{eq:analyticDMP}
  \begin{equation}
      \analyticDMPlowerbound_i \le \solutionletter_i^{n+1}
        \le \analyticDMPupperbound_i \eqc
  \end{equation}
  \begin{equation}
      \analyticDMPlowerbound_i
        \equiv \left\{\begin{array}{l l}
          \solutionletter_{\min,i}^n e^{-\speed\dt\reactioncoef_{\max,i}}
            + \frac{\scalarsource_{\min,i}}{\reactioncoef_{\max,i}}
            (1 - e^{-\speed\dt\reactioncoef_{\max,i}}) \eqc
          & \reactioncoef_{\max,i} \ne 0 \\
          \solutionletter_{\min,i}^n
            + \speed\dt\scalarsource_{\min,i} \eqc
          & \reactioncoef_{\max,i} = 0
        \end{array}\right.\eqc
  \end{equation}
  \begin{equation}
      \analyticDMPupperbound_i
        \equiv \left\{\begin{array}{l l}
          \solutionletter_{\max,i}^n e^{-\speed\dt\reactioncoef_{\min,i}}
            + \frac{\scalarsource_{\max,i}}{\reactioncoef_{\min,i}}
            (1 - e^{-\speed\dt\reactioncoef_{\min,i}}) \eqc
          & \reactioncoef_{\min,i} \ne 0 \\
          \solutionletter_{\max,i}^n
            + \speed\dt\scalarsource_{\max,i} \eqc
          & \reactioncoef_{\min,i} = 0
        \end{array}\right.\eqc
  \end{equation}
\end{subequations}
where $\solutionletter_{\max,i}^n
  \equiv\max\limits_{j\in\indices(\support_i)}\solutionletter_j^n$,
$\reactioncoef_{\max,i}
  \equiv\max\limits_{\x\in\support_i}\reactioncoef(\x)$, and
$\scalarsource_{\max,i}
  \equiv\max\limits_{\x\in\support_i}\scalarsource(\x)$,
with $\solutionletter_{\min,i}^n$, $\reactioncoef_{\min,i}$,
and $\scalarsource_{\min,i}$ defined similarly.
\end{theorem}
\begin{proof}
Due to the CFL condition, Equation \eqref{eq:cfl_analytic_dmp}, the support of
test function $i$ is a superset of the neighborhood $N(\x_i)$ defined by
Equation \eqref{eq:neighborhood}: $N(\x_i)\subset\support_i$. Thus
for an arbitrary function of space $f(\x)$,
\[
  \max\limits_{\x\in\support_i}f(\x)
    \geq \max\limits_{\x\in N(\x_i)}f(\x) \eqc \quad
  \min\limits_{\x\in\support_i}f(\x)
    \leq \min\limits_{\x\in N(\x_i)}f(\x) \eqc
\]
and
\[
  \approximatescalarsolution_{\max,\support_i}
    \geq \approximatescalarsolution_{\max,N} \eqc \quad
  \approximatescalarsolution_{\min,\support_i}
    \leq \approximatescalarsolution_{\min,N} \eqp
\]
Since $\approximatescalarsolution$ is a convex combination of nodal solution
values, the local extremum are obtained only at nodal values:
\[
  \approximatescalarsolution_{\max,\support_i}
    = \solutionletter_{\max,i} \eqc \quad
  \approximatescalarsolution_{\min,\support_i}
    = \solutionletter_{\min,i} \eqp \qed
\]
\end{proof}

The following corollary extends the analytic discrete maximum principle given
in Theorem \ref{thm:analytic_dmp} to the steady-state case and is given
without proof, as it follows the same logic as Theorem \ref{thm:analytic_dmp}.

\begin{corollary}{Analytic Steady-State Discrete Maximum Principle}
If one uses a parameter $s$ such that $s \leq \celldiameter_{\min}$, where
$\celldiameter_{\min}$ is defined by Equation \eqref{eq:cfl_analytic_dmp}, then
the following analytic discrete maximum principle bounds apply to the
steady-state problem:
\begin{subequations}\label{eq:analyticDMP_ss}
  \begin{equation}
      \analyticDMPlowerbound_i \le \solutionletter_i
        \le \analyticDMPupperbound_i \eqc
  \end{equation}
  \begin{equation}
      \analyticDMPlowerbound_i
        \equiv \left\{\begin{array}{l l}
          \solutionletter_{\min,i} e^{-s\reactioncoef_{\max,i}}
            + \frac{\scalarsource_{\min,i}}{\reactioncoef_{\max,i}}
            (1 - e^{-s\reactioncoef_{\max,i}}) \eqc
          & \reactioncoef_{\max,i} \ne 0 \\
          \solutionletter_{\min,i}
            + s\scalarsource_{\min,i} \eqc
          & \reactioncoef_{\max,i} = 0
        \end{array}\right.\eqc
  \end{equation}
  \begin{equation}
      \analyticDMPupperbound_i
        \equiv \left\{\begin{array}{l l}
          \solutionletter_{\max,i} e^{-s\reactioncoef_{\min,i}}
            + \frac{\scalarsource_{\max,i}}{\reactioncoef_{\min,i}}
            (1 - e^{-s\reactioncoef_{\min,i}}) \eqc
          & \reactioncoef_{\min,i} \ne 0 \\
          \solutionletter_{\max,i}
            + s\scalarsource_{\max,i} \eqc
          & \reactioncoef_{\min,i} = 0
        \end{array}\right.\eqp
  \end{equation}
\end{subequations}
\end{corollary}

\begin{remark}
In practice, one can approximate the maximum/minimum operations
by taking the maximum/minimum over quadrature points: e.g.,
$\max\limits_{\x\in\support_i} \approx \max\limits_{\x\in Q(\support_i)}$,
where $Q(\support_i)$ is the set of quadrature points in $\support_i$.
\end{remark}
