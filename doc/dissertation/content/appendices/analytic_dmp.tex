%================================================================================
\section{Introduction}
%================================================================================
In this section, an analytic discrete maximum principle (DMP) is derived for
scalar conservation laws having a constant, linear flux $\consfluxscalar$,
i.e., $\consfluxscalar = \velocity u$ with $\nabla\cdot(\velocity u) =
\velocity\cdot\nabla u$, where $\velocity$ is the constant velocity field. This
analysis is valid for radiation transport, where the constant velocity field is
$\velocity=\speed\directionvector$, with $\speed$ being the radiation speed.

The analytic DMPs are derived using the method of characteristics, whereby
paths in the $x-t$ plane are found, along which the governing PDE becomes an ODE.
This is simple for the case of constant linear transport because in this case
the characteristics are constant.

%================================================================================
\section{Integral Form of the Linear Transport Equation}
%================================================================================
\begin{theorem}{Integral Form of the Linear Transport Equation}{}
   An implicit solution to the initial value problem
   \begin{equation}\label{PDE}
      \ppt{\scalarsolution} + \velocity\cdot\nabla\scalarsolution(\x,t)
      + \speed\reactioncoef(\x)\scalarsolution(\x,t) = \speed\scalarsource(\x,t),
      \qquad \scalarsolution(\x,0) = \scalarsolution_0(\x)
   \end{equation}
   is the following:
   \begin{equation}\label{eq:integral_form}
      \scalarsolution(\x,t) = \scalarsolution_0(\x - \velocity t)
         e^{-\int\limits_0^t \speed\reactioncoef(\x - \velocity(t -t'))dt'} +
         \int\limits_0^t \speed\scalarsource(\x - \velocity(t -t'),t')
         e^{-\int\limits_{t'}^t\speed\reactioncoef(\x
         - \velocity(t -\bar{t}))d\bar{t}} dt' \eqp
   \end{equation}
\end{theorem}

\begin{proof}
   This proof will proceed by using the method of characteristics. The position
   $\x$ will be regarded as a function of time: $\x=\x(t)$.
   The characteristic $\x(t)$ is the solution of the following initial value problem:
   \[
      \frac{d\x}{dt} = \velocity,\qquad \x(0) = \x_0,
   \]
   which is
   \[
      \x(t) = \x_0 + \velocity t.
   \]
   Taking the derivative of $\scalarsolution(\x(t),t)$ gives
   \begin{eqnarray*}
      \frac{d\scalarsolution}{dt} & = &\ppt{\scalarsolution}
        + \nabla\cdot\scalarsolution(\x(t),t) \frac{d\x}{dt}\\
        & = & \ppt{\scalarsolution}
        + \velocity\cdot\nabla\scalarsolution(\x(t),t) \eqc
   \end{eqnarray*}
   which when combined with the PDE in Equation \eqref{PDE}, gives
   \[
      \ddt{\scalarsolution} + \speed\reactioncoef(\x(t))\scalarsolution(\x(t),t)
        = \speed\scalarsource(\x,t).
   \]
   This is a 1st-order linear ODE, which may be solved using an integrating factor
   \[
      \mu(t)=e^{\int\limits_0^t\speed\reactioncoef(\x(t'))dt'}.
   \]
   Multiplying both sides by this integrating factor and using the product rule,
   \[
      \frac{d}{dt}\left[\scalarsolution(\x(t),t)\mu(t)\right]
        = \speed\scalarsource(\x(t),t) \mu(t),
   \]
   and integrating from $0$ to $t$ gives
   \[
      \scalarsolution(\x(t),t)\mu(t)-\scalarsolution(\x(0),0)\mu(0) =
         \int\limits_0^t \speed\scalarsource(\x(t'),t') \mu(t') dt'.
   \]
   Simplifying,
   \[
      \scalarsolution(\x(t),t) = \scalarsolution(\x(0),0)
         e^{-\int\limits_0^t \speed\reactioncoef(\x(t'))dt'} +
         \left(\int\limits_0^t \speed\scalarsource(\x(t'),t')
         e^{\int\limits_0^{t'}\speed\reactioncoef(\x(\bar{t}))d\bar{t}} dt'\right)
         e^{-\int\limits_0^t\speed\reactioncoef(\x(t'))dt'},
   \]
   \[
      \scalarsolution(\x(t),t) = \scalarsolution(\x(0),0)
         e^{-\int\limits_0^t \speed\reactioncoef(\x(t'))dt'} +
         \int\limits_0^t \speed\scalarsource(\x(t'),t')
         e^{-\int\limits_{t'}^t\speed\reactioncoef(\x(\bar{t}))d\bar{t}} dt'.
   \]
   Finally, expressing $\x(t)$ in terms of $\x$, $\velocity$,
   and $t$ gives
   \[
      \scalarsolution(\x,t) = \scalarsolution_0(\x - \velocity t)
         e^{-\int\limits_0^t \speed\reactioncoef(\x - \velocity(t -t'))dt'} +
         \int\limits_0^t \speed\scalarsource(\x - \velocity(t -t'),t')
         e^{-\int\limits_{t'}^t\speed\reactioncoef(\x
         - \velocity(t -\bar{t}))d\bar{t}} dt'.\qed
   \]
\end{proof}
%================================================================================
\section{Local Maximum Principles}
%================================================================================
Before giving an analytic local discrete maximum principle, a local maximum
principle applying to a general region is given by the following theorem.

\begin{theorem}{Analytic Local Maximum Principle}
   The following local maximum principle is valid for the solution to the
   problem given by Equation \eqref{PDE}:
   \begin{subequations}\label{eq:local_max_principle}
   \begin{equation}
      \scalarsolution_{\text{min}} \le \scalarsolution(\x,\tau)
        \le \scalarsolution_{\text{max}} \eqc
   \end{equation}
   \begin{equation}
      \scalarsolution_{\text{min}}
        \equiv \left\{\begin{array}{l l}
          \scalarsolution_{\min,N}^0 e^{-\tau\speed\reactioncoef_{\max,N}}
            + \frac{\scalarsource_{\min,N}}{\reactioncoef_{\max,N}}
             (1 - e^{-\speed\reactioncoef_{\max,N}\tau}) \eqc
          & \reactioncoef_{\max,N} \ne 0 \\
          \scalarsolution_{\min,N}^0
            + \tau\speed\scalarsource_{\min,N} \eqc
          & \reactioncoef_{\max,N} = 0
        \end{array}\right.\eqc
   \end{equation}
   \begin{equation}
      \scalarsolution_{\text{max}}
        \equiv \left\{\begin{array}{l l}
          \scalarsolution_{\max,N}^0 e^{-\tau\speed\reactioncoef_{\min,N}}
            + \frac{\scalarsource_{\max,N}}{\reactioncoef_{\min,N}}
            (1 - e^{-\speed\reactioncoef_{\min,N}\tau}) \eqc
          & \reactioncoef_{\min,N} \ne 0 \\
          \scalarsolution_{\max,N}^0
            + \tau\speed\scalarsource_{\max,N} \eqc
          & \reactioncoef_{\min,N} = 0
        \end{array}\right.\eqc
   \end{equation}
   \end{subequations}
   where $\scalarsolution_{\max,N}^0\equiv\max\limits_{\mathbf{y}\in N(\x)}
   \scalarsolution(\mathbf{y},0)$,
   $\reactioncoef_{\max,N}\equiv\max\limits_{\mathbf{y}\in N(\x)}
   \reactioncoef(\mathbf{y})$, and
   $\scalarsource_{\max,N}\equiv\max\limits_{\mathbf{y}\in N(\x)}
   \scalarsource(\mathbf{y})$, 
   with $\scalarsolution_{\min,N}^0$, $\reactioncoef_{\min,N}$, and
   $\scalarsource_{\min,N}$ defined similarly, and the neighborhood $N$ is a
   sphere centered at $\x$ with radius $\speed\tau$:
   \begin{equation}
      N(\x)\equiv\left\{\mathbf{y}\in\mathbb{R}^d : 
         \|\mathbf{y} - \x\| \le \speed\tau\right\}.
   \end{equation}
\end{theorem}

\begin{proof}
   Rewriting Equation \eqref{eq:integral_form} with $t=\tau$ gives
   \[
      \scalarsolution(\x,\tau) = \scalarsolution_0(\x - \velocity\tau)
         e^{-\int\limits_0^\tau \speed\reactioncoef(\x - \velocity(\tau -t'))dt'} +
         \int\limits_0^\tau \speed\scalarsource(\x - \velocity(\tau -t'),t')
         e^{-\int\limits_{t'}^\tau\speed\reactioncoef(\x
         - \velocity(\tau -\bar{t}))d\bar{t}} dt'.
   \]
   Let $L(\x)$ be the line segment that spans between 
   $\x-\velocity\tau$ and $\x$:
   \[
      L(\x)\equiv \left\{\mathbf{y}\in\mathbb{R}^d : \mathbf{y}
         = \x-\velocity t,\qquad t\in(0,\tau) \right\}.
   \]
   One can bound the first term in the right hand side of Equation
   \eqref{eq:integral_form} as follows:
   \[
      \scalarsolution_{\min,L}^0 e^{-\tau\speed\reactioncoef_{\max,L}} \le
      \scalarsolution_0(\x - \velocity\tau)
         e^{-\int\limits_0^\tau \speed\reactioncoef(\x - \velocity(\tau -t'))dt'} \le
      \scalarsolution_{\max,L}^0 e^{-\tau\speed\reactioncoef_{\min,L}},
   \]
   where $\scalarsolution_{\max,L}^0 \equiv\max\limits_{\mathbf{y}\in L(\x)}
   \scalarsolution_0(\mathbf{y})$,
   $\reactioncoef_{\max,L}\equiv\max\limits_{\mathbf{y}\in L(\x)}
   \reactioncoef(\mathbf{y})$, and
   $\scalarsource_{\max,L}\equiv\max\limits_{\mathbf{y}\in L(\x)}
   \scalarsource(\mathbf{y})$,
   with $\scalarsolution_{\min,L}^0$, $\reactioncoef_{\min,L}$,
   and $\scalarsource_{\min,L}$ defined similarly.
   The source term can be bounded as follows:
   \begin{eqnarray*}
      \int\limits_0^\tau \speed\scalarsource(\x - \velocity(\tau -t'),t')
         e^{-\int\limits_{t'}^\tau\speed\reactioncoef(\x
         - \velocity(\tau -\bar{t}))d\bar{t}} dt' & \le &
         \speed\scalarsource_{\max,L}\int\limits_0^\tau 
         e^{-\int\limits_{t'}^\tau\speed\reactioncoef(\x
         - \velocity(\tau -\bar{t}))d\bar{t}} dt'\\
      & \le & \speed\scalarsource_{\max,L}\int\limits_0^\tau 
         e^{-\speed\reactioncoef_{\min,L}\int\limits_{t'}^\tau d\bar{t}} dt'\\
      & = & \speed\scalarsource_{\max,L} \int\limits_0^\tau
         e^{-\speed\reactioncoef_{\min,L}(\tau-t')} dt'\\
      & = & \speed\scalarsource_{\max,L}e^{-\speed\reactioncoef_{\min,L}\tau}
         \int\limits_0^\tau e^{\speed\reactioncoef_{\min,L}t'} dt'\\
      & = & \left\{\begin{array}{l l}
            \frac{\scalarsource_{\max,L}}{\reactioncoef_{\min,L}}
              (1 - e^{-\speed\reactioncoef_{\min,L}\tau})
               & \reactioncoef_{\min,L} \ne 0\\
            \tau \speed\scalarsource_{\max,L} & \reactioncoef_{\min,L} = 0
            \end{array}\right.
   \end{eqnarray*}
   A similar analysis is performed for the lower bound. Putting everything together,
   the following is a maximum principle on $L(\x)$:
   \[
      \scalarsolution_{\text{min}} \le \scalarsolution(\x,\tau)
        \le \scalarsolution_{\text{max}} \eqc
   \]
   \[
      \scalarsolution_{\text{min}}
        \equiv \left\{\begin{array}{l l}
          \scalarsolution_{\min,L}^0 e^{-\tau\speed\reactioncoef_{\max,L}}
            + \frac{\scalarsource_{\min,L}}{\reactioncoef_{\max,L}}
             (1 - e^{-\speed\reactioncoef_{\max,L}\tau}) \eqc
          & \reactioncoef_{\max,L} \ne 0 \\
          \scalarsolution_{\min,L}^0
            + \tau\speed\scalarsource_{\min,L} \eqc
          & \reactioncoef_{\max,L} = 0
        \end{array}\right.\eqc
   \]
   \[
      \scalarsolution_{\text{max}}
        \equiv \left\{\begin{array}{l l}
          \scalarsolution_{\max,L}^0 e^{-\tau\speed\reactioncoef_{\min,L}}
            + \frac{\scalarsource_{\max,L}}{\reactioncoef_{\min,L}}
            (1 - e^{-\speed\reactioncoef_{\min,L}\tau}) \eqc
          & \reactioncoef_{\min,L} \ne 0 \\
          \scalarsolution_{\max,L}^0
            + \tau\speed\scalarsource_{\max,L} \eqc
          & \reactioncoef_{\min,L} = 0
        \end{array}\right.\eqp
   \]
   Since $L(\x)\subset N(\x)$, the following is true:
   \[
      \scalarsolution_{\text{min}} \le \scalarsolution(\x,\tau)
        \le \scalarsolution_{\text{max}} \eqc
   \]
   \[
      \scalarsolution_{\text{min}}
        \equiv \left\{\begin{array}{l l}
          \scalarsolution_{\min,N}^0 e^{-\tau\speed\reactioncoef_{\max,N}}
            + \frac{\scalarsource_{\min,N}}{\reactioncoef_{\max,N}}
             (1 - e^{-\speed\reactioncoef_{\max,N}\tau}) \eqc
          & \reactioncoef_{\max,N} \ne 0 \\
          \scalarsolution_{\min,N}^0
            + \tau\speed\scalarsource_{\min,N} \eqc
          & \reactioncoef_{\max,N} = 0
        \end{array}\right.\eqc
   \]
   \[
      \scalarsolution_{\text{max}}
        \equiv \left\{\begin{array}{l l}
          \scalarsolution_{\max,N}^0 e^{-\tau\speed\reactioncoef_{\min,N}}
            + \frac{\scalarsource_{\max,N}}{\reactioncoef_{\min,N}}
            (1 - e^{-\speed\reactioncoef_{\min,N}\tau}) \eqc
          & \reactioncoef_{\min,N} \ne 0 \\
          \scalarsolution_{\max,N}^0
            + \tau\speed\scalarsource_{\max,N} \eqc
          & \reactioncoef_{\min,N} = 0
        \end{array}\right.\eqp \qed
   \]
\end{proof}
If one takes the parameter $\tau$ in Equation \eqref{eq:local_max_principle}
to be the time step size $\dt$, the initial data $\scalarsolution^0$ to
be the previous time solution $\approximatescalarsolution^n$, and
the region $N$ to be the support of test function $i$, then the following
local discrete maximum principle results:
\begin{subequations}\label{eq:analyticDMP}
  \begin{equation}
      \analyticDMPlowerbound_i \le \solutionletter_i^{n+1}
        \le \analyticDMPupperbound_i \eqc
  \end{equation}
  \begin{equation}
      \analyticDMPlowerbound_i
        \equiv \left\{\begin{array}{l l}
          \solutionletter_{\text{min},i}^n e^{-\dt\speed\reactioncoef_{\max,i}}
            + \frac{\scalarsource_{\min,i}}{\reactioncoef_{\max,i}}
            (1 - e^{-\speed\reactioncoef_{\max,i}\dt}) \eqc
          & \reactioncoef_{\max,i} \ne 0 \\
          \solutionletter_{\text{min},i}^n
            + \dt\speed\scalarsource_{\min,i} \eqc
          & \reactioncoef_{\max,i} = 0
        \end{array}\right.\eqc
  \end{equation}
  \begin{equation}
      \analyticDMPupperbound_i
        \equiv \left\{\begin{array}{l l}
          \solutionletter_{\text{max},i}^n e^{-\dt\speed\reactioncoef_{\min,i}}
            + \frac{\scalarsource_{\max,i}}{\reactioncoef_{\min,i}}
            (1 - e^{-\speed\reactioncoef_{\min,i}\dt}) \eqc
          & \reactioncoef_{\min,i} \ne 0 \\
          \solutionletter_{\text{max},i}^n
            + \dt\speed\scalarsource_{\max,i} \eqc
          & \reactioncoef_{\min,i} = 0
        \end{array}\right.\eqc
  \end{equation}
\end{subequations}
where $\solutionletter_{\text{max},i}^n
  \equiv\max\limits_{j\in\indices(\support_i)}\solutionletter_j^n$,
$\reactioncoef_{\text{max},i}
  \equiv\max\limits_{\x\in\support_i}\reactioncoef(\x)$, and
$\scalarsource_{\text{max},i}
  \equiv\max\limits_{\x\in\support_i}\scalarsource(\x)$,
with $\solutionletter_{\text{min},i}^n$, $\reactioncoef_{\text{max},i}$,
and $\scalarsource_{\text{max},i}$ defined similarly.
\begin{remark}
In practice, one can approximate the maximum/minimum operations
by taking the maximum/minimum over quadrature points: e.g.,
$\max\limits_{\x\in\support_i} \approx \max\limits_{\x\in Q(\support_i)}$,
where $Q(\support_i)$ is the set of quadrature points in $\support_i$.
\end{remark}

