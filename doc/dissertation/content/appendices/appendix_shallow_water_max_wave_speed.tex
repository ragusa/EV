\chapter{DERIVATION OF THE MAX WAVE SPEED FOR THE SHALLOW WATER EQUATIONS
\label{app:shallow_water_max_wave_speed}}

The maximum wave speed is the maximum of the absolute values of the
left-most and right-most wave speeds:
\begin{equation}
  \maxwavespeed(\vectorsolution_L, \vectorsolution_R)
  = \max(|\wavespeed_1^-(\vectorsolution_L, \vectorsolution_R)|
  ,|\wavespeed_2^+(\vectorsolution_L, \vectorsolution_R)|)
  \eqc
\end{equation}
where the ``+'' and ``-'' allow for the differentiation of the head
and tail speeds in the case of a rarefaction.

For the 1-D shallow water equations, the Riemann problem divides the
$x$-$\timevalue$ plane into 3 sectors, separated by 2 waves, which
each may be either a shock or rarefaction. The left sector shall be
denoted with ``L'', the middle with ``*'', and the right with ``R''.
The left-most and right-most wave speeds are

\begin{equation}
  \wavespeed_1^-(\vectorsolution_L, \vectorsolution_R)
    = \velocityx_L - \speedofsound_L\pr{1 + \pr{
    \frac{(\height_* - \height_L)(\height_* + 2\height_L)}{2\height_L^2}}_+}
    \eqc
\end{equation}
\begin{equation}
  \wavespeed_2^+(\vectorsolution_L, \vectorsolution_R)
    = \velocityx_R + \speedofsound_R\pr{1 + \pr{
    \frac{(\height_* - \height_R)(\height_* + 2\height_R)}{2\height_R^2}}_+}
    \eqc
\end{equation}

where $(z)_+=\max(z,0)$. These definitions are completely general in that
they apply to both shocks and rarefactions. In the case of a rarefaction
for the left side, $\height_L \leq \height_*$, and similarly for the right
side, $\height_R \leq \height_*$. The wave speed of the head of the rarefaction
in each case is

\begin{equation}
  \wavespeed_1^-(\vectorsolution_L, \vectorsolution_R)
    = \velocityx_L - \speedofsound_L
    \eqc
\end{equation}
\begin{equation}
  \wavespeed_2^+(\vectorsolution_L, \vectorsolution_R)
    = \velocityx_R + \speedofsound_R
    \eqp
\end{equation}

Otherwise (when $\height_L > \height_*$ or $\height_R > \height_*$), the
wave is a shock, and the shock speed in each case is

\begin{equation}
  \wavespeed_1^-(\vectorsolution_L, \vectorsolution_R)
    = \velocityx_L - \speedofsound_L\pr{1 + \pr{
    \frac{(\height_* - \height_L)(\height_* + 2\height_L)}{2\height_L^2}}}
    \eqc
\end{equation}
\begin{equation}
  \wavespeed_2^+(\vectorsolution_L, \vectorsolution_R)
    = \velocityx_R + \speedofsound_R\pr{1 + \pr{
    \frac{(\height_* - \height_R)(\height_* + 2\height_R)}{2\height_R^2}}}
    \eqp
\end{equation}

Combining these equations with the rarefaction speeds gives the general
definitions of the left-most and right-most wave speeds.

The height in the star (*) region is the solution of the nonlinear equation

\begin{equation}
  \phi(\height) \equiv
    \wavestrength_L(\height,\vectorsolution_L)
    + \wavestrength_R(\height,\vectorsolution_R)
    + \velocityx_R - \velocityx_L
    = 0 \eqc
\end{equation}

where $\wavestrength_L(\height,\vectorsolution_L)$ and
$\wavestrength_R(\height,\vectorsolution_R)$
are the left and right wave strengths, each corresponding to either a shock
or rarefaction. In the case of a shock,

\begin{equation}
  \wavestrength_K(\height,\vectorsolution_K) = \wavestrength_K^{\text{shock}} = 
    (\height_* - \height_K)\sqrt{\half\gravity\frac{\height_* + \height_K}
    {\height_*\height_K}}
    \eqc
\end{equation}

while in the case of a rarefaction,

\begin{equation}
  \wavestrength_K(\height,\vectorsolution_K) = \wavestrength_K^{\text{rarefaction}} = 
    2(\speedofsound_* - \speedofsound_K)
    \eqp
\end{equation}

These wave strength functions are derived in the following sections.

\section{Shock Wave}
%------------------------------------------------------------------------------
This derivation will correspond to the left wave; the right wave derivation
proceeds similarly.

In the case of a shock, the discontinuous wave front moves with speed
$\shockspeed_L$, separating the left solution $\vectorsolution_L$ and the right
solution $\vectorsolution_*$. Transforming to a reference frame moving with the
shock, the reference frame velocities are

\begin{equation}\label{eq:ref_vL}
  \hat{\velocityx}_L = \velocityx_L - \shockspeed_L \eqc
\end{equation}
\begin{equation}\label{eq:ref_vstar}
  \hat{\velocityx}_* = \velocityx_* - \shockspeed_L \eqp
\end{equation}

Applying the Rankine-Hugoniot condition for both the continuity equation and
momentum equation gives

\begin{equation}
  \height_L\hat{\velocityx}_L = \height_*\hat{\velocityx}_* \eqc
\end{equation}
\begin{equation}\label{eq:RH_momentumL}
  \height_L\hat{\velocityx}_L^2 + \pressure_L
    = \height_*\hat{\velocityx}_*^2 + \pressure_*
    \eqc
\end{equation}

where $\pressure = \half\gravity\height^2$.
Defining the reference discharge as

\begin{equation}
  \hat{\dischargex}_L = \height_L\hat{\velocityx}_L = \height_*\hat{\velocityx}_*
\end{equation}

and substituting into Equation \eqref{eq:RH_momentumL} gives

\begin{equation}\label{eq:dischargeL_hat}
  \hat{\dischargex}_L = \frac{\pressure_L - \pressure_*}
    {\hat{\velocityx}_* - \hat{\velocityx}_L}
    \eqc
\end{equation}

which when combined with Equations \eqref{eq:ref_vL} and \eqref{eq:ref_vstar}
gives

\begin{equation}\label{eq:dischargeL_h}
  \hat{\dischargex}_L = \sqrt{\frac{\height_*\height_L
    (\pressure_L - \pressure_*)}
    {\height_L - \height_*}}
    \eqp
\end{equation}

Using $\hat{\velocityx}_* - \hat{\velocityx}_L = \velocityx_* - \velocityx_L$
with Equation \eqref{eq:dischargeL_hat} gives

\begin{equation}
  \hat{\dischargex}_L = \frac{\pressure_L - \pressure_*}
    {\velocityx_* - \velocityx_L}
\end{equation}

and combining with Equation \eqref{eq:dischargeL_h} gives, after a bit of
algebra,

\begin{equation}\label{eq:ustarLshock}
  \velocityx_* = \velocityx_L
    - \wavestrength^{\text{shock}}_L(\height_*,\vectorsolution_L)
    \eqc
\end{equation}

where

\begin{equation}
  \wavestrength^{\text{shock}}_L(\height_*,\vectorsolution_L)
    = (\height_* - \height_L)
    \sqrt{\half\gravity\frac{\height_* + \height_L}{\height_*\height_L}}
    \eqp
\end{equation}

Performing a similar analysis for the right wave gives

\begin{equation}\label{eq:ustarRshock}
  \velocityx_* = \velocityx_R
    + \wavestrength^{\text{shock}}_R(\height_*,\vectorsolution_R)
    \eqc
\end{equation}

where

\begin{equation}
  \wavestrength^{\text{shock}}_R(\height_*,\vectorsolution_R)
    = (\height_* - \height_R)
    \sqrt{\half\gravity\frac{\height_* + \height_R}{\height_*\height_R}}
    \eqp
\end{equation}

\section{Rarefaction Wave}
%------------------------------------------------------------------------------
\begin{equation}\label{eq:ustarLrarefaction}
  \velocityx_* = \velocityx_L
    - \wavestrength^{\text{rarefaction}}_L(\height_*,\vectorsolution_L)
    \eqc
\end{equation}

where

\begin{equation}
  \wavestrength_L^{\text{rarefaction}}(\height_*,\vectorsolution_L)
    = 2(\speedofsound_* - \speedofsound_L)
\end{equation}

Performing a similar analysis for the right wave gives

\begin{equation}\label{eq:ustarRrarefaction}
  \velocityx_* = \velocityx_R
    + \wavestrength^{\text{rarefaction}}_R(\height_*,\vectorsolution_R)
    \eqc
\end{equation}

where

\begin{equation}
  \wavestrength_R^{\text{rarefaction}}(\height_*,\vectorsolution_R)
    = 2(\speedofsound_* - \speedofsound_R)
\end{equation}

\section{Obtaining the Solution in the Star Region}
%------------------------------------------------------------------------------

Combining Equation \eqref{eq:ustarLshock} or \eqref{eq:ustarLrarefaction}
with \eqref{eq:ustarRshock} or \eqref{eq:ustarRrarefaction} by eliminating
$\velocityx$ gives the nonlinear
equation to solve for $\height_*$, where the LHS is defined to be $\phi(\height)$:

\begin{equation}
  \phi(\height) \equiv
    \wavestrength_L(\height,\vectorsolution_L)
    + \wavestrength_R(\height,\vectorsolution_R)
    + \velocityx_R - \velocityx_L
    = 0 \eqp
\end{equation}

Then adding either Equation \eqref{eq:ustarLshock} or \eqref{eq:ustarLrarefaction}
with \eqref{eq:ustarRshock} or \eqref{eq:ustarRrarefaction} gives the equation
for $\velocityx_*$:

\begin{equation}
  \velocityx_* = \half\pr{\velocityx_L + \velocityx_R
    + \wavestrength_R(\height_*,\vectorsolution_R)
    - \wavestrength_L(\height_*,\vectorsolution_L)}
    \eqp
\end{equation}

\pagebreak{}
