The maximum wave speed in a multidimensional problem is equal to
the maximum wave speed of the one-dimensional problem in the direction
given by the normal vector $\normalvector$:
\begin{equation}
  \maxwavespeed(\normalvector,
    [\height,\heightmomentum]_L\transpose,
    [\height,\heightmomentum]_R\transpose)
  = 
  \maxwavespeed(
    [\height,\heightmomentumletter_n]_L\transpose,
    [\height,\heightmomentumletter_n]_R\transpose)
  \eqc
\end{equation}
where $\heightmomentumletter_n\equiv\heightmomentum\cdot\normalvector$
denotes the component of $\heightmomentum$ along $\normalvector$.
The maximum wave speed in the one-dimensional Riemann problem
is the maximum of the absolute values of the
left-most and right-most wave speeds:
\begin{equation}\label{eq:maxwavespeed}
  \maxwavespeed(\vectorsolution_L, \vectorsolution_R)
  = \max(|\wavespeed_1^-(\vectorsolution_L, \vectorsolution_R)|
  ,|\wavespeed_2^+(\vectorsolution_L, \vectorsolution_R)|)
  \eqc
\end{equation}
where $\vectorsolution_K\equiv[\height,\heightmomentumletter_n]_K\transpose$,
and the ``+'' and ``-'' allow for the differentiation of the head
and tail speeds in the case of a rarefaction.

For the 1-D shallow water equations, the Riemann problem divides the
$x$-$\timevalue$ plane into 3 sectors, separated by 2 waves, which
each may be either a shock or rarefaction. The left sector shall be
denoted with ``L'', the middle with ``*'', and the right with ``R''.
The left-most and right-most wave speeds are
\begin{equation}\label{eq:left_wave_speed}
  \wavespeed_1^-(\vectorsolution_L, \vectorsolution_R)
    = \velocityx_L - \speedofsound_L\pr{1 + \pr{
    \frac{(\height_* - \height_L)(\height_* + 2\height_L)}{2\height_L^2}}_+}^{\frac{1}{2}}
    \eqc
\end{equation}
\begin{equation}\label{eq:right_wave_speed}
  \wavespeed_2^+(\vectorsolution_L, \vectorsolution_R)
    = \velocityx_R + \speedofsound_R\pr{1 + \pr{
    \frac{(\height_* - \height_R)(\height_* + 2\height_R)}{2\height_R^2}}_+}^{\frac{1}{2}}
    \eqc
\end{equation}
where $(z)_+=\max(z,0)$. These definitions are completely general in that
they apply to both shocks and rarefactions. In the case of a rarefaction
for the left side, $\height_L \leq \height_*$, and similarly for the right
side, $\height_R \leq \height_*$. The wave speed of the head of the rarefaction
in each case is
\begin{equation}\label{eq:leftwavespeed_rarefaction}
  \wavespeed_1^-(\vectorsolution_L, \vectorsolution_R)
    = \velocityx_L - \speedofsound_L
    \eqc
\end{equation}
\begin{equation}\label{eq:rightwavespeed_rarefaction}
  \wavespeed_2^+(\vectorsolution_L, \vectorsolution_R)
    = \velocityx_R + \speedofsound_R
    \eqp
\end{equation}
Otherwise (when $\height_L > \height_*$ or $\height_R > \height_*$), the
wave is a shock, and the shock speed in each case is
\begin{equation}\label{eq:left_shockspeed}
  \wavespeed_1^-(\vectorsolution_L, \vectorsolution_R)
    = \velocityx_L - \speedofsound_L\pr{1 + \pr{
    \frac{(\height_* - \height_L)(\height_* + 2\height_L)}{2\height_L^2}}}^{\frac{1}{2}}
    \eqc
\end{equation}
\begin{equation}
  \wavespeed_2^+(\vectorsolution_L, \vectorsolution_R)
    = \velocityx_R + \speedofsound_R\pr{1 + \pr{
    \frac{(\height_* - \height_R)(\height_* + 2\height_R)}{2\height_R^2}}}^{\frac{1}{2}}
    \eqp
\end{equation}
Combining these equations with the rarefaction speeds gives the general
definitions of the left-most and right-most wave speeds.

The height in the star (*) region is the solution of the nonlinear equation
\begin{equation}\label{eq:objective_function}
  \phi(\height) \equiv
    \wavestrength_L(\height,\vectorsolution_L)
    + \wavestrength_R(\height,\vectorsolution_R)
    + \velocityx_R - \velocityx_L
    = 0 \eqc
\end{equation}
where $\wavestrength_L(\height,\vectorsolution_L)$ and
$\wavestrength_R(\height,\vectorsolution_R)$
are the left and right wave strengths, each corresponding to either a shock
or rarefaction. The derivation of this equation is given in Section \ref{sec:solution_star}.

In the case of a shock,
\begin{equation}
  \wavestrength_K(\height,\vectorsolution_K) = \wavestrength_K^{\text{shock}} = 
    (\height - \height_K)\sqrt{\half\gravity\frac{\height + \height_K}
    {\height\height_K}}
    \eqc
\end{equation}
while in the case of a rarefaction,
\begin{equation}
  \wavestrength_K(\height,\vectorsolution_K) = \wavestrength_K^{\text{rarefaction}} = 
    2(\speedofsound - \speedofsound_K)
    \eqp
\end{equation}
These wave strength functions are derived in the following sections.

\section{Shock Wave}
%------------------------------------------------------------------------------
This derivation will correspond to the left wave; the right wave derivation
proceeds similarly.

In the case of a shock, the discontinuous wave front moves with speed
$\shockspeed_L$, separating the left solution $\vectorsolution_L$ and the right
solution $\vectorsolution_*$. Transforming to a reference frame moving with the
shock, the reference frame velocities are
\begin{equation}\label{eq:ref_vL}
  \hat{\velocityx}_L = \velocityx_L - \shockspeed_L \eqc
\end{equation}
\begin{equation}\label{eq:ref_vstar}
  \hat{\velocityx}_* = \velocityx_* - \shockspeed_L \eqp
\end{equation}
Applying the Rankine-Hugoniot condition for both the continuity equation and
momentum equation gives
\begin{equation}
  \height_L\hat{\velocityx}_L = \height_*\hat{\velocityx}_* \eqc
\end{equation}
\begin{equation}\label{eq:RH_momentumL}
  \height_L\hat{\velocityx}_L^2 + \pressure_L
    = \height_*\hat{\velocityx}_*^2 + \pressure_*
    \eqc
\end{equation}
where $\pressure = \half\gravity\height^2$.
Defining the reference discharge as
\begin{equation}\label{eq:ref_qL}
  \hat{\dischargex}_L = \height_L\hat{\velocityx}_L = \height_*\hat{\velocityx}_*
\end{equation}
and substituting into Equation \eqref{eq:RH_momentumL} gives
\begin{equation}\label{eq:dischargeL_hat}
  \hat{\dischargex}_L = \frac{\pressure_L - \pressure_*}
    {\hat{\velocityx}_* - \hat{\velocityx}_L}
    \eqc
\end{equation}
which when combined with Equations \eqref{eq:ref_vL} and \eqref{eq:ref_vstar}
gives
\begin{equation}\label{eq:dischargeL_h}
  \hat{\dischargex}_L = \sqrt{\frac{\height_*\height_L
    (\pressure_L - \pressure_*)}
    {\height_L - \height_*}}
    \eqp
\end{equation}
\begin{remark}
Combining Equations \eqref{eq:ref_vL}, \eqref{eq:ref_qL}, and \eqref{eq:dischargeL_h}
and performing some algebra gives the expression for the shock speed $\shockspeed_L$
given by Equation \eqref{eq:left_shockspeed}.
\end{remark}
Using $\hat{\velocityx}_* - \hat{\velocityx}_L = \velocityx_* - \velocityx_L$
with Equation \eqref{eq:dischargeL_hat} gives
\begin{equation}
  \hat{\dischargex}_L = \frac{\pressure_L - \pressure_*}
    {\velocityx_* - \velocityx_L}
\end{equation}
and combining with Equation \eqref{eq:dischargeL_h} gives, after a bit of
algebra,
\begin{equation}\label{eq:ustarLshock}
  \velocityx_* = \velocityx_L
    - \wavestrength^{\text{shock}}_L(\height_*,\vectorsolution_L)
    \eqc
\end{equation}
where
\begin{equation}
  \wavestrength^{\text{shock}}_L(\height,\vectorsolution_L)
    = (\height - \height_L)
    \sqrt{\half\gravity\frac{\height + \height_L}{\height\height_L}}
    \eqp
\end{equation}
Performing a similar analysis for the right wave gives
\begin{equation}\label{eq:ustarRshock}
  \velocityx_* = \velocityx_R
    + \wavestrength^{\text{shock}}_R(\height_*,\vectorsolution_R)
    \eqc
\end{equation}
where
\begin{equation}
  \wavestrength^{\text{shock}}_R(\height,\vectorsolution_R)
    = (\height - \height_R)
    \sqrt{\half\gravity\frac{\height + \height_R}{\height\height_R}}
    \eqp
\end{equation}

\section{Rarefaction Wave}
%------------------------------------------------------------------------------
\begin{equation}\label{eq:ustarLrarefaction}
  \velocityx_* = \velocityx_L
    - \wavestrength^{\text{rarefaction}}_L(\height_*,\vectorsolution_L)
    \eqc
\end{equation}
where
\begin{equation}
  \wavestrength_L^{\text{rarefaction}}(\height,\vectorsolution_L)
    = 2(\speedofsound - \speedofsound_L)
\end{equation}
Performing a similar analysis for the right wave gives
\begin{equation}\label{eq:ustarRrarefaction}
  \velocityx_* = \velocityx_R
    + \wavestrength^{\text{rarefaction}}_R(\height_*,\vectorsolution_R)
    \eqc
\end{equation}
where
\begin{equation}
  \wavestrength_R^{\text{rarefaction}}(\height,\vectorsolution_R)
    = 2(\speedofsound - \speedofsound_R)
\end{equation}

\section{Obtaining the Solution in the Star Region}\label{sec:solution_star}
%------------------------------------------------------------------------------

Combining Equation \eqref{eq:ustarLshock} or \eqref{eq:ustarLrarefaction}
with \eqref{eq:ustarRshock} or \eqref{eq:ustarRrarefaction} by eliminating
$\velocityx$ gives the nonlinear
equation to solve for $\height_*$, where the LHS is defined to be $\phi(\height)$
this is Equation \eqref{eq:objective_function}.

Then adding either Equation \eqref{eq:ustarLshock} or \eqref{eq:ustarLrarefaction}
with \eqref{eq:ustarRshock} or \eqref{eq:ustarRrarefaction} gives the equation
for $\velocityx_*$:
\begin{equation}
  \velocityx_* = \half\pr{\velocityx_L + \velocityx_R
    + \wavestrength_R(\height_*,\vectorsolution_R)
    - \wavestrength_L(\height_*,\vectorsolution_L)}
    \eqp
\end{equation}

\section{Fast Estimate of Maximum Wave Speed}
%------------------------------------------------------------------------------

The fast algorithm given in this section attempts to ease the computational
burden of computing the height in the star region $\height_*$ (which requires
a nonlinear solve that may require several iterations), which is needed
in the computation of the wave speeds given by Equations \eqref{eq:left_wave_speed}
\eqref{eq:right_wave_speed}.

The first condition needed by this algorithm is that the objective function
$\phi(\height)$ defined in Equation \eqref{eq:objective_function} be monotone
increasing. This is given in the following theorem.
\begin{theorem}{Monotonicity of the Objective Function}
The objective function $\phi(\height)$ defined in Equation
\eqref{eq:objective_function} is monotone increasing: $\phi'(\height) \geq 0$.
\end{theorem}
\begin{proof}
It is sufficient to prove that
each wave strength function $\wavestrength_K^{\text{rarefaction}}$ and
$\wavestrength_K^{\text{shock}}$ are monotone increasing with respect to
$\height$. This is trivial to prove for rarefaction waves:
\begin{equation}
  \pd{\wavestrength_K^{\text{rarefaction}}}{\height} = \sqrt{\frac{\gravity}{\height}} \geq 0 \eqp
\end{equation}
For shock waves, the proof is more complicated. To simplify algebra, the following
definition is made:
\begin{equation}
  \alpha \equiv \sqrt{\half\gravity\pr{\frac{1}{\height} + \frac{1}{\height_K}}}
  \eqc
\end{equation}
making the expression for wave strength the following:
\begin{equation}
  \wavestrength_K^{\text{shock}} = \pr{\height - \height_K}\alpha
  \eqp
\end{equation}
Taking the derivative gives
\begin{eqnarray*}
  \pd{\wavestrength_K^{\text{shock}}}{\height} & = &
    \alpha + \pr{\height - \height_K}\pd{\alpha}{\height} \eqc\\
  \pd{\wavestrength_K^{\text{shock}}}{\height} & = &
    \alpha - \fourth\gravity\frac{\height - \height_K}{\height^2}\frac{1}{\alpha}
  \eqp
\end{eqnarray*}
This proof now proceeds by assumption that $\pd{\wavestrength_K^{\text{shock}}}{\height}\geq0$:
\begin{eqnarray*}
  \alpha - \fourth\gravity\frac{\height - \height_K}{\height^2}\frac{1}{\alpha}
    & \geq & 0 \eqc\\
  \alpha^2 & \geq & \fourth\gravity\frac{\height - \height_K}{\height^2} \eqp
\end{eqnarray*}
Note that the last equation assumes that $\alpha\geq0$.
\begin{eqnarray*}
  \half\gravity\frac{\height + \height_K}{\height\height_K} & \geq &
    \fourth\gravity\frac{\height - \height_K}{\height^2} \eqc\\
  \height^3 + \height_K\height^2 & \geq &
    \half\pr{\height_K\height^2 - \height_K^2\height} \eqc\\
  \height^3 + \half\height_K\height^2 + \half\height_K^2\height & \geq & 0 \eqp
\end{eqnarray*}
This last statement is proved using the entropy condition. Thus
the assumption $\pd{\wavestrength_K^{\text{shock}}}{\height}\geq0$
is verified, and the proof is complete.\qed
\end{proof}

\begin{algorithm}[htb]
\caption{Initialization}
\begin{algorithmic}
\State $\hmin\gets \min(\height_L,\height_R)$
\State $\hmax\gets \max(\height_L,\height_R)$
\If{$\objective(\hmin) \geq 0$} \Comment{Both waves are rarefactions}
  \State Compute $\maxwavespeed$ using Equations
    \eqref{eq:maxwavespeed},
    \eqref{eq:leftwavespeed_rarefaction}, and \eqref{eq:rightwavespeed_rarefaction}
  \State \Return
\EndIf
\If{$\objective(\hmax) = 0$} \Comment{$\height_*$ is already known to be $\hmax$}
  \State $\height_*\gets \hmax$
  \State Compute $\maxwavespeed$ using Equations
    \eqref{eq:maxwavespeed}, \eqref{eq:left_wave_speed},
    and \eqref{eq:right_wave_speed}
  \State \Return
\ElsIf{$\objective(\hmax) < 0$} \Comment{Both waves are shocks}
  \State $\hhigh^{(0)}\gets \hrarefaction$
  \State $\hlow^{(0)}\gets \max\pr{\hmax,\hhigh^{(0)}
    - \frac{\objective(\hhigh^{(0)})}{\objective'(\hhigh^{(0)})}}$
  \State Call Algorithm \ref{alg:computemaxwavespeed}
    with $(\hlow^{(0)},\hhigh^{(0)})$
  \State \Return
\Else \Comment{One wave is rarefaction, one wave is shock}
  \State $\hhigh^{(0)}\gets \min\pr{\hmax,\hrarefaction}$
  \State $\hlow^{(0)}\gets \max\pr{\hmin,\hhigh^{(0)}
    - \frac{\objective(\hhigh^{(0)})}{\objective'(\hhigh^{(0)})}}$
  \State Call Algorithm \ref{alg:computemaxwavespeed}
    with $(\hlow^{(0)},\hhigh^{(0)})$
  \State \Return
\EndIf
\end{algorithmic}
\end{algorithm}

\begin{algorithm}[htb]
\caption{Computation of $\maxwavespeed$}
\label{alg:computemaxwavespeed}
\begin{algorithmic}
\State Input: $(\hlow^{(0)},\hhigh^{(0)},\tol)$
\Loop
  \State Compute $\maxwavespeed[k]$ using Equation \eqref{eq:maxwavespeedbound}
  \State Compute $\minwavespeed^{(k)}$ using Equation \eqref{eq:minwavespeedbound}
  \If{$\minwavespeed^{(0)} > 0$}
    \If{$\frac{\maxwavespeed[k]}{\minwavespeed^{(k)}} - 1 \leq \tol$}
      \State \Return
    \EndIf
  \EndIf
  \If{$\objective(\hlow^{(k)}) > 0$ or $\objective(\hhigh^{(k)}) < 0$}
    \State \Return
  \EndIf
  \State $\hlow^{(k+1)}\gets \hinterplow(\hlow^{(k)},\hhigh^{(k)})$
  \State $\hhigh^{(k+1)}\gets \hinterphigh(\hlow^{(k)},\hhigh^{(k)})$
\EndLoop
\State $\maxwavespeed\gets \maxwavespeed[k]$
\end{algorithmic}
\end{algorithm}

\begin{equation}\label{eq:maxwavespeedbound}
  \maxwavespeed[k] = \max\pr{(\highwavespeedtwo^{(k)})_+,
    (\lowwavespeedone^{(k)})_-}
  \eqc
\end{equation}
\begin{equation}\label{eq:minwavespeedbound}
  \minwavespeed^{(k)} = \pr{\max\pr{(\lowwavespeedtwo^{(k)})_+,
    (\highwavespeedone^{(k)})_-}}_+
  \eqc
\end{equation}
where $z_+\equiv\max(z,0)$, $z_-\equiv\max(-z,0)$, and the bounds on the individual
wave speeds are given by the following equations:
\begin{equation}
  \lowwavespeedone^{(k)}
    = \velocityx_L - \speedofsound_L\pr{1 + \pr{
    \frac{(\hhigh^{(k)} - \height_L)(\hhigh^{(k)}
    + 2\height_L)}{2\height_L^2}}_+}^{\frac{1}{2}}
    \eqc
\end{equation}
\begin{equation}
  \highwavespeedone^{(k)}
    = \velocityx_L - \speedofsound_L\pr{1 + \pr{
    \frac{(\hlow^{(k)} - \height_L)(\hlow^{(k)}
    + 2\height_L)}{2\height_L^2}}_+}^{\frac{1}{2}}
    \eqc
\end{equation}
\begin{equation}
  \lowwavespeedtwo^{(k)}
    = \velocityx_R + \speedofsound_R\pr{1 + \pr{
    \frac{(\hlow^{(k)} - \height_R)(\hlow^{(k)}
    + 2\height_R)}{2\height_R^2}}_+}^{\frac{1}{2}}
    \eqc
\end{equation}
\begin{equation}
  \highwavespeedtwo^{(k)}
    = \velocityx_R + \speedofsound_R\pr{1 + \pr{
    \frac{(\hhigh^{(k)} - \height_R)(\hhigh^{(k)}
    + 2\height_R)}{2\height_R^2}}_+}^{\frac{1}{2}}
    \eqc
\end{equation}
The interpolant functions $\hinterplow$ and $\hinterphigh$ are given by
the following equations:
\begin{equation}
  \hinterplow(\height_1,\height_2) = \height_1
    - \frac{2\objective(\height_1)}{\objective'(\height_1)
    + \sqrt{\objective'(\height_1)^2 - 4\objective(\height_1)\objective[\height_1,\height_1,\height_2]}}
  \eqc
\end{equation}
\begin{equation}
  \hinterphigh(\height_1,\height_2) = \height_2
    - \frac{2\objective(\height_2)}{\objective'(\height_2)
    + \sqrt{\objective'(\height_2)^2 - 4\objective(\height_2)\objective[\height_1,\height_2,\height_2]}}
  \eqc
\end{equation}
where $\objective[x,y,z]$ denotes divided differences:
\begin{equation}
  \objective[x,y,z] = \frac{\frac{1}{x-y}\pr{\objective(x)-\objective(y)}
    - \frac{1}{y-z}\pr{\objective(y)-\objective(z)}}{x-z}
  \eqp
\end{equation}

\pagebreak{}
