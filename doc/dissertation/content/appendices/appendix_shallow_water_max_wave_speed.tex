\chapter{DERIVATION OF THE MAX WAVE SPEED FOR THE SHALLOW WATER EQUATIONS
\label{app:shallow_water_max_wave_speed}}

The maximum wave speed is the maximum of the absolute values of the
left-most and right-most wave speeds:
\begin{equation}
  \maxwavespeed(\vectorsolution_L, \vectorsolution_R)
  = \max(|\wavespeed_1^-(\vectorsolution_L, \vectorsolution_R)|
  ,|\wavespeed_2^+(\vectorsolution_L, \vectorsolution_R)|)
  \eqc
\end{equation}
where the ``+'' and ``-'' allow for the differentiation of the head
and tail speeds in the case of a rarefaction.

For the 1-D shallow water equations, the Riemann problem divides the
$x$-$\timevalue$ plane into 3 sectors, separated by 2 waves, which
each may be either a shock or rarefaction. The left sector shall be
denoted with ``L'', the middle with ``*'', and the right with ``R''.
The left-most and right-most wave speeds are

\begin{equation}
  \wavespeed_1^-(\vectorsolution_L, \vectorsolution_R)
    = \velocityx_L - \speedofsound_L\pr{1 + \pr{
    \frac{(\height_* - \height_L)(\height_* + 2\height_L)}{2\height_L^2}}_+}
    \eqc
\end{equation}
\begin{equation}
  \wavespeed_2^+(\vectorsolution_L, \vectorsolution_R)
    = \velocityx_R + \speedofsound_R\pr{1 + \pr{
    \frac{(\height_* - \height_R)(\height_* + 2\height_R)}{2\height_R^2}}_+}
    \eqc
\end{equation}

where $(z)_+=\max(z,0)$. These definitions are completely general in that
they apply to both shocks and rarefactions. In the case of a rarefaction
for the left side, $\height_L \leq \height_*$, and similarly for the right
side, $\height_R \leq \height_*$. The wave speed of the head of the rarefaction
in each case is

\begin{equation}
  \wavespeed_1^-(\vectorsolution_L, \vectorsolution_R)
    = \velocityx_L - \speedofsound_L
    \eqc
\end{equation}
\begin{equation}
  \wavespeed_2^+(\vectorsolution_L, \vectorsolution_R)
    = \velocityx_R + \speedofsound_R
    \eqp
\end{equation}

Otherwise (when $\height_L > \height_*$ or $\height_R > \height_*$), the
wave is a shock, and the shock speed in each case is

\begin{equation}
  \wavespeed_1^-(\vectorsolution_L, \vectorsolution_R)
    = \velocityx_L - \speedofsound_L\pr{1 + \pr{
    \frac{(\height_* - \height_L)(\height_* + 2\height_L)}{2\height_L^2}}}
    \eqc
\end{equation}
\begin{equation}
  \wavespeed_2^+(\vectorsolution_L, \vectorsolution_R)
    = \velocityx_R + \speedofsound_R\pr{1 + \pr{
    \frac{(\height_* - \height_R)(\height_* + 2\height_R)}{2\height_R^2}}}
    \eqp
\end{equation}

Combining these equations with the rarefaction speeds gives the general
definitions of the left-most and right-most wave speeds.

The height in the star (*) region is the solution of the nonlinear equation

\begin{equation}
  \phi(\height) \equiv
    f_L(\height,\vectorsolution_L) + f_R(\height,\vectorsolution_R)
    + \velocityx_R - \velocityx_L
    = 0 \eqc
\end{equation}

where $f_L(\height,\vectorsolution_L)$ and $f_R(\height,\vectorsolution_R)$
are the left and right wave strengths, each corresponding to either a shock
or rarefaction. In the case of a shock,

\begin{equation}
  f_K(\height,\vectorsolution_K) = f_K^{\text{shock}} = 
    (\height_* - \height_K)\sqrt{\half\gravity\frac{\height_* + \height_K}
    {\height_*\height_K}}
    \eqc
\end{equation}

while in the case of a rarefaction,

\begin{equation}
  f_K(\height,\vectorsolution_K) = f_K^{\text{rarefaction}} = 
    2(\speedofsound_* - \speedofsound_K)
    \eqp
\end{equation}

These wave strength functions are derived in the following sections.

\section{Left Shock Wave}
%------------------------------------------------------------------------------
In the case of a shock, the discontinuous wave front moves with speed
$\shockspeed_L$, separating the left solution $\vectorsolution_L$ and the right
solution $\vectorsolution_*$. Transforming to a reference frame moving with the
shock, the reference frame velocities are

\begin{equation}
  \hat{\velocityx}_L = \velocityx_L - \shockspeed_L \eqc
\end{equation}
\begin{equation}
  \hat{\velocityx}_* = \velocityx_* - \shockspeed_L \eqp
\end{equation}

Applying the Rankine-Hugoniot condition for both the continuity equation and
momentum equation gives

\begin{equation}
  \height_L\hat{\velocityx}_L = \height_*\hat{\velocityx}_* \eqc
\end{equation}
\begin{equation}\label{eq:RH_momentumL}
  \height_L\hat{\velocityx}_L^2 + \pressure_L
    = \height_*\hat{\velocityx}_*^2 + \pressure_* \eqc
\end{equation}

Defining the reference discharge as

\begin{equation}
  \hat{\discharge} = \height_L\hat{\velocityx}_L = \height_*\hat{\velocityx}_*
\end{equation}

and substituting into Equation \eqref{eq:RH_momentumL} gives

\begin{equation}
  \hat{\discharge} = -\frac{\pressure_* - \pressure_L}
    {\hat{\velocityx}_* - \hat{\velocityx}_L}}
    \eqp
\end{equation}

\pagebreak{}
