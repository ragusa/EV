To construct a high-order scheme, the concept of entropy viscosity is used in
conjunction with the bilinear form introduced in Equation
\eqref{eq:bilinearform}.  The high-order viscosity
$\highordercellviscosity[\timeindex]$ is computed as the minimum of the
low-order viscosity $\lowordercellviscosity$ and the entropy viscosity
$\entropycellviscosity[\timeindex]$:
\begin{equation}\label{eq:high_order_viscosity}
   \highordercellviscosity[\timeindex] = \min(\lowordercellviscosity,
   \entropycellviscosity[\timeindex]) \eqc
\end{equation}
where the entropy viscosity is defined as
\begin{equation}
   \entropycellviscosity[n] = \frac{\entropyresidualcoef
   \entropyresidual_\cellindex^\timeindex(\approximatescalarsolution^\timeindex,
   \approximatescalarsolution^{\timeindex-1})
   + \entropyjumpcoef\max\limits_{F\in\partial \cellindex}\entropyjump_F(
   \approximatescalarsolution^\timeindex)}
   {\|\entropy(\approximatescalarsolution^\timeindex)
   -\bar{\entropy}(\approximatescalarsolution^\timeindex)\|_{L^\infty(\domain)}}
   \eqp
\end{equation}
The entropy is defined to be some convex function of $\scalarsolution$ such as
$\entropy(\scalarsolution)=\frac{1}{2}\scalarsolution^2$. The entropy residual
$\entropyresidual_\cellindex^\timeindex(\approximatescalarsolution^\timeindex,
\approximatescalarsolution^{\timeindex-1})$ is the following:
\begin{equation}
  \entropyresidual_\cellindex^\timeindex(\approximatescalarsolution^\timeindex,
  \approximatescalarsolution^{\timeindex-1})
  = \left\|\frac{\entropy(\approximatescalarsolution^\timeindex)
  - \entropy(\approximatescalarsolution^{\timeindex-1})} 
  {\timestepsize^\timeindex}
  + \entropy'(\approximatescalarsolution^\timeindex)\pr{
  \divergence\consfluxscalar[\approximatescalarsolution^\timeindex]
  + \reactioncoef \approximatescalarsolution^\timeindex
  - \scalarsource}\right\|_{L^\infty(\cellindex)} \eqc
\end{equation}
where the $L^\infty(\cellindex)$ norm is approximated as the maximum of the
norm operand evaluated at each quadrature point on $\cellindex$.  Because the
entropy residual only measures cell-wise entropy production, it is useful to
include entropy flux \emph{jumps} in the definition of the entropy viscosity,
since these jumps are a measure of edge-wise entropy production.  The jump
$\entropyjump_F$ for a face $F$ measures the jump in the normal component of
the entropy flux across the cell interface:
\begin{equation}
  \entropyjump_F(\approximatescalarsolution^\timeindex)
  = \|\mathbf{\consfluxletter}'(\approximatescalarsolution)\cdot\normalvector_F
  [\![\partial_n \entropy(\approximatescalarsolution^\timeindex)
  ]\!]\|_{L^\infty(F)} \eqc
\end{equation}
where $\normalvector_F$ is the outward unit vector for face $F$, the
$L^\infty(F)$ norm is approximated as the maximum of the norm operand evaluated
at each quadrature point on $F$, and the term $[\![\partial_n \entropy(
\approximatescalarsolution^\timeindex)]\!]$ is computed as
\begin{eqnarray}
  [\![\partial_n \entropy(\approximatescalarsolution^\timeindex)]\!]
  & = & [\![\nabla\entropy(\approximatescalarsolution^\timeindex)
    \cdot\normalvector_F]\!]\\
  & = & [\![\entropy'(\approximatescalarsolution^\timeindex)
    \nabla\approximatescalarsolution^\timeindex\cdot\normalvector_F]\!]\\
  & = & (\entropy'(\approximatescalarsolution^\timeindex
    |_\cellindex)\nabla\approximatescalarsolution^\timeindex|_\cellindex
    - \entropy'(\approximatescalarsolution^\timeindex
    |_{\cellindex'})\nabla\approximatescalarsolution^\timeindex|_{\cellindex'})
    \cdot\normalvector_F
\end{eqnarray}
where $\cdot|_\cellindex$ denotes the computation of $\cdot$ from $\cellindex$,
and $\cdot|_{\cellindex'}$ denotes the computation of $\cdot$ from the neighbor
$\cellindex'$ sharing the face $F$.
