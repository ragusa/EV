The entropy function for the shallow water equations is defined to be
the sum of the kinetic and potential energy terms, in terms of the
conservative variables and the bathymetry function $\bathymetry$:
\begin{equation}\label{eq:entropy_swe}
  \entropy(\vectorsolution,\bathymetry)
  = \half\frac{\heightmomentum\cdot\heightmomentum}
  {\height} + \half\gravity\height\pr{\height+\bathymetry}
  \eqp
\end{equation}
% REMARK ======================================================================
\begin{remark}
Omission of the bathymetry term in the potential energy
term in the entropy definition has no effect on either the entropy
residual or entropy jump due to the assumption that $\bathymetry$ is
not a function of time. Thus in implementation, the following definition
can be used:
\begin{equation}
  \entropy(\vectorsolution)
  = \half\frac{\heightmomentum\cdot\heightmomentum}
  {\height} + \half\gravity\height^2
  \eqc
\end{equation}
which is often more convenient since it is a function of the conservative
variables only.
\end{remark}
%==============================================================================
The entropy flux, derived in Appendix \ref{sec:shallow_water_entropy_flux}, is the
following:
\begin{equation}
  \mathbf{\consfluxletter}^\entropy(\vectorsolution,\bathymetry)
  = \gravity(\height + \bathymetry)\heightmomentum
  + \half\frac{\pr{\heightmomentum\cdot\heightmomentum}\heightmomentum} 
  {\height^2}
  \eqp
\end{equation}
The entropy residual is defined to be the left hand side of 
Equation \eqref{eq:shallowwater_entropy_equality}:
\begin{equation}
  \entropyresidual(\vectorsolution^\timeindex, \vectorsolution^{\timeindex-1})
    \equiv \frac{\entropy(\vectorsolution^\timeindex)
      - \entropy(\vectorsolution^{\timeindex-1})}{\timestepsize^{\timeindex-1}}
    + \divergence\mathbf{\consfluxletter}^\entropy
      (\vectorsolution^\timeindex,\bathymetry)
  \eqp
\end{equation}
As in the scalar case, it can be helpful to define an \emph{entropy jump}
for an interior face $F$:
\begin{equation}
  \entropyjump_F(\vectorsolution)
  \equiv \left|
    (\jumpbrackets{\nabla\mathbf{\consfluxletter}^\entropy(\vectorsolution)}
    \cdot\normalvector_F)\cdot\normalvector_F
  \right| \eqc
\end{equation}
which can be interpreted as the jump in the gradient in the normal direction
of the normal component of the entropy flux.

To compute the high-order diffusion matrix, one option is to follow the approach
of the scalar case and compute a cell-wise entropy viscosity as in Equation
\eqref{eq:entropy_viscosity}:
\begin{equation}
   \entropycellviscosity[\timeindex] = \frac{\entropyresidualcoef
   \entropyresidual_\cellindex^\timeindex
   + \entropyjumpcoef\entropyjump_\cell^\timeindex}
   {\entropynormalization^n_\cell}
   \eqc
\end{equation}
where here $\entropynormalization^n_\cell$ represents a more general
normalization coefficient for a cell $\cell$ than the global normalization
presented for the scalar case,
\begin{equation}
  \entropynormalization^n_\cell = \entropynormalization^n =
    \|\entropy(\approximatescalarsolution^n)
   -\bar{\entropy}(\approximatescalarsolution^n)\|_{L^\infty(\domain)} \eqp
\end{equation}
For the shallow water equations, a possible alternative normalization
is the following, which unlike the above normalization, is \emph{local}:
\begin{equation}
  \entropynormalization^n_\cell =
    \max\limits_{\x_q\in\quadraturepoints_\cell} \gravity \approximate{\height}_q^2
  \eqc
\end{equation}
where $\x_q$ denotes a quadrature point in the set $\quadraturepoints_\cell$
of quadrature points in a cell $\cell$ and
$\approximate{h}_q$ denotes $\approximate{h}(\x_q)$. The high-order diffusion matrix
could then be assembled as in Equation \eqref{eq:high_order_Dij}.

Alternatively, one could compute an artificial diffusion matrix without
computing cell-wise entropy viscosities, as was presented for the low-order
case for systems in Equation \eqref{eq:low_order_Dij_system}.
For example, one could define an entry of the high-order diffusion matrix
as
\begin{equation}
  \diffusionmatrixletter^{\high,\timeindex}\ij \equiv
    \min\pr{\diffusionmatrixletter^{\entropy,\timeindex}\ij,
      \diffusionmatrixletter^{\low,\timeindex}\ij}
  \eqc
\end{equation}
where similar to before, the low-order diffusion is used as an upper
bound of the entropy diffusion $\diffusionmatrixletter^{\entropy,\timeindex}\ij$,
which could be defined in the form
\begin{equation}
  \diffusionmatrixletter^{\entropy,\timeindex}\ij \equiv
    \frac{\entropyresidualcoef\entropyresidual\ij^n +
      \entropyjumpcoef\entropyjump\ij^n}
      {\entropynormalization^\timeindex\ij}
  \eqc
\end{equation}
where for example $\entropyresidual\ij^\timeindex$ might be defined as
\begin{equation}
  \entropyresidual\ij^\timeindex \equiv \left|
    \int\limits_{\support\ij}
      \entropyresidual(\vectorsolution^\timeindex,\vectorsolution^{\timeindex-1})
      \testfunction_i(\x)
      \testfunction_j(\x) \dvolume
    \right|
  \eqp
\end{equation}
Possible definitions of $\entropyjump\ij^n$ and
$\entropynormalization^\timeindex\ij$ are omitted here.
The results obtained in this dissertation used the first approach, in
which cell-wise entropy viscosities were computed.

%\begin{equation}
%  \entropyjump\ij^\timeindex \equiv \maxcelldiameter
%    \sum\limits_{F:\facedomain_F\subset\support\ij} \,
%    \int\limits_{\facedomain_F}\entropyjump_F(\vectorsolution^\timeindex)
%      \testfunction_i(\x)
%      \testfunction_j(\x) \darea
%  \eqc
%\end{equation}
%\begin{equation}
%  \entropynormalization^\timeindex\ij \equiv
%    \max\limits_{\cell:\celldomain\subset\support\ij}
%    \max\limits_{\qpoint\in\quadraturepoints(\celldomain)}
%    \entropynormalization(\vectorsolution^\timeindex(\qpoint))
%  \eqp
%\end{equation}

