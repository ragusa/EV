The entropy function for the shallow water equations is defined to be
the sum of the kinetic and potential energy terms, in terms of the
conservative variables and the bathymetry function $\bathymetry$:
\begin{equation}
  \entropy(\vectorsolution,\bathymetry)
  = \half\frac{\heightmomentum\cdot\heightmomentum}
  {\height} + \half\gravity\height\pr{\height+\bathymetry}
  \eqp
\end{equation}
% REMARK ======================================================================
\begin{remark}
Omission of the bathymetry term in the potential energy
term in the entropy definition has no effect on either the entropy
residual or entropy jump due to the assumption that $\bathymetry$ is
not a function of time. Thus in implementation, the following definition
can be used:
\begin{equation}
  \entropy(\vectorsolution)
  = \half\frac{\heightmomentum\cdot\heightmomentum}
  {\height} + \half\gravity\height^2
  \eqc
\end{equation}
which is often more convenient since it is a function of the conservative
variables only.
\end{remark}
%==============================================================================
The entropy flux, derived in Appendix \ref{sec:shallow_water_entropy_flux}, is the
following:
\begin{equation}
  \mathbf{\consfluxletter}^\entropy(\vectorsolution,\bathymetry)
  = \gravity(\height + \bathymetry)\heightmomentum
  + \half\frac{\pr{\heightmomentum\cdot\heightmomentum}\heightmomentum} 
  {\height^2}
  \eqp
\end{equation}
The entropy residual is defined to be the left hand side of 
Equation \eqref{eq:shallowwater_entropy_equality}:
\begin{equation}
  \entropyresidual(\vectorsolution^\timeindex, \vectorsolution^{\timeindex-1})
    \equiv \frac{\entropy(\vectorsolution^\timeindex)
      - \entropy(\vectorsolution^{\timeindex-1})}{\timestepsize^{\timeindex-1}}
    + \divergence\mathbf{\consfluxletter}^\entropy(\vectorsolution^\timeindex)
  \eqp
\end{equation}
The entropy jump for a face $F$ is defined to be
\begin{equation}
  \entropyjump_F(\vectorsolution)
  \equiv \left|
    \jumpbrackets{\nabla\mathbf{\consfluxletter}^\entropy(\vectorsolution)}
    \cdot\normalvector_F\cdot\normalvector_F
  \right| \eqp
\end{equation}
For the conservation law system case, opposed to the scalar conservation
law case, an entropy \emph{viscosity} is not computed; instead an entropy diffusion
matrix is computed directly. This is because the domain-invariant low-order scheme
described in Section \ref{sec:low_order_scheme_system} also makes
this approach, and the low-order scheme diffusion must be used as an upper bound
for the high-order diffusion. The entropy diffusion matrix is computed as
\begin{subequations}
\begin{equation}
  \diffusionmatrixletter^{\entropy,\timeindex}\ij \equiv
    \nodequantity{\diffusionmatrixletter}^{\entropy,\timeindex}\nodeij \eqc
\end{equation}
\begin{equation}
  \nodequantity{\diffusionmatrixletter}^{\entropy,\timeindex}\kl \equiv
    \frac{\entropyresidualcoef\nodequantity{\entropyresidual}\kl +
      \entropyjumpcoef\nodequantity{\entropyjump}\kl}
      {\nodequantity{\entropynormalization}^\timeindex\kl}
  \eqc
\end{equation}
\end{subequations}
where
\begin{equation}
  \nodequantity{\entropyresidual}\kl^\timeindex \equiv \left|
    \int\limits_{\nodequantity{\support}\kl}
      \entropyresidual(\vectorsolution^\timeindex,\vectorsolution^{\timeindex-1})
      \nodequantity{\testfunction}_k(\x)
      \nodequantity{\testfunction}_\ell(\x) \dvolume
    \right|
  \eqc
\end{equation}
% TODO: I think the following quantity is always zero because the product
% of the supports on the faces are zero, so the jumps may need
% to be redefined or omitted.
\begin{equation}
  \nodequantity{\entropyjump}\kl^\timeindex \equiv \maxcelldiameter
    \sum\limits_{F:\facedomain_F\subset\nodequantity{\support}\kl} \,
    \int\limits_{\facedomain_F}\entropyjump_F(\vectorsolution^\timeindex)
      \nodequantity{\testfunction}_k(\x)
      \nodequantity{\testfunction}_\ell(\x) \darea
  \eqc
\end{equation}
\begin{equation}
  \nodequantity{\entropynormalization}^\timeindex\kl \equiv
    \max\limits_{\cell:\celldomain\subset\nodequantity{\support}\kl}
    \max\limits_{\qpoint\in\quadraturepoints(\celldomain)}
    \entropynormalization(\vectorsolution^\timeindex(\qpoint))
  \eqp
\end{equation}
Similarly to the scalar case, the high-order diffusion uses the low-order
diffusion as an upper bound:
\begin{subequations}
\begin{equation}
  \diffusionmatrixletter^{\high,\timeindex}\ij \equiv
    \nodequantity{\diffusionmatrixletter}^{\high,\timeindex}\nodeij \eqc
\end{equation}
\begin{equation}
  \nodequantity{\diffusionmatrixletter}^{\high,\timeindex}\kl \equiv
    \max\pr{\nodequantity{\diffusionmatrixletter}^{\entropy,\timeindex}\kl,
      \nodequantity{\diffusionmatrixletter}^{\low,\timeindex}\kl}
  \eqc
\end{equation}
\end{subequations}


