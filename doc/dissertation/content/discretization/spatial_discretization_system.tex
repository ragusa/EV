This section gives the spatial discretization for conservation law systems.
Again, CGFEM is used for
spatial discretization, so the numerical solution (now vector-valued:
$\approximatevectorsolution$) is again approximated using an expansion of basis
functions. However, since there are now multiple degrees of freedom at each
node, some discussion on notation is given here to distinguish between node
indexing and degree of freedom indexing.  First, let the number of
\emph{scalar} solution components be denoted by $\ncomponents$.  Thus, as an
example, the 2-D shallow water equations, which consist of a continuity
equation and a conservation of momentum equation, have $\ncomponents=3$ because
the multi-dimensional conservation of momentum equation is comprised of two
scalar conservation of momentum equations.  There are a number of ways to view
the basis function expansion of the approximate solution. For example, one may
use vector-valued basis functions:
\begin{equation}
  \approximatevectorsolution\xt = \sum\limits_{j=1}^{\ndofs}
    \solutionletter_j(\timevalue)\vectortestfunction_j(\x) \eqc
\end{equation}
where the coefficients $\solutionletter_j(\timevalue)$ are the basis function
expansion coefficients at time $\timevalue$, and $\vectortestfunction_j(\x)$
are the vector-valued basis functions
\begin{equation}
  \vectortestfunction_j(\x) = 
    \unitvector{\componentindex(j)}
      \testfunction^{\componentindex(j)}_{\nodeindex(j)}(\x)
  \eqc
\end{equation}
where $\unitvector{\componentindex(j)}$ is the unit vector for dimension
$\componentindex(j)$, $\componentindex(j)$ returns the component index associated with degree
of freedom $j$, and $\nodeindex(j)$ returns the node index associated with
degree of freedom $j$. The basis function $\vectortestfunction_j$ is thus
zero for all components except the one corresponding to $j$ which uses
the scalar component basis function corresponding to component $\componentindex(j)$ of
the solution at node $\nodeindex(j)$.
Alternately, one may view each solution component as its own expansion:
\begin{equation}
  \approximatescalarsolution^m\xt = \sum\limits_{k=1}^{\nnodes}
    \solutionletter_{\dofindex(k,m)}(\timevalue)\testfunction^m_k(\x)
  \eqc
\end{equation}
where $\dofindex(k,m)$ is the global degree of freedom index associated
with node $k$ and solution component $m$.
Substituting the approximate
solution into Equation \eqref{eq:scalar_transport} and testing with basis
function $\testfunction_i(\x)$ gives
\begin{equation}
   \intSi\ppt{\approximatescalarsolution}\testfunction_i(\x) d\volume
      + \intSi\left(\divergence\consfluxscalar[\approximatescalarsolution]
      + \reactioncoef(\x)\approximatescalarsolution\xt\right)
      \testfunction_i(\x) d\volume
      = \intSi \scalarsource\xt \testfunction_i(\x) d\volume \eqc
\end{equation}
where $\support_i$ is the support of $\testfunction_i(\x)$. If the flux
function $\consfluxscalar$ is linear with respect to $\scalarsolution$, i.e.,
$\consfluxscalar = \velocity\scalarsolution$ for some uniform velocity field
$\velocity$, then the system to be solved is linear:
\begin{equation}\label{eq:semidiscrete}
  \consistentmassmatrix\ddt{\solutionvector}+\ssmatrix\solutionvector(t)
  = \ssrhs(\timevalue) \eqc
\end{equation}
with the elements of $\ssmatrix$ being the following:
\begin{equation}\label{eq:Aij}
  \ssmatrixletter\ij \equiv \intSij\left(
  \velocity\cdot\nabla\testfunction_j(\x) +
  \reactioncoef(\x)\testfunction_j(\x)\right)\testfunction_i(\x) d\volume \eqc
\end{equation}
where $\support\ij$ is the dual support of $\testfunction_i(\x)$ and
$\testfunction_j(\x)$.
If the flux function $\consfluxscalar$ is nonlinear, then the system is
nonlinear, but it may be expressed in a quasilinear form:
\begin{equation}\label{eq:semi_quasilinear}
   \consistentmassmatrix\ddt{\solutionvector}
   + \ssmatrix(\approximatescalarsolution)\solutionvector(\timevalue)
   = \ssrhs(\timevalue) \eqc
\end{equation}
where $\approximatescalarsolution\xt$ is the numerical solution, and the
quasilinear matrix (i.e., the Jacobian matrix) entries are
\begin{equation}\label{eq:Aij_nonlinear}
  \ssmatrixletter\ij(\approximatescalarsolution) \equiv \intSij\left(
  \mathbf{\consfluxletter}'(\approximatescalarsolution)\cdot\nabla
  \testfunction_j(\x) +
  \reactioncoef(\x)\testfunction_j(\x)\right)
  \testfunction_i(\x) d\volume \eqp
\end{equation}
The elements of $\ssrhs(\timevalue)$ are
\begin{equation}
  \ssrhsletter_i(\timevalue) \equiv \intSi \scalarsource\xt\testfunction_i(\x)
  d\volume \eqp
\end{equation}
$\consistentmassmatrix$ is the consistent mass matrix, which has the entries
\begin{equation}\label{eq:massmatrix}
  \massmatrixletter^C\ij \equiv \intSij
  \testfunction_j(\x)\testfunction_i(\x) d\volume \eqp
\end{equation}
Similarly, for the linear steady-state case, the linear system is
\begin{equation}
  \ssmatrix\solutionvector = \ssrhs \eqc
\end{equation}
or for the nonlinear case,
\begin{equation}
  \ssmatrix(\approximatescalarsolution)\solutionvector = \ssrhs \eqp
\end{equation}
