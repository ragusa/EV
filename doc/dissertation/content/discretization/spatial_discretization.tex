The continuous Galerkin (CG) finite element method (FEM) is used for spatial
discretization.  The numerical solution is thus approximated using an
expansion of basis functions $\testfunction_j(\x)$:
\begin{equation}
  \approximatescalarsolution\xt = \sumj \solutionletter_j(\timevalue)
  \testfunction_j(\x) \eqc
\end{equation}
where the coefficients $\solutionletter_j(\timevalue)$ are the basis function
expansion coefficients at time $\timevalue$. Substituting the approximate
solution into Equation \eqref{eq:cons_law} and testing with basis function
$\testfunction_i(\x)$ gives
\begin{equation}
   \intSi\ppt{\approximatescalarsolution}\testfunction_i(\x) d\volume
      + \intSi\left(\divergence\consfluxscalar[\approximatescalarsolution]
      + \reactioncoef(\x)\approximatescalarsolution\xt\right)
      \testfunction_i(\x) d\volume
      = \intSi \scalarsource\xt \testfunction_i(\x) d\volume \eqc
\end{equation}
where $\support_i$ is the support of $\testfunction_i(\x)$. If the flux function
$\consfluxscalar$ is linear, then the system to be solved is linear:
\begin{equation}\label{eq:semidiscrete}
  \consistentmassmatrix\ddt{\solutionvector}+\ssmatrix\solutionvector(t)
  = \ssrhs(\timevalue) \eqc
\end{equation}
where $\ssmatrix$ is the steady-state system matrix and $\solutionvector$ is the
vector of discrete solution values. If the velocity field $\velocity$ is
constant, then the elements of $\ssmatrix$ are
\begin{equation}\label{eq:Aij}
  \ssmatrixletter\ij \equiv \intSij\left(
  \velocity\cdot\nabla\testfunction_j(\x) +
  \reactioncoef(\x)\testfunction_j(\x)\right)\testfunction_i(\x) d\volume \eqc
\end{equation}
where $\support\ij$ is the dual support of $\testfunction_i(\x)$ and
$\testfunction_j(\x)$.  Otherwise, the divergence of the velocity field
appears:
\begin{equation}\label{eq:Aij_nonconstant_v}
  \ssmatrixletter\ij \equiv \intSij\left(
  \velocity\cdot\nabla\testfunction_j(\x) +
  (\reactioncoef(\x)+\divergence\velocity)\testfunction_j(\x)\right)
  \testfunction_i(\x) d\volume \eqp
\end{equation}
If the flux function $\consfluxscalar$ is nonlinear, then the system is
nonlinear, but it may be expressed in a quasilinear form:
\begin{equation}\label{eq:semi_quasilinear}
   \consistentmassmatrix\ddt{\solutionvector}
   + \ssmatrix(\approximatescalarsolution)\solutionvector(\timevalue)
   = \ssrhs(\timevalue) \eqc
\end{equation}
where $\approximatescalarsolution\xt$ is the numerical solution, and the
quasilinear matrix (i.e., the Jacobian matrix) entries are
\begin{equation}\label{eq:Aij_nonlinear}
  \ssmatrixletter\ij(\approximatescalarsolution) \equiv \intSij\left(
  \mathbf{\consfluxletter}'(\approximatescalarsolution)\cdot\nabla
  \testfunction_j(\x) +
  (\reactioncoef(\x)+\nabla\cdot\velocity)\testfunction_j(\x)\right)
  \testfunction_i(\x) d\volume \eqp
\end{equation}
The elements of $\ssrhs(\timevalue)$ are
\begin{equation}
  \ssrhsletter_i(\timevalue) \equiv \intSi \scalarsource\xt\testfunction_i(\x)
  d\volume \eqp
\end{equation}
$\consistentmassmatrix$ is the consistent mass matrix, which has the entries
\begin{equation}\label{eq:massmatrix}
  \massmatrixletter^C\ij \equiv \intSij
  \testfunction_j(\x)\testfunction_i(\x) d\volume \eqp
\end{equation}
Similarly, for the linear steady-state case, the linear system is
\begin{equation}
  \ssmatrix\solutionvector = \ssrhs \eqc
\end{equation}
or for the nonlinear case,
\begin{equation}
  \ssmatrix(\approximatescalarsolution)\solutionvector = \ssrhs \eqp
\end{equation}
