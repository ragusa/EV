In this section, some options for the implementation of the incoming flux
boundary condition, given by Equation \eqref{eq:incoming_flux}, are
presented. This research considers three options for this task:
\begin{enumerate}
  \item Strongly impose the Dirichlet boundary conditions,
  \item Weakly impose the Dirichlet boundary conditions, and
  \item Weakly impose the Dirichlet boundary conditions with the boundary
    penalty method.
\end{enumerate}
Most of the scalar results in this dissertation will use strongly imposed
Dirichlet boundary conditions, but for some cases, the other two approaches
are considered.

Consider the linear system $\mathbf{B}\solutionvector = \mathbf{r}$.
To strongly impose Dirichlet boundary conditions on this linear system,
the system matrix and system right-hand-side vector are directly
modified:
\begin{subequations}
\begin{equation}
  \tilde{\mathbf{B}}\solutionvector = \tilde{\mathbf{r}} \eqc
\end{equation}
\begin{equation}
  \tilde{B}\ij = \left\{\begin{array}{l l}
    \left\{\begin{array}{c l}
      \alpha_i & j   = i\\
      0        & j \ne i\\
    \end{array}\right. & i \in \incomingindices\\
    B\ij & i \notin \incomingindices
  \end{array}\right. \eqc
\end{equation}
\begin{equation}
  \tilde{r}_i = \left\{\begin{array}{l l}
    \alpha_i u^{\textup{inc}}_i & i \in    \incomingindices\\
    r_i                         & i \notin \incomingindices
  \end{array}\right. \eqc
\end{equation}
\end{subequations}
where $\alpha_i$ is some positive value, $u^{\textup{inc}}_i$ is the incoming
value to be imposed, and $\incomingindices$ is the set of
indices of degrees of freedom (DoF) for which incoming boundary conditions apply.
To summarize, for an incoming DoF $i$, the row $i$ of the system matrix
$\mathbf{B}$ is zeroed, except for the diagonal value $\tilde{B}_{i,i}$,
which is set to some positive value $\alpha_i$. The right-hand-side value
$r_i$ is then replaced with $\tilde{r}_i=\alpha u^{\textup{inc}}_i$.
In this way, the $i$th equation is replaced with the equation
$\alpha_i\solutionletter_i = \alpha_i u^{\textup{inc}}_i$. Optionally, one
may also eliminate all of the off-diagonal entries corresponding to $i$
by bringing them over to the right-hand-side:
\begin{subequations}
\begin{equation}
  \tilde{B}\ij = \left\{\begin{array}{l l}
    \left\{\begin{array}{c l}
      \alpha_i & j   = i\\
      0        & j \ne i\\
    \end{array}\right. & i \in \incomingindices\\
    \left\{\begin{array}{c l}
      0        & j \in \incomingindices\\
      B\ij     & j \notin \incomingindices\\
    \end{array}\right. & i \notin \incomingindices\\
  \end{array}\right. \eqp
\end{equation}
\begin{equation}
  \tilde{r}_i = \left\{\begin{array}{l l}
    \alpha_i u^{\textup{inc}}_i & i \in    \incomingindices\\
    r_i - \sum\limits_{j\in\incomingindices}B\ij u^{\textup{inc}}_j
      & i \notin \incomingindices
  \end{array}\right. \eqp
\end{equation}
\end{subequations}
This approach has the advantage of preserving symmetry of the original
matrix $\mathbf{B}$.

To weakly impose the incoming boundary conditions for a degree
of freedom $i\in\incomingindices$, one starts by
integrating the advection term in Equation \eqref{eq:Aij} by parts:
\begin{align}
  \intSij \testfunction_i\velocity\cdot\nabla\testfunction_j\dvolume
    &= -\int\limits_{\support\ij} \testfunction_j\velocity
      \cdot\nabla\testfunction_i\dvolume + 
    \int\limits_{\partial\support\ij} \testfunction_j\testfunction_i
      \velocity\cdot\normalvector d\area\\
    &= -\int\limits_{\support\ij} \testfunction_j\velocity
      \cdot\nabla\testfunction_i\dvolume
    + \sum\limits_{F\in\partial\support\ij}\int\limits_F \testfunction_j\testfunction_i
      \velocity\cdot\normalvector d\area
\end{align}
The interior face terms cancel with their skew-symmetric counterparts
and can thus be ignore, leaving only face terms for exterior faces
$F\in\partial\support\ij^{\textup{ext}}$:
\begin{align}
  \intSij \testfunction_i\velocity\cdot\nabla\testfunction_j\dvolume
    &= -\int\limits_{\support\ij} \testfunction_j\velocity
      \cdot\nabla\testfunction_i\dvolume
    + \sum\limits_{F\in\partial\support\ij^{\textup{ext}}}
      \int\limits_F \testfunction_j\testfunction_i
      \velocity\cdot\normalvector d\area
\end{align}
One then brings these terms to the right-hand-side, multiplied by the boundary
data at $\x_j$. In summary, the modified steady-state matrix and
steady-state right-hand-side vector are
\begin{subequations}
\begin{equation}
  \tilde{\ssmatrixletter}\ij = \intSij\left(-\testfunction_j\velocity
    \cdot\nabla\testfunction_i +
  \reactioncoef\testfunction_j\testfunction_i\right) \dvolume \eqc
\end{equation}
\begin{equation}
  \tilde{b}_i = \left\{\begin{array}{l l}
    b_i - \sumj\pr{\sum\limits_{F\in\partial\support\ij^{\textup{ext}}}
          \int\limits_F \testfunction_j\testfunction_i
            \velocity\cdot\normalvector d\area}
          u^{\textup{inc}}_j $$ & i \in    \incomingindices\\
    b_i                         & i \notin \incomingindices
  \end{array}\right. \eqp
\end{equation}
\end{subequations}

For the boundary penalty method, to impose a Dirichlet boundary condition
for degree of freedom $i$, instead of replacing equation $i$ with
$\alpha_i\solutionletter_i=\alpha_i u^{\textup{inc}}_i$, as is done
with strongly imposed Dirichlet boundary conditions, this equation is
\emph{added} to the existing equation. The strength of the imposition
of the boundary condition increases with increasing $\alpha_i$. The
modified system matrix and right-hand-side vector are the following:
\begin{subequations}
\begin{equation}
  \tilde{B}\ij = \left\{\begin{array}{l l}
    \left\{\begin{array}{c l}
      B_{i,i} + \alpha_i & j   = i\\
      B\ij               & j \ne i\\
    \end{array}\right. & i \in \incomingindices\\
    B\ij & i \notin \incomingindices
  \end{array}\right. \eqc
\end{equation}
\begin{equation}
  \tilde{r}_i = \left\{\begin{array}{l l}
    r_i + \alpha_i u^{\textup{inc}}_i & i \in    \incomingindices\\
    r_i                               & i \notin \incomingindices
  \end{array}\right. \eqp
\end{equation}
\end{subequations}
This approach may be used with or without applying weak Dirichlet
boundary conditions.

