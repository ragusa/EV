Strong Stability-Preserving Runge Kutta (SSPRK) methods
\cite{gottlieb}\cite{macdonald} are a subclass of
Runge Kutta methods that offer high-order accuracy while preserving stability.
Suppose that one wants to integrate the following ODE system:
\begin{equation}
  \ddt{\solutionvector} = \mathbf{G}(t,\solutionvector(t)) \eqp
\end{equation}
For example, the function $\mathbf{G}(t,\solutionvector(t))$ corresponding
to Equation \eqref{eq:semidiscrete} is
\begin{equation}
  \mathbf{G}(t,\solutionvector(t)) = (\consistentmassmatrix)^{-1}
    \pr{\ssrhs(t) - \ssmatrix\solutionvector(t)} \eqp
\end{equation}
In $\alpha$-$\beta$ notation, SSPRK methods can be described by the following
steps:
\begin{subequations}
\begin{align}
  & \RKstagesolution^0 = \solutionvector^n \eqc \\
  & \RKstagesolution^i = \sum\limits_{j=1}^{i-1} \sq{
      \alpha\ij\RKstagesolution^{j-1}
      + \dt\beta\ij\mathbf{G}(t^n+c_j\dt, \RKstagesolution^{j-1})}
    \eqc \quad
    i = 1,\ldots,s
    \eqc \\
  & \solutionvector^{n+1} = \RKstagesolution^s \eqc
\end{align}
\end{subequations}
where $s$ is the number of stages of the method, and $\alpha$, $\beta$, and $c$
are coefficient arrays corresponding to the particular SSPRK method.
Results in this dissertation use two SSPRK methods: the forward (explicit) Euler
method, and the 3-stage, 3rd-order-accurate Shu-Osher scheme, hereafter
referred to as SSPRK33. The SSPRK33 method has the following coefficients:
\begin{equation}
  \alpha = \left[\begin{array}{c c c}
    1           &             & \\
    \frac{3}{4} & \frac{1}{4} & \\
    \frac{1}{3} & 0           & \frac{2}{3}
    \end{array}\right] \eqc \quad
  \beta = \left[\begin{array}{c c c}
    1           &             & \\
    0           & \frac{1}{4} & \\
    0           & 0           & \frac{2}{3}
    \end{array}\right] \eqc \quad
  c = \left[\begin{array}{c}
    0           \\
    1           \\
    \frac{1}{2}
    \end{array}\right] \eqp
\end{equation}
Alternatively, the SSPRK methods used in this dissertation can be expressed
in the form
\begin{subequations}
\begin{align}
  & \RKstagesolution^0 = \solutionvector^n \eqc \\
  & \RKstagesolution^i = \gamma_i \solutionvector^n + \zeta_i \sq{
      \RKstagesolution^{i-1}
      + \dt\mathbf{G}(t^n+c_i\dt, \RKstagesolution^{i-1})}
    \eqc \quad
    i = 1,\ldots,s
    \eqc \\
  & \solutionvector^{n+1} = \RKstagesolution^s \eqp
\end{align}
\end{subequations}
This form makes it clear that these methods can be expressed as a linear
combination of steps resembling a forward Euler step, except that quantities
with explicit time dependence are not always evaluated at the time
corresponding to the beginning of the step. This allows the methodology
developed for forward Euler in this research to be directly extended to these
SSPRK methods.  This includes, for example, the FCT methodology described in
Sections \ref{sec:fct_scalar} and \ref{sec:fct_systems}.
In the $\gamma$-$\zeta$ notation, the coefficient arrays for SSPRK33 are
\begin{equation}
  \gamma = \left[\begin{array}{c}
    0\\\frac{3}{4}\\\frac{1}{3}\end{array}\right]
  \eqc \quad
  \zeta = \left[\begin{array}{c}
    1\\\frac{1}{4}\\\frac{2}{3}\end{array}\right]
  \eqc \quad
  c = \left[\begin{array}{c}0\\1\\\frac{1}{2}\end{array}\right] \eqp
\end{equation}
