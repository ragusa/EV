In 1-D, the boundary integrals in Equations \eqref{eq:shallowwater_height_weak_form}
and \eqref{eq:shallowwater_momentumx_weak_form} reduce to differences between
the right and left boundaries:
\begin{subequations}
\begin{equation}\label{eq:boundary_height_1d}
  \intboundary{\testfunction_i^\height\tilde{\heightmomentum}\cdot\normalvector}
  = \pr{\height\velocityx}_R - \pr{\height\velocityx}_L \eqc 
\end{equation}
\begin{equation}\label{eq:boundary_momentumx_1d}
  \intboundary{\testfunction_i^{\heightmomentumx}
    \pr{\frac{\tilde{\heightmomentumx}^2}{\height}
    + \frac{1}{2}\gravity\tilde{\height}^2}\normalx}
  = \pr{\height\velocityx^2 + \half\gravity\height^2}_R
    -\pr{\height\velocityx^2 + \half\gravity\height^2}_L
  \eqc
\end{equation}
\end{subequations}
where $R$ and $L$ denote right and left boundaries, respectively.

In 1-D, there are 2 characteristics,
\begin{subequations}
\begin{equation}
  d\velocityx - 2d\speedofsound = 0\eqc
\end{equation}
\begin{equation}
  d\velocityx + 2d\speedofsound = 0 \eqc
\end{equation}
\end{subequations}
which correspond to the eigenvalues $\lambda_1=\velocityx - \speedofsound$ and
$\lambda_2=\velocityx + \speedofsound$, respectively.
Integrating these from the boundary position $x\BC$ to the adjacent
node position $x\interior$ gives
\begin{subequations}
\begin{equation}\label{eq:characteristic_bc_1}
  \velocityx\interior - \velocityx\BC
  = 2\pr{\speedofsound\interior - \speedofsound\BC} \eqc
\end{equation}
\begin{equation}\label{eq:characteristic_bc_2}
  \velocityx\interior - \velocityx\BC
  = 2\pr{\speedofsound\BC - \speedofsound\interior} \eqc
\end{equation}
\end{subequations}
associated with $\lambda_1$ and $\lambda_2$, respectively.
At each boundary, one
must determine whether the waves associated with each eigenvalue are coming
into the domain or going out of the domain; this determines how many external
boundary conditions must be applied at each boundary.
If $\lambda_i\normalx \leq 0$, then an external boundary condition must be
applied for the $i$-wave; otherwise, internal information is used for that
boundary condition.

For subcritical flow, i.e., $|\velocityx|\leq\speedofsound$,
the signs of each eigenvalue are $\lambda_1\leq 0$ and $\lambda_2\geq 0$.
For supercritical flow, the signs are $\lambda_1\geq 0$
and $\lambda_2\geq 0$ for $\velocityx<0$, and for $\velocityx\geq 0$,
the signs are $\lambda_1\leq 0$ and $\lambda_2\leq 0$. Thus for supercritical
flow, inlets, i.e., boundaries for which $\velocityx\normalx<0$, require
2 external boundary conditions, whereas outlets use 2 internal boundary
conditions.

As an example, consider a problem with subcritical flow in which both
boundaries are outlets. At each boundary it will be necessary to provide one
external boundary condition. Suppose at both boundaries one chooses to fulfill
this requirement by providing boundary height values: $\height_L$ and $\height_R$.
Thus the task remains to compute the boundary velocities
$\velocityx_L$ and $\velocityx_R$, which is achieved by using Equations
\eqref{eq:characteristic_bc_1} and \eqref{eq:characteristic_bc_2}.
The sound speeds $\speedofsound_L$ and $\speedofsound_R$ used in these
equations are computed directly from the provided boundary height values
$\height_L$ and $\height_R$.
For the left boundary, the 2-wave comes from the exterior and thus Equation
\eqref{eq:characteristic_bc_2} is used to compute the left boundary
velocity:
\begin{equation}
  \velocityx_L
  = \velocityx_1 + 2\pr{\speedofsound_1 - \speedofsound_L} \eqc
\end{equation}
where the numbering of degrees of freedom starts from zero and
thus $\velocityx_1$ and $\speedofsound_1$ correspond to the node
adjacent to the left boundary node.
For the right boundary, the 1-wave comes from the exterior, and
thus Equation \eqref{eq:characteristic_bc_1} is used instead of
\eqref{eq:characteristic_bc_2}:
\begin{equation}
  \velocityx_R
  = \velocityx_{N-1} + 2\pr{\speedofsound_R - \speedofsound_{N-1}} \eqc
\end{equation}
where $N$ is the total number of nodes.
With the boundary values determined, Equations \eqref{eq:boundary_height_1d}
and \eqref{eq:boundary_momentumx_1d} are ready to be applied.

