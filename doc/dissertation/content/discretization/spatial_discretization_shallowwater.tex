Rearranging the continuity equation, substituting the approximate FEM
solution and testing with a test function $\testfunction_i^\height$ gives its
weak form for degree of freedom $i$:
\begin{equation}
  \intdomain{\testfunction_i^\height\ppt{\tilde{\height}}}
  = - \intdomain{\testfunction_i^\height\nabla\cdot\tilde{\heightmomentum}}
  \eqp
\end{equation}
Integrating by parts gives
\begin{equation}\label{eq:shallowwater_height_weak_form}
  \intdomain{\testfunction_i^\height\ppt{\tilde{\height}}}
  = \intdomain{\nabla\testfunction_i^\height\cdot\tilde{\heightmomentum}}
  - \intboundary{\testfunction_i^\height\tilde{\heightmomentum}\cdot\normalvector}
  \eqp
\end{equation}
Rearranging the momentum equation for the $d$ direction, substituting the
approximate FEM solution, and testing with a test function
$\testfunction_i^{\heightmomentumd}$ gives its weak form for degree of freedom
$i$:
\begin{equation}
  \intdomain{\testfunction_i^{\heightmomentumd}\ppt{\tilde{\heightmomentumd}}}
  =
  - \intdomain{\testfunction_i^{\heightmomentumd}
      \divergence\pr{\frac{\tilde{\heightmomentumd}}{\height}
        \approximate{\heightmomentum}
      + \frac{1}{2}\gravity\tilde{\height}^2\unitvector{d}}}
  - \intdomain{\testfunction_i^{\heightmomentumd}
      \gravity\tilde{\height}\pd{\bathymetry}{x_d}}
  \eqp
\end{equation}
Integrating by parts for the flux terms gives
\begin{multline}\label{eq:shallowwater_momentumx_weak_form}
  \intdomain{\testfunction_i^{\heightmomentumd}\ppt{\tilde{\heightmomentumd}}}
  =
  + \intdomain{\nabla\testfunction_i^{\heightmomentumd}
      \cdot\pr{\frac{\tilde{\heightmomentumd}}{\height}
        \approximate{\heightmomentum}
      + \frac{1}{2}\gravity\tilde{\height}^2\unitvector{d}}}
  \\
  - \intboundary{\testfunction_i^{\heightmomentumd}
      \pr{\frac{\tilde{\heightmomentumd}}{\height}
        \approximate{\heightmomentum}
      + \frac{1}{2}\gravity\tilde{\height}^2\unitvector{d}}\cdot\normalvector}
  - \intdomain{\testfunction_i^{\heightmomentumd}
      \gravity\tilde{\height}\pd{\bathymetry}{x_d}}
  \eqp
\end{multline}
In a more compact vector format, the momentum equations may be expressed as
\begin{multline}
  \intdomain{\testfunction_i^{\heightmomentum}\cdot\ppt{\tilde{\heightmomentum}}}
  = \intdomain{\nabla\testfunction_i^{\heightmomentum}:
    \left(\tilde{\heightmomentum}\otimes\tilde{\velocity}
    + \frac{1}{2}\gravity\tilde{\height}^2\mathbf{I}\right)}
  \\
  - \intboundary{\testfunction_i^{\heightmomentum}\cdot
    \left(\tilde{\heightmomentum}\otimes\tilde{\velocity}
    + \frac{1}{2}\gravity \tilde{\height}^2\mathbf{I}\right)\cdot\normalvector}
  - \intdomain{\testfunction_i^{\heightmomentum}\cdot
    \gravity\tilde{\height}\nabla\bathymetry}
  \eqp
\end{multline}
This yields a discrete system
\begin{equation}
  \consistentmassmatrix\ddt{\solutionvector} = \ssres \eqc
\end{equation}
where $\consistentmassmatrix$ is the mass matrix and the steady-state residual
$\ssres$ is given by
\begin{multline}
  r_i =
  \intdomain{\nabla\testfunction_i^\height\cdot\tilde{\heightmomentum}}
  - \intboundary{\testfunction_i^\height\tilde{\heightmomentum}\cdot\normalvector}
  + \intdomain{\nabla\testfunction_i^{\heightmomentum}:
    \left(\tilde{\heightmomentum}\otimes\tilde{\velocity}
    + \frac{1}{2}\gravity\tilde{\height}^2\mathbf{I}\right)}
  \\
  - \intboundary{\testfunction_i^{\heightmomentum}\cdot
    \left(\tilde{\heightmomentum}\otimes\tilde{\velocity}
    + \frac{1}{2}\gravity \tilde{\height}^2\mathbf{I}\right)\cdot\normalvector}
  - \intdomain{\testfunction_i^{\heightmomentum}\cdot
    \gravity\tilde{\height}\nabla\bathymetry}
\end{multline}
