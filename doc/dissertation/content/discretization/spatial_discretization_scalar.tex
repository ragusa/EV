The continuous Galerkin (CG) finite element method (FEM) is used for spatial
discretization.
In this research, linear piecewise polynomials are used to approximate
the solution; formally, the approximation space is the following:
\begin{equation}
  \approximationspace = \left\{v\in C^0(\domain;\realspace);
    \, v|_K\circ\referenceelementmap_K \in \qonespace
    \quad \forall K \in \triangulation\right\} \eqc
\end{equation}
where $\referenceelementmap_K$ is a map from the reference element
to an element $K$, and $\triangulation$ is the triangulation.
When the incoming flux boundary condition of Equation \eqref{eq:incoming_flux}
is strongly imposed, the approximation space reduces to
\begin{equation}
  \approximationspaceinc = \left\{v\in \approximationspace;
    \, v(\x) = u^{\text{inc}}(\x)
    \quad \forall \x \in \domainboundary^-\right\} \eqc
\end{equation}
where $u^{\text{inc}}(\x)$ is the incoming flux function.
The approximate solution is an expansion of basis functions $\testfunction_j(\x)$:
\begin{equation}
  \approximatescalarsolution\xt = \sumj \solutionletter_j(\timevalue)
  \testfunction_j(\x) \eqc
\end{equation}
where the coefficients $\solutionletter_j(\timevalue)$ are the basis function
expansion coefficients at time $\timevalue$. Substituting the approximate
solution into Equation \eqref{eq:scalar_transport} and testing with basis
function $\testfunction_i(\x)$ gives
\begin{equation}
   \intSi\ppt{\approximatescalarsolution}\testfunction_i(\x) \dvolume
      + \intSi\left(\divergence\consfluxscalar[\approximatescalarsolution]
      + \reactioncoef(\x)\approximatescalarsolution\xt\right)
      \testfunction_i(\x) \dvolume
      = \intSi \scalarsource\xt \testfunction_i(\x) \dvolume \eqc
\end{equation}
where $\support_i$ is the support of $\testfunction_i(\x)$. If the flux
function $\consfluxscalar$ is linear with respect to $\scalarsolution$, i.e.,
$\consfluxscalar = \velocity\scalarsolution$ for some uniform velocity field
$\velocity$, then the system to be solved is linear:
\begin{equation}\label{eq:semidiscrete}
  \consistentmassmatrix\ddt{\solutionvector}+\ssmatrix\solutionvector(t)
  = \ssrhs(\timevalue) \eqc
\end{equation}
with the elements of $\ssmatrix$ being the following:
\begin{equation}\label{eq:Aij}
  \ssmatrixletter\ij \equiv \intSij\left(
  \velocity\cdot\nabla\testfunction_j(\x) +
  \reactioncoef(\x)\testfunction_j(\x)\right)\testfunction_i(\x) \dvolume \eqc
\end{equation}
where $\support\ij$ is the shared support of $\testfunction_i(\x)$ and
$\testfunction_j(\x)$. Figure \ref{fig:shared_support} illustrates an
example of this definition.
%-------------------------------------------------------------------------------
\begin{figure}[ht]
   \centering
     \def\pointsize{2pt}

\begin{tikzpicture}[
  scale=1]

\coordinate (p1) at (1,1);
\coordinate (p2) at (2.1,0.4);
\coordinate (p3) at (3.3,0.4);
\coordinate (p4) at (4.3,0.9);
\coordinate (p5) at (5.2,1.3);
\coordinate (p6) at (5.8,2.1);
\coordinate (p7) at (5.8,3.5);
\coordinate (p8) at (5.2,4);
\coordinate (p9) at (4.5,4.8);
\coordinate (p10) at (3.6,5);
\coordinate (p11) at (2.7,4.5);
\coordinate (p12) at (1.6,4.6);
\coordinate (p13) at (0.8,3.5);
\coordinate (p14) at (1.2,2.5);
\coordinate (p15) at (2,1.5);
\coordinate (p16) at (3.3,1.2);
\coordinate (p17) at (4,1.9);
\coordinate (p18) at (5,2.2);
\coordinate (p19) at (5,3.3);
\coordinate (p20) at (4.2,3.8);
\coordinate (p21) at (3.3,3.6);
\coordinate (p22) at (2,3.7);
\coordinate (i) at (3,2.5);
\coordinate (j) at (4,3);

\fill (p1) circle (\pointsize);
\fill (p2) circle (\pointsize);
\fill (p3) circle (\pointsize);
\fill (p4) circle (\pointsize);
\fill (p5) circle (\pointsize);
\fill (p6) circle (\pointsize);
\fill (p7) circle (\pointsize);
\fill (p8) circle (\pointsize);
\fill (p9) circle (\pointsize);
\fill (p10) circle (\pointsize);
\fill (p11) circle (\pointsize);
\fill (p12) circle (\pointsize);
\fill (p13) circle (\pointsize);
\fill (p14) circle (\pointsize);
\fill (p15) circle (\pointsize);
\fill (p16) circle (\pointsize);
\fill (p17) circle (\pointsize);
\fill (p18) circle (\pointsize);
\fill (p19) circle (\pointsize);
\fill (p20) circle (\pointsize);
\fill (p21) circle (\pointsize);
\fill (p22) circle (\pointsize);
\fill (i) circle (\pointsize);
\fill (j) circle (\pointsize);

\draw (p1) -- (p2);
\draw (p1) -- (p14);
\draw (p1) -- (p15);
\draw (p2) -- (p3);
\draw (p2) -- (p15);
\draw (p2) -- (p16);
\draw (p3) -- (p4);
\draw (p3) -- (p16);
\draw (p4) -- (p5);
\draw (p4) -- (p16);
\draw (p4) -- (p17);
\draw (p4) -- (p18);
\draw (p5) -- (p6);
\draw (p5) -- (p18);
\draw (p6) -- (p7);
\draw (p6) -- (p18);
\draw (p6) -- (p19);
\draw (p7) -- (p8);
\draw (p7) -- (p19);
\draw (p8) -- (p9);
\draw (p8) -- (p19);
\draw (p8) -- (p20);
\draw (p9) -- (p10);
\draw (p9) -- (p20);
\draw (p10) -- (p11);
\draw (p10) -- (p20);
\draw (p10) -- (p21);
\draw (p11) -- (p12);
\draw (p11) -- (p21);
\draw (p11) -- (p22);
\draw (p12) -- (p13);
\draw (p12) -- (p22);
\draw (p13) -- (p14);
\draw (p13) -- (p22);
\draw (p14) -- (p15);
\draw (p14) -- (p22);
\draw (p15) -- (i);
\draw (p15) -- (p16);
\draw (p15) -- (p22);
\draw (p16) -- (i);
\draw (p16) -- (p17);
\draw (p17) -- (i);
\draw (p17) -- (j);
\draw (p17) -- (p18);
\draw (p18) -- (j);
\draw (p18) -- (p19);
\draw (p19) -- (j);
\draw (p19) -- (p20);
\draw (p20) -- (j);
\draw (p20) -- (p21);
\draw (p21) -- (i);
\draw (p21) -- (j);
\draw (p21) -- (p22);
\draw (p22) -- (i);
\draw (i) -- (j);

\draw[thick, draw=black, fill=red, fill opacity=0.25] (p15)--(p16)--(p17)--(j)
  --(p21)--(p22)--cycle;
\draw[thick, draw=black, fill=blue, fill opacity=0.25] (p17)--(p18)--(p19)--(p20)
  --(p21)--(i)--cycle;

\node at ($(i)-(0.25,0)$) {$i$};
\node at ($(j)+(0.2,0.25)$) {$j$};

\node at (2.5,3) {\Large $\textcolor{gray!20!red}{\support_i}$};
\node at (3.5,2.8) {\Large $\textcolor{violet}{\support\ij}$};
\node at (4.5,2.5) {\Large $\textcolor{gray!20!blue}{\support_j}$};

\end{tikzpicture}

      \caption{Illustration of Shared Support between Test Functions}
   \label{fig:shared_support}
\end{figure}
%-------------------------------------------------------------------------------
If the flux function $\consfluxscalar$ is nonlinear, then the system is
nonlinear, but it may be expressed in a quasi-linear form:
\begin{equation}\label{eq:semi_quasilinear}
   \consistentmassmatrix\ddt{\solutionvector}
   + \ssmatrix(\approximatescalarsolution)\solutionvector(\timevalue)
   = \ssrhs(\timevalue) \eqc
\end{equation}
where $\approximatescalarsolution\xt$ is the numerical solution, and the
quasi-linear matrix (i.e., the Jacobian matrix) entries are
\begin{equation}\label{eq:Aij_nonlinear}
  \ssmatrixletter\ij(\approximatescalarsolution) \equiv \intSij\left(
  \mathbf{\consfluxletter}'(\approximatescalarsolution)\cdot\nabla
  \testfunction_j(\x) +
  \reactioncoef(\x)\testfunction_j(\x)\right)
  \testfunction_i(\x) \dvolume \eqp
\end{equation}
The elements of $\ssrhs(\timevalue)$ are
\begin{equation}
  \ssrhsletter_i(\timevalue) \equiv \intSi \scalarsource\xt\testfunction_i(\x)
  \dvolume \eqp
\end{equation}
$\consistentmassmatrix$ is the consistent mass matrix, which has the entries
\begin{equation}\label{eq:massmatrix}
  \massmatrixletter^C\ij \equiv \intSij
  \testfunction_j(\x)\testfunction_i(\x) \dvolume \eqp
\end{equation}
Similarly, for the linear steady-state case, the linear system is
\begin{equation}
  \ssmatrix\solutionvector = \ssrhs \eqc
\end{equation}
or for the nonlinear case,
\begin{equation}
  \ssmatrix(\approximatescalarsolution)\solutionvector = \ssrhs \eqp
\end{equation}
