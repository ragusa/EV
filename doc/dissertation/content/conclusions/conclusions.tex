This research investigated a number of different numerical methods for
the solution of hyperbolic PDEs with the continuous finite element
method.

A first-order, positivity-preserving, DMP-satisfying method
for scalar hyperbolic PDEs, recently developed by Guermond and Nazarov
\cite{guermond_firstorder} for scalar hyperbolic PDEs, was extended
to hyperbolic PDEs including a reaction term and extraneous source term.

A first-order, positivity-preserving, and domain-invariant method for systems
of hyperbolic PDEs, recently developed by Guermond and Popov
\cite{guermond_invariantdomain} for systems of hyperbolic PDEs,
has been applied to the shallow water equations with flat bottom topography.

The entropy viscosity method developed by Guermond and others \cite{guermond_ev}
was applied to scalar transport and the shallow water equations, with flat
or non-flat bottom topography. Results show that addition of this entropy-based
artificial dissipation results in convergence to the entropy solution
and reduces the onset of spurious oscillations but in general does not eliminate
them completely, and thus the entropy viscosity method is not immune to
solution negativities.

The flux corrected transport (FCT) algorithm, originally developed by Boris
and Book \cite{borisbook} was implemented in conjunction with the entropy
viscosity method, as in \cite{guermond_secondorder}. In addition to the
family of explicit SSPRK methods, steady-state and implicit $\theta$
time discretization methods were employed. For all time discretizations,
the FCT algorithm could be used to guarantee the absence of solution
negativities, and spurious oscillations are not observed in practice
although have not been proven impossible in the FCT algorithm. The formation
of spurious plateaus (also known as ``terracing'' or ``stair-stepping'')
remains an open issue for FCT, particularly when using explicit Euler.

The selection of the solution bounds to impose in the FCT algorithm
was found to be vital; usage of the low-order scheme DMP was found in
general to produce first-order spatial convergence for radiation transport.
Using solution bounds derived from the method of characteristics allowed
second-order spatial convergence to be achieved. These analytic solution
bounds have the advantage of being fully explicit; however, this is only
valid for CFL less than one. Increasing the CFL above one requires widening
the stencil in the min/max operations in the solution bounds, thus making
it less restrictive. Thus for large CFL numbers, one must use the low-order
DMP solution bounds, which would be implicit. The implicitness of the
solution bounds necessitates the usage of nonlinear iteration, which can
be problematic; severe convergence difficulties have been noted in many
cases, and the success of the iteration process is sometimes dependent
on the initial guess. These issues make implicit FCT unreliable; a remedy
to these challenges has not yet been found.

The FCT algorithm was also applied to the shallow water equations,
again in conjunction with the entropy viscosity method. In this case,
no discrete maximum principle applies as in the scalar case. Therefore,
the approach taken was to transform the system into characteristic variables
to allow scalar FCT methodology to be applied. This was found to have some
success; however, this approach was limited to 1-D because characteristic
transformations could not be applied simulatenously in multiple directions.
