The solution of conservation law equations such as the neutron transport
equation presents a number of unique challenges; in the vicinity of strong
gradients and discontinuities, numerical solutions are prone to spurious
oscillations that may generate unphysical values. For example,
numerical schemes may generate negative solution values for physically
non-negative quantities such as scalar flux or angular flux
if adequate precautions are not taken.
These negativities are not only undesirable because they are physically
incorrect, but also because often numerical solution algorithms completely break
down, causing simulations to terminate prematurely. Even more consequential
is the possibility that these negative solution values go undiscovered
and cause significant inaccuracies in quantities of interest.
This is a particularly serious possibility, as these erroneous results may
lead to poor design choices, thus presenting significant safety concerns.

The formation of spurious oscillations and negativities is a well-known issue
in numerical discretizations of hyperbolic partial differential equations (PDEs), which
include, for example, linear advection, Burger's equation, the inviscid Euler
equations of gas dynamics, and the shallow water (or Saint-Venant) equations.
These PDEs result from manipulating the corresponding integral conservation law
equations; however, these manipulations are only valid when the solution is
smooth - in the presence of shocks, the PDE form breaks down
\cite{leveque2002}\refsec{11.6}. Thus it becomes necessary to work with these
equations in a weak form, which holds in the presence of shocks.  However, the
mathematical formulations for these problems do not necessarily yield unique
weak solutions; this is a manifestation of the lack of some physics in the
underlying hyperbolic PDE model \cite{leveque2002}\refsec{11.13}.

%Attempts to address these issues have taken a number
%of different approaches for different discretization schemes.

To produce a unique, physically meaningful solution, it is necessary to
enforce additional conditions, often called \emph{admissibility conditions}
or \emph{entropy conditions}, which filter out spurious weak solutions,
leaving only the physical, \emph{entropy-satisfying} weak solution
\cite{leveque2002}\refsec{11.13}.
There are a number of entropy conditions that may be applied: some
examples are the Lax entropy condition and the Oleinik entropy
condition\cite{leveque2002}\refsec{11.13}; however, it is typically impractical
to apply these conditions in a numerical simulation. The research in this
dissertation employs the notion of an entropy-based artificial viscosity,
based on the recent work of Guermond et al.\cite{guermond_ev}.
The notion of entropy stems from
thermodynamics, in which entropy is a non-decreasing function of time, whereas
the concept of entropy in mathematics is usually viewed as the opposite: it is
a non-\emph{increasing} function of time.
