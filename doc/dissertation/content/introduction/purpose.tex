The solution of hyperbolic conservation law equations
presents a number of unique challenges; in the vicinity of strong
gradients and discontinuities, numerical solutions are prone to spurious
oscillations that may generate unphysical values. For example,
numerical schemes may generate negative solution values for physically
non-negative quantities such as scalar flux or angular flux
if adequate precautions are not taken.
These negativities are not only undesirable because they are physically
incorrect, but also because often numerical solution algorithms completely break
down, causing simulations to terminate prematurely. Even more consequential
is the possibility that these negative solution values go undiscovered
and cause significant inaccuracies in quantities of interest.
This is a particularly serious possibility, as these erroneous results may
lead to poor design choices, thus presenting significant safety concerns.

The formation of spurious oscillations and negativities is a well-known issue
in numerical discretizations of hyperbolic partial differential equations (PDEs), which
include, for example, linear advection, Burger's equation, the inviscid Euler
equations of gas dynamics, and the shallow water (or Saint-Venant) equations.
These PDEs result from manipulating the corresponding integral conservation law
equations; however, these manipulations are only valid when the solution is
smooth - in the presence of shocks, the PDE form breaks down
\cite{leveque2002}\refsec{11.6}. Thus it becomes necessary to work with these
equations in a weak form, which holds in the presence of shocks.  However, the
mathematical formulations for these problems do not necessarily yield unique
weak solutions; this is a manifestation of the omission of some physics in the
approximate hyperbolic PDE model \cite{leveque2002}\refsec{11.13}.

To produce a unique, physically meaningful solution, it is necessary to
enforce additional conditions, often called \emph{admissibility conditions}
or \emph{entropy conditions}, which filter out spurious weak solutions,
leaving only the physical, \emph{entropy-satisfying} weak solution
\cite{leveque2002}\refsec{11.13}.
There are a number of entropy conditions that may be applied: some
examples are the Lax entropy condition and the Oleinik entropy
condition\cite{leveque2002}\refsec{11.13}; however, it is typically impractical
to apply these conditions in a numerical simulation. The research in this
dissertation employs the notion of an entropy-based artificial viscosity,
based on the recent work of Guermond et al.\cite{guermond_ev}.
%The notion of entropy stems from
%thermodynamics, in which entropy is a non-decreasing function of time, whereas
%in mathematics, the concept of entropy is usually viewed as the opposite: it is
%a non-\emph{increasing} function of time.

While entropy-based methods mitigate the issues of spurious oscillations
and negativities, they still do not resolve the issues entirely, even
though such methods help in convergence to the entropy solution; spurious
oscillations and negativities are still present, although smaller in
magnitude. To further address these issues, one can employ the Flux-Corrected Transport (FCT)
algorithm, which has shown some success in the solution of hyperbolic
conservation laws for several decades.
The FCT algorithm was introduced in 1973 by Boris and Book\cite{borisbook} for
finite difference discretizations of transport problems, and it has
been applied to the finite element method more recently. The idea of FCT
is to blend a low-order scheme that is monotone with a high-order
scheme using a nonlinear limiting procedure. FCT takes the difference
of the high-order and low-order schemes to define antidiffusive
fluxes (or \emph{correction} fluxes), which when added to the low-order
scheme as a source, becomes equivalent to the high-order scheme. However,
the FCT algorithm limits these antidiffusive fluxes to satisfy some
physically-motivated criteria, such as discrete local-extremum-diminishing
(LED) bounds.

For the monotone, low-order method required by the FCT algorithm, this research
uses the discrete maximum principle (DMP) preserving method introduced by Guermond
\cite{guermond_firstorder} for the scalar transport model, and the invariant
domain method for the case of conservation law systems,
also introduced by Guermond \cite{guermond_invariantdomain}.
The invariant domain property is a key
property in ensuring monotonicity; it essentially ensures that the numerical
solution will not leave a domain determined by the initial data \cite{hoff_1985}.
For the case of a scalar conservation law, this property reduces to a discrete
maximum principle.
For the high-order method required by the FCT algorithm, FEM-FCT traditionally
has used the Galerkin method without any artificial dissipation \cite{kuzmin_FCT},
which in some cases is adequate; however, in
this work, an entropy-based dissipation is added to the scheme to enforce
an entropy inequality, as performed by Guermond
\cite{guermond_ev}\cite{guermond_secondorder}.

