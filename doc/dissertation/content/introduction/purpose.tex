The solution of conservation law equations such as the neutron transport
equation presents a number of unique challenges; in the vicinity of strong
gradients and discontinuities, numerical solutions are prone to spurious
oscillations that may generate unphysical values. For example, physically
non-negative quantities such as scalar flux or angular flux may have negative
numerical solution values if care is not taken in the numerical scheme.
These negativities are not only undesirable because they are physically
incorrect, but also because often numerical algorithms completely break
down, causing simulations to abort, or worse, altering results without
discovery that unphysical values were encountered. The consequences of
these results may lead to poor design choices, which can pose serious safety
risks when a design is implemented.

These issues of spurious oscillations and negativities are a well-known
phenomenon in the simulation of hyperbolic partial differential equations,
for example, linear advection, Burger's equation, the inviscid Euler equations
of gas dynamics, and the shallow water (or Saint-Venant) equations.
These PDEs result from taking the differential form of the corresponding integral
conservation law equations; however, the differential forms of these equations
are not physically correct - they do not hold in the differential case
because they break down in the presence of a discontinuity.
Moreover, some physics is omitted in these models, such as the
lack of an entropy condition.
The mathematical formulations for these problems
do not always guarantee a unique solution; this is a manifestation of the
neglect of physics in the underlying PDE model.

Attempts to address these issues have taken a number
of different approaches for different discretization schemes.
