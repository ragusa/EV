Negativities have long been an outstanding issue in the numerical solution
of the neutron transport equation \cite{lanthrop}. While negativities
may not significantly degrade the accuracy of a method, they may cause
drastic, unintended consequences where assumptions of non-negativity
are employed.

There have been a number of attempts to address this issue for various
spatial discretizations. Many rely on ad-hoc fixups, such as the classic
set-to-zero fixup for the diamond difference scheme \cite{lewis}.
Hamilton \cite{hamilton} introduced a similar fixup method for the
linear discontinuous (LD) finite element method (FEM) that conserves local
particle balance and keeps the third-order accuracy of the standard
LD FEM. Walters and Wareing \cite{walters} developed a nonlinear
spatial differencing scheme for one-dimensional slab geometry.
So-called characteristic methods were developed by Walters and Wareing
\cite{walters_NC} and Minor \cite{minor}.
Wareing notes in \cite{wareing} that these characteristic methods
are difficult to derive and implement and offers instead
a nonlinear positive spatial differencing scheme called
the exponential discontinuous scheme, which was applicable in 1-D,
2-D, and 3-D cartesian meshes. More recently, Maginot has
developed a consistent set-to-zero method for LD FEM \cite{maginot},
as well as a non-negative, Bilinear Discontinuous (BLD) method
\cite{maginot_mc2015}.
