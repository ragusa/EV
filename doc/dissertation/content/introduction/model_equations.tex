This research considers two physical models, both of which are of
great importance in the field of nuclear engineering:
linear transport, and the shallow water equations (SWE).
The linear transport model considered in this research is the following:
\begin{equation}\label{eq:rad_transport}
  \frac{1}{\speed}\ppt{\angularflux} + \directionvector\cdot\nabla\angularflux\xt
  + \totalcrosssection(\x)\angularflux\xt = \radiationsource\xt \eqc
\end{equation}
where $\angularflux\xt$ is the angular flux in direction $\directionvector$,
$\speed$ is the transport speed, $\totalcrosssection(\x)$
is the macroscopic total cross-section, and $\radiationsource\xt$ is an
extraneous source.
This model is a special case of the full neutron transport equation,
\begin{multline}
  \frac{1}{\speed(E)}\ppt{\angularflux} + \directionvector\cdot\nabla\angularflux\xdet
    + \totalcrosssection\xet\angularflux\xdet = \radiationsource\xdet\\
    + \frac{\chi_\text{p}(E)}{4\pi}\int\limits_0^\infty
      dE'\nu_\text{p}(\x,E',t)\Sigma_\text{f}(\x,E',t)\phi(\x,\di,E',t)
    + \sum\limits_{i=1}^{n_\text{d}}\frac{\chi_{\text{d},i}(E)}{4\pi}\lambda_i C_i\xt\\
    + \int\limits_0^\infty dE'\int\limits_{4\pi}d\di'
      \Sigma_\text{s}(\x,E'\rightarrow E,\di'\rightarrow\di,t)\angularflux(\x,\di',E',t)
  \eqc
\end{multline}
where $E$ is energy, $\Sigma_\text{f}$ is the fission cross section, $\Sigma_\text{s}$ is the
double-differential scattering cross section, $\chi_\text{p}$ is the prompt
neutron energy spectrum, $\nu_\text{p}$ is the prompt
neutron yield, $n_\text{d}$ is the number of delayed neutron precursors,
and $\chi_{\text{d},i}$, $\lambda_i$, and $C_i$ are the delayed neutron
energy spectrum, decay constant, and concentration of precursor $i$, respectively.
Equation \eqref{eq:rad_transport} is obtained by assuming that the transport
medium does not scatter, fission, or change with time. The arguments
$E$ and $\di$ are dropped for the remainder of the dissertation
because this dependence is not important since the omission of scattering
and fission terms decouples the equation in energy and direction.
The units are thus intentionally left ambiguous.

The transport equation is often referred to as the Boltzmann equation
due to its resemblance to the transport theory developed by Boltzmann
in the 1800s to study kinetic theory of gases \cite{duderstadt}\cite{glasstone}.
In general, transport theory is used in describing the transport of particles
or waves through some background media. Such particles might include neutrons,
electrons, ions, gas molecules, or photons.
Since its origination, transport theory has evolved indepedently in many
different disciplines and applications such as nuclear reactors,
atmospheric science, radiation therapy, radiation shielding, and stars.
In fact, much of the early transport theory was developed to meet the needs
of astrophysical research in the study of stellar and planetary
atmospheres \cite{duderstadt}.
In the field of nuclear engineering, the transport equation is
especially important, as the transport equation is used to guide nuclear
reactor design and operation as well as design and analysis of
radiation shielding.

This research also considers the shallow water equations (SWE):
\begin{subequations}
\begin{align}
  \ppt{\height} + \ppx{(\height\velocityx)} + \ppy{(\height\velocityy)} &= 0
  \eqc\\
  \ppt{(\height\velocityx)}
    + \ppx{}\pr{\height\velocityx^2 + \frac{1}{2}\gravity\height^2}
    + \ppy{}\pr{\height\velocityx\velocityy} &= 0
  \eqc\\
  \ppt{(\height\velocityy)}
    + \ppx{}\pr{\height\velocityx\velocityy}
    + \ppy{}\pr{\height\velocityy^2 + \frac{1}{2}\gravity\height^2} &= 0
  \eqc
\end{align}
\end{subequations}
where $\height$ is the height, $\velocityx$ is the x-velocity,
$\velocityy$ is the y-velocity, and $\gravity$ is acceleration due to
gravity.
The SWE are derived from the Navier-Stokes
equations by making the assumption that the horizontal length
scale is much larger than the vertical length scale. Under this
assumption, the vertical velocity is small (not necessarily zero),
and upon depth-integrating the equations, the vertical velocity
is removed. The SWE have many applications, including coastal
flows, lakes, rivers, dam breaks, and tsunamis \cite{kirby}.

The importance of the tsunami modelling in nuclear engineering has unfortunately
become very clear in recent years.
The magnitude 9.0 earthquake that occurred off the coast of Tohoku in 2011,
often referred to as the Great East Japan earthquake \cite{USGS_tohoku},
produced a tsunami that not only
resulted in an enormous loss of life and infrastructure, but
also resulted in the most severe nuclear accident since the Chernobyl
disaster in 1986 \cite{IAEAfukushima}, becoming only the second
accident in history to receive the highest rating of 7 on the International
Nuclear and Radiological Event Scale (INES) \cite{IAEA_fukushima_scale}.
Kanayama \cite{Kanayama2013} performed a tsunami simulation of
Hakata Bay using the viscous shallow water equations, which was
used to evaluate damage to coastal areas of the Tohoku district.
