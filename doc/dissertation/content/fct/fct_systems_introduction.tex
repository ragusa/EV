Recall that for scalar conservation laws, a discrete maximum principle (DMP) can
be derived for the low-order scheme and used for the evaluation of the
imposed FCT bounds. In the case of \emph{systems} of conservation laws,
no DMP is available to use for this purpose. Thus the bounds to impose on
the FCT solution are unclear. The invariant domain property of the low-order
scheme for systems is not useful in the manner that the DMP property was
for scalar conservation laws; this is because while one can prove that the
low-order solution is in some invariant domain, one does not necessarily
have knowledge of the extent of the domain itself and thus it cannot
be translated to solution bounds.

Direct extension of scalar FCT methodology to the systems case is often met
with poor results because this extension is typically not physically valid.
Kuzmin states that limitation on alternative sets of variables such as
primitive variables or characteristic variables, rather than conservative
variables, typically produces better quality FCT results \cite{kuzmin_FCT}.
For example, for the shallow water equations, one may consider the
following sets of variables:
\begin{itemize}
  \item Conservative:
    $\vectorsolution \equiv [\height,\height\velocityx]\transpose$,
  \item Primitive:
    $\check{\vectorsolution} \equiv [\height,\velocityx]\transpose$,
  \item Characteristic:
    $\hat{\vectorsolution} \equiv [\velocityx-2\speedofsound,
      \velocityx+2\speedofsound]\transpose$,
      where $\speedofsound\equiv\sqrt{\gravity\height}$.
\end{itemize}
Section \ref{sec:fct_systems_scheme} gives the general FCT scheme for systems,
with applicability to those FCT schemes limiting non-conservative sets of
variables.
