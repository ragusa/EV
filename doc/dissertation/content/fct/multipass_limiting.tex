In this section, a novel approach for FCT limitation is desribed, in which
multiple passes are taken through any limiter to maximize the amount of
antidiffusion that is accepted without violating the imposed bounds.

To date, no practical limiter has been developed that perfectly solves
the optimization problem; solving an optimization problem exactly would
be too computationally expensive to be used in practical calculations.
Thus all limiters are suboptimal; additional antidiffusion can be accepted
without violating the imposed bounds. For example, recall that the Zalesak
limiter described in Section \ref{sec:zalesak_limiter} makes the safe
choice that when considering the upper antidiffusion bound
$\antidiffusionbound_i^+$ for node $i$, the limiting coefficients
for the positive antidiffusive fluxes, $\limiterletter_i^+$, are computed
while assuming that there is no contribution to the limited antidiffusion sum
$\limitedcorrectionfluxletter_i$ from \emph{negative} antidiffusive fluxes:
\[
  \limiterletter_i^+ = \min\pr{1,\frac{\antidiffusionbound_i^+}
    {\correctionfluxletter_i^+}} \eqp
\]
A more optimal limiter would consider the negative fluxes here, for example
by the following approach:
\[
  \limiterletter_i^+ = \min\pr{1,\frac{\antidiffusionbound_i^+
    - \limitedcorrectionfluxletter_i^-}
    {\correctionfluxletter_i^+}} \eqc
\]
where
\[
  \limitedcorrectionfluxletter_i^- = \sum\limits_{j:\correctionfluxij<0}
    \limiterletter\ij\correctionfluxij \eqp
\]
Of course, the problem with this hypothetical limiter is that one does not
know the limiting coefficients for the negative antidiffusive fluxes yet;
thus Zalesak designed the limiter to assume those limiting coefficients
are all zero. Zalesak's limiter is suboptimal whenever
positive antidiffusive fluxes are limited
(this happens when $\correctionfluxletter_i^+ > \antidiffusionbound^+$)
and there are some nonzero limiting coefficients for the negative antidiffusive
fluxes. Furthermore, in the end, symmetry is enforced on the limiting
coefficients, and thus the actual limiting coefficients applied to the
positive antidiffusive fluxes are not $\limiterletter_i^+$ but instead,
$\limiterletter\ij$, which is either lesser or equal:
$\limiterletter\ij \leq\limiterletter_i^+$ (see Equation
\eqref{eq:limiting_coefficients} to see why). With these arguments considered
(and they apply similarly to the lower bound), Zalesak's limiter has
room for improvement.

The idea of multi-pass limiting is to take multiple passes through a
limiter to overcome the suboptimality.
A limiter may be considered as a black box in which the inputs are the
antidiffusive fluxes $\{\correctionfluxij\}\ij$ and the antidiffusion bounds
$\{\antidiffusionbound_i^\pm\}_i$ and the outputs are the limiting coefficients
$\{\limiterletter\ij\}\ij$. See Figure \ref{eq:limiting_coefficients} for
an illustration of this concept.

%-------------------------------------------------------------------------------
\begin{figure}[htb]
   \centering
     % Basic limiter diagram
\begin{tikzpicture}[scale=1]
  %\draw[-{Latex[length=\arrowheadlength,width=\arrowheadwidth]}]
  \draw[->]
    (0,0.5*\boxheight) --
    (\arrowlength,0.5*\boxheight);
  \draw (\arrowlength,0) rectangle
    (\arrowlength+\boxwidth,\boxheight);
  %\draw[-{Latex[length=\arrowheadlength,width=\arrowheadwidth]}]
  \draw[->]
    (\arrowlength+\boxwidth,0.5*\boxheight) -- 
    (2*\arrowlength+\boxwidth,0.5*\boxheight);
  \node at (\arrowlength+0.5*\boxwidth,0.5*\boxheight) {Limiter};
  \node at (0.5*\arrowlength,0.5*\boxheight+\textpad)
    {$\{\correctionfluxij\}, \{\antidiffusionbound_i^\pm\}$};
  \node at (1.5*\arrowlength+\boxwidth,0.5*\boxheight+\textpad)
    {$\{\limiterletter\ij\}$};
\end{tikzpicture}

      \caption{Limiter Input and Output}
   \label{fig:limiting_coefficients}
\end{figure}
%-------------------------------------------------------------------------------

In multi-pass limiting, a loop is formed around the limiter. After each pass
through the limiter, the remainder of the antidiffusive fluxes
$\correctionfluxmatrixremainder$ is computed and passed back to the limiter
in place of the original antidiffusive fluxes $\correctionfluxmatrix$. In
this next iteration, some antidiffusion $\limitedcorrectionfluxletter_i$
has now already been accepted, and thus in the limiter, one does not start
from $\limitedcorrectionfluxletter_i=0$ when considering antidiffusion bounds but instead
starts from $\limitedcorrectionfluxletter_i=\limitedcorrectionfluxletter_i^{(\ell-1)}$.
For example if $\limitedcorrectionfluxletter_i>0$, then the remaining positive
antidiffusion fluxes have less room to be accepted, and the remaining negative
fluxes have \emph{more} room to be accepted. For example, Zalesak's limiter
would be modified as
\[
  \limiterletter_i^+ = \min\pr{1,\frac{\antidiffusionbound_i^+
    - \limitedcorrectionfluxletter_i^{(\ell-1)}}
    {\correctionfluxletter_i^+}} \eqp
\]
Perhaps a better approach is simply to pass into the limiter
$\{\antidiffusionbound_i^\pm-\limitedcorrectionfluxletter_i^{(\ell-1)}\}_i$ 
in place of $\{\antidiffusionbound_i^\pm\}_i$. This achieves the same effect but
avoids any modification of the limiter function.
With each iteration, the accepted antidiffusion becomes smaller,
and the iteration is terminated when the change in the total accepted
antidiffusion source becomes sufficiently small. This is illustrated in
Figure \ref{fig:multipass_limiting}.

%-------------------------------------------------------------------------------
\begin{figure}[htb]
   \centering
     \def\mynewboxwidth{5}
\def\yone{4}
\def\textdistance{0.6}
\def\nodedistance{3}

% Multi-pass limiter diagram
\begin{tikzpicture}[
  scale=1,
  transform shape,
  %show background rectangle,
  every text node part/.style={align=center},
  mybox/.style={
    draw,
    fill=white,
    rectangle,
    minimum width={10}}]

  \coordinate (start) at (0,\yone);
  \node[draw] (dP) at ($(start.center)+(\nodedistance,0)$)
    {$\ell \gets 0$\\
    $\correctionfluxmatrixremainder^{(\ell)} \gets \correctionfluxmatrix$};
  \node[mybox] (lim)  at ($(dP.center)+(1.3*\nodedistance,0)$) {Limiter};
  \node[draw, diamond] (if)  at ($(lim.center)+(1.1*\nodedistance,0)$)
    {$\limitedcorrectionfluxletter^{\text{tot},(\ell)} = 0$};
  \coordinate (end)     at ($(if.center)+(1.3*\nodedistance,0)$);

  \node (Ptext) at ($(start.center)+(0.25*\nodedistance,\textpad)$)
    {$\{\antidiffusionbound_i^\pm\}$};
  \node at ($(Ptext.center)+(0,\textdistance)$)
    {$\{\correctionfluxij\}$};

  \node (dPtext) at ($(dP.center)+(0.75*\nodedistance,\textpad)$)
    {$\{\antidiffusionbound_i^\pm\}$};
  \node at ($(dPtext.center)+(0,\textdistance)$)
    {$\{\correctionfluxremainder^{(\ell)}\ij\}$};

  \node at ($(lim.center)+(0.45*\nodedistance,\textpad)$)
    {$\{\limiterletter\ij^{(\ell)}\}$};
  \node at ($(if.center)+(\nodedistance,\textpad)$) {$\{\limiterletter\ij\}$};
  \node at ($(if.center)+(0.55*\nodedistance,\textpad)$) {true};
  \node at ($(if.center)+(2*\textpad,-0.5*\nodedistance)$) {false};
  %\node at ($(if.center)+(0.75*\nodedistance,\textpad)$)
  %  {$\{\limiterletter\ij\}$};
  \coordinate (reentry) at ($(dP.center)+(0.5*\nodedistance,0)$);
  \coordinate (belowif) at ($(if.center)+(0,-0.75*\nodedistance)$);
  \node[draw] (dPupdate) at ($(reentry.center)+(0,-0.75*\nodedistance)$)
    {$\correctionfluxremainder\ij^{(\ell+1)} \gets
      \correctionfluxremainder\ij^{(\ell)}
      - \limiterletter\ij^{(\ell)}\correctionfluxremainder\ij^{(\ell)}$\\
    $\ell \gets \ell+1$};
  \draw[->] (start)     -- (dP);
  \draw[->] (dP)        -- (lim);
  \draw[->] (lim)       -- (if);
  \draw[->] (if)        -- (end);
  \draw[->] (if)        -- (belowif) -- (dPupdate);
  \draw[->] (dPupdate)  -- (reentry);

  \draw[style=dashed] ($(start.center)+(0.5*\nodedistance,-1*\nodedistance)$)
    rectangle ($(if.center)+(0.75*\nodedistance,0.5*\nodedistance)$);
\end{tikzpicture}

\[
  \limitedcorrectionfluxletter^{\text{tot},(\ell)} \gets \sum\limits_{i,j}
    \left|\limiterletter\ij^{(\ell)}\correctionfluxremainder\ij^{(\ell)}\right|
\]

      \caption{Multi-Pass Limiting Diagram}
   \label{fig:multipass_limiting}
\end{figure}
%-------------------------------------------------------------------------------

