The FCT algorithm requires that the antidiffusion bounds
$\limitedfluxbound_i^\pm$,
\[
  \limitedfluxbound_i^- \leq \limitedfluxsumi \leq \limitedfluxbound_i^+ \eqc
\]
must have the following properties:
\begin{equation}\label{eq:antidiffusion_bounds_properties}
  \limitedfluxbound_i^-\leq 0 \eqc \quad \limitedfluxbound_i^+\geq 0 \eqp
\end{equation}
Otherwise, there would not be a fail-safe definition of the limiting
coefficients; the FCT solution could not necessarily revert to the low-order
solution, which corresponds to $\limiterletter\ij = 0 \;\forall i,j$.
If either of the conditions are violated, then the default choice
$\limiterletter\ij = 0 \;\forall i,j$ does not even satisfy the
antidiffusion bounds, so there is no guarantee that there is any
combination of limiting coefficients that could satisfy the bounds.

Most limiters assume these conditions are satisfied; if they are
used despite the conditions not being satisfied, then negative limiting
coefficients may be produced.
While one may decide that negative limiting coefficients are acceptable,
this is usually considered undesirable because it is usually the goal
to produce limiting coefficients between 0 and 1:
\begin{equation}
  0 \leq \limiterletter\ij \leq 1 \eqc \quad \forall i,j \eqc
\end{equation}
representing a scale between complete rejection and complete acceptance
of an antidiffusion flux, respectively. A negative limiting coefficient
would represent a \emph{reversal} of an antidiffusive flux.

The following theorems give conditions under which Equation
\eqref{eq:antidiffusion_bounds_properties} is satisfied for each temporal
discretization.

%Recall that the antidiffusion bounds have a fixed definition that take the
%imposed solution bounds as input:
%$\limitedfluxbound_i^\pm(\solutionbound_i^\pm)$, which are given
%by Equations
%\eqref{eq:antidiffusion_bounds_steady_state},
%\eqref{eq:limited_flux_bounds_explicit_euler}, and
%\eqref{eq:antidiffusion_bounds_theta},
%for steady-state, explicit Euler, and theta time discretizations, respectively.
%In the case of steady-state and theta time discretizations, the
%antidiffusion bounds are implicit, even if the solution bounds
%$\solutionbound_i^\pm$ are computed explicitly: thus the antidiffusion
%bounds for theta methods, for example, are the functions
%$\limitedfluxbound_i^\pm(\solutionbound_i^\pm,\solutionvector^{n+1})$.

%------------------------------------------------------------------------------
\begin{theorem}{Signs of Antidiffusion Bounds for Steady-State Scheme}
The antidiffusion bounds $\limitedfluxbound_i^\pm$ for the steady-state
scheme, given by Equation \eqref{eq:antidiffusion_bounds_steady_state},
satisfy the conditions given by Equation
\eqref{eq:antidiffusion_bounds_properties} when the DMP solution bounds given
by Equation \eqref{eq:DMP_ss} are used
and both the antidiffusion bounds
and DMP solution bounds are evaluated with the same solution $\solutionvector$:
\begin{subequations}
\begin{equation}
  \limitedfluxbound_i^+ = \ssmatrixletter_{i,i}^L \DMPupperbound_i(\solutionvector)
    + \sumjnoti\ssmatrixletter\ij^L\solutionletter_j - \ssrhsletter_i \geq 0\eqc
\end{equation}
\begin{equation}
  \limitedfluxbound_i^- = \ssmatrixletter_{i,i}^L \DMPlowerbound_i(\solutionvector)
    + \sumjnoti\ssmatrixletter\ij^L\solutionletter_j - \ssrhsletter_i \leq 0\eqp
\end{equation}
\end{subequations}
\end{theorem}

\begin{proof}
Starting with Equation \eqref{eq:antidiffusion_bounds_steady_state},
\[
     \limitedfluxbound_i^\pm \equiv \ssmatrixletter_{i,i}^L \solutionbound_i^\pm
       + \sumjnoti\ssmatrixletter\ij^L\solutionletter_j - \ssrhsletter_i \eqp
\]
Evaluating Equation \eqref{eq:DMP_ss} with an arbitrary solution
$\solutionvector$ instead of the low-order solution $\solutionvector^L$ gives
\[
   \DMPbounds_i(\solutionvector)
     \equiv -\frac{1}{\ssmatrixletter^L_{i,i}}
      \sumjnoti\ssmatrixletter^L_{i,j}
      \solutionletter_{\substack{\max\\\min},j\ne i}
      + \frac{\ssrhsletter_i}{\ssmatrixletter^L_{i,i}} \eqc
\]
where
\[
  \solutionletter_{\min,j\ne i} \equiv \min\limits_{j\ne i\in\indicescell[\support_i]}
    \solutionletter_j
  \eqc\quad
  \solutionletter_{\max,j\ne i} \equiv \max\limits_{j\ne i\in\indicescell[\support_i]}
    \solutionletter_j
  \eqp
\]
Substituting this expression for $\solutionbound_i^\pm$ into the antidiffusion
bounds expression gives
\[
     \limitedfluxbound_i^\pm =
       - \sumjnoti\ssmatrixletter\ij^L
         (\solutionletter_{\substack{\max\\\min},j\ne i} - \solutionletter_j) \eqp
\]
By Lemma \ref{lem:offdiagonalnegative}, the off-diagonal matrix entries
$\ssmatrixletter\ij^L$ are non-positive,
and by definition, $\solutionletter_{\max,j\ne i} \geq \solutionletter_j$ and
$\solutionletter_{\min,j\ne i} \leq \solutionletter_j$, so the signs of
Equation \eqref{eq:antidiffusion_bounds_properties} are verified.\qed
\end{proof}
%------------------------------------------------------------------------------
%------------------------------------------------------------------------------
\begin{theorem}{Signs of Antidiffusion Bounds for Explicit Euler Scheme}
If the time step size $\dt$ satisfies Equation \eqref{eq:explicit_cfl},
\[
  \timestepsize \leq \frac{\massmatrixletter_{i,i}^{L}}
    {\ssmatrixletter_{i,i}^{L,\timeindex}}
  \eqc\quad\forall i \eqc
\]
then the antidiffusion bounds $\limitedfluxbound_i^\pm$ for the explicit Euler
scheme, given by Equation \eqref{eq:limited_flux_bounds_explicit_euler},
satisfy the conditions given by Equation
\eqref{eq:antidiffusion_bounds_properties} when the DMP solution bounds given
by Equation \eqref{eq:explicit_dmp} are used:
\begin{subequations}
\begin{equation}
  \limitedfluxbound_i^+ = \lumpedmassentry
    \frac{\DMPupperbound_i-\solutionletter_i^\timeindex}{\timestepsize}
  + \sumj \ssmatrixletter\ij^L \solutionletter_j^\timeindex
  - \ssrhsletter_i^\timeindex \geq 0\eqc
\end{equation}
\begin{equation}
  \limitedfluxbound_i^- = \lumpedmassentry
    \frac{\DMPlowerbound_i-\solutionletter_i^\timeindex}{\timestepsize}
  + \sumj \ssmatrixletter\ij^L \solutionletter_j^\timeindex
  - \ssrhsletter_i^\timeindex \leq 0\eqp
\end{equation}
\end{subequations}
\end{theorem}

\begin{proof}
Starting with Equation \eqref{eq:limited_flux_bounds_explicit_euler},
\[
  \limitedfluxbound_i^\pm \equiv \lumpedmassentry
    \frac{\solutionbound_i^\pm-\solutionletter_i^\timeindex}{\timestepsize}
  + \sumj \ssmatrixletter\ij^L \solutionletter_j^\timeindex
  - \ssrhsletter_i^\timeindex \eqp
\]
Substituting Equation \eqref{eq:explicit_dmp} into this expression gives
\[
  \limitedfluxbound_i^\pm = \frac{\lumpedmassentry}{\dt}
    \left(\pr{1-\frac{\timestepsize}{\massmatrixletter_{i,i}^L}
       \sumj\ssmatrixletter^{L}_{i,j}}
     \solutionletter_{\substack{\max\\\min},i}^\timeindex
     + \frac{\timestepsize}{\massmatrixletter_{i,i}^L}\ssrhsletter_i^\timeindex
      -\solutionletter_i^n\right)
  + \sumj \ssmatrixletter\ij^L \solutionletter_j^n
  - \ssrhsletter_i^\timeindex \eqc
\]
\[
  \limitedfluxbound_i^\pm = 
    \pr{\frac{\lumpedmassentry}{\dt}
       -\sumj\ssmatrixletter^{L}_{i,j}}
     \solutionletter_{\substack{\max\\\min},i}^\timeindex
      -\frac{\lumpedmassentry}{\dt}\solutionletter_i^n
  + \sumj \ssmatrixletter\ij^L \solutionletter_j^n
  \eqc
\]
\[
  \limitedfluxbound_i^\pm = 
    \pr{\frac{\lumpedmassentry}{\dt}
       -\ssmatrixletter^L_{i,i}}
     \pr{\solutionletter_{\substack{\max\\\min},i}^n-\solutionletter_i^n}
    -\sumjnoti\ssmatrixletter^L_{i,j}
     \pr{\solutionletter_{\substack{\max\\\min},i}^n-\solutionletter_j^n}
  \eqp
\]
Due to the time step size restriction of Equation \eqref{eq:explicit_cfl},
\[
  \frac{\lumpedmassentry}{\dt}
       -\ssmatrixletter^L_{i,i} \geq 0
  \eqc
\]
and by Lemma \ref{lem:offdiagonalnegative}, the off-diagonal matrix entries
$\ssmatrixletter\ij^L$ are non-positive.
By definition, $\solutionletter_{\max,i}^n \geq \solutionletter_j^n$ and
$\solutionletter_{\min,i}^n \leq \solutionletter_j^n$, so the signs of
Equation \eqref{eq:antidiffusion_bounds_properties} are verified.\qed
\end{proof}
%------------------------------------------------------------------------------
%------------------------------------------------------------------------------
\begin{theorem}{Signs of Antidiffusion Bounds for Theta Scheme}
If the time step size $\dt$ satisfies Equation \eqref{eq:theta_cfl},
\[
   \timestepsize \leq \frac{\massmatrixletter^L_{i,i}}{(1-\theta)
     \ssmatrixletter_{i,i}^{L,\timeindex}}
   \eqc\quad\forall i \eqc
\]
then the antidiffusion bounds $\limitedfluxbound_i^\pm$ for the theta
scheme, given by Equation \eqref{eq:antidiffusion_bounds_theta},
satisfy the conditions given by Equation
\eqref{eq:antidiffusion_bounds_properties} when the DMP solution bounds given
by Equation \eqref{eq:theta_dmp} are used
and both the antidiffusion bounds
and DMP solution bounds are evaluated with the same solution $\solutionvector^{n+1}$:
\begin{subequations}
  \begin{multline}
    \limitedfluxbound_i^+ =
     \pr{\frac{\lumpedmassentry}{\dt}+\theta\ssmatrixletter_{i,i}^L}
       \DMPupperbound_i(\solutionvector^{n+1})
      + \pr{(1-\theta)\ssmatrixletter_{i,i}^L-\frac{\lumpedmassentry}{\dt}}
       \solutionletter_i^n\\
    +\sumjnoti \ssmatrixletter\ij^L \solutionletter_j^\theta
    -\ssrhsletter_i^\theta \geq 0
    \eqc
  \end{multline}
  \begin{multline}
    \limitedfluxbound_i^- =
     \pr{\frac{\lumpedmassentry}{\dt}+\theta\ssmatrixletter_{i,i}^L}
       \DMPlowerbound_i(\solutionvector^{n+1})
      + \pr{(1-\theta)\ssmatrixletter_{i,i}^L-\frac{\lumpedmassentry}{\dt}}
       \solutionletter_i^n\\
    +\sumjnoti \ssmatrixletter\ij^L \solutionletter_j^\theta
    -\ssrhsletter_i^\theta \leq 0
    \eqp
  \end{multline}
\end{subequations}
\end{theorem}

\begin{proof}
Starting with Equation \eqref{eq:antidiffusion_bounds_theta},
  \[
    \limitedfluxbound_i^\pm =
     \pr{\frac{\lumpedmassentry}{\dt}+\theta\ssmatrixletter_{i,i}^L}
       \solutionbound_i^\pm
      + \pr{(1-\theta)\ssmatrixletter_{i,i}^L-\frac{\lumpedmassentry}{\dt}}
       \solutionletter_i^n
    +\sumjnoti \ssmatrixletter\ij^L \solutionletter_j^\theta
    -\ssrhsletter_i^\theta
    \eqp
  \]
Substituting Equation \eqref{eq:theta_dmp} into this expression gives
  \[
    \limitedfluxbound_i^\pm =
    -\sumjnoti \ssmatrixletter\ij^L \pr{
      \theta\pr{\solutionletter_{\substack{\max\\\min},j\ne i}^{n+1} - \solutionletter_j^{n+1}}
      + (1-\theta)\pr{\solutionletter_{\substack{\max\\\min},j\ne i}^n - \solutionletter_j^n}}
    \eqc
  \]
and by Lemma \ref{lem:offdiagonalnegative}, the off-diagonal matrix entries
$\ssmatrixletter\ij^L$ are non-positive.
By definition, $\solutionletter_{\max,j\ne i}^n \geq \solutionletter_j^n$ and
$\solutionletter_{\min,j\ne i}^n \leq \solutionletter_j^n$, so the signs of
Equation \eqref{eq:antidiffusion_bounds_properties} are verified.\qed
\end{proof}
%------------------------------------------------------------------------------

If one wishes to use solution bounds $\solutionbound_i^\pm$ other than the
discrete maximum principle bounds, then
one needs to take care to ensure that the imposed bounds yield the properties
given by Equation \eqref{eq:antidiffusion_bounds_properties}.
One way to achieve this is to ensure that the solution bounds
$\solutionbound_i^\pm$ themselves bound the solution bounds for which the
properties given by Equation
\eqref{eq:antidiffusion_bounds_properties}
are known to hold:
\begin{subequations}
\begin{equation}
  \solutionbound_i^-
    \leq \DMPlowerbound_i
  \eqc
\end{equation}
\begin{equation}
  \solutionbound_i^+
    \geq \DMPupperbound_i
  \eqp
\end{equation}
\end{subequations}
These conditions may be enforced by the following operation:
\begin{subequations}\label{eq:solution_bounds_operation}
\begin{equation}
  \tilde{\solutionbound}_i^-
    \equiv \min(\solutionbound_i^-,
      \DMPlowerbound_i)
  \eqc
\end{equation}
\begin{equation}
  \tilde{\solutionbound}_i^+
    \equiv \max(\solutionbound_i^+,
      \DMPupperbound_i)
  \eqp
\end{equation}
\end{subequations}
Then when the antidiffusion bounds $\limitedfluxbound_i^\pm$ are computed from
the solution bounds $\tilde{\solutionbound}_i^\pm$, the properties of Equation
\eqref{eq:antidiffusion_bounds_properties} will still hold.  Similarly, a more
direct approach to enforcing the properties given by Equation
\eqref{eq:antidiffusion_bounds_properties} is not to perform the operation
given by Equation \eqref{eq:solution_bounds_operation} but instead to compute
$\limitedfluxbound_i^\pm$ with the unmodified bounds $\solutionbound_i^\pm$
and then perform the following operation:
\begin{subequations}\label{eq:antidiffusion_bounds_operation}
\begin{equation}
  \tilde{\antidiffusionbound}_i^-
    \equiv \min(\antidiffusionbound_i^-, 0)
  \eqc
\end{equation}
\begin{equation}
  \tilde{\antidiffusionbound}_i^+
    \equiv \max(\antidiffusionbound_i^+, 0)
  \eqp
\end{equation}
\end{subequations}
%This operation potentially gives tighter bounds than the operation of Equation
%\eqref{eq:solution_bounds_operation}.

