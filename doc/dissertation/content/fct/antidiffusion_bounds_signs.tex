The FCT algorithm requires that the antidiffusion bounds
$\limitedfluxbound_i^\pm$ must have the following properties:
\begin{equation}\label{eq:antidiffusion_bounds_properties}
  \limitedfluxbound_i^-\leq 0 \eqc \quad \limitedfluxbound_i^+\geq 0 \eqp
\end{equation}
Otherwise, there would not be a fail-safe definition of the limiting
coefficients; the FCT solution could not necessarily revert to the low-order
solution.
Recall that the antidiffusion bounds have a fixed definition that take the
imposed solution bounds as input:
$\limitedfluxbound_i^\pm(\solutionbound_i^\pm)$, which are given
by Equations
\eqref{eq:antidiffusion_bounds_steady_state},
\eqref{eq:limited_flux_bounds_explicit_euler}, and
\eqref{eq:antidiffusion_bounds_theta},
for steady-state, explicit Euler, and theta time discretizations, respectively.

There are certain $\solutionbound_i^\pm$ that guarantee the properties given
by Equation
\eqref{eq:antidiffusion_bounds_properties}, for example, the
discrete maximum principle bounds given in Equations
\eqref{eq:DMP_ss},
\eqref{eq:explicit_dmp},
\eqref{eq:theta_dmp},
for steady-state, explicit Euler, and theta time discretizations, respectively.
However, it should be noted that the steady-state and theta methods that are
not fully explicit (theta methods other than explicit Euler), have discrete
maximum principle bounds that are implicit with the solution:
$\DMPboundsss_i(\solutionvector)$ and $\DMPboundstheta_i(\solutionvector^{n+1})$,
for steady-state and theta time discretizations, respectively.
It is only proven that the properties given by Equation
\eqref{eq:antidiffusion_bounds_properties}
hold when these expressions are evaluated with the low-order solution.
The following equations summarize these statements
for steady-state, explicit Euler, and theta time discretizations, respectively:
\begin{equation}
  \antidiffusionlowerboundss_i(\DMPlowerboundss_i(\solutionvector^L))
    \leq 0
  \eqc \quad
  \antidiffusionupperboundss_i(\DMPupperboundss_i(\solutionvector^L))
    \geq 0
  \eqc
\end{equation}
\begin{equation}
  \antidiffusionlowerboundee_i(\DMPlowerboundee_i)
    \leq 0
  \eqc \quad
  \antidiffusionupperboundee_i(\DMPupperboundee_i)
    \geq 0
  \eqc
\end{equation}
\begin{equation}
  \antidiffusionlowerboundtheta_i(\DMPlowerboundtheta_i(\solutionvector^{L,n+1}))
    \leq 0
  \eqc \quad
  \antidiffusionupperboundtheta_i(\DMPupperboundtheta_i(\solutionvector^{L,n+1}))
    \geq 0
  \eqp
\end{equation}
If one wishes to use solution bounds $\solutionbound_i^\pm$ other than the
discrete maximum principle bounds evaluated with the low-order solution, then
one needs to take care to ensure that the imposed bounds yield the properties
given by Equation
\eqref{eq:antidiffusion_bounds_properties}.
One way to achieve this is to ensure that the solution bounds
$\solutionbound_i^\pm$ themselves bound the solution bounds for which the
properties given by Equation
\eqref{eq:antidiffusion_bounds_properties}
are known to hold.
For example, for steady-state, this corresponds to the following conditions:
\begin{subequations}
\begin{equation}
  \solutionbound_i^-
    \leq \DMPlowerboundss_i(\DMPlowerboundss_i(\solutionvector^L))
  \eqc
\end{equation}
\begin{equation}
  \solutionbound_i^+
    \geq \DMPupperboundss_i(\DMPupperboundss_i(\solutionvector^L))
  \eqp
\end{equation}
\end{subequations}
These conditions simply may be enforced by the following operation:
\begin{subequations}
\begin{equation}
  \solutionbound_i^-
    \gets \min(\solutionbound_i^-,
      \DMPlowerboundss_i(\solutionvector^L))
  \eqc
\end{equation}
\begin{equation}
  \solutionbound_i^+
    \gets \max(\solutionbound_i^+,
      \DMPupperboundss_i(\solutionvector^L))
  \eqp
\end{equation}
\end{subequations}
Similarly, an even simpler approach to enforcing the property given by Equation
\eqref{eq:antidiffusion_bounds_properties}
is to directly impose the properties:
\begin{subequations}
\begin{equation}
  \antidiffusionbound_i^-
    \gets \min(\antidiffusionbound_i^-, 0)
  \eqc
\end{equation}
\begin{equation}
  \antidiffusionbound_i^+
    \gets \max(\antidiffusionbound_i^+, 0)
  \eqp
\end{equation}
\end{subequations}
Thus when either of the conditions of Equation
\eqref{eq:antidiffusion_bounds_properties}
are violated, then by this operation, the respective solution bound is forced
to be the low order solution:
\begin{subequations}
\begin{equation}
  \antidiffusionbound_i^- = 0 \quad \Rightarrow \quad
  \solutionbound_i^-
    \gets \solutionletter_i^{L}
  \eqc
\end{equation}
\begin{equation}
  \antidiffusionbound_i^+ = 0 \quad \Rightarrow \quad
  \solutionbound_i^+
    \gets \solutionletter_i^{L}
  \eqc
\end{equation}
\end{subequations}
where in the transient case $\solutionletter_i^{L}$ represents
$\solutionletter_i^{L,n+1}$.



