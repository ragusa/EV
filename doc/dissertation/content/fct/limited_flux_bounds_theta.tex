\begin{theorem}{Limited Flux Bounds for Theta Scheme}
  Using the Theta FCT scheme
  \begin{equation}
    \lumpedmassmatrix\frac{\solutionvector^{\timeindex+1}
      - \solutionvector^\timeindex}{\timestepsize}
    + (1-\theta)\loworderssmatrix\solutionvector^\timeindex
    + \theta\loworderssmatrix\solutionvector^{\timeindex+1}
    = (1-\theta)\ssrhs^\timeindex + \theta\ssrhs^{\timeindex+1}
    + \limitedfluxsum \eqc
  \end{equation}
  the following limited flux bounds $\limitedfluxbound_i^\pm$ correspond to the
  discrete maximum principle
  $\DMPbound_i^+\le \solutionletter_i^{\timeindex+1}\le \DMPbound_i^-$:
  \begin{equation}
    \limitedfluxbound_i^\pm \equiv \lumpedmassentry\frac{\DMPbound_i^\pm
      - \solutionletter_i^\timeindex}{\timestepsize}
    +(1-\theta)\sumj \ssmatrixletter\ij^L \solutionletter_j^\timeindex
    +\theta\sumj \ssmatrixletter\ij^L \solutionletter_j^{\timeindex+1}
    -(1-\theta)\ssrhsletter_i^\timeindex
    -\theta \ssrhsletter_i^{\timeindex+1} \eqp
  \end{equation}
\end{theorem}

\begin{proof}
  Starting with row $i$ of Equation \eqref{eq:fct_theta},
   \[
      \lumpedmassentry\frac{\solutionletter_i^{\timeindex+1}
        - \solutionletter_i^\timeindex}{\timestepsize}
      + (1-\theta)\sumj \ssmatrixletter\ij^L \solutionletter_j^\timeindex
      + \theta\sumj \ssmatrixletter\ij^L \solutionletter_j^{\timeindex+1}
      = (1-\theta)\ssrhsletter_i^\timeindex
      + \theta \ssrhsletter_i^{\timeindex+1}
      + \limitedfluxsumi \eqp
   \]
   Solving for $\limitedfluxsumi$ gives
   \[
      \limitedfluxsumi =
      \lumpedmassentry\frac{\solutionletter_i^{\timeindex+1}
        - \solutionletter_i^\timeindex}{\timestepsize}
      + (1-\theta)\sumj \ssmatrixletter\ij^L \solutionletter_j^\timeindex
      + \theta\sumj \ssmatrixletter\ij^L \solutionletter_j^{\timeindex+1}
      - (1-\theta)\ssrhsletter_i^\timeindex
      - \theta \ssrhsletter_i^{\timeindex+1} \eqp
   \]
   The discrete maximum principle is
   \[
      \DMPbound_i^-\le \solutionletter_i^{\timeindex+1}\le \DMPbound_i^+ \eqp
   \]
   Through addition/subtraction and multiplication/division operations, this
   principle can be made to look like the following:
   \begin{multline*}
   \lumpedmassentry\frac{\DMPbound_i^-
     -\solutionletter_i^\timeindex}{\timestepsize}
      + (1-\theta)\sumj \ssmatrixletter\ij^L \solutionletter_j^\timeindex
      + \theta\sumj \ssmatrixletter\ij^L \solutionletter_j^{\timeindex+1}
      - (1-\theta)\ssrhsletter_i^\timeindex
      - \theta \ssrhsletter_i^{\timeindex+1}\\
   \le \lumpedmassentry\frac{\solutionletter_i^{\timeindex+1}
     -\solutionletter_i^\timeindex}{\timestepsize}
      + (1-\theta)\sumj \ssmatrixletter\ij^L \solutionletter_j^\timeindex
      + \theta\sumj \ssmatrixletter\ij^L \solutionletter_j^{\timeindex+1}
      - (1-\theta)\ssrhsletter_i^\timeindex
      - \theta \ssrhsletter_i^{\timeindex+1}\\
   \le \lumpedmassentry\frac{\DMPbound_i^+
     -\solutionletter_i^\timeindex}{\timestepsize}
      + (1-\theta)\sumj \ssmatrixletter\ij^L \solutionletter_j^\timeindex
      + \theta\sumj \ssmatrixletter\ij^L \solutionletter_j^{\timeindex+1}
      - (1-\theta)\ssrhsletter_i^\timeindex
      - \theta \ssrhsletter_i^{\timeindex+1},
   \end{multline*}
   which by substituting equations above gives
   \[
     \limitedfluxbound_i^-\le\limitedfluxsumi\le \limitedfluxbound_i^+ \eqp\qed
   \]
\end{proof}
