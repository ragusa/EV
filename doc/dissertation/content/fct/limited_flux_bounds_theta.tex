\begin{theorem}{Antidiffusion Bounds for Theta Scheme}
  Using the Theta FCT scheme given by Equation \eqref{eq:fct_theta},
  the following antidiffusion bounds $\limitedfluxbound_i^\pm$ correspond to the
  solution bounds
  $\solutionbound_i^-\le \solutionletter_i^{\timeindex+1}\le \solutionbound_i^+$:
  \begin{subequations}
  \begin{equation}\label{eq:antidiffusion_bounds_theta}
    \limitedfluxbound_i^\pm \equiv
     \pr{\frac{\lumpedmassentry}{\dt}+\theta\ssmatrixletter_{i,i}^L}
       \solutionbound_i^\pm
      + \pr{(1-\theta)\ssmatrixletter_{i,i}^L-\frac{\lumpedmassentry}{\dt}}
       \solutionletter_i^n
    +\sumjnoti \ssmatrixletter\ij^L \solutionletter_j^\theta
    -\ssrhsletter_i^\theta
    \eqc
  \end{equation}
  \begin{equation}
    \solutionletter_j^\theta \equiv
      \theta\solutionletter_j^{n+1} + (1-\theta)\solutionletter_j^n
    \eqc
  \end{equation}
  \begin{equation}
    \ssrhsletter_i^\theta \equiv
      \theta\ssrhsletter_i^{n+1} + (1-\theta)\ssrhsletter_i^n
    \eqp
  \end{equation}
  \end{subequations}
\end{theorem}

\begin{proof}
  Starting with row $i$ of Equation \eqref{eq:fct_theta},
  \[
     \pr{\frac{\lumpedmassentry}{\dt}+\theta\ssmatrixletter_{i,i}^L}
       \solutionletter_i^{n+1}
      + \pr{(1-\theta)\ssmatrixletter_{i,i}^L-\frac{\lumpedmassentry}{\dt}}
       \solutionletter_i^n
    +\sumjnoti \ssmatrixletter\ij^L \solutionletter_j^\theta
    = \ssrhsletter_i^\theta
      + \limitedfluxsumi
    \eqp
  \]
  Solving for $\limitedfluxsumi$ gives
  \[
    \limitedfluxsumi =
     \pr{\frac{\lumpedmassentry}{\dt}+\theta\ssmatrixletter_{i,i}^L}
       \solutionletter_i^{n+1}
      + \pr{(1-\theta)\ssmatrixletter_{i,i}^L-\frac{\lumpedmassentry}{\dt}}
       \solutionletter_i^n
    +\sumjnoti \ssmatrixletter\ij^L \solutionletter_j^\theta
    - \ssrhsletter_i^\theta
    \eqp
  \]
   The solution bounds for degree of freedom $i$ are
   \[
      \solutionbound_i^-\le \solutionletter_i^{\timeindex+1}\le \solutionbound_i^+ \eqp
   \]
   Through addition/subtraction and multiplication/division operations, this
   principle can be made to look like the following:
   \begin{multline*}
     \pr{\frac{\lumpedmassentry}{\dt}+\theta\ssmatrixletter_{i,i}^L}
       \solutionbound_i^-
      + \pr{(1-\theta)\ssmatrixletter_{i,i}^L-\frac{\lumpedmassentry}{\dt}}
       \solutionletter_i^n
    +\sumjnoti \ssmatrixletter\ij^L \solutionletter_j^\theta
    -\ssrhsletter_i^\theta\\
   \le
     \pr{\frac{\lumpedmassentry}{\dt}+\theta\ssmatrixletter_{i,i}^L}
       \solutionletter_i^{n+1}
      + \pr{(1-\theta)\ssmatrixletter_{i,i}^L-\frac{\lumpedmassentry}{\dt}}
       \solutionletter_i^n
    +\sumjnoti \ssmatrixletter\ij^L \solutionletter_j^\theta
    -\ssrhsletter_i^\theta\\
   \le
     \pr{\frac{\lumpedmassentry}{\dt}+\theta\ssmatrixletter_{i,i}^L}
       \solutionbound_i^+
      + \pr{(1-\theta)\ssmatrixletter_{i,i}^L-\frac{\lumpedmassentry}{\dt}}
       \solutionletter_i^n
    +\sumjnoti \ssmatrixletter\ij^L \solutionletter_j^\theta
    -\ssrhsletter_i^\theta
   \eqc
   \end{multline*}
   which by substituting equations above gives
   \[
     \limitedfluxbound_i^-\le\limitedfluxsumi\le \limitedfluxbound_i^+ \eqp\qed
   \]
\end{proof}
