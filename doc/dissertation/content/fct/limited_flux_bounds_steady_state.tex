\begin{theorem}{Antidiffusion Bounds for Steady-State Scheme}
   Using the steady-state FCT scheme given by Equation
   \eqref{eq:fct_steady_state},
   the following antidiffusion bounds $\limitedfluxbound_i^\pm$ correspond to the
   solution bounds
   $\solutionbound_i^-\le \solutionletter_i\le \solutionbound_i^+$:
   \begin{subequations}\label{eq:antidiffusion_bounds_steady_state}
   \begin{equation}
     \limitedfluxbound_i^- \leq \limitedfluxsumi \leq \limitedfluxbound_i^+ \eqc
   \end{equation}
   \begin{equation}
     \limitedfluxbound_i^\pm \equiv \ssmatrixletter_{i,i}^L \solutionbound_i^\pm
       + \sumjnoti\ssmatrixletter\ij^L\solutionletter_j - \ssrhsletter_i \eqp
   \end{equation}
   \end{subequations}
\end{theorem}

\begin{proof}
   Starting with row $i$ of Equation \eqref{eq:fct_steady_state},
   \[
      \sumj \ssmatrixletter\ij^L \solutionletter_j
      = \ssrhsletter_i + \limitedfluxsumi \eqp
   \]
   Solving for $\limitedfluxsumi$ gives
   \[
      \limitedfluxsumi = \sumj\ssmatrixletter\ij^L\solutionletter_j
      - \ssrhsletter_i \eqp
   \]
   The solution bounds for $i$ are
   \[
      \solutionbound_i^-\le \solutionletter_i\le \solutionbound_i^+ \eqp
   \]
   Through addition/subtraction and multiplication/division operations, this
   principle can be made to look like the following:
   \begin{multline*}
   \ssmatrixletter_{i,i}^L \solutionbound_i^-
     + \sumjnoti\ssmatrixletter\ij^L\solutionletter_j - \ssrhsletter_i 
     = \limitedfluxbound_i^-\\
   \le \sumj \ssmatrixletter\ij^L \solutionletter_j - \ssrhsletter_i
     = \limitedfluxsumi\\
   \le \ssmatrixletter_{i,i}^L \solutionbound_i^+
     + \sumjnoti\ssmatrixletter\ij^L \solutionletter_j - \ssrhsletter_i
     = \limitedfluxbound_i^+ \eqp\qed
   \end{multline*}
\end{proof}
