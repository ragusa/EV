A classic multi-dimensional limiter used in the FCT algorithm is the
limiter of Zalesak \cite{zalesak}.
Zalesak's limiter separately considers positive and negative antidiffusive
fluxes $\correctionfluxij$ for a degree of freedom $i$ and makes the conservative choice that
the upper antidiffusion bound $\antidiffusionbound_i^+$ does not consider
the possibility of negative fluxes when assigning limiting coefficients
for the positive fluxes, and similarly for the lower antidiffusion bound
$\antidiffusionbound_i^-$.
The following theorem states and proves that the classic Zalesak
limiter produces limiting coefficients that satisfy the imposed solution
bounds.
\begin{theorem}{Zalesak Limiting Coefficients}
   Suppose that that the solution bounds
   $\solutionbound_i^+\le \solutionletter_i^{n+1}\le \solutionbound_i^-$
   correspond to the following inequality for the antidiffusion fluxes:
   \begin{equation}
      \limitedfluxbound_i^- \le \limitedfluxsumi \le \limitedfluxbound_i^+ \eqc
   \end{equation}
   where $\limitedfluxbound_i^\pm$ are bounds that depend on the temporal
   discretization and satisfy the conditions of Equation
   \eqref{eq:antidiffusion_bounds_properties}:
\[
  \limitedfluxbound_i^-\leq 0 \eqc \quad \limitedfluxbound_i^+\geq 0 \eqp
\]
   Then
   the following limiting coefficient definitions satisfy the solution bounds:
   \begin{equation}\label{eq:flux_sums}
      \correctionfluxletter_i^+ \equiv \sumj\max(0,\correctionfluxij) \eqc\qquad
      \correctionfluxletter_i^- \equiv \sumj\min(0,\correctionfluxij) \eqc
   \end{equation}
   \begin{equation}\label{eq:single_node_limiting_coefficients}
      \limiterletter_i^\pm \equiv\left\{
         \begin{array}{l l}
            1 & \correctionfluxletter_i^\pm = 0\\
            \min\left(1,\frac{\limitedfluxbound_i^\pm}
              {\correctionfluxletter_i^\pm}\right) & \correctionfluxletter_i^\pm
              \ne 0
         \end{array}
         \right. \eqc
   \end{equation}
   \begin{equation}\label{eq:limiting_coefficients}
      \limiterletter\ij \equiv\left\{
         \begin{array}{l l}
            \min(\limiterletter_i^+,\limiterletter_j^-)
              & \correctionfluxij \geq 0\\
            \min(\limiterletter_i^-,\limiterletter_j^+)
              & \correctionfluxij < 0
         \end{array}
         \right..
   \end{equation}  
\end{theorem}

\begin{proof}
   First, note some properties of the above definitions:
   \begin{gather*}
      \correctionfluxletter_i^+ \geq 0
        \eqc \qquad \correctionfluxletter_i^- \leq 0 \eqc\\
      \limitedfluxbound_i^+ \geq 0
        \eqc \qquad \limitedfluxbound_i^- \leq 0 \eqc\\
      0 \leq \limiterletter_i^\pm \leq 1 \eqc\\
      0 \leq \limiterletter\ij \leq 1 \eqp
   \end{gather*}
   The proof will be given for the upper bound.
   \[
      \sumj\limiterletter\ij\correctionfluxij
      \leq \sum\limits_{j:\correctionfluxij\geq 0}
        \limiterletter\ij\correctionfluxij
      = \sum\limits_{j:\correctionfluxij\geq 0}
        \min(\limiterletter_i^+,\limiterletter_j^-)\correctionfluxij
      \leq \sum\limits_{j:\correctionfluxij\geq 0}
        \limiterletter_i^+\correctionfluxij
      =\limiterletter_i^+\correctionfluxletter_i^+ \eqp
   \]
   For the case $\correctionfluxletter_i^+ = 0$,
   \[
     \limiterletter_i^+\correctionfluxletter_i^+
       = 0 \leq \limitedfluxbound_i^+ \eqp
   \]
   For the case $\correctionfluxletter_i^+ \ne 0$,
   \[
      \limiterletter_i^+\correctionfluxletter_i^+
      \leq \frac{\limitedfluxbound_i^+}{\correctionfluxletter_i^+}
        \correctionfluxletter_i^+
      = \limitedfluxbound_i^+ \eqp
   \]
   Thus,
   \[
      \sumj \limiterletter\ij\correctionfluxij \leq \limitedfluxbound_i^+ \eqp
   \]
   The lower bound is proved similarly.
   \qed
\end{proof}
