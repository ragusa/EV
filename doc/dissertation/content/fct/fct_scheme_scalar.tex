The crux of the flux-corrected transport scheme is to define an antidiffusive
correction flux $\correctionfluxvector$ from a monotone, low-order scheme to a
high-order scheme.  Thus to define $\correctionfluxvector$ for a particular
temporal discretization, $\correctionfluxvector$ is added as a source to the
respective low-order system given in Section \ref{sec:low_order_scheme_scalar},
except that the solution at $\timeindex+1$ is no longer the low-order solution
$\lowordersolution$, but instead, the high-order solution $\highordersolution$.
The systems defining $\correctionfluxvector$ for each temporal discretization
follow.
\begin{center}{\textbf{Steady-state scheme}:}\end{center}
\begin{equation}\label{eq:correction_flux_definition_steady_state}
  \loworderssmatrix\highordersolution = \ssrhs + \correctionfluxvector
\end{equation}
\begin{center}{\textbf{Semi-discrete scheme}:}\end{center}
\begin{equation}\label{eq:correction_flux_definition_semidiscrete}
  \lumpedmassmatrix\ddt{\highordersolution}
  + \loworderssmatrix\highordersolution(\timevalue)
  = \ssrhs(\timevalue) + \correctionfluxvector
\end{equation}
\begin{center}{\textbf{Explicit Euler scheme}:}\end{center}
\begin{equation}\label{eq:correction_flux_definition_explicit_euler}
  \lumpedmassmatrix\frac{\highordersolution-\solutionvector^\timeindex}
    {\timestepsize}
  + \loworderssmatrix\solutionvector^\timeindex
  = \ssrhs^\timeindex + \correctionfluxvector
\end{equation}
\begin{center}{\textbf{Theta scheme}:}\end{center}
\begin{equation}\label{eq:correction_flux_definition_theta}
  \lumpedmassmatrix\frac{\highordersolution-\solutionvector^\timeindex}
    {\timestepsize}
  + (1-\theta)\loworderssmatrix\solutionvector^\timeindex
  + \theta\loworderssmatrix\highordersolution
  = (1-\theta)\ssrhs^\timeindex + \theta\ssrhs^{\timeindex+1}
  + \correctionfluxvector
\end{equation}
The object of FCT is to limit these correction fluxes to satisfy
physically-motivated bounds imposed on the solution. To adopt the limiting
procedure used in this dissertation, it is necessary to decompose the
correction flux for a node $i$ into a sum of correction fluxes coming into node
$i$ from its neighbors. These decomposed fluxes are conveniently represented by
a correction flux matrix, denoted by $\correctionfluxmatrix$, e.g., entry
$\MakeUppercase{\correctionfluxletter}\ij$ is the correction flux going into node
$i$ from node $j$, and $\sumj\MakeUppercase{\correctionfluxletter}\ij =
\correctionfluxletter_i$. Thus the question remains of how to distribute, or
\emph{decompose}, the correction flux $\correctionfluxletter_i$ among its
neighbors.  A convenient decomposition reveals itself when the correction flux
definitions given by Equations
\eqref{eq:correction_flux_definition_steady_state}
- \eqref{eq:correction_flux_definition_theta} are combined with the respective
high-order system equations given by Equations \eqref{eq:high_ss} -
\eqref{eq:high_theta}. This yields new expressions for $\correctionfluxvector$,
which follow.
\begin{center}{\textbf{Steady-state scheme}:}\end{center}
\begin{equation}\label{eq:correction_flux_steady_state}
  \correctionfluxvector \equiv \pr{\loworderdiffusionmatrix
    - \highorderdiffusionmatrix}\highordersolution
\end{equation}
\begin{center}{\textbf{Semi-discrete scheme}:}\end{center}
\begin{equation}\label{eq:correction_flux_semidiscrete}
  \correctionfluxvector \equiv -\pr{\consistentmassmatrix
    - \lumpedmassmatrix}\ddt{\highordersolution}
  + \pr{\loworderdiffusionmatrix - \highorderdiffusionmatrix(\timevalue)}
    \highordersolution(\timevalue)
\end{equation}
\begin{center}{\textbf{Explicit Euler scheme}:}\end{center}
\begin{equation}\label{eq:correction_flux_explicit_euler}
  \correctionfluxvector \equiv -\pr{\consistentmassmatrix
    - \lumpedmassmatrix}\frac{\highordersolution-\solutionvector^\timeindex}
    {\timestepsize^{\timeindex+1}}
  + \pr{\loworderdiffusionmatrix
    - \highorderdiffusionmatrix[\timeindex]}\solutionvector^\timeindex
\end{equation}
\begin{center}{\textbf{Theta scheme}:}\end{center}
\begin{equation}\label{eq:correction_flux_theta}
  \correctionfluxvector \equiv -\pr{\consistentmassmatrix
  - \lumpedmassmatrix}\frac{\highordersolution-\solutionvector^\timeindex}
    {\timestepsize^{\timeindex+1}}
  + (1-\theta)\pr{\loworderdiffusionmatrix
    - \highorderdiffusionmatrix[\timeindex]}\solutionvector^\timeindex 
  + \theta    \pr{\loworderdiffusionmatrix
    - \highorderdiffusionmatrix[\timeindex+1]}\highordersolution
\end{equation}
These definitions suggest convenient decompositions because
$\consistentmassmatrix-\lumpedmassmatrix$ and
$\loworderdiffusionmatrix-\highorderdiffusionmatrix$ are symmetric
and feature zero row sums:
\begin{align*}
  \sumj\pr{\massmatrixletter^L\ij - \consistentmassentry} &= 0 \eqc\\
  \lumpedmassentry - \sumj\consistentmassentry      &= 0 \eqc
\end{align*}
and
\[
  \sumj\pr{\diffusionmatrixletter^L\ij - \diffusionmatrixletter^H\ij} = 0 \eqp
\]
The following decompositions result for each temporal discretization:
\begin{center}{\textbf{Steady-state scheme}:}\end{center}
\begin{equation}
  \MakeUppercase{\correctionfluxletter}\ij
  = \pr{\diffusionmatrixletter\ij^L-\diffusionmatrixletter\ij^H}
    \pr{\solutionletter_j^H - \solutionletter_i^H}
\end{equation}
\begin{center}{\textbf{Semi-discrete scheme}:}\end{center}
\begin{equation}
  \MakeUppercase{\correctionfluxletter}\ij
  = -\massmatrixletter^C\ij\pr{\ddt{\solutionletter_j^H}
    - \ddt{\solutionletter_i^H}}
  + \pr{\diffusionmatrixletter\ij^L-\diffusionmatrixletter\ij^H(\timevalue)}
    \pr{\solutionletter_j^H(\timevalue) - \solutionletter_i^H(\timevalue)}
\end{equation}
\begin{center}{\textbf{Explicit Euler scheme}:}\end{center}
\begin{equation}
  \MakeUppercase{\correctionfluxletter}\ij
  = -\massmatrixletter^C\ij\pr{\frac{\solutionletter_j^H
    - \solutionletter_j^\timeindex}{\timestepsize}
    - \frac{\solutionletter_i^H - \solutionletter_i^\timeindex}
    {\timestepsize}}
  + \pr{\diffusionmatrixletter\ij^L-\diffusionmatrixletter\ij^{H,n}}
    \pr{\solutionletter_j^\timeindex - \solutionletter_i^\timeindex}
\end{equation}
\begin{center}{\textbf{Theta scheme}:}\end{center}
\begin{multline}
  \MakeUppercase{\correctionfluxletter}\ij
  = -\massmatrixletter^C\ij\pr{\frac{\solutionletter_j^H
    - \solutionletter_j^\timeindex}{\timestepsize}
    - \frac{\solutionletter_i^H - \solutionletter_i^\timeindex}
      {\timestepsize}}
  + (1-\theta)\pr{\diffusionmatrixletter\ij^L
    - \diffusionmatrixletter\ij^{H,\timeindex}}
    \pr{\solutionletter_j^\timeindex - \solutionletter_i^\timeindex}\\
  + \theta    \pr{\diffusionmatrixletter\ij^L
    - \diffusionmatrixletter\ij^{H,\timeindex+1}}
    \pr{\solutionletter_j^H - \solutionletter_i^H}
\end{multline}
Note that the decompositions for the time-dependent schemes above use the fact
\[
  \sumj\consistentmassentry\ddt{\solutionletter_i}
    = \lumpedmassentry\ddt{\solutionletter_i} \eqp
\]
\begin{remark}
If Dirichlet boundary conditions are strongly imposed on a degree of freedom
$i$, then antidiffusive flux decompositions above do not apply. The total
antidiffusive flux into $i$ is $\correctionfluxletter_i=0$, and the antidiffusive
flux decomposition for degrees of freedom $j$ neighboring $i$ are no longer
valid because there is no longer an equal and opposite antidiffusive flux
$\MakeUppercase{\correctionfluxletter}\ij$ to cancel
$\MakeUppercase{\correctionfluxletter}\ji$. Therefore, one must either accept
the lack of the conservation property, or one must completely cancel
$\MakeUppercase{\correctionfluxletter}\ji$.
\end{remark}
Up until this point, no limiting has been applied; using the schemes as defined
in Equations \eqref{eq:correction_flux_definition_steady_state}
- \eqref{eq:correction_flux_definition_theta} would simply reproduce the
high-order solution $\highordersolution$. As stated previously, FCT applies a
limiting procedure to the antidiffusive correction fluxes to satisfy the bounds
that are imposed. This is achieved by assigning each \emph{internodal}
correction flux $\MakeUppercase{\correctionfluxletter}\ij$ its own limiting
coefficient $\limiterletter\ij$, which is applied as a scaling factor. Again,
it is convenient to store these limiting coefficients in a matrix
$\limitermatrix$. Instead of adding the full correction flux to a node $i$,
$\correctionfluxletter_i=\sumj\MakeUppercase{\correctionfluxletter}\ij$, the
limited correction flux sum
$\sumj\limiterletter\ij\MakeUppercase{\correctionfluxletter}\ij$ is added. In
vector form, this row-wise product is denoted by $\limitedfluxsum$, i.e.,
$(\limitedfluxsum)_i =
\limitedfluxsumi$.
The FCT scheme for each temporal discretization follows.
\begin{center}{\textbf{Steady-state scheme}:}\end{center}
\begin{equation}\label{eq:fct_steady_state}
  \loworderssmatrix\solutionvector = \ssrhs + \limitedfluxsum
\end{equation}
\begin{center}{\textbf{Semi-discrete scheme}:}\end{center}
\begin{equation}\label{eq:fct_semidiscrete}
  \lumpedmassmatrix\ddt{\solutionvector}
  + \loworderssmatrix\solutionvector(\timevalue)
  = \ssrhs(\timevalue) + \limitedfluxsum
\end{equation}
\begin{center}{\textbf{Explicit Euler scheme}:}\end{center}
\begin{equation}\label{eq:fct_explicit_euler}
  \lumpedmassmatrix\frac{\solutionvector^{\timeindex+1}
    - \solutionvector^\timeindex}{\timestepsize}
  + \loworderssmatrix\solutionvector^\timeindex
  = \ssrhs^\timeindex + \limitedfluxsum
\end{equation}
\begin{center}{\textbf{Theta scheme}:}\end{center}
\begin{equation}\label{eq:fct_theta}
  \lumpedmassmatrix\frac{\solutionvector^{\timeindex+1}
    - \solutionvector^\timeindex}{\timestepsize}
  + (1-\theta)\loworderssmatrix\solutionvector^\timeindex
  + \theta\loworderssmatrix\solutionvector^{\timeindex+1}
  = (1-\theta)\ssrhs^\timeindex
  + \theta\ssrhs^{\timeindex+1} + \limitedfluxsum
\end{equation}
Each of these limiting coefficients is a number from zero to one; if the
coefficient is one, then no limiting is applied, and if it is zero, then full
limiting is applied, i.e., the internodal correction flux
$\MakeUppercase{\correctionfluxletter}\ij$ is completely canceled. If all of
the limiting coefficients are zero, then the low-order solution
$\lowordersolution$ is recovered, and if all of the limiting coefficients are
one, then the high-order solution $\highordersolution$ is recovered. The
definition of the limiting coefficients is given in Section
\ref{sec:limiting_coefficients_scalar}.
