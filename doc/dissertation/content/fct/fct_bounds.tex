The main idea of the FCT algorithm is to enforce some physically-motivated
bounds on the solution values:
\begin{equation}
  \DMPbound^{-}_i \leq \solutionletter_i^{n+1} \leq \DMPbound^{+}_i \eqc
  \qquad \forall i \eqp
\end{equation}
These bounds should be satisfied by the low-order scheme; this allows the
low-order scheme to act as a fail-safe for the FCT algorithm. Therefore,
the FCT algorithm requires that the chosen bounds above satisfy
$\DMPbound^{+}_i \geq \DMPbound^{L,+}_i$ and
$\DMPbound^{-}_i \leq \DMPbound^{L,-}_i$, where $\DMPbound^{L,\pm}$
are the low-order scheme discrete maximum principle (DMP) bounds given in
Section \ref{sec:DMP}. Therefore, one could choose to use the
low-order scheme DMP bounds as the bounds to impose to impose on the
FCT solution. However, it is sometimes the case that the low-order scheme
bounds do not envelop the most desirable FCT solution; for example, sometimes
the exact solution to a problem lies outside of the bounds. Therefore, it
becomes advantageous to widen the FCT bounds to allow such a solution:
one could derive some alternative physical
solution bounds, such as those derived in Appendix \ref{sec:analytic_dmp},
calling these $\analyticDMPbounds_i$, and then taking the minimum/maximum of
each: $\DMPbound^{+}_i = \max(\DMPbound^{L,+}_i,\analyticDMPupperbound_i)$ and
$\DMPbound^{-}_i = \min(\DMPbound^{L,-}_i,\analyticDMPlowerbound_i)$.
For convenience, the local discrete maximum principle bounds for the
different considered time discretizations, as well as the analytic local
discrete maximum principle bounds, are summarized in Table \ref{tab:dmp}.

\begin{mytable}{Discrete Maximum Principles}{dmp}{l c}
{\emph{Case} & \emph{Discrete Maximum Principle}}
\\
Steady-State   &
      \(\displaystyle
      \DMPbound_i^{L,\pm} \equiv -\frac{1}{\ssmatrixletter^L_{i,i}}
      \sumjnoti\ssmatrixletter^L_{i,j}
      \solutionletter_{\substack{\max\\\min},i}^L
      + \frac{\ssrhsletter_i}{\ssmatrixletter^L_{i,i}}\)\\
Explicit Euler &
     \(\displaystyle
     \DMPbound_i^{L,\pm}\equiv \solutionletter_{\substack{\max\\\min},i}^\timeindex
     \pr{1-\frac{\timestepsize}{\massmatrixletter_{i,i}^L}
       \sumj\ssmatrixletter^{L,\timeindex}_{i,j}}
     + \frac{\timestepsize}{\massmatrixletter_{i,i}^L}\ssrhsletter_i^\timeindex\) \\
Theta Method   &
   \(\displaystyle
   \DMPbound_i^{L,\pm}
   \equiv \frac{1}{1+\frac{\theta\timestepsize}{\massmatrixletter^L_{i,i}}
     \ssmatrixletter_{i,i}^{L,\timeindex+1}}\Bigg[\pr{1
     - \frac{(1-\theta)\timestepsize}{\massmatrixletter^L_{i,i}}
       \sumj\ssmatrixletter_{i,j}^{L,\timeindex}}
       \solutionletter_{\substack{\max\\\min},i}^\timeindex\)\\
   & $\displaystyle
     - \frac{\theta\timestepsize}{\massmatrixletter^L_{i,i}}
       \sumjnoti\ssmatrixletter_{i,j}^{L,\timeindex+1}
       \solutionletter_{\substack{\max\\\min},i}^{L,\timeindex+1}
     + \frac{\timestepsize}{\massmatrixletter^L_{i,i}}\pr{(1-\theta)
       \ssrhsletter_i^\timeindex + \theta\ssrhsletter_i^{\timeindex+1}}\Bigg]
   $\\
Analytic       &
      \(\displaystyle
        \analyticDMPlowerbound_i
        \equiv \solutionletter_{\text{min},i}^n e^{-\dt\reactioncoef_{\max,i}}
        + \frac{\scalarsource_{\min,i}}{\reactioncoef_{\max,i}}
        (1 - e^{-\reactioncoef_{\max,i}\dt}) \)\\
      & \(\displaystyle
        \analyticDMPupperbound_i
        \equiv \solutionletter_{\text{max},i}^n e^{-\dt\reactioncoef_{\min,i}}
        + \frac{\scalarsource_{\max,i}}{\reactioncoef_{\min,i}}
        (1 - e^{-\reactioncoef_{\min,i}\dt})\) \\
\end{mytable}
