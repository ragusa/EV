The main idea of the FCT algorithm is to enforce some physically-motivated
bounds on the each solution degree of freedom:
\begin{equation}
  \solutionbound^{-}_i
    \leq \solutionletter_i^{n+1}
    \leq \solutionbound^{+}_i \eqc
  \qquad \forall i \eqp
\end{equation}
For example, one may use the discrete maximum principle (DMP) bounds
derived for the low-order schemes in Section \ref{sec:DMP} as FCT solution
bounds. Alternatively, if the
physics are relatively simple, one could use the method of characteristics
to derive solution bounds. This approach is performed for the scalar transport
equation in Appendix \ref{sec:analytic_dmp}. For solution variables that
are physically non-negative, an obvious lower bound is zero:
$\solutionbound^-_i=0$; if one only wants to prevent negativities, then
one can simply set the upper bound to be an arbitrary large number
$\solutionbound^+_i=c_{\text{large}}$. However, often one wants other
properties such as monotonicity, in which case it is necessary to use
more restrictive bounds.

Equation numbers for the local discrete maximum principle bounds for the
different considered time discretizations, as well as the analytic local
discrete maximum principle bounds, are given in Table \ref{tab:dmp}.

\begin{mytable}{Discrete Maximum Principles}{dmp}{l c}
{\emph{Case} & \emph{Equation}}
Steady-State   & Equation \eqref{eq:DMP_ss} \\
Explicit Euler & Equation \eqref{eq:explicit_dmp} \\
Theta Method   & Equation \eqref{eq:theta_dmp} \\
Analytic       & Equation \eqref{eq:analyticDMP} \\
\end{mytable}
