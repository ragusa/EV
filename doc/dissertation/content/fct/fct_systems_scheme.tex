The general FCT strategy is the same in the systems case as in the scalar
case. One first defines antidiffusive correction fluxes such that
\begin{equation}
  \lumpedmassentry
    \frac{\solutionvector_i^{H}-\solutionvector_i^n}{\dt}
    + \sum_j\gradiententry\cdot\consfluxinterpolant_j^n
    + \sum_j\diffusionmatrixletter\ij^{L,n}\solutionvector_j^n
    = \ssrhs_i^n + \correctionfluxvector_i
  \eqp
\end{equation}
Then subtracting the high-order scheme equation from this gives the
definition of $\correctionfluxvector$:
\begin{equation}
  \correctionfluxvector_i \equiv
    \lumpedmassentry
      \frac{\solutionvector_i^{H}-\solutionvector_i^n}{\dt}
    -\sum_j\consistentmassentry
      \frac{\solutionvector_j^{H}-\solutionvector_j^n}{\dt}
    +\sum_j(\diffusionmatrixletter\ij^{L,n}-\diffusionmatrixletter\ij^{H,n})
      \solutionvector_j^n
  \eqp
\end{equation}
As in the scalar case, these fluxes are decomposed into internodal fluxes
$\correctionfluxmatrix\ij$ such that $\sum_j \correctionfluxmatrix\ij =
\correctionfluxvector_i$:
\begin{equation}
  \correctionfluxmatrix\ij = -M\ij^C\pr{
      \frac{\solutionvector_j^{H}-\solutionvector_j^n}{\dt}
      -\frac{\solutionvector_i^{H}-\solutionvector_i^n}{\dt}
    }
    + (D\ij^{L,n}-D\ij^{H,n})(\solutionvector_j^n-\solutionvector_i^n) \eqp
\end{equation}
Applying a limiting coefficient to each internodal antidiffusive correction
flux gives
\begin{equation}\label{eq:fct_scheme_conservative}
  \lumpedmassentry
    \frac{\solutionvector_i^{n+1}-\solutionvector_i^n}{\dt}
    + \sum_j\gradiententry\cdot\consfluxinterpolant_j^n
    + \sum_j\diffusionmatrixletter\ij^{L,n}\solutionvector_j^n
    = \ssrhs_i^n + \sum_j\limitermatrix\ij\odot\correctionfluxmatrix\ij \eqc
\end{equation}
where the notation $\limitermatrix\ij\odot\correctionfluxmatrix\ij$ denotes an
element-wise multiplication of $\limitermatrix\ij$ and $\correctionfluxmatrix\ij$:
$(\limitermatrix\ij\odot\correctionfluxmatrix\ij)^k
= \limiterletter\ij^k\correctionfluxentry\ij^k$.

As discussed in Section \ref{sec:fct_systems_introduction}, FCT limitation
for systems of conservation laws may benefit from transformations to other
sets of variables such as primitive or characteristic variables.
Consider some set of variables $\hat{\vectorsolution}$, which is produced
using a transformation matrix $\transformationmatrix(\vectorsolution)$:
$\hat{\vectorsolution} = \transformationmatrix^{-1}(\vectorsolution)
\vectorsolution$.
%To use the original, conservative variables
%($\hat{\vectorsolution} = \vectorsolution$), the transformation matrix is the
%identity matrix: $\transformationmatrix(\vectorsolution)=\mathbb{I}$.
For example, with the shallow water equations, to transform to the characteristic
variables, one uses the matrix of right eigenvectors of the Jacobian
$\jacobianx(\vectorsolution)$ as the transformation matrix:
$\transformationmatrix(\vectorsolution) = \eigenvectormatrix(\vectorsolution)$.
Applying a local transformation $\transformationmatrix^{-1}(\solutionvector_i^n)$
to Equation \eqref{eq:fct_scheme_conservative} gives
\begin{equation}
  \lumpedmassentry
    \frac{\hat{\solutionvector}_i^{n+1}-\hat{\solutionvector}_i^n}{\dt}
    + \sum_j\gradiententry\cdot\hat{\consfluxinterpolant}_j^n
    + \sum_j\diffusionmatrixletter\ij^{L,n}\hat{\solutionvector}_j^n
    = \hat{\ssrhs}_i^n + \sum_j\limitermatrix\ij\odot\hat{\correctionfluxmatrix}\ij
    \eqc
\end{equation}
where accents denote transformed quantities:
\begin{equation}
  \hat{\solutionvector}_j^{n+1} = \transformationmatrix^{-1}(\solutionvector_i^n)
    \solutionvector_j^{n+1} \eqc
\end{equation}
\begin{equation}
  \hat{\solutionvector}_j^n = \transformationmatrix^{-1}(\solutionvector_i^n)
    \solutionvector_j^n \eqc
\end{equation}
\begin{equation}
  \hat{\correctionfluxmatrix}\ij = \transformationmatrix^{-1}(\solutionvector_i^n)
    \correctionfluxmatrix\ij \eqp
\end{equation}
One can then choose the limiting coefficients $\limitermatrix\ij$ to satisfy
bounds on the transformed variables: 
\begin{equation}
  \hat{\DMPbound}_i^- \leq
  \hat{\solutionvector}_i^{n+1} \leq
  \hat{\DMPbound}_i^+ \quad \forall i \eqp
\end{equation}
Imposition of these bounds corresponds to bounding the transformed antidiffusion
flux sums:
\begin{equation}
  \hat{\mathbf{\limitedfluxbound}}^-_i \leq
  \sum\limits_j \limitermatrix\ij\odot\hat{\correctionfluxmatrix}\ij \leq
  \hat{\mathbf{\limitedfluxbound}}^+_i \eqc
\end{equation}
where the bounds are obtained by performing some algebra on the transformed
system:
\begin{equation}
  \hat{\mathbf{\limitedfluxbound}}_i^\pm \equiv
    \lumpedmassentry\frac{\hat{\solutionvector}_i^\pm
      -\hat{\solutionvector}_i^n}{\dt}
    + \sum_j\gradiententry\cdot\hat{\consfluxinterpolant}_j^n
    + \sum_j\diffusionmatrixletter\ij^{L,n}\hat{\solutionvector}_j^n
    - \hat{\ssrhs}_i^n \eqp
\end{equation}

Zalesak's limiter then takes the same form as in the scalar case, but
now the transformed quantities $\hat{\mathbf{\limitedfluxbound}}^\pm$
and $\hat{\correctionfluxmatrix}$ are used instead of
$\mathbf{\limitedfluxbound}^\pm$ and $\correctionfluxmatrix$:
\begin{equation}
  \hat{\correctionfluxletter}_i^{-,\componentindex} \equiv
    \sum\limits_{j:\hat{\correctionfluxentry}\ij^\componentindex<0}
    \hat{\correctionfluxentry}\ij^\componentindex \eqc \qquad
  \hat{\correctionfluxletter}_i^{+,\componentindex} \equiv
    \sum\limits_{j:\hat{\correctionfluxentry}\ij^\componentindex>0}
    \hat{\correctionfluxentry}\ij^\componentindex \eqc
\end{equation}
\begin{equation}
  \limiterletter_i^{\pm,\componentindex} \equiv\left\{
    \begin{array}{l l}
      1 & \hat{\correctionfluxletter}_i^{\pm,\componentindex} = 0\\
      \min\left(1,\frac{Q_i^{\pm,\componentindex}}
        {\hat{\correctionfluxletter}_i^{\pm,\componentindex}}\right) &
      \hat{\correctionfluxletter}_i^{\pm,\componentindex} \ne 0
    \end{array}
  \right. \eqc
\end{equation}
\begin{equation}
  \limiterletter\ij^\componentindex \equiv\left\{
    \begin{array}{l l}
      \min(L_i^{+,\componentindex},L_j^{-,\componentindex}) &
        \hat{\correctionfluxentry}\ij^\componentindex \geq 0\\
      \min(L_i^{-,\componentindex},L_j^{+,\componentindex}) &
        \hat{\correctionfluxentry}\ij^\componentindex < 0
    \end{array}
  \right. \eqp
\end{equation}

Kuzmin states that for the systems case, the limiting coefficients may require
synchronization between the components, such as
\begin{equation}
  \limiterletter\ij^\componentindex \mapsfrom
    \min\limits_k\limiterletter\ij^k \quad \forall\componentindex \eqp
\end{equation}
Otherwise, antidiffusive fluxes in one component can violate the
conditions of another component \cite{kuzmin_FCT}.
