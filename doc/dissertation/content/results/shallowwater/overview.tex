This section presents the results for the shallow water equations.
For transient simulations, the time step size used is given as a
``CFL'' number $\nu$:
\begin{subequations}
\begin{equation}
  \nu \equiv \frac{\dt}{\dt_{\textup{CFL}}} \eqc \quad
  \dt_{\textup{CFL}} \equiv \frac{\dx_{\textup{min}}}
    {\wavespeed_{\textup{max}}} \eqc
  \eqp
\end{equation}
\begin{equation}
  \dx_{\textup{min}} \equiv \min\limits_\cell \dx_\cell \eqc
\end{equation}
\begin{equation}
  \wavespeed_{\textup{max}} \equiv \max\limits_{\x\in\domain}
    \pr{\|\velocity(\x)\| + \speedofsound(\x)} \eqc
\end{equation}
\end{subequations}
where the maximum over the domain is approximated by the maximum over
all quadrature points in the domain.

Note that to guarantee the invariant domain property for the low-order
scheme, one still needs to verify the time step size requirement given by Equation
\eqref{eq:dt_invariant_domain}; the condition $\nu < 1$ is not
sufficient.
