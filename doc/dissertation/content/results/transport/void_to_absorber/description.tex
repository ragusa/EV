This problem simulates the interface between a void region and a strong, pure absorber
region, with an incoming flux boundary condition applied.
Table \ref{tab:void_to_absorber} summarizes the problem parameters.

%-------------------------------------------------------------------------------
\begin{table}[htb]\caption{Normal Void-to-Absorber Test Problem Summary}
\label{tab:void_to_absorber}
\centering
\begin{tabular}{l l}\toprule
\emph{Parameter} & \emph{Value}\\\midrule
Domain & $\mathcal{D} = (0,1)^d$\\
Initial Conditions & $u_0(\x)=0$\\
Boundary Conditions & $u(\x,t)=1,\quad \x\in\partial\mathcal{D}^-,\quad t>0$,\\
   & $\quad\partial\mathcal{D}^-=\{\x\in\partial\mathcal{D}:\mathbf{n}(\x)
       \cdot\mathbf{\Omega}<0\}$\\
Direction & $\mathbf{\Omega} = \mathbf{e}_x$\\
Cross Section & $\sigma(\x)=\left\{\begin{array}{c l}
   10, & \x\in(\frac{1}{2},1)^d\\
   0,  & \mbox{otherwise}\end{array}\right.$\\
Source & $q(\x,t)=0$\\
Speed & $\speed=1$\\
Exact Solution & $u(\x,t)=\left\{\begin{array}{l l}
   \scalarsolution_{\text{ss}}(\x), & x-t<0\\
   0, & \mbox{otherwise}
   \end{array}\right.$ \\
   & $\scalarsolution_{\text{ss}}(\x) =
       \left\{\begin{array}{l l}
          e^{-10(x-\frac{1}{2})}, & x\ge\frac{1}{2}, y\ge\frac{1}{2}, z\ge\frac{1}{2}\\
          1,                      & \mbox{otherwise}
       \end{array}\right.$\\
\bottomrule\end{tabular}
\end{table}
%-------------------------------------------------------------------------------

Table \ref{tab:void_to_absorber_run_parameters} shows the run parameters used
to obtain the results in this section, Figures \ref{fig:void_to_absorber_2D_fe}
and \ref{fig:void_to_absorber_2D_ssprk33} show 2-D results for explicit Euler
and SSPRK33 time discretizations, respectively, and Figure
\ref{fig:void_to_absorber_3D} shows 3-D results.

From Figure \ref{fig:void_to_absorber_2D_ssprk33}, one can see that the
Galerkin scheme (which has no artificial dissipation) generates significant
spurious oscillations perpendicular to the transport direction, even below the
absorber region. The oscillations are particularly severe along the lower edge
of the absorber region, where particles/photons are traveling parallel to the
absorber; this edge has a sharper gradient in the solution than the left edge
of the absorber region due to the lack of attenuation in this direction, which
is present for the left edge. Figure \ref{fig:void_to_absorber_2D_ssprk33},
which uses explicit Euler instead of SSPRK33 does not show the Galerkin plot
because the oscillations grew without bound, leading to infinite solution
values. The entropy viscosity scheme is also vulnerable to spurious
oscillations, although to a lesser extent than the Galerkin scheme.

Note that all numerical schemes except the Galerkin scheme involve some
dissipation; this can be seen at the outgoing (right) boundary of the void
region, where there is a solution gradient despite the lack of
absorption. This is because the simulation was run to $t=1$, and the transport
speed is $\speed=1$, so the wave front should be located at the right
boundary of the domain since the domain width is equal to 1;
the diffusivity at the right boundary is due to artificial diffusion
along the wave front.
For steady-state computations, where there is no transient and
thus no wave front, one would not see this diffusivity
at the right boundary.

Figure \ref{fig:void_to_absorber_visc} shows the low-order and entropy
viscosity profiles using SSPRK33, both on linear scales but separately
scaled. One can see that the entropy viscosity
is highest along the incident edge of the absorber region, and in
particular, the corner of the absorber region in the center of the
domain.

One can visually compare the width of the diffusive region to infer the
diffusivity of each numerical scheme. For example, one can see that the
low-order solution is a bit more diffusive than the high-order schemes and FCT
schemes.  Both the Galerkin-FCT and EV-FCT solutions show a lack of
oscillations and less diffusivity than the low-order solution.

The 3-D results are included here to show a proof of principle
that the FCT algorithm used is not restricted to 1-D or 2-D.

%-------------------------------------------------------------------------------
\begin{table}[ht]\caption{Normal Void-to-Absorber Test Problem Run Parameters}
\label{tab:void_to_absorber_run_parameters}
\centering
\begin{tabular}{l l}\toprule
\emph{Parameter} & \emph{Value}\\\midrule
Number of Cells & $N_{cell} = 16384$\\
Time Discretization & Explicit Euler, SSPRK33, Steady-State\\
End Time & $t = 1$\\
CFL Number & $\nu = 0.5$\\
Boundary Conditions & Strong Dirichlet with
  $\limiterletter_i^-=\limiterletter_i^+=1$\\\midrule
Entropy Function & $\entropy(\scalarsolution) = \frac{1}{2}\scalarsolution^2$\\
Entropy Residual Coefficient & $\entropyresidualcoef = 0.1$\\
Entropy Jump Coefficient & $\entropyjumpcoef = 0.1$\\\midrule
FCT Solution Bounds & DMP\\
\bottomrule\end{tabular}
\end{table}
%-------------------------------------------------------------------------------
\begin{figure}[ht]
   \centering
   \begin{subfigure}{0.3\textwidth}
      \includegraphics[width=\textwidth]
        {\contentdir/results/transport/void_to_absorber/images/Exact.png}
      \caption{Exact}
   \end{subfigure}
   \begin{subfigure}{0.3\textwidth}
      \includegraphics[width=\textwidth]
        {\contentdir/results/transport/void_to_absorber/images/GalFCT_FE.png}
      \caption{Galerkin FCT}
   \end{subfigure}
   \begin{subfigure}{0.3\textwidth}
      \includegraphics[width=\textwidth]
        {\contentdir/results/transport/void_to_absorber/images/Low_FE.png}
      \caption{Low-Order}
   \end{subfigure}
   \begin{subfigure}{0.3\textwidth}
      \includegraphics[width=\textwidth]
        {\contentdir/results/transport/void_to_absorber/images/EV_FE.png}
      \caption{Entropy Viscosity}
   \end{subfigure}
   \begin{subfigure}{0.3\textwidth}
      \includegraphics[width=\textwidth]
        {\contentdir/results/transport/void_to_absorber/images/EVFCT_FE.png}
      \caption{Entropy Viscosity FCT}
   \end{subfigure}
   \caption{Comparison of Solutions for 2-D Normal Void-to-Absorber Test
     Problem Using Explicit Euler Time Discretization and Low-Order DMP
     Solution Bounds}
   \label{fig:void_to_absorber_2D_fe}
\end{figure}
%-------------------------------------------------------------------------------
\begin{figure}[ht]
   \centering
   \begin{subfigure}{0.3\textwidth}
      \includegraphics[width=\textwidth]
        {\contentdir/results/transport/void_to_absorber/images/Exact.png}
      \caption{Exact}
   \end{subfigure}
   \begin{subfigure}{0.3\textwidth}
      \includegraphics[width=\textwidth]
        {\contentdir/results/transport/void_to_absorber/images/Gal_SSPRK33.png}
      \caption{Galerkin}
   \end{subfigure}
   \begin{subfigure}{0.3\textwidth}
      \includegraphics[width=\textwidth]
        {\contentdir/results/transport/void_to_absorber/images/GalFCT_SSPRK33.png}
      \caption{Galerkin FCT}
   \end{subfigure}
   \begin{subfigure}{0.3\textwidth}
      \includegraphics[width=\textwidth]
        {\contentdir/results/transport/void_to_absorber/images/Low_SSPRK33.png}
      \caption{Low-Order}
   \end{subfigure}
   \begin{subfigure}{0.3\textwidth}
      \includegraphics[width=\textwidth]
        {\contentdir/results/transport/void_to_absorber/images/EV_SSPRK33.png}
      \caption{Entropy Viscosity}
   \end{subfigure}
   \begin{subfigure}{0.3\textwidth}
      \includegraphics[width=\textwidth]
        {\contentdir/results/transport/void_to_absorber/images/EVFCT_SSPRK33.png}
      \caption{Entropy Viscosity FCT}
   \end{subfigure}
   \caption{Comparison of Solutions for 2-D Normal Void-to-Absorber Test
     Problem Using SSPRK33 Time Discretization and Low-Order DMP
     Solution Bounds}
   \label{fig:void_to_absorber_2D_ssprk33}
\end{figure}
%-------------------------------------------------------------------------------
\begin{figure}[ht]
   \centering
   \begin{subfigure}{0.45\textwidth}
      \includegraphics[width=\textwidth]
        {\contentdir/results/transport/void_to_absorber/images/low_viscosity_SSP3_linearscale.png}
      \caption{Low-order viscosity}
   \end{subfigure}
   \begin{subfigure}{0.45\textwidth}
      \includegraphics[width=\textwidth]
        {\contentdir/results/transport/void_to_absorber/images/entropy_viscosity_SSP3_linearscale.png}
      \caption{Entropy viscosity}
   \end{subfigure}
   \caption{Viscosity Profiles for the 2-D Void-to-Absorber Test
     Problem Using SSPRK33 Time Discretization}
   \label{fig:void_to_absorber_visc}
\end{figure}
%-------------------------------------------------------------------------------
\begin{figure}[ht]
   \centering
   \begin{subfigure}{0.45\textwidth}
      \includegraphics[width=\textwidth]
        {\contentdir/results/transport/void_to_absorber/images/Gal_3D.png}
      \caption{Galerkin}
   \end{subfigure}
   \begin{subfigure}{0.45\textwidth}
      \includegraphics[width=\textwidth]
        {\contentdir/results/transport/void_to_absorber/images/GalFCT_3D.png}
      \caption{Galerkin with FCT}
   \end{subfigure}
   \caption{Comparison of Solutions for the 3-D Normal Void-to-Absorber Test
     Problem Using SSPRK33 Time Discretization and Low-Order DMP
     Solution Bounds}
   \label{fig:void_to_absorber_3D}
\end{figure}
%-------------------------------------------------------------------------------

The steady-state FCT solution was found to have significant convergence
difficulties, and for example, the steady-state solution with 4096 cells
did not converge. The main difficulty of implicit and steady-state FCT is
that in general the solution bounds, such as those given by
Equation \eqref{eq:analyticDMP}, are implicit:
\begin{subequations}
  \begin{equation}
      \solutionbound_i^- \le \solutionletter_i
        \le \solutionbound_i^+ \eqc
  \end{equation}
  \begin{equation}
      \solutionbound_i^-
        \equiv \left\{\begin{array}{l l}
          \solutionletter_{\min,i} e^{-\dx\reactioncoef_{\max,i}}
            + \frac{\scalarsource_{\min,i}}{\reactioncoef_{\max,i}}
            (1 - e^{-\dx\reactioncoef_{\max,i}}) \eqc
          & \reactioncoef_{\max,i} \ne 0 \\
          \solutionletter_{\min,i}
            + \dx\scalarsource_{\min,i} \eqc
          & \reactioncoef_{\max,i} = 0
        \end{array}\right.\eqc
  \end{equation}
  \begin{equation}
      \solutionbound_i^+
        \equiv \left\{\begin{array}{l l}
          \solutionletter_{\max,i} e^{-\dx\reactioncoef_{\min,i}}
            + \frac{\scalarsource_{\max,i}}{\reactioncoef_{\min,i}}
            (1 - e^{-\dx\reactioncoef_{\min,i}}) \eqc
          & \reactioncoef_{\min,i} \ne 0 \\
          \solutionletter_{\max,i}
            + \dx\scalarsource_{\max,i} \eqc
          & \reactioncoef_{\min,i} = 0
        \end{array}\right.\eqp
  \end{equation}
\end{subequations}
Note that for this test problem, the source $\scalarsource$ is zero;
it is included in these expressions for generality.

To illustrate the issue of using implicit solution bounds,
the problem was run again but with \emph{explicitly}
computed solution bounds, which use the known exact solution for this problem
instead of the numerical solution values:
\begin{subequations}
  \begin{equation}
      \solutionbound_i^- \le \solutionletter_i
        \le \solutionbound_i^+ \eqc
  \end{equation}
  \begin{equation}
      \solutionbound_i^-
        \equiv \left\{\begin{array}{l l}
          u_{\min,i}^{\textup{exact}} e^{-\dx\reactioncoef_{\max,i}}
            + \frac{\scalarsource_{\min,i}}{\reactioncoef_{\max,i}}
            (1 - e^{-\dx\reactioncoef_{\max,i}}) \eqc
          & \reactioncoef_{\max,i} \ne 0 \\
          u_{\min,i}^{\textup{exact}}
            + \dx\scalarsource_{\min,i} \eqc
          & \reactioncoef_{\max,i} = 0
        \end{array}\right.\eqc
  \end{equation}
  \begin{equation}
      \solutionbound_i^+
        \equiv \left\{\begin{array}{l l}
          u_{\max,i}^{\textup{exact}} e^{-\dx\reactioncoef_{\min,i}}
            + \frac{\scalarsource_{\max,i}}{\reactioncoef_{\min,i}}
            (1 - e^{-\dx\reactioncoef_{\min,i}}) \eqc
          & \reactioncoef_{\min,i} \ne 0 \\
          u_{\max,i}^{\textup{exact}}
            + \dx\scalarsource_{\max,i} \eqc
          & \reactioncoef_{\min,i} = 0
        \end{array}\right.\eqp
  \end{equation}
  \begin{equation}
    u_{\min,i}^{\textup{exact}} \equiv \min\limits_{j\in\indices(\support_i)}
      u^{\textup{exact}}(\x_j) \eqc \quad
    u_{\max,i}^{\textup{exact}} \equiv \max\limits_{j\in\indices(\support_i)}
      u^{\textup{exact}}(\x_j) \eqp
  \end{equation}
\end{subequations}
When these bounds were used, the steady-state FCT solution did converge,
with 50 iterations.
Figure \ref{fig:void_to_absorber_ss} compares the steady-state solutions
computed with 4096 cells, with the FCT solution using the exact solution
bounds given above. 

%-------------------------------------------------------------------------------
\begin{figure}[ht]
   \centering
   \begin{subfigure}{0.49\textwidth}
      \includegraphics[width=\textwidth]
        {\contentdir/results/transport/void_to_absorber_ss/images/exact.png}
      \caption{Exact}
   \end{subfigure}
   \begin{subfigure}{0.49\textwidth}
      \includegraphics[width=\textwidth]
        {\contentdir/results/transport/void_to_absorber_ss/images/low.png}
      \caption{Low-Order}
   \end{subfigure}
   \begin{subfigure}{0.49\textwidth}
      \includegraphics[width=\textwidth]
        {\contentdir/results/transport/void_to_absorber_ss/images/EV.png}
      \caption{Entropy Viscosity}
   \end{subfigure}
   \begin{subfigure}{0.49\textwidth}
      \includegraphics[width=\textwidth]
        {\contentdir/results/transport/void_to_absorber_ss/images/EVFCT.png}
      \caption{Entropy Viscosity FCT}
   \end{subfigure}
   \caption{Comparison of Steady-State Solutions for 2-D Normal
     Void-to-Ab\-sorber Test Problem with 4096 Cells with Exact Solution Bounds}
   \label{fig:void_to_absorber_ss}
\end{figure}
%-------------------------------------------------------------------------------

\clearpage
