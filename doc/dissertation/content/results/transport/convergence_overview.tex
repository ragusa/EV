A number of test problems were used to evaluate spatial and temporal
convergence rates. Before describing these problems, the methodology for
evaluating these rates will be described. The error in the numerical solution
$\approximatescalarsolution\xt$ has a number of components:
\begin{itemize}
  \item Spatial discretization error,
  \item Temporal discretization error, and
  \item Computer precision (round-off) error.
\end{itemize}
Round-off error arises from limited precision of floating point numbers and
becomes relevant here only when measuring errors on the order of the computer
precision, $\sim\order(10^{-15})$; this becomes the bottleneck of improvement
of convergence at fine refinements. For larger magnitude error, the important
components are thus spatial and temporal discretization error.

Of course, for steady-state problems, there is no temporal discretization
error. Omitting higher order terms, for a steady-state problem, the error of the
approximate solution in some norm has the form
\begin{equation}
  \err = c_{x}\dx^m \eqc
\end{equation}
where $\Delta x$ is the spatial element size, $m$ is the spatial convergence
rate, and $c_{x}$ is the leading coefficient. Taking the logarithm of
this equation gives
\begin{equation}
  \log(\err) = m\log(\dx) + c \eqc
\end{equation}
where $c$ is some constant. Thus one can make two measurements $(\dx_i,\err_i)$
and $(\dx_{i+1},\err_{i+1})$ to compute a slope $m_{i+\half}$ on a log-log
plot to estimate the convergence rate:
\begin{equation}\label{eq:convrate}
  m_{i+\half} = \frac{\log(\err_{i+1}) - \log(\err_i)}
    {\log(\dx_{i+1}) - \log(\dx_i)}
  \eqp
\end{equation}

For time-dependent solutions, the situation is more complicated:
\begin{equation}
  \err = c_{x}\dx^m + c_{t}\dt^p \eqc
\end{equation}
where $\dt$ is the time step size, $p$ is the temporal convergence rate,
and $c_{t}$ is the leading coefficient for temporal error. Taking the
logarithm of both sides of this equation does not yield the same
linear relationship as in the steady-state case:
\begin{equation}\label{eq:logeq}
  \log(\err) = \log(c_{x}\dx^m + c_{t}\dt^p) \eqp
\end{equation}
One may be interested in measuring the spatial convergence rate $m$,
but temporal errors may prevent the rate from being recovered.
Similarly, One may be interested in measuring the temporal convergence rate $p$,
but spatial errors may prevent the rate from being recovered.
There are three main strategies for overcoming this difficulty:
\begin{itemize}
  \item \textbf{For measuring spatial/temporal convergence rates, choose a test problem
    such that no temporal/spatial error arises.} For example, for measuring
    convergence rates, one can choose a problem with a solution that is not
    a function of time or that is linear in time, since the time discretization
    should be able to exactly integrate linear functions of time.
    Similarly, if one knows that a spatial discretization can exactly
    approximate a linear solution, then there is no spatial error and thus
    temporal convergence rates can be measured.
  \item \textbf{For measuring spatial/temporal convergence rates, use a very
    fine temporal/spatial refinement level.}
    This is an obvious approach; however, this is often undesirable because it is
    a computationally costly approach.
  \item \textbf{Refine both space and time using knowledge of expected
    convergence rates.}
    The idea of this approach is to use a certain relation between mesh size
    and time step size to recover either the spatial or temporal convergence
    rate. If one would like to recover the spatial convergence rate $m$, then
    one can assume that time step size has the relation
    \begin{equation}
      \dt^p = \dx^m \eqc
    \end{equation}
    so that $\dt = \dx^{m/p}$. Then Equation \eqref{eq:logeq} becomes
    \begin{equation}
      \log(\err) = \log(c_{x}\dx^m + c_{t}\dx^m) = m\log(\dx) + c \eqp
    \end{equation}
    Similarly one can use the relation $\dx = \dt^{p/m}$ to recover
    \begin{equation}
      \log(\err) = \log(c_{x}\dt^p + c_{t}\dt^p) = p\log(\dt) + c \eqp
    \end{equation}
    Then rates can be measured just as given in Equation \eqref{eq:convrate}.
    For example, suppose that one wishes to measure temporal convergence rate
    and refines mesh size by a factor of $\half$ in each cycle:
    $\dx_{i+1} = \half\dx_i$. Suppose that the spatial discretization is of
    order $m$ and the temporal discretization is supposedly of order $p$.
    Then using $\dt_{i+1} = \dx_{i+1}^{m/p} = (\half\dx_i)^{m/p} =
    (\half)^{m/p}\dt_i$. Therefore the temporal refinement factor should
    be $(\half)^{m/p}$. Then the temporal convergence rate can be measured:
    \begin{equation}
      p_{i+\half} = \frac{\log(\err_{i+1}) - \log(\err_i)}
        {\log(\dt_{i+1}) - \log(\dt_i)}
      \eqp
    \end{equation}
\end{itemize}
The first and third of these approaches are used in sections that follow.

