This section presents results for scalar transport.
Section \ref{sec:convergence} shows convergence results that
demonstrate second-order spatial accuracy for the entropy
viscosity (EV) method, as well as the FCT scheme.
Results in subsequent sections give various results exploring a number of
dimensions of EV and FCT schemes.

The FCT algorithm can be built on either the entropy viscosity
method or the standard Galerkin high-order scheme given in
Section \ref{sec:spatial_discretization_scalar}. The resulting
schemes will be referred to in this section as the EV-FCT scheme
and the Galerkin-FCT scheme, respectively.

Incoming flux boundary conditions can be applied in a number of
different ways, as discussed in Section \ref{sec:transport_bc}.
When these boundary conditions are applied as strong Dirichlet,
recall from Section \ref{sec:fct_scheme_scalar} that the antidiffusion
flux decomposition does not apply in the neighborhood of nodes
for which Dirichlet boundary conditions are strongly imposed.
Therefore, to keep the conservation property, antidiffusive fluxes
involving Dirichlet nodes must be completely canceled:
\begin{equation}
  \limiterletter\ij \gets 0 \eqc \quad \forall j \eqc
    \forall i \in \incomingindices \eqp
\end{equation}
Recall from Section \ref{sec:zalesak_limiter} that Zalesak's limiter
computes upper bounds on the limiting coefficients for all positive
and negative antidiffusive fluxes for a node $i$: $\limiterletter_i^+$
and $\limiterletter_i^-$. Then for a positive antidiffusive flux
$\antidiffusiveflux\ij$, the limiting coefficient is
$\limiterletter\ij=\min(\limiterletter_i^+,\limiterletter_j^-)$.
Thus to cancel the antidiffusive fluxes for Dirichlet nodes, one can
perform the following operation:
\begin{equation}
  \limiterletter_i^+ \gets 0 \eqc \quad
  \limiterletter_i^- \gets 0 \eqc \quad
    \forall i \in \incomingindices \eqp
\end{equation}
If one does not deem the conservation property important, then one may
instead decide that antidiffusive fluxes from Dirichlet nodes should
not necessarily be canceled and thus may instead perform the following:
\begin{equation}
  \limiterletter_i^+ \gets 1 \eqc \quad
  \limiterletter_i^- \gets 1 \eqc \quad
    \forall i \in \incomingindices \eqp
\end{equation}
This has the advantage that more antidiffusion can be accepted and
thus that the solution may have greater accuracy. These two
paradigms are considered for a number of test problems and will
typically be labeled as
``strong Dirichlet BC with $\limiterletter_i^-=\limiterletter_i^+=0$''
and ``strong Dirichlet BC with $\limiterletter_i^-=\limiterletter_i^+=1$'',
respectively. Other options considered include weak Dirichlet and
weak Dirichlet with a boundary penalty. When results in this
section use a boundary penalty, the penalty coefficient (as described
in Section \ref{sec:transport_bc} takes the value $\alpha=1000$.

A number of options for the solution bounds to impose in FCT
will be considered. Firstly, the low-order DMP bounds given in
Section \ref{sec:DMP} for each temporal discretization are
considered. Alternatively, analytic bounds derived from the
method of characteristics (MoC) in Appendix \ref{sec:analytic_dmp}
are used. The analytic bounds are derived along an upwind line
segment $L(\x,\x-\speed\dt\di)$; this is referred to as the
``\emph{upwind} analytic solution bounds''. However, a more relaxed
set of solution bounds are used as the default solution bounds,
which considers a spherical
neighborhood around each node instead of just the upwind line
segment; this will be referred to as the ``analytic solution bounds''.
Certain sections will also consider a modification of these bounds,
referred to as the ``modified analytic solution bounds'',
which as of yet has no analytic proof. These modified bounds are
discussed when used. Note that the corresponding antidiffusion
bounds $\antidiffusionbound_i^\pm$ for a given set of solution
bounds $\solutionbound_i^\pm$ must obey the conditions discussed
in Section \ref{sec:antidiffusion_bounds_signs}, and thus the
following operation given by Equation \eqref{eq:antidiffusion_bounds_operation}
is performed for all results in this section:
\begin{subequations}
\begin{equation*}
  \antidiffusionbound_i^-
    \gets \min(\antidiffusionbound_i^-, 0)
  \eqc
\end{equation*}
\begin{equation*}
  \antidiffusionbound_i^+
    \gets \max(\antidiffusionbound_i^+, 0)
  \eqp
\end{equation*}
\end{subequations}

Some results in this section also try the multi-pass limiting
procedure described in Section \ref{sec:multipass_limiting}.

For transient simulations, the time step size used is given as a
``CFL'' number $\nu$, which is defined in terms of the maximum
time step size for Explicit Euler time discretization of the
low-order scheme, given by Equation \eqref{eq:explicit_cfl}:
\begin{equation}
  \nu \equiv \frac{\dt}{\dt_{\textup{CFL}}} \eqc \quad
  \dt_{\textup{CFL}} \equiv \min\limits_i
    \frac{\massmatrixletter_{i,i}^{L}}{\ssmatrixletter_{i,i}^{L,n}}
  \eqp
\end{equation}
Note that the time step size for the Theta method is less
restrictive for $\theta > 0$ (see Equation \eqref{eq:theta_cfl});
however, for Theta method results, the definition of $\nu$ above is
still used.

In summary, results in this section have the following dimensions, where
the default options in this section are underlined:
\begin{itemize}
  \item temporal discretization
  \item time step size/CFL number
  \item mesh size
  \item BC method:
    \begin{itemize}
      \item strong Dirichlet BC with $\limiterletter_i^-=\limiterletter_i^+=0$
      \item \underline{strong Dirichlet BC with $\limiterletter_i^-=\limiterletter_i^+=1$}
      \item weak Dirichlet BC
      \item weak Dirichlet BC with a boundary penalty
    \end{itemize}
  \item FCT type:
    \begin{itemize}
      \item \underline{EV-FCT}
      \item Galerkin-FCT
    \end{itemize}
  \item FCT solution bounds:
    \begin{itemize}
      \item low-order DMP bounds
      \item \underline{analytic bounds from method of characteristics}
      \item \emph{modified} analytic bounds from method of characteristics
      \item upwind analytic bounds from method of characteristics
    \end{itemize}
  \item FCT limiter:
    \begin{itemize}
      \item \underline{single-pass Zalesak}
      \item multi-pass Zalesak
    \end{itemize}
  \item Initial guess for implicit FCT and steady-state FCT solutions:
    \begin{itemize}
      \item zeroes: $u^{(0)} = 0$
      \item \underline{low-order solution: $u^{(0)} = u^L$}
      \item high-order solution: $u^{(0)} = u^H$
    \end{itemize}
\end{itemize}

Section \ref{sec:scalar_summary} will attempt to make some conclusions about
some of these options.

