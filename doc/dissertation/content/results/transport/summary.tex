This section provides a summary of the scalar transport results.
As discussed in Section \ref{sec:scalar_overview}, there are a number
of parameters that influence the success of the scalar FCT algorithm,
so this section attempts to analyze what the optimal FCT configuration
might be.

Firstly, it is strongly recommended to use the entropy viscosity method
as the high-order scheme instead of the standard Galerkin method that
lacks any artificial dissipation. It should be noted that entropy-based
artificial viscosity is the main component of a successful CFEM
scalar conservation law solution; this is the component which ensures
convergence to the entropy solution. The FCT algorithm is best viewed
as a conservative clean-up procedure for mitigating spurious oscillations
and preventing negativities. As the mesh is refined, the entropy viscosity
solution converges to the entropy solution, which lacks any spurious
oscillations or negativities, but for coarser meshes, some of these
artifacts are still present, hence the need for FCT. Thus it is preferred
to use EV-FCT over Galerkin-FCT. One can find cases/configurations in which
the Galerkin-FCT solution is more accurate EV-FCT solution, but this is
not the general case. Also, one might find that an oscillatory solution
has a smaller $L^1$ and/or $L^2$ error than a non-oscillatory solution,
but that the non-oscillatory is more favorable qualitatively. Galerkin-FCT
is more vulnerable to the well-known FCT phenomenon known as ``terracing'',
``stair-stepping'', or ``plateauing''; this effect is associated with
the limitation of large oscillations. The EV method mitigates or eliminates
oscillations that the Galerkin method encounters, and thus the FCT
algorithm for EV encounters smaller magnitude oscillations in the
high-order solution, and the stair-stepping effect is decreased from
that of Galerkin FCT.

For explicit time discretizations, it is strongly recommended to use a
higher-order SSPRK scheme as opposed to explicit Euler. Explicit Euler
is particularly vulnerable to the formation of spurious oscillations
of large magnitude and thus FCT solutions are more vulnerable to stair-stepping.
High-order SSPRK schemes such as SSPRK33 are expressed as a sequence of
explicit Euler steps so that the same methodology can be used in each
step, and usage of these high-order time discretizations reduces spurious
oscillations even without artificial viscosity or FCT.

Implicit time discretizations and steady-state often suffer from convergence
difficulties. The nonlinear scheme used for the EV solution and the FCT
solution is a type of fixed-point iteration, which is arguably the simplest
and slowest-converging nonlinear iteration scheme, but the lack of convergence
in some cases is due to the fact that the imposed solution bounds are implicit.
When the FCT solution iterate changes, the solution bounds change as well.
In some cases an FCT solution iterate produces solution bounds that lead to
a reversal of antidiffusive fluxes made in the previous iteration, and then
this can repeat indefinitely, preventing convergence. One can try using
a relaxation factor in solution updates, but this is not a solution to
the underlying issue, and thus it may or may not be a successful approach
for a particular problem. For implicit time discretizations, convergence
difficulties are most prominent when high CFL numbers are used; in this
case, the solution bounds are wider, and the limiting coefficient values
have more opportunity to vary. For both implicit time discretizations and
steady-state, convergence difficulties are most prominent for multi-dimensional
problems; this is because the solution bounds have a greater number of
degrees of freedom coupled for a given node - the neighborhoods around
each node are larger. With these issues considered, implicit FCT
currently requires more research to become a reliable method.

Recall that there were a number of different approaches for imposing
incoming flux boundary conditions. Strong Dirichlet BC for example
have the issue in FCT that the conservation property is not satisfied
unless the antidiffusive fluxes from Dirichlet nodes are completely
canceled. Accurate solutions can still be obtained by ignoring this
requirement, and often the cancellation of Dirichlet antidiffusive fluxes
leads to less accurate solutions.
Weak Dirichlet BC are found to be very inaccurate,
especially when the values in the vicinity are deemed very important;
thus they are not recommended for general use. However, using weak
Dirichlet BC with a boundary penalty was found to be effective, without
sacrificing the conservation property or the opportunity to accept
antidiffusion from incoming boundary nodes. It is thus recommended
for general use to use weak Dirichlet BC with a boundary penalty.

There were a number of different solution bounds considered. The
low-order DMP bounds were initially considered because for fully
explicit time discretization, they automatically satisfied the
fail-safe conditions on the antidiffusion bounds, as discussed in Section
\ref{sec:antidiffusion_bounds_signs}; however, they were found to prevent
second-order spatial accuracy in many cases because these bounds
were proven to be inaccurate when a reaction term is present.
It was found that the fail-safe condition could be enforced, no
matter what solution bounds are imposed, by using Equation
\eqref{eq:antidiffusion_bounds_operation}, and the bounds derived
using the method of characteristics (MoC) were found to overcome
the issues of the low-order DMP bounds and thus obtain second-order
accuracy. For general use, it is recommended to use the MoC bounds
in favor of the low-order DMP bounds. For the MoC bounds, there
are two additional dimensions. Dimension one is the possibility of only
evaluating the bounds along the upwind line segment instead of the spherical
neighborhood, and dimension two is the modification presented by
Equation \eqref{eq:modified_analytic_bounds}. Usage of upwind solution
bounds gave mixed results. The upwind line solution bounds are very
tight; for some cases, the only reason that the upper and lower bounds
differ is due to the enforcement of the fail-safe condition, which
effectively enforces that the lower bound is not greater than the low-order
solution and that the upper bound is not lesser than the low-order
solution. Tight solution bounds have advantages and disadvantages. The
obvious advantage is that there is less room for unphysical oscillations,
so these remaining artifacts, including the stair-stepping effect,
are reduced. Also, tighter bounds give smaller ranges for limiting
coefficient values, which leads to a decrease in nonlinear iterations
for implicit and steady-state FCT. However, a disadvantage of tight
solution bounds is that sub-optimal limiters (all known single-pass limiters)
will not be able to accept as much antidiffusion, even though there
may be a combination of limiting coefficients that produce superior
results. Thus in some cases the FCT solution can look only marginally
better than the low-order solution. A remedy to this fact is to use
a more optimal limiter; to achieve this, one can employ the multi-pass
limiting procedure introduced in Section \ref{sec:multipass_limiting}.
This usually produces the best results; however, the computational
expense of the additional passes may need to be considered.
In summary, for general use, the upwind MoC solution bounds
are recommended, especially if multi-pass limiting is able to be applied.
The second dimension of the solution bounds is to make the modification
presented by Equation \eqref{eq:modified_analytic_bounds}, which
gives a less conservative estimate for the bounds of $\scalarsource/\reactioncoef$.
This modification is not supported by an analytic proof, but for the
test problems in which it is applied, the solution bounds appear much
more sensible. The modification only makes a difference for nodes for
which both the reaction coefficient and source vary in the surrounding
neighborhood. Without the modification, there are large peaks in the
solution bounds of such nodes, but these peaks disappear with the
modification. In some cases studied, the modification of the solution
bounds had little or no effect because limitation requirements of
nodes adjacent to these interface nodes were the bottleneck; however,
in other test problems, this modification makes a significant difference.
The recommendation here is to make the modification if the rigorous
mathematical support of the solution bounds is deemed less important.


