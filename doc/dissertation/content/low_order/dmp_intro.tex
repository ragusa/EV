In this section, the low-order schemes described in Section
\ref{sec:low_order_scheme_scalar} are each shown to satisfy a local discrete
maximum principle (DMP). The DMP is analogous to local extremum diminishing
(LED) constraints such as
\begin{equation}
   \solutionletter_{\textup{min},i}^n \leq \solutionletter_i^{n+1}
     \leq \solutionletter_{\textup{max},i}^n \eqc
\end{equation}
where $\solutionletter_{\textup{min},i}^n$ and $\solutionletter_{\textup{max},i}^n$
are the minimum and maximum, respectively, of the old solution in the
neighborhood of $i$. Thus if $i$ corresponds to a local minimum,
it cannot shrink, and if it corresponds to a local maximum, it cannot
grow. However, this particular constraint does not apply to general scalar
conservation law given by Equation \eqref{eq:scalar_transport} due
to the presence of the reaction term and source term; without these
terms, the DMP given in this section for explicit Euler will reduce
to this constraint. The DMP gives proof that the solution local minima
will not decrease without a reaction term and that the solution local
maxima will not increase without a source. In addition, the lower DMP
bound can provide proof that a particular scheme is positivity-preserving.
