For assembly of the low-order system, a modification of the
the local viscous bilinear form introduced in Section \ref{sec:low_order_viscosity}
is defined.
%--------------------------------------------------------------------------------
\begin{definition}{Local Viscous Bilinear Form for Systems}
   The local viscous bilinear form for cell $\cell$ and solution component
   $m$ is defined as follows:
   \begin{equation}\label{eq:bilinearform_system}
     \localvisc_{\cell}^m(\vectortestfunction_i,\vectortestfunction_j)
       \equiv \left\{\begin{array}{l l}
         -\frac{1}{\cardsystem_\cell - 1}\cellvolume\eqc & j\ne i\eqc
       \quad i,j\in \indicescell^m\eqc \\
       \cellvolume\eqc & j = i \eqc \quad i,j\in \indicescell^m\eqc \\
       0          \eqc & \mbox{otherwise}\eqc
     \end{array}\right.
   \end{equation}
   with $\indicescell^m$ being the set of indices of degrees of
   freedom corresponding to solution component $m$ whose basis functions
   have support on cell $\cell$:
   \begin{equation}
     \indicescell^m \equiv \{ j :
       \componentindex(j) = m \eqc \quad|\support_j\cap \celldomain|\ne 0\}
     \eqc
   \end{equation}
   $\cardsystem_\cell$ being the number of elements in the set $\indicescell^m$
   (for any $m$), i.e., the set's cardinality:
   \begin{equation}
     \cardsystem_\cell \equiv \textup{card}(\indicescell^m) =
       \frac{\cardinality}{\ncomponents}
     \eqc
   \end{equation} 
   and $\cellvolume$ being the cell volume.
\end{definition}
%--------------------------------------------------------------------------------
However, for the computation of the low-order viscosity, it is more convenient
to use a scalar version of this bilinear form, given in the following definition.
%--------------------------------------------------------------------------------
\begin{definition}{Scalar Version of the Local Viscous Bilinear Form for Systems}
   The local viscous bilinear form for cell $\cell$ is
   \begin{equation}\label{eq:bilinearform_system}
     \localvisc_{\cell}(\testfunction_i,\testfunction_j)
       \equiv \left\{\begin{array}{l l}
         -\frac{1}{\cardsystem_\cell - 1}\cellvolume\eqc & j\ne i\eqc
       \quad i,j\in \indicesnode\eqc \\
       \cellvolume\eqc & j = i \eqc \quad i,j\in \indicesnode\eqc \\
       0          \eqc & \mbox{otherwise}\eqc
     \end{array}\right.
   \end{equation}
   with $\indicesnode$ being the set of node indices, not degree of
   freedom indices, whose scalar basis functions have support on cell $\cell$:
   \begin{equation}
     \indicesnode \equiv \{ j :
       |\support(\testfunction_j)\cap \celldomain|\ne 0\}
     \eqp
   \end{equation}
\end{definition}
%--------------------------------------------------------------------------------
The definition of the low-order viscosity follows.
%--------------------------------------------------------------------------------
\begin{definition}{Low-Order Viscosity}
   The low-order viscosity for cell $\cell$ is defined as follows:
   \begin{equation}
     \lowordercellviscosity[\timeindex] \equiv
       \max\limits_{j\ne i\in\indicesnode}
         \frac{\maxwavespeed(\normalvector_{i,j},\vectorsolution^\timeindex_i,
           \vectorsolution^\timeindex_j)\|\gradiententry\|_{\ell^2}}
         {-\mkern-20mu\sumKSij[T]\mkern-20mu\localviscbilinearform{T}{i}{j}}
     \eqc
   \end{equation}
   where $\support\ij=\support_i\cap \support_j$ is the dual-support of test
   functions $\testfunction_i$ and $\testfunction_j$.
\end{definition}
%--------------------------------------------------------------------------------
\begin{remark}
For the shallow water equations, it can be advantageous to multiply the low-order
viscosity by the local Froude number $\froude_\cell\equiv\max\limits_q\froude_q$,
where the Froude number is defined as $\froude\equiv\frac{\speed}{\speedofsound}$.
\end{remark}
