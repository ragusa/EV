In the case of \emph{systems} of conservation laws, the discrete maximum
principle is no longer an applicable tool. Instead, the desired property
is \emph{domain-invariance}.
Thus the low-order scheme for conservation law systems is centered around
this property.
 The approach given in this section is taken
from recent work by Guermond \cite{guermond_invariantdomain}. This section
will begin by making definitions necessary to describe the domain-invariance
property of the low-order scheme. Subsequent sections will define the scheme,
including the low-order diffusion terms necessary to ensure the
invariant domain property.

It is desired that the solution process produce admissible
(physical, entropy-satisfying) solutions; let the space of these solutions be
$\admissibleset\subset\realspace[\ncomponents]$, where $\ncomponents$ is the
number of components in the system. The definition of an \emph{invariant set}
follows.
%------------------------------------------------------------------------------
\begin{definition}{Invariant Set}
Consider the following Riemann initial value problem (IVP):
\begin{equation}\label{eq:riemannivp}
  \ppt{\vectorsolution} + \pd{}{x}(\consflux(\vectorsolution)\cdot\normalvector)
    = 0 \eqc
  \quad (x,t)\in\realspace\times\realspace_+ \eqc
  \quad \vectorsolution(x,0) = \left\{\begin{array}{l}
    \vectorsolution_L \eqc \quad x \leq 0 \eqc\\
    \vectorsolution_R \eqc \quad x > 0 \eqc
  \end{array}\right. \eqp
\end{equation}
This problem has a unique solution which is denoted by
$\vectorsolution(\vectorsolution_L,\vectorsolution_R,\normalvector)(x,t)$.
A set $\invariantset\subset\admissibleset\subset\realspace[\ncomponents]$ 
is called an invariant set for the Riemann problem given by Equation
\eqref{eq:riemannivp} if and only if
$\forall(\vectorsolution_L,\vectorsolution_R)\in\invariantset\times\invariantset$,
$\forall\normalvector$ on the unit sphere, and $\forall t > 0$, the average
of the entropy solution 
$\vectorsolution(\vectorsolution_L,\vectorsolution_R,\normalvector)$
over the Riemann fan, i.e.,
\begin{equation}
  \bar{\vectorsolution} \equiv
    \frac{1}{t(\wavespeed_{\ncomponents}^+ - \wavespeed_1^-)}
    \int\limits_{\wavespeed_1^- t}^{\wavespeed_{\ncomponents}^+ t}
    \vectorsolution(\vectorsolution_L,\vectorsolution_R,\normalvector)(x,t)dx \eqc
\end{equation}
is an element in $\invariantset$.
\end{definition}
%------------------------------------------------------------------------------
Let $\discreteprocess$ be defined as the discrete solution process,
which produces each subsequent approximate solution:
$\vectorsolution^{n+1} = \discreteprocess(\vectorsolution^n)$.

Recall the following definition for a \emph{convex set}:
%------------------------------------------------------------------------------
\begin{definition}{Convex Set}
A convex set $\invariantset$ is a set such that for any two elements in
the set, the line connecting the two remains completely in the set.
As a consequence, any convex combination of elements in the set remains in the
set: $\sum_i\convexcoefficient_i\convexelement_i\in\invariantset$, where
$\convexelement_i\in\invariantset\quad\forall i$,
$\convexcoefficient_i\geq 0\quad\forall i$, and $\sum_i\convexcoefficient_i=0$.
\end{definition}
%------------------------------------------------------------------------------
Finally the definition for an \emph{invariant domain} is made.
%------------------------------------------------------------------------------
\begin{definition}{Invariant Domain}
A convex invariant set $\invariantset$ is an invariant domain for the
process $\discreteprocess$ if and only if
$\forall\vectorsolution\in\invariantset$,
$\discreteprocess(\vectorsolution)\in\invariantset$.
\end{definition}
%------------------------------------------------------------------------------
Now all of the necessary definitions have been made. Proving the invariant
domain property with respect to the discrete process $\discreteprocess$ takes
the following approach:
\begin{enumerate}
  \item Assume the initial data $\vectorsolution^0\in\invariantset$, where
    $\invariantset$ is a convex invariant set.
  \item Prove that the discrete scheme $\discreteprocess$ is such that
    $\vectorsolution^{n+1}\equiv\discreteprocess(\vectorsolution^n)$
    can be expressed a convex combination of elements in $\invariantset$:
    $\vectorsolution^{n+1}=\sum_i\convexcoefficient_i\convexelement_i$.
    \begin{enumerate}
      \item Prove $\sum_i\convexcoefficient_i=0$.
      \item Prove $\convexcoefficient_i\geq 0\forall i$, which requires
        conditions on the time step size.
      \item Prove $\convexelement_i\in\invariantset\quad\forall i$.
    \end{enumerate}
  \item Invoke the definition of a convex set to prove that $\invariantset$
    is an invariant domain for the process $\discreteprocess$.
\end{enumerate}
These proofs are given in detail in \cite{guermond_invariantdomain} and
will not be reproduced here.

