In the case of \emph{systems} of conservation laws, the local discrete maximum
principle is no longer valid. Instead, the desired property
is \emph{domain-invariance}.
Thus the low-order scheme for conservation law systems is designed around
this property.
 The approach given in this section is taken
from recent work by Guermond \cite{guermond_invariantdomain}. This section
will begin by making definitions necessary to describe the domain-invariance
property of the low-order scheme. Subsequent sections will define the scheme,
including the low-order diffusion terms necessary to ensure the
invariant domain property.

It is desired that the solution process produce admissible
(physical, entropy-satisfying) solutions; let the space of these solutions be
$\admissibleset\subset\realspace[\ncomponents]$, where $\ncomponents$ is the
number of components in the system. The following definition of an
\emph{invariant set} comes from \cite{guermond_invariantdomain}:
%------------------------------------------------------------------------------
\begin{definition}{Invariant Set}
Consider the following Riemann initial value problem (IVP):
\begin{equation}\label{eq:riemannivp}
  \ppt{\vectorsolution} + \pd{}{x}(\consfluxsystem(\vectorsolution)\cdot\normalvector)
    = 0 \eqc
  \quad (x,t)\in\realspace\times\realspace_+ \eqc
  \quad \vectorsolution(x,0) = \left\{\begin{array}{l}
    \vectorsolution_L \eqc \quad x \leq 0 \eqc\\
    \vectorsolution_R \eqc \quad x > 0 \eqc
  \end{array}\right. \eqc
\end{equation}
where $\vectorsolution\in\realspace^\ncomponents$ is the $\ncomponents$-component solution,
$\consfluxsystem(\vectorsolution)\equiv[\consflux(\vectorsolution)_1,\ldots,
 \consflux(\vectorsolution)_{\ncomponents}]^T$ is a matrix of size $\ncomponents\times d$,
where each row $i$ is the $d$-dimensional conservation
law flux for component $i$, and $\normalvector$ is a unit vector
in $\realspace^d$.
This problem has a unique solution which is denoted by
$\vectorsolution(\vectorsolution_L,\vectorsolution_R,\normalvector)(x,t)$.
A set $\invariantset\subset\admissibleset\subset\realspace[\ncomponents]$ 
is called an invariant set for the Riemann problem given by Equation
\eqref{eq:riemannivp} if and only if
$\forall(\vectorsolution_L,\vectorsolution_R)\in\invariantset\times\invariantset$,
$\forall\normalvector$ on the unit sphere, and $\forall t > 0$, the average
of the entropy solution 
$\vectorsolution(\vectorsolution_L,\vectorsolution_R,\normalvector)$
over the Riemann fan, i.e.,
\begin{equation}
  \bar{\vectorsolution} \equiv
    \frac{1}{t(\wavespeed_{\ncomponents}^+ - \wavespeed_1^-)}
    \int\limits_{\wavespeed_1^- t}^{\wavespeed_{\ncomponents}^+ t}
    \vectorsolution(\vectorsolution_L,\vectorsolution_R,\normalvector)(x,t)dx \eqc
\end{equation}
is an element in $\invariantset$.
\end{definition}
%------------------------------------------------------------------------------
To summarize the definition above, an invariant set is one in which
a 1-D Riemann problem using any two elements in the set as left and right
data in any direction produces an entropy solution that remains in that set.

Before defining an invariant domain, first recall the following definition
for a \emph{convex set}:
%------------------------------------------------------------------------------
\begin{definition}{Convex Set}
A convex set $\invariantset$ is a set such that for any two elements in
the set, the line connecting the two remains completely in the set.
As a consequence, any convex combination of elements in the set remains in the
set: $\sum_k\convexcoefficient_k\convexelement_k\in\invariantset$, where
$\convexelement_k\in\invariantset\;\forall k$,
$\convexcoefficient_k\geq 0\,\,\forall k$, and $\sum_k\convexcoefficient_k=1$.
\end{definition}
%------------------------------------------------------------------------------
Now the definition for an \emph{invariant domain} is made:
%------------------------------------------------------------------------------
\begin{definition}{Invariant Domain}
Let $\discreteprocess$ be defined as the discrete solution process,
which produces each subsequent approximate solution:
$\vectorsolution^{n+1} = \discreteprocess(\vectorsolution^n)$.
A convex invariant set $\invariantset$ is called an invariant domain for the
process $\discreteprocess$ if and only if
$\forall\vectorsolution\in\invariantset$,
$\discreteprocess(\vectorsolution)\in\invariantset$.
\end{definition}
%------------------------------------------------------------------------------
Now all of the necessary definitions have been made. Proving the invariant
domain property with respect to the discrete process $\discreteprocess$ takes
the following approach:
\begin{enumerate}
  \item Assume the initial data $\vectorsolution^0\in\invariantset$, where
    $\invariantset$ is a convex invariant set.
  \item Prove that the discrete scheme $\discreteprocess$ is such that
    $\vectorsolution^{n+1}\equiv\discreteprocess(\vectorsolution^n)$
    can be expressed a convex combination of elements in $\invariantset$:
    $\vectorsolution^{n+1}=\sum_k\convexcoefficient_k\convexelement_k$.
    \begin{enumerate}
      \item Prove $\sum_k\convexcoefficient_k=1$.
      \item Prove $\convexcoefficient_k\geq 0\,\forall k$, which requires
        conditions on the time step size.
      \item Prove $\convexelement_k\in\invariantset\;\forall k$.
    \end{enumerate}
  \item Invoke the definition of a convex set to prove that $\invariantset$
    is an invariant domain for the process $\discreteprocess$.
\end{enumerate}
These proofs are given in detail in \cite{guermond_invariantdomain} and
will not be reproduced here. The time step size requirement is the
following:
\begin{subequations}
\label{eq:dt_invariant_domain}
\begin{equation}
  \dt \leq \frac{1}{2}\frac{\underline{\dx}}{\wavespeed_{\textup{max}}(\invariantset)}
    \frac{\mu_{\textup{min}}\vartheta_{\textup{min}}}{\mu_{\textup{max}}} \eqc
\end{equation}
\begin{equation}
  \underline{\dx} \equiv \min\limits_\cell \underline{\dx}_\cell \eqc \quad
    \underline{\dx}_\cell \equiv \frac{1}{\max\limits_{i\ne j\in\indicescell}
      \|\nabla\testfunction_i\|_{L^\infty(\support\ij)}} \eqc
\end{equation}
\begin{equation}
  \vartheta_{\textup{min}} \equiv \min\limits_\cell \vartheta_\cell \eqc \quad
    \vartheta_\cell \equiv \frac{1}{\cardinality-1} \eqc
\end{equation}
\begin{equation}
  \mu_{\textup{min}} \equiv \min\limits_\cell \min\limits_{i\in\indicescell}
    \frac{1}{\cellvolume}\int\limits_\cell \testfunction_i \dvolume \eqc \quad
  \mu_{\textup{max}} \equiv \max\limits_\cell \max\limits_{i\in\indicescell}
    \frac{1}{\cellvolume}\int\limits_\cell \testfunction_i \dvolume \eqc
\end{equation}
\begin{equation}
  \wavespeed^{\textup{max}}(\invariantset) \equiv
    \max\limits_{\normalvector\in S^d(\mathbf{0},1)}
    \max\limits_{\vectorsolution_L,\vectorsolution_R\in\invariantset}
    \wavespeed^{\textup{max}}(\normalvector,\vectorsolution_L,\vectorsolution_R) \eqc
\end{equation}
\end{subequations}
where $S^d(\mathbf{0},1)$ is the unit sphere in dimension $d$. For example, in 1-D,
$\underline{\dx}=\dx_{\textup{min}}\equiv\min_\cell\dx_\cell$, and
$\vartheta_{\textup{min}}=\mu_{\textup{min}}=\mu_{\textup{max}}=1$, so the
time step size requirement becomes
\begin{equation}
  \dt \leq \frac{1}{2}\frac{\dx_{\textup{min}}}{\wavespeed_{\textup{max}}(\invariantset)}
    \eqp
\end{equation}
From this expression, it is easier to see the underlying requirement: for a given
time step step, the elementary waves generated from each Riemann problem between
neighboring nodal data must not travel far enough such that they can interact.
Each of these Riemann problems is separated by a distance $\geq \dx_{\textup{min}}$,
so if for example for a pair of adjacent Riemann problems, a wave is traveling
to the right from the left Riemann problem and to the left from the right Riemann
problem, then they could interact at half the distance between them.
In practice, to determine $\wavespeed_{\textup{max}}(\invariantset)$ for a
transient, one needs to loop over all pairs of initial data $\vectorsolution_i$
and $\vectorsolution_j$ for which the associated test functions have
nonempty shared support $\support\ij$, and compute the maximum wave speeds
from the associated Riemann problem (or some upper bound to them).


