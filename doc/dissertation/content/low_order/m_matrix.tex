In this section, it will be shown that the low-order steady-state system matrix
defined in Equation \eqref{eq:low_order_ss_matrix} is an M-matrix, which allows
a discrete maximum principle for the low-order solution to be proven in Section
\ref{sec:DMP}.
%--------------------------------------------------------------------------------
\begin{lemma}[lem:offdiagonalnegative]{Non-Positivity of Off-Diagonal Elements}
   The off-diagonal elements of the linear system matrix are non-positive:
   \[
     \ssmatrixletter^{L,\timeindex}\ij\le 0, \quad j\ne i \eqp
   \]
\end{lemma}

\begin{proof}
This proof begins by bounding the term $\diffusionmatrixletter\ij^{L,\timeindex}$:
\begin{eqnarray*}
   \diffusionmatrixletter\ij^{L,\timeindex}=
     \sumKSij\lowordercellviscosity[\timeindex]
   \localviscbilinearform{\cell}{j}{i}
   & = & \sumKSij\max\limits_{k\ne \ell\in \indicescell}
     \pr{\frac{\max(0,\ssmatrixletter_{k,\ell}^\timeindex)}
       {\mkern10mu-\mkern-20mu\sum\limits_{T:\celldomain[T]\subset\support_{k,\ell}}
       \mkern-20mu\localviscbilinearform{T}{\ell}{k}}}
     \localviscbilinearform{\cell}{j}{i} \eqp
\end{eqnarray*}
Since $\localviscbilinearform{\cell}{j}{i} < 0$ for $j\ne i$ and $y_i \leq
\max_j y_j$,
\begin{eqnarray*}
   \diffusionmatrixletter\ij^{L,\timeindex} & \le &
     \sumKSij \frac{\max(0,\ssmatrixletter\ij^\timeindex)}
   {-\sumKSij[T]\localviscbilinearform{T}{j}{i}}
   \localviscbilinearform{\cell}{j}{i} \eqc\\
   &  =  & -\max(0,\ssmatrixletter\ij^\timeindex)
     \frac{\sumKSij\localviscbilinearform{\cell}{j}{i}}
     {\sumKSij[T]\localviscbilinearform{T}{j}{i}} \eqc\\
   &  =  & -\max(0,\ssmatrixletter\ij^\timeindex) \eqc\\
   & \le & -\ssmatrixletter\ij^\timeindex \eqp
\end{eqnarray*}
Applying this inequality to Equation \eqref{eq:low_order_ss_matrix} gives
\begin{eqnarray*}
  \ssmatrixletter^{L,\timeindex}\ij &  =  &
    \ssmatrixletter\ij^\timeindex + \diffusionmatrixletter\ij^{L,\timeindex}
    \eqc\\
  \ssmatrixletter^{L,\timeindex}\ij & \le &
    \ssmatrixletter\ij^\timeindex - \ssmatrixletter\ij^\timeindex
    \eqc\\
  \ssmatrixletter^{L,\timeindex}\ij & \le & 0 \eqp \qed
\end{eqnarray*}
\end{proof}
%--------------------------------------------------------------------------------
\begin{lemma}[lem:diagonalpositive]{Non-Negativity of Diagonal Elements}
   The diagonal elements  of the linear system matrix are non-negative:
   \[
     \ssmatrixletter^{L,\timeindex}_{i,i}\ge 0 \eqp
   \]
\end{lemma}

\begin{proof}
The diagonal elements of the low-order system matrix are
\[
  \ssmatrixletter^{L,\timeindex}_{i,i} =
    \intSi\mathbf{\consfluxletter}'(\approximatescalarsolution^\timeindex)\cdot
    \nabla\testfunction_i(\x)\testfunction_i(\x)\dvolume
  + \intSi\sigma(\x)\testfunction_i^2(\x)\dvolume
  + \sumKSi\mkern-15mu\lowordercellviscosity[\timeindex]
    \localviscbilinearform{\cell}{i}{i}
  \eqp
\]
To prove that $\ssmatrixletter^{L,\timeindex}_{i,i}$ is non-negative, it is sufficient to
prove that each term in the above expression is non-negative. The
non-negativity of the interaction term and viscous term are obvious
($\reactioncoef \ge 0, \, \lowordercellviscosity[\timeindex]\ge 0, \,
\localviscbilinearform{\cell}{i}{i}>0$), but the non-negativity of the divergence
term is not necessarily obvious. On the interior of the domain, the divergence
term gives zero contribution because the divergence integral may be transformed
into a surface integral
$\intSi\mathbf{\consfluxletter}'(\approximatescalarsolution^\timeindex)
\cdot\normalvector\frac{\testfunction_i^2}{2} d\area$ via the
divergence theorem; one can then recognize that the basis function
$\testfunction_i$ evaluates to zero on the boundary of its support
$\support_i$. On the outflow boundary of the domain, the term
$\mathbf{\consfluxletter}'(\approximatescalarsolution^\timeindex)
\cdot\normalvector \frac{\testfunction_i^2}{2}$ is positive because
$\mathbf{\consfluxletter}'(\approximatescalarsolution^\timeindex)
\cdot\normalvector > 0$ for an outflow boundary. This quantity is of
course negative for the inflow boundary, so one must consider the boundary
conditions applied for incoming boundary nodes to determine if this condition
is true and a discrete maximum principle applies. For instance, if a Dirichlet boundary condition is
applied, then a discrete maximum principle does not apply.
strongly imposed on the incoming boundary, so for degrees of freedom $i$ on the
incoming boundary, $\ssmatrixletter^{L,\timeindex}_{i,i}$ will be set equal to some positive
value such as 1 with a corresponding incoming value accounted for in the right
hand side $\ssrhs$ of the linear system.\qed
\end{proof}
%--------------------------------------------------------------------------------
\begin{lemma}{Non-Negativity of Row Sums}
   The sum of all elements in a row $i$ is non-negative:
   \[
     \sumj \ssmatrixletter^{L,\timeindex}\ij \ge 0 \eqp
   \]
\end{lemma}

\begin{proof}
Using the fact that $\sumj\testfunction_j(\x)=1$ and
$\sumj \localviscbilinearform{\cell}{j}{i}=0$,
\begin{eqnarray*}
   \sumj \ssmatrixletter^{L,\timeindex}\ij & = & \sumj \intSij
      \left(\mathbf{\consfluxletter}'(\approximatescalarsolution^\timeindex)
        \cdot\nabla\testfunction_j +
      \reactioncoef\testfunction_j\right)\testfunction_i \dvolume +
      \sumj\sumKSij\lowordercellviscosity[\timeindex]
        \localviscbilinearform{\cell}{j}{i}
      \eqc\\
   & = & \intSi\left(
      \mathbf{\consfluxletter}'(\approximatescalarsolution^\timeindex)\cdot
      \nabla\sumj\testfunction_j(\x) +
      \reactioncoef(\x)\sumj\testfunction_j(\x)\right)
      \testfunction_i(\x) \dvolume \eqc\\
   \label{eq:rowsum} & = & \intSi\reactioncoef(\x)\testfunction_i(\x) \dvolume
     \eqc\\
   &\ge& 0 \eqp \qed
\end{eqnarray*}
\end{proof}
%--------------------------------------------------------------------------------
\begin{lemma}[lem:diagonallydominant]{Diagonal Dominance}
   $\loworderssmatrix[\timeindex]$ is strictly diagonally dominant:
   \[
     \left|\ssmatrixletter^{L,\timeindex}_{i,i}\right|
     \ge \sumjnoti \left|\ssmatrixletter^{L,\timeindex}\ij\right| \eqp
   \]
\end{lemma}
\begin{proof}
Using the inequalities $\sumj \ssmatrixletter^{L,\timeindex}\ij \ge 0$ and
$\ssmatrixletter^{L,\timeindex}\ij\le 0, j\ne i$, it is proven that
$\loworderssmatrix[\timeindex]$ is strictly diagonally dominant:
\begin{eqnarray*}
  \sumj     \ssmatrixletter^{L,\timeindex}\ij       & \ge & 0 \eqc\\
  \sumjnoti \ssmatrixletter^{L,\timeindex}\ij
    + \ssmatrixletter^{L,\timeindex}_{i,i} & \ge & 0 \eqc\\
  \left|\ssmatrixletter^{L,\timeindex}_{i,i}\right| & \ge &
    \sumjnoti -\ssmatrixletter^{L,\timeindex}\ij
    \eqc\\
  \left|\ssmatrixletter^{L,\timeindex}_{i,i}\right| & \ge
    & \sumjnoti \left|\ssmatrixletter^{L,\timeindex}\ij\right| \eqp \qed
\end{eqnarray*}
\end{proof}
%--------------------------------------------------------------------------------
\begin{lemma}{M-Matrix}
  $\loworderssmatrix[\timeindex]$ is an M-Matrix.
\end{lemma}
\begin{proof}
To prove that a matrix is an M-Matrix, it is sufficient to prove that
the following 3 statements are true:
\begin{enumerate}
\item $\ssmatrixletter^{L,\timeindex}\ij\le 0, j\ne i$,
\item $\ssmatrixletter^{L,\timeindex}_{i,i}\ge 0$,
\item $\left|\ssmatrixletter^{L,\timeindex}_{i,i}\right|
      \ge \sumjnoti \left|\ssmatrixletter^{L,\timeindex}\ij\right|$.
\end{enumerate}
These conditions are proven by Lemmas \ref{lem:offdiagonalnegative},
\ref{lem:diagonalpositive}, and \ref{lem:diagonallydominant}, respectively.\qed
\end{proof}
%--------------------------------------------------------------------------------
