In this section, it will be shown that the low-order steady-state system matrix
$\loworderssmatrix$ defined in Equation \eqref{eq:low_order_ss_matrix} is an M-matrix, also
called an inverse-positive matrix or a monotone matrix. An inverse-positive
matrix $\mathbf{A}$ has the property
\begin{equation}\label{eq:m_matrix}
  \mathbf{A}\mathbf{x} \geq \mathbf{0} \Rightarrow \mathbf{x} \geq \mathbf{0} \eqp
\end{equation}
Thus if one has a linear system $\mathbf{A}\mathbf{x} = \mathbf{b}$ with
$\mathbf{b} \geq \mathbf{0}$, then the solution $\mathbf{x}$
is known to be non-negative.
This property is used in Section \ref{sec:positivity_preservation}
to prove the positivity-preservation of the low-order scheme in each
temporal discretization.

The non-positivity off-diagonal elements of $\loworderssmatrix$ and
positivity of its diagonal elements are proven in the following
lemmas. These properties are used in the proof that $\loworderssmatrix$
is an M-matrix, and are also used in Section \ref{sec:DMP} to
prove local discrete maximum principles for each temporal discretization
of the low-order scheme.
%--------------------------------------------------------------------------------
\begin{lemma}[lem:offdiagonalnegative]{Non-Positivity of Off-Diagonal Elements}
   The off-diagonal elements of the matrix $\loworderssmatrix[n]$ are non-positive:
   \[
     \ssmatrixletter^{L,\timeindex}\ij\le 0 \eqc \quad j\ne i
       \eqc \quad \forall i\eqp
   \]
\end{lemma}

\begin{proof}
This proof begins by bounding the term $\diffusionmatrixletter\ij^{L,\timeindex}$:
\begin{eqnarray*}
   \diffusionmatrixletter\ij^{L,\timeindex}=
     \sumKSij\lowordercellviscosity[\timeindex]
   \localviscbilinearform{\cell}{j}{i}
   & = & \sumKSij\max\limits_{k\ne \ell\in \indicescell}
     \pr{\frac{\max(0,\ssmatrixletter_{k,\ell}^\timeindex)}
       {-\sum\limits_{T\in\cellindices(\support_{k,\ell})}
       \mkern-20mu\localviscbilinearform{T}{\ell}{k}}}
     \localviscbilinearform{\cell}{j}{i} \eqp
\end{eqnarray*}
Recall $\localviscbilinearform{\cell}{j}{i} < 0$ for $j\ne i$.
For an arbitrary quantity $c_{k,\ell} \geq 0 \eqc \forall k\ne\ell\in\indices$,
the following is true for $i\ne j\in\indices$:
$\max\limits_{k\ne\ell\in\indices}c_{k,\ell} \geq c\ij$, and thus for $a\leq 0$,
$a\max\limits_{k\ne\ell\in\indices}c_{k,\ell} \leq a c\ij$.
Thus,
\begin{eqnarray*}
   \diffusionmatrixletter\ij^{L,\timeindex} & \le &
     \sumKSij \frac{\max(0,\ssmatrixletter\ij^\timeindex)}
   {-\sumKSij[T]\localviscbilinearform{T}{j}{i}}
   \localviscbilinearform{\cell}{j}{i} \eqc\\
   &  =  & -\max(0,\ssmatrixletter\ij^\timeindex)
     \frac{\sumKSij\localviscbilinearform{\cell}{j}{i}}
     {\sumKSij[T]\localviscbilinearform{T}{j}{i}} \eqc\\
   &  =  & -\max(0,\ssmatrixletter\ij^\timeindex) \eqc\\
   & \le & -\ssmatrixletter\ij^\timeindex \eqp
\end{eqnarray*}
Applying this inequality to Equation \eqref{eq:low_order_ss_matrix} gives
\begin{eqnarray*}
  \ssmatrixletter^{L,\timeindex}\ij &  =  &
    \ssmatrixletter\ij^\timeindex + \diffusionmatrixletter\ij^{L,\timeindex}
    \eqc\\
  \ssmatrixletter^{L,\timeindex}\ij & \le &
    \ssmatrixletter\ij^\timeindex - \ssmatrixletter\ij^\timeindex
    \eqc\\
  \ssmatrixletter^{L,\timeindex}\ij & \le & 0 \eqp \qed
\end{eqnarray*}
\end{proof}
%--------------------------------------------------------------------------------
%The following lemma proves that the diagonal elements of the low-order system
%matrix are non-negative.
%; however, this proof uses the assumption that the
%conservation law flux is linear with a uniform velocity field:
%$\consfluxscalar = \velocity\scalarsolution$, where $\velocity$ is some
%constant velocity field. This assumption is valid for the linear transport
%equation given by Equation \eqref{eq:rad_transport}, where
%$\consfluxscalar = \speed\directionvector\scalarsolution$.
%In addition, the proof of this lemma only applies for degrees of freedom
%that are not on the incoming boundary. Let $\indices(\triangulation)$
%denote the set of all degree of freedom indices in the domain
%($\triangulation$ is the triangulation), $\incomingindices$
%denote the subset of indices corresponding to degrees on freedom on the
%incoming boundary, and $\notincomingindices$ denote the remaining
%indices (those corresponding to degrees of freedom not on the incoming
%boundary):
%\begin{equation}
%  \notincomingindices = \left\{j\in\indices(\triangulation)
%    \, | \, j\notin \incomingindices \right\} \eqp
%\end{equation}
%--------------------------------------------------------------------------------
%\begin{lemma}[lem:diagonalpositive]{Non-Negativity of Diagonal Elements}
%  If the conservation law flux function  is
%  linear with a uniform velocity field $\velocity$:
%  $\consfluxscalar = \velocity\scalarsolution$, then
%   the diagonal elements of the linear system matrix are non-negative:
%   \[
%     \ssmatrixletter^{L,\timeindex}_{i,i}
%       = \ssmatrixletter^{L}_{i,i}\ge 0
%     \eqc \quad \forall i\notin\incomingindices\eqp
%   \]
%\end{lemma}
%
%\begin{proof}
%For $\consfluxscalar = \velocity\scalarsolution$,
%the diagonal elements of the low-order system matrix are
%\[
%  \ssmatrixletter^{L}_{i,i} =
%    \intSi\velocity\cdot
%    \nabla\testfunction_i(\x)\testfunction_i(\x)\dvolume
%  + \intSi\sigma(\x)\testfunction_i^2(\x)\dvolume
%  + \sumKSi\mkern-15mu\lowordercellviscosity[\timeindex]
%    \localviscbilinearform{\cell}{i}{i}
%  \eqp
%\]
%To prove that $\ssmatrixletter^{L}_{i,i}$ is non-negative, it is sufficient to
%prove that each term in the above expression is non-negative. The
%non-negativity of the interaction term and viscous term are obvious
%($\reactioncoef \ge 0, \, \lowordercellviscosity[\timeindex]\ge 0, \,
%\localviscbilinearform{\cell}{i}{i}>0$), but the non-negativity of the divergence
%term is not necessarily obvious.
%The divergence integral may be transformed into a surface integral via the
%divergence theorem:
%\begin{align*}
%  \intSi\velocity\cdot
%    \nabla\testfunction_i(\x)\testfunction_i(\x)\dvolume
%  &= \intSi\nabla\cdot\pr{
%    \velocity\testfunction_i(\x)}\testfunction_i(\x)\dvolume\\
%  &= \intSi\nabla\cdot\pr{
%    \velocity\frac{1}{2}\testfunction_i(\x)^2}\dvolume\\
%  &= \int\limits_{\partial \support_i}
%    \frac{1}{2}\testfunction_i(\x)^2 \velocity\cdot\normalvector d\area \eqp
%\end{align*}
%On the interior of the domain, 
% the basis function
%$\testfunction_i$ evaluates to zero on the boundary of its support
%$\support_i$. On the outflow boundary of the domain, the term
%$\frac{1}{2}\testfunction_i(\x)^2\velocity\cdot\normalvector$ is positive because
%$\velocity\cdot\normalvector > 0$ for an outflow boundary.
%Conversely, this quantity is
%negative for the inflow boundary, so one must consider the boundary
%conditions applied for incoming boundary nodes to determine if this condition
%is true.
%If the system corresponds to steady-state and a Dirichlet boundary condition
%is strongly imposed, then this proof
%is automatic, provided a positive value is chosen for the diagonal entry
%$\ssmatrixletter^L_{i,i}$.\qed
%\end{proof}
%--------------------------------------------------------------------------------
\begin{lemma}[lem:rowsumspositive]{Non-Negativity of Row Sums}
   The row-sums of the matrix $\loworderssmatrix[n]$ are non-negative:
   \[
     \sumj \ssmatrixletter^{L,\timeindex}\ij \ge 0
       \eqc \quad \forall i \eqp
   \]
\end{lemma}

\begin{proof}
Using the fact that $\sumj\testfunction_j(\x)=1$ and
$\sumj \localviscbilinearform{\cell}{j}{i}=0$,
\begin{eqnarray*}
   \sumj \ssmatrixletter^{L,\timeindex}\ij & = & \sumj \intSij
      \left(\mathbf{\consfluxletter}'(\approximatescalarsolution^\timeindex)
        \cdot\nabla\testfunction_j +
      \reactioncoef\testfunction_j\right)\testfunction_i \dvolume +
      \sumj\sumKSij\lowordercellviscosity[\timeindex]
        \localviscbilinearform{\cell}{j}{i}
      \eqc\\
   & = & \intSi\left(
      \mathbf{\consfluxletter}'(\approximatescalarsolution^\timeindex)\cdot
      \nabla\sumj\testfunction_j(\x) +
      \reactioncoef(\x)\sumj\testfunction_j(\x)\right)
      \testfunction_i(\x) \dvolume \eqc\\
   \label{eq:rowsum} & = & \intSi\reactioncoef(\x)\testfunction_i(\x) \dvolume
     \eqc\\
   &\ge& 0 \eqp \qed
\end{eqnarray*}
\end{proof}
%--------------------------------------------------------------------------------
\begin{lemma}[lem:diagonalpositive]{Non-Negativity of Diagonal Elements}
   The diagonal elements of the matrix $\loworderssmatrix[n]$ are non-negative:
   \[
     \ssmatrixletter^{L,n}_{i,i} \ge 0
       \eqc \quad \forall i\eqp
   \]
\end{lemma}

\begin{proof}
Using Lemma \ref{lem:rowsumspositive},
\[
  \sumj \ssmatrixletter^{L,n}\ij \ge 0 \eqp
\]
Thus,
\[
  \ssmatrixletter^{L,n}_{i,i} \ge -\sumjnoti \ssmatrixletter^{L,n}\ij \eqp
\]
From Lemma \ref{lem:offdiagonalnegative}, the off-diagonal elements are known
to be non-positive: $\ssmatrixletter^{L,n}\ij \leq 0$. Thus,
$-\ssmatrixletter^{L,n}\ij \geq 0$ and finally,
\[
  \ssmatrixletter^{L,n}_{i,i} \ge 0 \eqp \qed
\]
\end{proof}
%--------------------------------------------------------------------------------
\begin{lemma}[lem:diagonallydominant]{Diagonal Dominance}
   The matrix $\loworderssmatrix[\timeindex]$ is strictly diagonally dominant:
   \[
     \left|\ssmatrixletter^{L,\timeindex}_{i,i}\right|
     \ge \sumjnoti \left|\ssmatrixletter^{L,\timeindex}\ij\right|
     \eqc \quad \forall i\eqp
   \]
\end{lemma}
\begin{proof}
Using the inequalities $\sumj \ssmatrixletter^{L,\timeindex}\ij \ge 0$ and
$\ssmatrixletter^{L,\timeindex}\ij\le 0, j\ne i$, it is proven that
$\loworderssmatrix[\timeindex]$ is strictly diagonally dominant:
\begin{eqnarray*}
  \sumj     \ssmatrixletter^{L,\timeindex}\ij       & \ge & 0 \eqc\\
  \sumjnoti \ssmatrixletter^{L,\timeindex}\ij
    + \ssmatrixletter^{L,\timeindex}_{i,i} & \ge & 0 \eqc\\
  \left|\ssmatrixletter^{L,\timeindex}_{i,i}\right| & \ge &
    \sumjnoti -\ssmatrixletter^{L,\timeindex}\ij
    \eqc\\
  \left|\ssmatrixletter^{L,\timeindex}_{i,i}\right| & \ge
    & \sumjnoti \left|\ssmatrixletter^{L,\timeindex}\ij\right| \eqp \qed
\end{eqnarray*}
\end{proof}
%--------------------------------------------------------------------------------
\begin{theorem}[thm:m_matrix]{M-Matrix}
  The matrix $\loworderssmatrix[\timeindex]$ is an M-Matrix.
\end{theorem}
\begin{proof}
To prove that a matrix is an M-Matrix, it is sufficient to prove that
the following 3 statements are true:
\begin{enumerate}
\item $\ssmatrixletter^{L,\timeindex}\ij\le 0
      \eqc \quad j\ne i \eqc \forall i$,
\item $\ssmatrixletter^{L,\timeindex}_{i,i}\ge 0 \eqc \quad \forall i$,
\item $\left|\ssmatrixletter^{L,\timeindex}_{i,i}\right|
      \ge \sumjnoti \left|\ssmatrixletter^{L,\timeindex}\ij\right|
      \eqc \quad \forall i$.
\end{enumerate}
These conditions are proven by Lemmas \ref{lem:offdiagonalnegative},
\ref{lem:diagonalpositive}, and \ref{lem:diagonallydominant}, respectively.\qed
\end{proof}
%--------------------------------------------------------------------------------
