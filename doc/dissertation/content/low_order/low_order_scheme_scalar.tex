This section gives the low-order system for each of the considered
temporal discretizations. First, a low-order artificial diffusion matrix
is defined, which is built upon the definitions of the local viscous bilinear
form and low-order viscosity defined in Section \eqref{sec:low_order_viscosity_scalar}.
%--------------------------------------------------------------------------------
\begin{definition}{Low-Order Artificial Diffusion Matrix}
   The low-order artificial diffusion matrix $\loworderdiffusionmatrix[\timeindex]$
   is assembled using the low-order viscosity and local viscous bilinear
   form introduced in Section \ref{sec:low_order_viscosity_scalar}:
   \begin{equation}\label{eq:low_order_diffusion_matrix}
     \diffusionmatrixletter\ij^{L,\timeindex} \equiv
       \sumKSij\lowordercellviscosity[\timeindex]
       \localviscbilinearform{\cell}{j}{i} \eqp
   \end{equation}
\end{definition}
%--------------------------------------------------------------------------------
The low-order system matrix incorporates the low-order artificial diffusion
matrix and is given in the following definition.
%--------------------------------------------------------------------------------
\begin{definition}{Low-Order Steady-State System Matrix}
   The low-order steady-state system matrix is the sum of the inviscid 
   steady-state system matrix $\ssmatrix[\timeindex]$ and the low-order diffusion
   matrix $\loworderdiffusionmatrix[\timeindex]$:
   \begin{equation}\label{eq:low_order_ss_matrix}
      \loworderssmatrix[\timeindex] \equiv
        \ssmatrix[\timeindex] + \loworderdiffusionmatrix[\timeindex] \eqp
   \end{equation}
\end{definition}
%--------------------------------------------------------------------------------
For the low-order system, the mass matrix is lumped:
$\consistentmassmatrix\rightarrow\lumpedmassmatrix$, where
\begin{equation}
  \massmatrixletter^L_{i,j} = \left\{\begin{array}{l l}
    \sum\limits_k \massmatrixletter^C_{i,k} \eqc & j = i\\
    0                                       \eqc & \mbox{otherwise}
    \end{array}\right.
    \eqc
\end{equation}
and the low-order
steady-state system matrix defined in Equation \eqref{eq:low_order_ss_matrix}
is used. The low-order system is given here for different time discretizations:
\begin{center}{\textbf{Steady-state scheme}:}\end{center}
\begin{equation}\label{eq:low_ss}
   \loworderssmatrix\lowordersolution = \ssrhs
\end{equation}
\begin{center}{\textbf{Semidiscrete scheme}:}\end{center}
\begin{equation}\label{eq:low_semidiscrete}
   \lumpedmassmatrix\ddt{\lowordersolution}
    + \loworderssmatrix(\timevalue)\lowordersolution(\timevalue) 
    = \ssrhs(\timevalue)
\end{equation}
\begin{center}{\textbf{Explicit Euler scheme}:}\end{center}
\begin{equation}\label{eq:low_explicit_euler}
  \lumpedmassmatrix\frac{\lowordersolution-\solutionvector^{\timeindex}}
  {\timestepsize}
  + \loworderssmatrix[\timeindex]\solutionvector^{\timeindex}
  = \ssrhs^\timeindex
\end{equation}
\begin{center}{\textbf{Theta scheme}:}\end{center}
\begin{equation}\label{eq:low_theta}
  \lumpedmassmatrix\frac{\lowordersolution-\solutionvector^\timeindex}
  {\timestepsize}
  + (1-\theta)\loworderssmatrix[\timeindex]\solutionvector^\timeindex
  + \theta\loworderssmatrix[\timeindex+1]\lowordersolution
  = (1-\theta)\ssrhs^\timeindex + \theta\ssrhs^{\timeindex+1}
\end{equation}
Section \ref{sec:DMP} will prove that each of the fully discrete schemes
above satisfy a local discrete maximum principle.
