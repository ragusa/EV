In this section, it will be shown that the low-order scheme for each temporal
discretization preserves non-negativity of the solution, given that a
CFL-like time step size condition is satisfied. This section
builds upon the results of Section \ref{sec:m_matrix}, which proved that the
low-order system matrix $\loworderssmatrix[n]$ is an M-matrix, which to
recall Equation \eqref{eq:m_matrix}, has the property
\[
  \mathbf{A}\mathbf{x} \geq \mathbf{0} \Rightarrow \mathbf{x} \geq \mathbf{0} \eqc
\]
and thus for a linear system $\mathbf{A}\mathbf{x} = \mathbf{b}$,
proof of non-negativity of the right-hand-side vector proves non-negativity
of the solution $\mathbf{x}$. For each temporal discretization, it will be
shown that the system matrix inverted for the corresponding low-order system
is also an M-matrix and that the right-hand-side vector for each system is
non-negative. Thus positivity-preservation of the solution will be proven.
%--------------------------------------------------------------------------------
\begin{theorem}{Non-Negativity of the Steady-State Low-Order Solution}
  The solution of the steady-state low-order system given by Equation
  \eqref{eq:low_ss} is non-negative:
  \[
    \solutionletter^L_i \geq 0 \eqc \quad \forall i\eqp
  \]
\end{theorem}

\begin{proof}
By Theorem \ref{thm:m_matrix}, the system matrix $\loworderssmatrix$ is an
M-matrix, and by assumption in Section \ref{sec:scalar}, the source $\scalarsource$
is non-negative, and thus the steady-state right-hand-side vector entries
$\ssrhsletter_i$ are non-negative. Invoking the M-matrix property
concludes the proof.\qed
\end{proof}
%--------------------------------------------------------------------------------
\begin{theorem}{Non-Negativity Preservation of the Explicit Euler Low-Order Solution}
  If the old solution $\solutionvector^n$ is non-negative and
  the time step size $\dt$ satisfies
\begin{equation}\label{eq:explicit_cfl}
  \timestepsize \leq \frac{\massmatrixletter_{i,i}^{L}}
    {\ssmatrixletter_{i,i}^{L,\timeindex}}
  \eqc\quad\forall i \eqc
\end{equation}
  then the new solution $\solutionvector^{L,n+1}$ of the explicit Euler low-order
  system given by Equation \eqref{eq:low_explicit_euler} is non-negative:
  \[
    \solutionletter^{L,n+1}_i \geq 0 \eqc \quad \forall i\eqp
  \]
\end{theorem}

\begin{proof}
Rearranging Equation \eqref{eq:low_explicit_euler},
\[
  \lumpedmassmatrix\solutionvector^{L,n+1}
    = \dt\ssrhs^n
      + \lumpedmassmatrix\solutionvector^{n}
      + \dt\loworderssmatrix[n]\solutionvector^{n}
  \eqp
\]
Thus the system matrix to invert is the lumped mass matrix, which is
an M-matrix since it is diagonal and positive. The right-hand-side
vector $\mathbf{y}$ of this system has the entries
\[
  y_i
    = \dt\ssrhsletter^n_i
      + \pr{\lumpedmassentry - \dt\ssmatrixletter^{L,n}_{i,i}}
        \solutionletter^n_i
      - \dt\sumjnoti\ssmatrixletter^{L,n}\solutionletter^n_j
  \eqp
\]
It now just remains to prove that these entries are non-negative.
As stated previously, the source function $\scalarsource$ is assumed
to be non-negative and thus the steady-state right-hand-side
vector is non-negative. Due to the time step size assumption
given by Equation \eqref{eq:explicit_cfl} and Lemma \ref{lem:diagonalpositive},
\[
  \lumpedmassentry - \dt\ssmatrixletter^{L,n}_{i,i} \geq 0 \eqc
\]
and by Lemma \ref{lem:offdiagonalnegative}, the off-diagonal
sum term is also non-negative. Thus $y_i$ is a sum of non-negative
terms. Invoking the M-matrix property concludes the proof.\qed
\end{proof}
%--------------------------------------------------------------------------------
\begin{theorem}{Non-Negativity Preservation of the Theta Low-Order Solution}
  If the old solution $\solutionvector^n$ is non-negative and
  the time step size $\dt$ satisfies
\begin{equation}\label{eq:theta_cfl}
   \timestepsize \leq \frac{\massmatrixletter^L_{i,i}}{(1-\theta)
     \ssmatrixletter_{i,i}^{L,\timeindex}}
   \eqc\quad\forall i \eqc
\end{equation}
  then the new solution $\solutionvector^{L,n+1}$ of the Theta low-order
  system given by Equation \eqref{eq:low_theta} is non-negative:
  \[
    \solutionletter^{L,n+1}_i \geq 0 \eqc \quad \forall i\eqp
  \]
\end{theorem}

\begin{proof}
Rearranging Equation \eqref{eq:low_theta},
\[
  \pr{\lumpedmassmatrix+\theta\dt\loworderssmatrix[n+1]}\solutionvector^{L,n+1}
    = \dt\pr{(1-\theta)\ssrhs^n + \theta\ssrhs^{n+1}}
      + \lumpedmassmatrix\solutionvector^{n}
      - (1-\theta)\dt\loworderssmatrix[n]\solutionvector^n
  \eqp
\]
Thus the system matrix to invert is
$\lumpedmassmatrix+\dt\theta\loworderssmatrix[n+1]$, which is
an M-matrix since it is a linear combination of two M-matrices.
The right-hand-side vector $\mathbf{y}$ of this system has the entries
\[
  y_i
    = \dt\pr{(1-\theta)\ssrhsletter^n_i + \theta\ssrhsletter^{n+1}_i}
      + \pr{\lumpedmassentry - (1-\theta)\dt\ssmatrixletter^{L,n}_{i,i}}
        \solutionletter^n_i
      - (1-\theta)\dt\sumjnoti\ssmatrixletter^{L,n}\solutionletter^n_j
  \eqp
\]
It now just remains to prove that these entries are non-negative.
As stated previously, the source function $\scalarsource$ is assumed
to be non-negative and thus the steady-state right-hand-side
vector is non-negative. Due to the time step size assumption
given by Equation \eqref{eq:theta_cfl} and Lemma \ref{lem:diagonalpositive},
\[
  \lumpedmassentry - (1-\theta)\dt\ssmatrixletter^{L,n}_{i,i} \geq 0 \eqc
\]
and by Lemma \ref{lem:offdiagonalnegative}, the off-diagonal
sum term is also non-negative. Thus $y_i$ is a sum of non-negative
terms. Invoking the M-matrix property concludes the proof.\qed
\end{proof}
%--------------------------------------------------------------------------------
