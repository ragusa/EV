In this section, definitions of a graph-theoretic local viscous bilinear form
and a low-order viscosity from Guermond \cite{guermond_firstorder} are given.
First, some preliminary definitions are given.

Let $\indicescell$ denote the set of degree of freedom indices associated with
cell $\cell$, which is defined to be those degrees of freedom $j$ for which the
corresponding test function $\testfunction_j$ has nonzero support on cell
$\cell$:
\begin{equation}
  \indicescell \equiv \{j\in\{1,\ldots,N\}: |\support_j\cap \celldomain|\ne 0\}
  \eqc
\end{equation}
where $\celldomain[\cell]$ is the domain of cell $\cell$. An illustration of
this definition is given in Figure \ref{fig:cell_indices}.
Let $\cardinality$ denote the number of elements in the set $\indicescell$;
for example, in Figure \ref{fig:cell_indices}, $\cardinality=3$.
%-------------------------------------------------------------------------------
\begin{figure}[ht]
   \centering
     \begin{tikzpicture}[
  scale=1]

\def\pointsize{2pt}

\coordinate (p1) at (1,1);
\coordinate (p2) at (2.1,0.4);
\coordinate (p3) at (3.3,0.4);
\coordinate (p4) at (4.3,0.9);
\coordinate (p5) at (5.2,1.3);
\coordinate (p6) at (5.8,2.1);
\coordinate (p7) at (5.8,3.5);
\coordinate (p8) at (5.2,4);
\coordinate (p9) at (4.5,4.8);
\coordinate (p10) at (3.6,5);
\coordinate (p11) at (2.7,4.5);
\coordinate (p12) at (1.6,4.6);
\coordinate (p13) at (0.8,3.5);
\coordinate (p14) at (1.2,2.5);
\coordinate (p15) at (2,1.5);
\coordinate (p16) at (3.3,1.2);
\coordinate (p17) at (4,1.9);
\coordinate (p18) at (5,2.2);
\coordinate (p19) at (5,3.3);
\coordinate (p20) at (4.2,3.8);
\coordinate (p21) at (3.3,3.6);
\coordinate (p22) at (2,3.7);
\coordinate (i) at (3,2.5);
\coordinate (j) at (4,3);

%\fill (p1) circle (\pointsize);
%\fill (p2) circle (\pointsize);
%\fill (p3) circle (\pointsize);
%\fill (p4) circle (\pointsize);
%\fill (p5) circle (\pointsize);
%\fill (p6) circle (\pointsize);
%\fill (p7) circle (\pointsize);
%\fill (p8) circle (\pointsize);
%\fill (p9) circle (\pointsize);
%\fill (p10) circle (\pointsize);
%\fill (p11) circle (\pointsize);
%\fill (p12) circle (\pointsize);
%\fill (p13) circle (\pointsize);
%\fill (p14) circle (\pointsize);
%\fill (p15) circle (\pointsize);
%\fill (p16) circle (\pointsize);
%\fill (p17) circle (\pointsize);
%\fill (p18) circle (\pointsize);
%\fill (p19) circle (\pointsize);
%\fill (p20) circle (\pointsize);
%\fill (p21) circle (\pointsize);
%\fill (p22) circle (\pointsize);
%\fill (i) circle (\pointsize);
%\fill (j) circle (\pointsize);

\draw (p1) -- (p2);
\draw (p1) -- (p14);
\draw (p1) -- (p15);
\draw (p2) -- (p3);
\draw (p2) -- (p15);
\draw (p2) -- (p16);
\draw (p3) -- (p4);
\draw (p3) -- (p16);
\draw (p4) -- (p5);
\draw (p4) -- (p16);
\draw (p4) -- (p17);
\draw (p4) -- (p18);
\draw (p5) -- (p6);
\draw (p5) -- (p18);
\draw (p6) -- (p7);
\draw (p6) -- (p18);
\draw (p6) -- (p19);
\draw (p7) -- (p8);
\draw (p7) -- (p19);
\draw (p8) -- (p9);
\draw (p8) -- (p19);
\draw (p8) -- (p20);
\draw (p9) -- (p10);
\draw (p9) -- (p20);
\draw (p10) -- (p11);
\draw (p10) -- (p20);
\draw (p10) -- (p21);
\draw (p11) -- (p12);
\draw (p11) -- (p21);
\draw (p11) -- (p22);
\draw (p12) -- (p13);
\draw (p12) -- (p22);
\draw (p13) -- (p14);
\draw (p13) -- (p22);
\draw (p14) -- (p15);
\draw (p14) -- (p22);
\draw (p15) -- (i);
\draw (p15) -- (p16);
\draw (p15) -- (p22);
\draw (p16) -- (i);
\draw (p16) -- (p17);
\draw (p17) -- (i);
\draw (p17) -- (j);
\draw (p17) -- (p18);
\draw (p18) -- (j);
\draw (p18) -- (p19);
\draw (p19) -- (j);
\draw (p19) -- (p20);
\draw (p20) -- (j);
\draw (p20) -- (p21);
\draw (p21) -- (i);
\draw (p21) -- (j);
\draw (p21) -- (p22);
\draw (p22) -- (i);
\draw (i) -- (j);



\draw[draw=none, fill=red, fill opacity=0.5] (p15)--(p16)--(p17)--(j)
  --(p21)--(p22)--cycle;
\draw[draw=none, fill=blue, fill opacity=0.5] (p17)--(p18)--(p19)--(p20)
  --(p21)--(i)--cycle;
\draw[draw=none, fill=yellow, fill opacity=0.5] (p16)--(i)--(j)--(p18)
  --(p4)--cycle;

\node[black] at ($0.333*(i) + 0.333*(j) + 0.333*(p17)$) {\Large $K$};

\fill[black] (i) circle (\pointsize);
\fill[black] (j) circle (\pointsize);
\fill[black] (p17) circle (\pointsize);

\end{tikzpicture}

      \caption{Illustration of Cell Degree of Freedom Indices $\indicescell$}
   \label{fig:cell_indices}
\end{figure}
%-------------------------------------------------------------------------------
Let $\cellindices(\support\ij)$ denote the set of cell indices corresponding
to cells that lie in the shared support $\support\ij$:
\begin{equation}
  \cellindices(\support\ij) \equiv \{\cell: \celldomain \subset \support\ij\}
  \eqp
\end{equation}
For example, in Figure \ref{fig:shared_support}, $\cellindices(\support\ij)$
would consist of the indices of the two cells in $\support\ij$.

The following graph-theoretic local viscous bilinear form from
\cite{guermond_firstorder} is employed in computation of the artificial
diffusion terms, which are expressed in matrix form in Section
\ref{sec:low_order_scheme_scalar}:
%--------------------------------------------------------------------------------
\begin{definition}{Local Viscous Bilinear Form}
   The local viscous bilinear form for cell $\cell$ is defined as follows:
   \begin{equation}\label{eq:bilinearform}
     \localviscbilinearform{\cell}{j}{i} \equiv \left\{\begin{array}{l l}
       -\frac{1}{\cardinality - 1}\cellvolume\eqc & i\ne j\eqc
       \quad i,j\in \indicescell\eqc \\
       \cellvolume\eqc & i = j \eqc \quad i,j\in \indicescell\eqc \\
       0          \eqc & \mbox{otherwise}\eqc
     \end{array}\right. \eqc
   \end{equation}
   with $\cellvolume$ defined as the volume of cell $\cell$.
\end{definition}
%--------------------------------------------------------------------------------
Note some properties of this definition: the diagonal entries
$\localviscbilinearform{\cell}{i}{i}$ are positive, the off-diagonal entries
are negative, and the row-sum $\sum_j\localviscbilinearform{\cell}{i}{j}$
is zero. The signs of the entries are important in Section \ref{sec:m_matrix},
where this knowledge is invoked in the proof of inverse-positivity of the
system matrix. The zero row-sum is important in proving that the method
is conservative, and it is also used when defining antidiffusive fluxes
in the FCT scheme in Section \ref{sec:fct_scheme_scalar}; specifically, it
allows the antidiffusive source for a node $i$ to be decomposed into
skew-symmetric antidiffusive fluxes between adjacent nodes.

The definition of the low-order viscosity, also taken from
\cite{guermond_firstorder}, follows. The resulting piecewise
viscosity is constant over each cell. This definition is designed to
introduce the smallest amount of artificial diffusion possible such that
the inverse-positivity of the system matrix can be guaranteed;
specifically, this definition allows Lemma \ref{lem:offdiagonalnegative}
in Section \ref{sec:m_matrix} to be proven.
%--------------------------------------------------------------------------------
\begin{definition}{Low-Order Viscosity}
   The low-order viscosity for cell $\cell$ is defined as follows:
   \begin{equation}
     \lowordercellviscosity[\timeindex] \equiv \max\limits_{i\ne j\in\indicescell}
     \frac{\max(0,\ssmatrixletter\ij^\timeindex)}
     {-\sumKSij[T]\localviscbilinearform{T}{j}{i}}
     \eqc
   \end{equation}
   where $\ssmatrixletter\ij^\timeindex$ is the $i,j$th entry of the Galerkin
   steady-state matrix given by Equation \eqref{eq:Aij}.
\end{definition}
%--------------------------------------------------------------------------------
