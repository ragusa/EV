%================================================================================
\section{Steady-State Radiation Transport}
%================================================================================
%--------------------------------------------------------------------------------
\subsection{Deriving a Maximum Principle}
%--------------------------------------------------------------------------------
A maximum principle is satisfied by adding a viscous bilinear form with a local
viscosity coefficient given in the following definition.
\begin{definition}
The low-order viscosity for the radiative transfer equation on cell $K$ is
defined as follows:
\begin{equation}
	\nu_K^L = \max\limits_{i\ne j\in \mathcal{I}(K)}\frac{\left|\int\limits_{S_{ij}}
      \left(\mathbf{\Omega}\cdot\nabla\varphi_j +
		\sigma_t\varphi_j\right)\varphi_i d\mathbf{x}\right|}
		{-\sum\limits_{T\subset S_{ij}} b_T(\varphi_j, \varphi_i)},
\end{equation}
where $S_{ij}$ is the dual-support of degrees of freedom $i$ and $j$,
$\mathcal{I}(K)$ is the set of indices corresponding to degrees of freedom in
the support of cell $K$, and $b_K(\varphi_j, \varphi_i)$ is a local bilinear
form defined as follows:
\begin{equation}
   b_K(\varphi_j, \varphi_i) = \left\{\begin{array}{l l}
      -\frac{1}{n_K - 1}|K| & i\ne j, \quad i,j\in \mathcal{I}(K),\\
      |K|                   & i = j,  \quad i,j\in \mathcal{I}(K),\\
      0                     & i\notin\mathcal{I}(K) \quad | \quad j\notin\mathcal{I}(K),
   \end{array}\right.
\end{equation}
where $n_K = \mbox{card}(\mathcal{I}(K))$.
\end{definition}

Applying the Galerkin finite element method to Equation \ref{ss} and adding a
viscous bilinear form gives a linear system $\mathbf{A^L} \mathbf{U} = \mathbf{b}$.
The superscript $L$ denotes the low-order matrix (use of the low-order viscosity). 
The $i$-th equation is the following:
\begin{equation}
	\sum\limits_{K\subset S_i}\left[\int\limits_K\left(\mathbf{\Omega}\cdot\nabla\psi
      + \sigma_t\psi\right)\varphi_i d\mathbf{x} + \nu_K^L b_K(\psi, \varphi_i)\right]
      = \sum\limits_{K\subset S_i}\int\limits_K q \varphi_i d\mathbf{x} = b_i.
\end{equation}

\begin{lemma}
The coefficients of the resulting linear system are given by the following equation:
\begin{equation}\label{Aij}
	A^L_{i,j} = \int\limits_{S_{ij}}\left(\mathbf{\Omega}\cdot\nabla\varphi_j +
		\sigma_t\varphi_j\right)\varphi_i d\mathbf{x} +
		\sum\limits_{K\subset S_{ij}}\nu_K^L b_K(\varphi_j, \varphi_i).
\end{equation}
\end{lemma}

\begin{proof}
Expanding the solution gives
\[
	\sum\limits_{K\subset S_i}\left[\int\limits_K\left(\mathbf{\Omega}\cdot
      \nabla\sum\limits_{j\in \mathcal{I}(K)}U_j\varphi_j +
		\sigma_t\sum\limits_{j\in \mathcal{I}(K)}U_j\varphi_j\right)\varphi_i d\mathbf{x} +
		\nu_K^L \sum\limits_{j\in \mathcal{I}(K)}U_j b_K(\varphi_j, \varphi_i)\right] = b_i,
\]
and rearranging gives
\[
	\sum\limits_{j\in \mathcal{I}(S_{i})}U_j\sum\limits_{K\subset S_{ij}}
      \left(\int\limits_K\left(\mathbf{\Omega}\cdot\nabla\varphi_j +
		\sigma_t\varphi_j\right)\varphi_i d\mathbf{x} +
		\nu_K^L b_K(\varphi_j, \varphi_i)\right) = b_i
\]
Thus the elements of the linear system matrix are:
\begin{eqnarray*}
	A^L_{i,j} & = & \sum\limits_{K\subset S_{ij}}\left(\int\limits_K
      \left(\mathbf{\Omega}\cdot\nabla\varphi_j +
		\sigma_t\varphi_j\right)\varphi_i d\mathbf{x} +
		\nu_K^L b_K(\varphi_j, \varphi_i)\right)\\
  & = & \int\limits_{S_{ij}}\left(\mathbf{\Omega}\cdot\nabla\varphi_j +
		\sigma_t\varphi_j\right)\varphi_i d\mathbf{x} +
		\sum\limits_{K\subset S_{ij}}\nu_K^L b_K(\varphi_j, \varphi_i).\qed
\end{eqnarray*}
\end{proof}

\newpage
\begin{lemma}\label{offdiagonalnegative}
   The off-diagonal elements of the linear system matrix are non-positive:
   $A^L_{i,j}\le 0, j\ne i$.
\end{lemma}
\begin{proof}
This proof begins by bounding the term
$\sum\limits_{K\subset S_{ij}}\nu_K^L b_K(\varphi_j, \varphi_i)$:
\begin{eqnarray*}
   \sum\limits_{K\subset S_{ij}}\nu_K^L b_K(\varphi_j, \varphi_i)
   & = & \sum\limits_{K\subset S_{ij}} \max\limits_{i\ne j\in \mathcal{I}(K)}
      \frac{\left|\int\limits_{S_{ij}}\left(\mathbf{\Omega}\cdot\nabla\varphi_j +
		\sigma_t\varphi_j\right)\varphi_i d\mathbf{x}\right|}
		{-\sum\limits_{T\subset S_{ij}} b_T(\varphi_j, \varphi_i)}b_K(\varphi_j,\varphi_i)\\
   & \le & \sum\limits_{K\subset S_{ij}} \frac{\left|\int\limits_{S_{ij}}
      \left(\mathbf{\Omega}\cdot\nabla\varphi_j +
		\sigma_t\varphi_j\right)\varphi_i d\mathbf{x}\right|}
		{-\sum\limits_{T\subset S_{ij}} b_T(\varphi_j, \varphi_i)}b_K(\varphi_j,\varphi_i)\\
   & \le & -\left|\int\limits_{S_{ij}}\left(\mathbf{\Omega}\cdot\nabla\varphi_j +
		\sigma_t\varphi_j\right)\varphi_i d\mathbf{x}\right|
\end{eqnarray*}
Now, recalling Equation \ref{Aij},
\begin{eqnarray*}
	A^L_{i,j} & = & \int\limits_{S_{ij}}\left(\mathbf{\Omega}\cdot\nabla\varphi_j +
		\sigma_t\varphi_j\right)\varphi_i d\mathbf{x} +
		\sum\limits_{K\subset S_{ij}}\nu_K^L b_K(\varphi_j, \varphi_i)\\
      & \le & \int\limits_{S_{ij}}\left(\mathbf{\Omega}\cdot\nabla\varphi_j +
		\sigma_t\varphi_j\right)\varphi_i d\mathbf{x} -\left|\int\limits_{S_{ij}}
      \left(\mathbf{\Omega}\cdot\nabla\varphi_j +
		\sigma_t\varphi_j\right)\varphi_i d\mathbf{x}\right|\\
      & \le & 0.\qed
\end{eqnarray*}
\end{proof}

\begin{lemma}\label{diagonalpositive}
   The diagonal elements  of the linear system matrix are non-negative: $A^L_{i,i}\ge 0$.
\end{lemma}
\begin{proof}
\[
	A^L_{i,i} = \int\limits_{S_{i}}\left(\nabla\cdot\frac{\mathbf{\Omega}\varphi_i^2}{2} +
		\sigma_t\varphi_i^2\right) d\mathbf{x} +
		\sum\limits_{K\subset S_{i}}\nu_K^L b_K(\varphi_i, \varphi_i).
\]
To prove that $A^L_{i,i}$ is non-negative, it is sufficient to prove that
each term in the above expression is non-negative. The non-negativity of
the interaction term and viscous term are obvious ($\sigma_t \ge 0, 
\, \nu_K^L\ge 0, \, b_K(\varphi_i, \varphi_i)>0$), but
the non-negativity of the divergence term is not necessarily obvious. On the interior of
the domain, the divergence term gives zero contribution because the divergence integral may
be transformed into a surface integral $\int\limits_{\partial S_{i}}
\mathbf{\Omega}\cdot\mathbf{n}\frac{\varphi_i^2}{2} d\mathbf{x}$
via the divergence theorem; one can then recognize that
the basis function $\varphi_i$ evaluates to zero on the boundary of its support $S_{i}$.\\
On the outflow boundary of the domain, the term $\mathbf{\Omega}\cdot\mathbf{n}
\frac{\varphi_i^2}{2}$ is positive because the $\mathbf{\Omega}\cdot\mathbf{n} >0$
for an outflow boundary. This quantity is of course negative for the inflow boundary,
but a Dirichlet boundary condition is strongly imposed on the incoming boundary, so
for degrees of freedom $i$ on the incoming boundary, $A^L_{i,i}$ will be set equal
to some positive value such as 1 with a corresponding incoming value
accounted for in the right hand side $\mathbf{b}$ of the linear system.\qed
\end{proof}

\newpage
\begin{lemma}
   The sum of all elements in a row $i$ is non-negative: $\sum\limits_j A^L_{i,j} \ge 0$.
\end{lemma}

\begin{proof}
Using the fact that $\sum\limits_j\varphi_j=1$,
\begin{eqnarray*}
	\sum\limits_j A^L_{i,j} & = & \sum\limits_j\int\limits_{S_{ij}}
      \left(\mathbf{\Omega}\cdot\nabla\varphi_j +
		\sigma_t\varphi_j\right)\varphi_i d\mathbf{x} +
		\sum\limits_j\sum\limits_{K\subset S_{ij}}\nu_K b_K(\varphi_j, \varphi_i)\\
		& = & \int\limits_{S_{i}}\left(\mathbf{\Omega}\cdot\nabla\sum\limits_j\varphi_j +
		\sigma_t\sum\limits_j\varphi_j\right)\varphi_i d\mathbf{x}\\
		& = & \int\limits_{S_{i}}\sigma_t\varphi_i d\mathbf{x}\\
		&\ge& 0.\qed
\end{eqnarray*}
\end{proof}

\begin{lemma}\label{diagonallydominant}
   $\mathbf{A^L}$ is strictly diagonally dominant:
   $\left|A^L_{i,i}\right| \ge \sum\limits_{j\ne i} \left|A^L_{i,j}\right|$.
\end{lemma}
\begin{proof}
Using the inequalities $\sum\limits_j A^L_{i,j} \ge 0$ and $A^L_{i,j}\le 0, j\ne i$,
it is proven that $\mathbf{A^L}$ is strictly diagonally dominant:
\begin{eqnarray*}
	\sum\limits_j A^L_{i,j} & \ge & 0\\
	\sum\limits_{j\ne i} A^L_{i,j} + A^L_{i,i} & \ge & 0\\
	\left|A^L_{i,i}\right| & \ge & \sum\limits_{j\ne i} -A^L_{i,j}\\
	\left|A^L_{i,i}\right| & \ge & \sum\limits_{j\ne i} \left|A^L_{i,j}\right|.\qed
\end{eqnarray*}
\end{proof}

\begin{lemma}
   $\mathbf{A^L}$ is an M-Matrix.
\end{lemma}
\begin{proof}
To prove that a matrix is an M-Matrix, it is sufficient to prove that:
\[
\left\{\begin{array}{l}
A^L_{i,j}\le 0, j\ne i\\
A^L_{i,i}\ge 0\\
\left|A^L_{i,i}\right| \ge \sum\limits_{j\ne i} \left|A^L_{i,j}\right|\\
\end{array}
\right.,
\]
which are given by Lemmas \ref{offdiagonalnegative}, \ref{diagonalpositive}, and
\ref{diagonallydominant}, respectively.\qed
\end{proof}

\begin{theorem}
The solution $\mathbf{\psi}$ is non-negative and satisfies the following maximum principle:
\begin{equation}\label{ss_max_principle}
   \left(1 - \frac{1}{A^L_{i,i}}\sum\limits_j A^L_{i,j}\right)U_{min,i}
      + \frac{b_i}{A^L_{i,i}}\le U_i
   \le \left(1 - \frac{1}{A^L_{i,i}}\sum\limits_j A^L_{i,j}\right)U_{max,i}
      + \frac{b_i}{A^L_{i,i}},
\end{equation}
where $U_{min,i} = \min\limits_{j\in \mathcal{I}(S_i)}U_j$, $U_{max,i}
= \max\limits_{j\in \mathcal{I}(S_i)}U_j$,
and $\mathcal{I}(S_i)$ is the set of indices of degrees of freedom in the support
of degree of freedom $i$.
\end{theorem}
\begin{proof}
\begin{eqnarray*}
	\sum\limits_j A^L_{i,j}U_j & = & b_i\\
	A^L_{i,i}U_i & = & \sum\limits_{j\ne i} -A^L_{i,j}U_j + b_i\\
	A^L_{i,i}U_i & \le & \left(\sum\limits_{j\ne i} -A^L_{i,j}\right)U_{max,i} + b_i\\
	A^L_{i,i}(U_i - U_{max,i}) & \le & \left(-\sum\limits_j A^L_{i,j}\right)U_{max,i}
      + b_i\\
	U_i - U_{max,i} & \le & \left(-\frac{1}{A^L_{i,i}}\sum\limits_j A^L_{i,j}\right)
      U_{max,i} + \frac{b_i}{A^L_{i,i}}\\
	U_i & \le & \left(1 - \frac{1}{A^L_{i,i}}\sum\limits_j A^L_{i,j}\right)U_{max,i}
      + \frac{b_i}{A^L_{i,i}}
\end{eqnarray*}
A similar analysis is performed to prove the lower bound for $U_i$.\qed
\end{proof}