%================================================================================
\subsection{FCT}
%================================================================================
%================================================================================
\subsubsection{The Implicit System}
%================================================================================
Without strongly imposing Dirichlet boundary conditions, the high-order system is
\begin{equation}\label{im_high}
   \mathbf{A}^{tr,H}\mathbf{U}^H
      = \mathbf{M}_C\mathbf{U}^n + \Delta t\mathbf{b} \equiv \mathbf{b}^H,
\end{equation}
where $\mathbf{A}^{tr,H} \equiv \mathbf{M}_C-\Delta t\mathbf{K}$.
Strongly imposing Dirichlet boundary conditions on Equation \ref{im_high}
gives
\begin{equation}\label{im_highD}
   \tilde{\mathbf{A}}^{tr,H}\mathbf{U}^H
      = \tilde{\mathbf{b}}^H.
\end{equation}
The low-order system corresponding to Equation \ref{im_high} is
\begin{equation}\label{im_low}
   (\mathbf{M}_L-\Delta t(\mathbf{K}+\mathbf{D}))\mathbf{U}^L
      = \mathbf{M}_L\mathbf{U}^n + \Delta t\mathbf{b}.
\end{equation}
Note that the solution to this low-order system
is not computed in this Dr. Kuzmin's flux correction algorithm given in Section
\ref{im_kuzmin}; instead, this low-order system is just used
to define the correction step.
%--------------------------------------------------------------------------------
\subsubsection{Kuzmin's Implicit Flux Correction Algorithm}\label{im_kuzmin}
%--------------------------------------------------------------------------------
The flux correction vector $\mathbf{f}$ is defined such that:
\begin{equation}\label{im_kuzminfc}
   \mathbf{A}^{tr,F}\mathbf{U}^H
      = \mathbf{M}_L\mathbf{U}^n + \Delta t\mathbf{b} + \mathbf{f} \equiv \mathbf{b}^F,
\end{equation}
where $\mathbf{A}^{tr,F} \equiv \mathbf{M}_L-\Delta t(\mathbf{K}+\mathbf{D})$.
Subtracting Equation \ref{im_high} from \ref{im_kuzminfc} gives the definition of
$\mathbf{f}$:
\begin{equation}\label{im_kuzminFdef}
   \mathbf{f} \equiv -(\mathbf{M}_C-\mathbf{M}_L)\Delta\mathbf{U}^H
      -\Delta t\mathbf{D}\mathbf{U}^H,
\end{equation}
where $\Delta\mathbf{U}^H = \mathbf{U}^H - \mathbf{U}^n$. Since
$\mathbf{M}_C-\mathbf{M}_L$ and $\mathbf{D}$ are symmetric
and feature zero row and column sums, a valid decomposition for $\mathbf{f}$,
called $\mathbf{F}$, is
\begin{equation}
   F_{i,j} = -m_{i,j}(\Delta U^H_j - \Delta U^H_i) - \Delta t D_{i,j}(U^H_j - U^H_i)
\end{equation}
where $m_{i,j}$ is the $i,j$th element of the consistent mass matrix.
Applying a limiter to Equation \ref{im_kuzminfc} gives
\begin{equation}\label{im_limited}
   \mathbf{A}^{tr,F}\mathbf{U}^{n+1}
      = \mathbf{M}_L\mathbf{U}^n + \Delta t\mathbf{b} + \mathcal{L}[\mathbf{F}],
\end{equation}
where $(\mathcal{L}[\mathbf{F}])_i = \mathcal{L}_{i,:}F_{i,:}^T
= \sum\limits_j \mathcal{L}_{i,j}F_{i,j}$ and $\mathbf{U}^{n+1}$ is the FCT solution.

\begin{lemma}
   Suppose that with strongly imposed Dirichlet boundary conditions, the
   correction step is defined such that the Dirichlet boundary conditions
   are strongly imposed on Equation \ref{im_kuzminfc}:
   \begin{equation}\label{im_kuzminfcD}
      \tilde{\mathbf{A}}^{tr,F}\hat{\mathbf{U}}^H
         = \tilde{\mathbf{b}}^F,
   \end{equation}
   and the definition for
   the flux correction vector $\mathbf{f}$ from
   Equation \ref{im_kuzminFdef} is used without modification. Then the solution
   to this correction step, $\hat{\mathbf{U}}^H$, will yield the
   correct high-order solution $\mathbf{U}^H$ given by Equation \ref{im_highD}.
\end{lemma}
\begin{proof}
   For $i\in\mathcal{D}$, $\hat{U}^H_i = U^H_i = g_i$. For $i\notin\mathcal{D}$,
   we take the $i$th equation of the linear system given by Equation \ref{im_kuzminfcD}:
   \begin{equation}\label{i_eq}
      (\mathbf{A}^{tr,F}\hat{\mathbf{U}}^H)_i
         = m_i U^n_i + \Delta t b_i - (\mathbf{M}_C\Delta\mathbf{U}^H)_i
         + m_i(U^H_i-U^n_i) - \Delta t(\mathbf{D}\mathbf{U}^H)_i.
   \end{equation}
   Examining the $i$th equation of the linear system given by Equation \ref{im_highD}
   and solving for $\Delta t b_i$ gives
   \begin{equation}
      \Delta t b_i = (\mathbf{M}_C\Delta\mathbf{U}^H)_i
         - \Delta t(\mathbf{K}\mathbf{U}^H)_i.
   \end{equation}
   Substituting this back into Equation \ref{i_eq} gives
   \begin{eqnarray}
      (\mathbf{A}^{tr,F}\hat{\mathbf{U}}^H)_i
         & = & m_i U^H_i - \Delta t(\mathbf{K}\mathbf{U}^H)_i -
            \Delta t(\mathbf{D}\mathbf{U}^H)_i.\\
         & = & \left((\mathbf{M}_L - \Delta t(\mathbf{K}+\mathbf{D}))\mathbf{U}^H\right)_i\\
         & = &(\mathbf{A}^{tr,F}\mathbf{U}^H)_i.
   \end{eqnarray}
   The system of equations
   \begin{equation}
      \left\{
         \begin{array}{l l}
            \hat{U}^H_i = U^H_i & i\in\mathcal{D}\\
            (\mathbf{A}^{tr,F}\hat{\mathbf{U}}^H)_i = (\mathbf{A}^{tr,F}\mathbf{U}^H)_i
               & i\notin\mathcal{D}
         \end{array}
      \right.
   \end{equation}
   determines that $\hat{\mathbf{U}}^H$ is uniquely equal to $\mathbf{U}^H$.
   \qed
\end{proof}
%================================================================================
\subsubsection{Deriving the Limiting Coefficients}
%================================================================================
\begin{lemma}
   The definitions given by Equations \ref{P_defs}, \ref{R_defs}, and \ref{L_defs},
   and the following definitions yield maximum-principle preserving limiting coefficients,
   where $W_i^\pm$ are defined to be to be upper and lower bounds of the maximum
   principle given by Equation \ref{implicit_max_principle}:
   \begin{equation}
      W_i^+ \equiv \frac{1}{1+\frac{\Delta t}{m_i}A_{i,i}}\left(U_i^n
         - \left(\frac{\Delta t}{m_i}\sum\limits_{j\ne i} A^L_{i,j}\right)U_{max,i}^{L,n+1}
         + \frac{\Delta t}{m_i}b_i\right),
   \end{equation}
   \begin{equation}
      W_i^- \equiv \frac{1}{1+\frac{\Delta t}{m_i}A_{i,i}}\left(U_i^n
         - \left(\frac{\Delta t}{m_i}\sum\limits_{j\ne i} A^L_{i,j}\right)U_{min,i}^{L,n+1}
         + \frac{\Delta t}{m_i}b_i\right),
   \end{equation}
   \begin{equation}
      Q_i^\pm \equiv (m_i+\Delta t A_{i,i}^L)W_i^\pm + \Delta t
         \sum\limits_{j\ne i} A_{i,j}^L U_j^{n+1} - m_i U_i^n
         - \Delta t b_i,
   \end{equation}
\end{lemma}

\begin{proof}
   The properties given in the proof for Lemma \ref{ss_coef} for $P_i^\pm$,
   $Q_i^\pm$, $R_i^\pm$, and $\mathcal{L}_{i,j}$ hold with these definitions
   of $W_i^\pm$ and $Q_i^\pm$.
   The proof will be given for the upper bound. 
   As in the proof of Lemma \ref{ss_coef}, the following inequality is found:
   \[
      \sum\limits_j \mathcal{L}_{i,j}F_{i,j} \leq Q_i^+
   \]
   Taking row $i$ of the linear system given by Equation \ref{im_limited} gives
   \begin{eqnarray*}
     (m_i + \Delta t A^L{i,i})U_i^{n+1} + \Delta t\sum\limits_{j\ne i}A^L_{i,j}U_j^{n+1}
         - m_i U_i^n - \Delta t b_i & = & \sum\limits_j \mathcal{L}_{i,j}F_{i,j}\\
         & \leq & Q_i^+\\
         & \leq & (m_i+\Delta t A_{i,i}^L)W_i^+ + \Delta t
         \sum\limits_{j\ne i} A_{i,j}^L U_j^{n+1} - m_i U_i^n
         - \Delta t b_i\\
      U_i^{n+1} & \leq & W_i^+ 
   \end{eqnarray*}
   The lower bound is proved similarly.
   \qed
\end{proof}