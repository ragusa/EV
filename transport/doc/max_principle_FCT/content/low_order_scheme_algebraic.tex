\section{Low-Order Schemes}
\subsection{Algebraic Low-Order Scheme}\label{algebraicloworder}
%================================================================================
In this section, an algebraic approach is taken to develop a low-order scheme,
which is based on a low-order artificial diffusion matrix:
%--------------------------------------------------------------------------------
\begin{definition}{Low-Order Artificial Diffusion Matrix}
   The low-order artificial diffusion matrix $\mathbf{D}$ is
   \begin{equation}
      D_{i,j} \equiv \left\{\begin{array}{c c}
         -\max(0,A_{i,j},A_{j,i}) & j\ne i\\
         -\sum\limits_{j\ne i} D_{i,j} & j = i
      \end{array}\right.
   \end{equation}
\end{definition}
%--------------------------------------------------------------------------------
The low-order system matrix is then defined as the sum of the Galerkin system
matrix $\mathbf{A}$ and the low-order artificial diffusion matrix $\mathbf{D}$:
%--------------------------------------------------------------------------------
\begin{definition}{Low-Order System Matrix}
   The low-order system matrix is
   \begin{equation}\label{systemmatrixdef}
      \mathbf{A}^L \equiv \mathbf{A} + \mathbf{D}.
   \end{equation}
\end{definition}
%--------------------------------------------------------------------------------
The low-order semidiscrete scheme is the following:
\begin{equation}\label{semidiscretelow}
   \mathbf{M}^L\frac{d\mathbf{U}^L}{dt}+\mathbf{A}^L \mathbf{U}^L(t) = \mathbf{b},
\end{equation}
where $\mathbf{M}^L$ is the lumped mass matrix and $\mathbf{U}^L$ is the low-order
solution. With explicit Euler time-stepping, the low-order semidiscrete scheme becomes
\begin{equation}\label{loworderexplicit}
   \mathbf{M}_L\frac{\mathbf{U}^{L,n+1}-\mathbf{U}^n}{\Delta t}
      +\mathbf{A}^L\mathbf{U}^n = \mathbf{b},
\end{equation}
where $\mathbf{U}^{L,n+1}$ is the low-order solution at time $t^{n+1}$.
%--------------------------------------------------------------------------------
\subsubsection{M-Matrix Property of the Steady-State System Matrix}
%================================================================================
In this section, it will be shown that the low-order steady-state
system matrix defined in Equation \eqref{systemmatrixdef} is an M-matrix, which
allows a discrete maximum principle for the low-order solution to be proven in
Section \ref{DMP}.
%--------------------------------------------------------------------------------
\begin{lemma}[label={offdiagonalnegative}]{Non-Positivity of Off-Diagonal Matrix Entries}
   The off-diagonal elements of the low-order system matrix are non-positive:
   $A^L_{i,j}\le 0, j\ne i$.
\end{lemma}
\begin{proof}
From Equation \eqref{systemmatrixdef}
\begin{eqnarray*}
	A^L_{i,j} & = & A_{i,j} + D_{i,j}\\
             & = & A_{i,j} - \max(0,A_{i,j},A_{j,i})
\end{eqnarray*}
The non-positivity of the off-diagonal coefficients is proven by examining the three possible
outputs of the $\max()$ function:

\begin{tabular}{l l}
   case $D_{i,j}=0$:        & $A^L_{i,j} = A_{i,j} \leq 0$ because
      $D_{i,j}=0\geq A_{i,j}$.\\
   case $D_{i,j}=-A_{i,j}$: & $A^L_{i,j} = 0$.\\
   case $D_{i,j}=-A_{j,i}$: & $A^L_{i,j} = A_{i,j} - A_{j,i} \leq 0$ because
      $A_{j,i} \geq A_{i,j}$.\qed
\end{tabular}
\end{proof}
%--------------------------------------------------------------------------------
\begin{lemma}[label={diagonalpositive}]{Non-Negativity of Diagonal Matrix Entries}
   The diagonal elements of the linear system matrix are non-negative: $A^L_{i,i}\ge 0$.
\end{lemma}
\begin{proof}
The diagonal elements  of the low-order system matrix are
\[
	A^L_{i,i} = c\int\limits_{S_{i}}\nabla\cdot
   \frac{\mathbf{\Omega}\varphi_i^2(\mathbf{x})}{2} d\mathbf{x} 
      + c\int\limits_{S_{i}}\sigma(\mathbf{x})\varphi_i(\mathbf{x})d\mathbf{x}
      + D_{i,i}.
\]
To prove that $A^L_{i,i}$ is non-negative, it is sufficient to prove that
each term in the above expression is non-negative. The non-negativity of
the interaction term and viscous term are obvious ($\sigma(\mathbf{x}) \ge 0$,
$D_{i,i}\geq 0$), but
the non-negativity of the divergence term is not necessarily obvious. On the interior of
the domain, the divergence term gives zero contribution because the divergence integral may
be transformed into a surface integral $\int\limits_{\partial S_{i}}
\mathbf{\Omega}\cdot\mathbf{n}\frac{\varphi_i^2}{2} d\mathbf{x}$
via the divergence theorem; one can then recognize that
the basis function $\varphi_i$ evaluates to zero on the boundary of its support $S_{i}$.\\
On the outflow boundary of the domain, the term $\mathbf{\Omega}\cdot\mathbf{n}
\frac{\varphi_i^2}{2}$ is positive because $\mathbf{\Omega}\cdot\mathbf{n} >0$
for an outflow boundary. This quantity is of course negative for the inflow boundary,
but a Dirichlet boundary condition is strongly imposed on the incoming boundary, so
for degrees of freedom $i$ on the incoming boundary, $A^L_{i,i}$ will be set equal
to some positive value such as 1 with a corresponding incoming value
accounted for in the right hand side $\mathbf{b}$ of the linear system.\qed
\end{proof}
%--------------------------------------------------------------------------------
\begin{lemma}{Non-Negativity of Row Sum}
   The sum of all elements in a row $i$ is non-negative: $\sum\limits_j A^L_{i,j} \ge 0$.
\end{lemma}

\begin{proof}
Using the fact that $\sum\limits_j\varphi_j(\mathbf{x})=1$,
\begin{eqnarray*}
	\sum\limits_j A^L_{i,j} & = & \sum\limits_j c\int\limits_{S_{ij}}
      \left(\mathbf{\Omega}\cdot\nabla\varphi_j(\mathbf{x}) +
		\sigma(\mathbf{x})\varphi_j(\mathbf{x})\right)\varphi_i(\mathbf{x}) d\mathbf{x} +
		D_{i,i} + \sum\limits_{j\neq i}D_{i,j}\\
		& = & c\int\limits_{S_{i}}\left(\mathbf{\Omega}\cdot
         \nabla\sum\limits_j\varphi_j(\mathbf{x}) +
         \sigma(\mathbf{x})\sum\limits_j\varphi_j(\mathbf{x})\right)
         \varphi_i(\mathbf{x}) d\mathbf{x}\\
		\label{rowsum} & = & c\int\limits_{S_{i}}\sigma(\mathbf{x})
         \varphi_i(\mathbf{x}) d\mathbf{x}\\
		&\ge& 0.\qed
\end{eqnarray*}
\end{proof}
%--------------------------------------------------------------------------------
\begin{lemma}[label={diagonallydominant}]{Diagonal Dominance}
   $\mathbf{A}^L$ is strictly diagonally dominant:
   $\left|A^L_{i,i}\right| \ge \sum\limits_{j\ne i} \left|A^L_{i,j}\right|$.
\end{lemma}
\begin{proof}
Using the inequalities $\sum\limits_j A^L_{i,j} \ge 0$ and $A^L_{i,j}\le 0, j\ne i$,
it is proven that $\mathbf{A}^L$ is strictly diagonally dominant:
\begin{eqnarray*}
	\sum\limits_j A^L_{i,j} & \ge & 0\\
	\sum\limits_{j\ne i} A^L_{i,j} + A^L_{i,i} & \ge & 0\\
	\left|A^L_{i,i}\right| & \ge & \sum\limits_{j\ne i} -A^L_{i,j}\\
	\left|A^L_{i,i}\right| & \ge & \sum\limits_{j\ne i} \left|A^L_{i,j}\right|.\qed
\end{eqnarray*}
\end{proof}
%--------------------------------------------------------------------------------
\begin{lemma}{M-Matrix}
   $\mathbf{A}^L$ is an M-Matrix.
\end{lemma}
\begin{proof}
To prove that a matrix is an M-Matrix, it is sufficient to prove that:
\[
\left\{\begin{array}{l}
A^L_{i,j}\le 0, j\ne i\\
A^L_{i,i}\ge 0\\
\left|A^L_{i,i}\right| \ge \sum\limits_{j\ne i} \left|A^L_{i,j}\right|\\
\end{array}
\right.,
\]
which are given by Lemmas \ref{offdiagonalnegative}, \ref{diagonalpositive}, and
\ref{diagonallydominant}, respectively.\qed
\end{proof}
%--------------------------------------------------------------------------------