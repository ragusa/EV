%================================================================================
\section{Explicit Transient Radiation Transport}
%================================================================================
%--------------------------------------------------------------------------------
\subsection{Deriving a Maximum Principle}
%--------------------------------------------------------------------------------
Here we seek to find a maximum principle for the transient problem, given by
Equation \ref{tr}.
Using explicit Euler for time stepping, the following equation relates the new and old
time step solutions for the low-order scheme:
\begin{equation}\label{newstep}
   U_i^L = U_i^n - \frac{\Delta t}{m_i}\sum\limits_j U_j^n A^L_{i,j}
      + \frac{\Delta t}{m_i}b_i,
\end{equation}
where $\Delta t$ is the time step size, $m_i$ is the $i$th element of the \emph{lumped} mass
matrix, i.e., $M^L_{i,i}$, $\mathbf{A}^L$ is the steady-state system matrix, and
$\mathbf{b}$ is the right-hand side.

\begin{theorem}
If the CFL-like condition
\begin{equation}\label{ex_CFL}
   \Delta t \leq \frac{m_i}{A_{i,i}^L}\quad\forall i
   \qquad\Longleftrightarrow\qquad
   1 - \frac{\Delta t}{m_i}A_{i,i}^L \geq 0\quad\forall i,
\end{equation}
then the low-order solution at the new time step, $\mathbf{U}^L$ satisfies the following maximum principle:
\begin{equation}\label{explicit_max_principle}
   W_i^-\leq U_i^L\leq W_i^+\quad\forall i,
\end{equation}
where
\begin{equation}\label{ex_bounds}
   W_i^\pm\equiv U_{\substack{\max\\\min},i}^n\left(1-\frac{\Delta t}{m_i}\sum\limits_j A^L_{i,j}\right)
      + \frac{\Delta t}{m_i}b_i,
\end{equation}
where $U_{\substack{\max\\\min},i}^n = \substack{\max\\\min\limits_{j\in \mathcal{I}(S_i)}}U_j^n$,
and $\mathcal{I}(S_i)$ is the set of indices of degrees of freedom in the
support of degree of freedom $i$.
\end{theorem}
\begin{proof}
Rearranging Equation \ref{newstep},
\[
   U_i^L = \left(1-\frac{\Delta t}{m_i}A^L_{i,i}\right)U_i^n - \frac{\Delta t}{m_i}
      \sum\limits_{j\ne i} U_j^n A^L_{i,j} + \frac{\Delta t}{m_i}b_i,
\]
The CFL-like condition in Equation \ref{ex_CFL} gives that $1-\frac{\Delta t}{m_i}A^L_{i,i} \ge 0$, and by
Lemma \ref{offdiagonalnegative}, it is known that the off-diagonal
elements $A^L_{i,j}, j\ne i$, are non-positive. Thus, the following inequality is
able to be applied:
\[
   U_i^L \le
   U_{\max,i}^n\left(1-\frac{\Delta t}{m_i}\sum\limits_j A^L_{i,j}\right)
      + \frac{\Delta t}{m_i}b_i,
\]
and similarly for the lower bound.\qed
\end{proof}