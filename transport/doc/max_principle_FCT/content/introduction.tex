%================================================================================
\section{Introduction}
%================================================================================
The radiation transport equation is the following:
\begin{equation}\label{tr}
	\frac{1}{c}\frac{\partial \psi}{\partial t} + \mathbf{\Omega}\cdot\nabla\psi(\mathbf{x},t)
      + \sigma(\mathbf{x})\psi(\mathbf{x},t) = q(\mathbf{x},t).
\end{equation}
Here $q(\mathbf{x},t)$ represents the total source (extraneous plus scattering).
%If necessary, source iteration can be employed to converge the angular fluxes:
%\begin{equation}
	%(\mathbf{\Omega}\cdot\nabla + \sigma)\psi^{\ell+1} = q^\ell,
%\end{equation}
%
%Note that for implicit time stepping, we solve an equation that is similar
%to the steady-state case:
%\begin{equation}
	%(\mathbf{\Omega}\cdot\nabla + \sigma^\ast)\psi^{n+1} =
      %q^{n+1}+\tfrac{\psi^n}{c\Delta t},
%\end{equation}
%with $\sigma^\ast = \sigma + \tfrac{1}{c\Delta t}$. Source iteration can be applied as well if scattering is present:
%\begin{equation}
	%(\mathbf{\Omega}\cdot\nabla + \sigma_t^\ast)\psi^{n+1,\ell+1} = q^{n+1,\ell}+\tfrac{\psi^n}{c\Delta t}.
%\end{equation}
The problem definition is completed with an incoming flux boundary condition:
\begin{equation}
   \psi(\mathbf{x}) = \psi^{inc}(\mathbf{x})  \quad \forall \mathbf{x}\in \partial V^-,
      \quad \partial V^- = \{\mathbf{x}\in\partial V: \mathbf{\Omega}\cdot\mathbf{n}(\mathbf{x})<0\}.
\end{equation}
For transient problems, the following initial condition applies:
\begin{equation}
   \psi(\mathbf{x},t) = \psi^0(\mathbf{x})  \quad \forall \mathbf{x}\in V.
\end{equation}
Note that
\[
   \sigma(\mathbf{x}) \ge 0
\]
with $\sigma(\mathbf{x})=0$ occurring only in voids.