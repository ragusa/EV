%================================================================================
\section{Introduction}
%================================================================================
The purpose of this document is to present a maximum-principle preserving
finite element method and the corresponding maximum principle for both the
steady-state and transient radiation transport equations,
\begin{equation}\label{ss}
	\mathbf{\Omega}\cdot\nabla\psi + \sigma_t\psi = q,
\end{equation}
and
\begin{equation}\label{tr}
	\frac{1}{c}\frac{\partial \psi}{\partial t} + \mathbf{\Omega}\cdot\nabla\psi
      + \sigma_t\psi = q,
\end{equation}
respectively.

Here $q$ represents the total source (external+scattering). If necessary, source
iteration can be employed to converge the angular intensities:
\begin{equation}
	(\mathbf{\Omega}\cdot\nabla + \sigma_t)\psi^{\ell+1} = q^\ell,
\end{equation}

Note that for implicit time stepping, we solve an equation that is similar
to the steady-state case:
\begin{equation}
	(\mathbf{\Omega}\cdot\nabla + \sigma_t^\ast)\psi^{n+1} =
      q^{n+1}+\tfrac{\psi^n}{c\Delta t},
\end{equation}
with $\sigma_t^\ast = \sigma_t + \tfrac{1}{c\Delta t}$. Source iteration can be applied as well if scattering is present:
\begin{equation}
	(\mathbf{\Omega}\cdot\nabla + \sigma_t^\ast)\psi^{n+1,\ell+1} = q^{n+1,\ell}+\tfrac{\psi^n}{c\Delta t}.
\end{equation}

The problem definition is completed with an incoming flux boundary condition:
\begin{equation}
   \psi(\mathbf{x}) = \psi^{inc}(\mathbf{x})  \quad \forall \mathbf{x}\in \partial V^-,
      \quad \partial V^- = \{\mathbf{x}\in\partial V: \mathbf{\Omega}\cdot\mathbf{n}(\mathbf{x})<0\}.
\end{equation}

For transient problems, the following initial condition applies:
\begin{equation}
   \psi(\mathbf{x},t) = \psi^0(\mathbf{x})  \quad \forall \mathbf{x}\in V.
\end{equation}

Note that
\begin{equation}
\sigma_t \ge 0
\end{equation}
with $\sigma_t=0$ occurring only in voids.