\documentclass{article}
\usepackage{color}
\usepackage{alltt}
\usepackage{amsmath}
\title{Maximum-Principle Preserving Finite Element Method for the Radiation
   Transport Equation}
\author{}
%------------------------------------------------------------

\setlength{\oddsidemargin}{0in}
\setlength{\evensidemargin}{0in}
\setlength{\topmargin}{0in}
\setlength{\headheight}{0in}
\setlength{\headsep}{0in}
\setlength{\textwidth}{6.5in}
\setlength{\textheight}{9in}
\setlength{\footskip}{0.5in}
\setlength{\parskip}{\baselineskip}

% Math theorem commands
%------------------------------------------------------------
\newtheorem{theorem}{Theorem}[section]
\newtheorem{lemma}[theorem]{Lemma}
\newtheorem{proposition}[theorem]{Proposition}
\newtheorem{corollary}[theorem]{Corollary}

\newenvironment{proof}[1][Proof]{\begin{trivlist}
\item[\hskip \labelsep {\bfseries #1}]}{\end{trivlist}}
\newenvironment{definition}[1][Definition]{\begin{trivlist}
\item[\hskip \labelsep {\bfseries #1}]}{\end{trivlist}}
\newenvironment{example}[1][Example]{\begin{trivlist}
\item[\hskip \labelsep {\bfseries #1}]}{\end{trivlist}}
\newenvironment{remark}[1][Remark]{\begin{trivlist}
\item[\hskip \labelsep {\bfseries #1}]}{\end{trivlist}}

\newcommand{\qed}{\nobreak \ifvmode \relax \else
      \ifdim\lastskip<1.5em \hskip-\lastskip
      \hskip1.5em plus0em minus0.5em \fi \nobreak
      \vrule height0.75em width0.5em depth0.25em\fi}
%------------------------------------------------------------
      
\begin{document}
\maketitle
\tableofcontents
The transport equation, also called the Boltzmann equation, describes the
transport of particles or waves through some background media and some
of its applications include nuclear reactors, atmospheric science, radiation
therapy, astrophysics, radiation shielding, and high energy density physics.
In this paper, focus is on solution techniques applicable to the first-order
form of the transport equation, discretized in angle with discrete ordinates,
which gives what is commonly called the $S_N$ equations:
\begin{equation}\label{eq:transport_scalar}
  \frac{1}{v(E)}\ppt{\aflux} + \di\cdot\nabla\aflux\xdet
    + \totalxsec\xet\aflux\xdet = \Qtot\xdet
  \eqc
\end{equation}
where $\Qtot\xdet$ denotes the sum of the extraneous source, prompt and delayed
fission sources, and scattering source:
\begin{multline}
  \Qtot\xdet \equiv \Qext\xdet\\
    + \frac{\chi_\text{p}(E)}{4\pi}\int\limits_0^\infty
      dE'\nu_\text{p}(\x,E',t)\fissionxsec(\x,E',t)\phi(\x,\di,E',t)
    + \sum\limits_{i=1}^{n_\text{d}}\frac{\chi_{\text{d},i}(E)}{4\pi}\lambda_i C_i\xt\\
    + \int\limits_0^\infty dE'\int\limits_{4\pi}d\di'
      \scatteringxsec(\x,E'\rightarrow E,\di'\rightarrow\di,t)\aflux(\x,\di',E',t)
  \eqp
\end{multline}
The $S_N$ equations are an attractive form of the transport equation because
the $S_N$ equations can be decoupled by using iterative techniques for the
scattering source, an approach called source iteration \cite{glasstone}:
\begin{equation}
  \frac{1}{v}\ppt{\aflux^{(\ell)}}
    + \di\cdot\nabla\aflux^{(\ell)}
    + \totalxsec\aflux^{(\ell)} = \Qtot^{(\ell-1)} \eqc
\end{equation}
where $\ell$ is the iteration index. The decoupling of the equations allows
scalar solution techniques to be leveraged.
Traditionally, the preferred spatial discretization method for the $S_N$
equations is the Discontinuous Galerkin finite element method (DGFEM)
\cite{Lesaint1974}\cite{Reed_Hill_1973}. Here, however, the
Continuous Galerkin finite element method (CGFEM) is applied. There
has been some recent work by Guermond and Popov \cite{guermond_ev} on
solution techniques for conservation laws with CGFEM, which addresses some
of the main disadvantages of CGFEM versus DGFEM, including the formation
of spurious oscillations. This work aims to demonstrate a proof of concept
for the application of these solution techniques to the transport equation.
Furthermore, some or all of the methodology explored in this paper may be
later extended to DGFEM as well \cite{zingan_2013}.

One of the main objectives of this paper is to present a method that precludes
the formation of spurious oscillations and the negativities that result from
these oscillations. The occurrence of negativities in the numerical solution of
the transport equation has been a long-standing issue \cite{lanthrop}.
Not only are these negativities physically inaccurate, but they can cause
simulations to terminate prematurely. Many attempts to remedy this
issue rely on ad-hoc fix-ups, such as the set-to-zero fix-up for the
classic diamond difference scheme \cite{lewis}. Recent work by Hamilton
introduced a similar fix-up for the linear discontinuous finite element
method (LDFEM) that conserves local balance and preserves third-order accuracy.
Walters and Wareing developed characteristic methods \cite{walters_NC}, but
Wareing later notes that these characteristic methods are difficult to
implement and offers a nonlinear positive spatial differencing scheme
known as the exponential discontinuous scheme \cite{wareing}.
Maginot has recently developed a consistent set-to-zero (CSZ) LDFEM
method \cite{maginot}, as well as a non-negative method for bilinear
discontinuous FEM \cite{maginot_mc2015}.

Traditional approaches to remedy the spurious oscillation issue included
the flux-corrected transport (FCT) algorithm, introduced in 1973 for finite
difference discretizations
by Boris and Book \cite{borisbook}, which has since been applied to the finite
element method \cite{kuzmin_FCT}. The idea of FCT is to blend a low-order scheme
having desirable properties with a scheme of a higher order of accuracy.

Recent work by Guermond and Popov addresses the issue of spurious oscillations
for general conservation laws by using artificial dissipation based on
local entropy production, a method known as entropy viscosity \cite{guermond_ev}.
The idea of entropy viscosity is to enforce an entropy inequality on the weak solution,
and thus filter out weak solutions containing spurious oscillations. However,
entropy viscosity solutions may still contain spurious
oscillations, albeit smaller in magnitude, and consequently negativities
are not precluded. To circumvent this deficiency, Guermond proposed using
the entropy viscosity method in conjunction with the FCT
algorithm \cite{guermond_secondorder}; the high-order scheme component in FCT,
traditionally the unmodified Galerkin scheme, is replaced with the entropy
viscosity scheme.
For the low-order
scheme, Guermond also introduced
a discrete maximum principle (DMP) preserving (and positivity-preserving)
scheme for scalar
conservation laws \cite{guermond_firstorder}.

This paper presents an FCT scheme that is largely rooted in the work by Guermond
and Popov, but is extended to allow application to the transport equation,
which does not fit the prototype of a conservation law but is instead a
balance law, which includes sinks and sources, namely the reaction term
$\totalxsec\aflux$ and the source term $\Qtot$. The presence
of these terms is also a novelty in the context of the FCT algorithm.
In addition, much of the present work on FCT has been for fully explicit time
discretizations, although there has been some work on implicit time discretizations
as well. Because speeds in radiation transport (such as the speed of light)
are so large, implicit and steady-state time discretization are important
considerations, given the CFL time step size restriction for fully explicit
methods. Thus this paper also considers implicit and steady-state FCT, which
has been implemented before \cite{implicit_FCT}.

This paper is organized as follows. Section \ref{sec:preliminaries} gives
some preliminaries such as the problem formulation and discretization.
Recall that the FCT algorithm uses a low-order scheme and a high-order scheme.
Section \ref{sec:low} presents the low-order scheme, Section \ref{sec:high}
presents the high-order scheme (which is based on entropy viscosity),
and Section \ref{sec:fct} presents the FCT scheme that combines the two. Then, Section
\ref{sec:results} presents results for a number of test problems, and
Section \ref{sec:conclusions} gives conclusions.

%================================================================================
\section{Steady-State Radiation Transport}
%================================================================================
%--------------------------------------------------------------------------------
\subsection{Deriving a Maximum Principle}
%--------------------------------------------------------------------------------
A maximum principle is satisfied by adding a viscous bilinear form with a local
viscosity coefficient given in the following definition.
\begin{definition}
The low-order viscosity for the radiative transfer equation on cell $K$ is
defined as follows:
\begin{equation}
	\nu_K^L = \max\limits_{i\ne j\in \mathcal{I}(K)}\frac{\left|\int\limits_{S_{ij}}
      \left(\mathbf{\Omega}\cdot\nabla\varphi_j +
		\sigma_t\varphi_j\right)\varphi_i d\mathbf{x}\right|}
		{-\sum\limits_{T\subset S_{ij}} b_T(\varphi_j, \varphi_i)},
\end{equation}
where $S_{ij}$ is the dual-support of degrees of freedom $i$ and $j$,
$\mathcal{I}(K)$ is the set of indices corresponding to degrees of freedom in
the support of cell $K$, and $b_K(\varphi_j, \varphi_i)$ is a local bilinear
form defined as follows:
\begin{equation}
   b_K(\varphi_j, \varphi_i) = \left\{\begin{array}{l l}
      -\frac{1}{n_K - 1}|K| & i\ne j, \quad i,j\in \mathcal{I}(K),\\
      |K|                   & i = j,  \quad i,j\in \mathcal{I}(K),\\
      0                     & i\notin\mathcal{I}(K) \quad | \quad j\notin\mathcal{I}(K),
   \end{array}\right.
\end{equation}
where $n_K = \mbox{card}(\mathcal{I}(K))$.
\end{definition}

Applying the Galerkin finite element method to Equation \ref{ss} and adding a
viscous bilinear form gives a linear system $\mathbf{A^L} \mathbf{U} = \mathbf{b}$.
The superscript $L$ denotes the low-order matrix (use of the low-order viscosity). 
The $i$-th equation is the following:
\begin{equation}
	\sum\limits_{K\subset S_i}\left[\int\limits_K\left(\mathbf{\Omega}\cdot\nabla\psi
      + \sigma_t\psi\right)\varphi_i d\mathbf{x} + \nu_K^L b_K(\psi, \varphi_i)\right]
      = \sum\limits_{K\subset S_i}\int\limits_K q \varphi_i d\mathbf{x} = b_i.
\end{equation}

\begin{lemma}
The coefficients of the resulting linear system are given by the following equation:
\begin{equation}\label{Aij}
	A^L_{i,j} = \int\limits_{S_{ij}}\left(\mathbf{\Omega}\cdot\nabla\varphi_j +
		\sigma_t\varphi_j\right)\varphi_i d\mathbf{x} +
		\sum\limits_{K\subset S_{ij}}\nu_K^L b_K(\varphi_j, \varphi_i).
\end{equation}
\end{lemma}

\begin{proof}
Expanding the solution gives
\[
	\sum\limits_{K\subset S_i}\left[\int\limits_K\left(\mathbf{\Omega}\cdot
      \nabla\sum\limits_{j\in \mathcal{I}(K)}U_j\varphi_j +
		\sigma_t\sum\limits_{j\in \mathcal{I}(K)}U_j\varphi_j\right)\varphi_i d\mathbf{x} +
		\nu_K^L \sum\limits_{j\in \mathcal{I}(K)}U_j b_K(\varphi_j, \varphi_i)\right] = b_i,
\]
and rearranging gives
\[
	\sum\limits_{j\in \mathcal{I}(S_{i})}U_j\sum\limits_{K\subset S_{ij}}
      \left(\int\limits_K\left(\mathbf{\Omega}\cdot\nabla\varphi_j +
		\sigma_t\varphi_j\right)\varphi_i d\mathbf{x} +
		\nu_K^L b_K(\varphi_j, \varphi_i)\right) = b_i
\]
Thus the elements of the linear system matrix are:
\begin{eqnarray*}
	A^L_{i,j} & = & \sum\limits_{K\subset S_{ij}}\left(\int\limits_K
      \left(\mathbf{\Omega}\cdot\nabla\varphi_j +
		\sigma_t\varphi_j\right)\varphi_i d\mathbf{x} +
		\nu_K^L b_K(\varphi_j, \varphi_i)\right)\\
  & = & \int\limits_{S_{ij}}\left(\mathbf{\Omega}\cdot\nabla\varphi_j +
		\sigma_t\varphi_j\right)\varphi_i d\mathbf{x} +
		\sum\limits_{K\subset S_{ij}}\nu_K^L b_K(\varphi_j, \varphi_i).\qed
\end{eqnarray*}
\end{proof}

\newpage
\begin{lemma}\label{offdiagonalnegative}
   The off-diagonal elements of the linear system matrix are non-positive:
   $A^L_{i,j}\le 0, j\ne i$.
\end{lemma}
\begin{proof}
This proof begins by bounding the term
$\sum\limits_{K\subset S_{ij}}\nu_K^L b_K(\varphi_j, \varphi_i)$:
\begin{eqnarray*}
   \sum\limits_{K\subset S_{ij}}\nu_K^L b_K(\varphi_j, \varphi_i)
   & = & \sum\limits_{K\subset S_{ij}} \max\limits_{i\ne j\in \mathcal{I}(K)}
      \frac{\left|\int\limits_{S_{ij}}\left(\mathbf{\Omega}\cdot\nabla\varphi_j +
		\sigma_t\varphi_j\right)\varphi_i d\mathbf{x}\right|}
		{-\sum\limits_{T\subset S_{ij}} b_T(\varphi_j, \varphi_i)}b_K(\varphi_j,\varphi_i)\\
   & \le & \sum\limits_{K\subset S_{ij}} \frac{\left|\int\limits_{S_{ij}}
      \left(\mathbf{\Omega}\cdot\nabla\varphi_j +
		\sigma_t\varphi_j\right)\varphi_i d\mathbf{x}\right|}
		{-\sum\limits_{T\subset S_{ij}} b_T(\varphi_j, \varphi_i)}b_K(\varphi_j,\varphi_i)\\
   & \le & -\left|\int\limits_{S_{ij}}\left(\mathbf{\Omega}\cdot\nabla\varphi_j +
		\sigma_t\varphi_j\right)\varphi_i d\mathbf{x}\right|
\end{eqnarray*}
Now, recalling Equation \ref{Aij},
\begin{eqnarray*}
	A^L_{i,j} & = & \int\limits_{S_{ij}}\left(\mathbf{\Omega}\cdot\nabla\varphi_j +
		\sigma_t\varphi_j\right)\varphi_i d\mathbf{x} +
		\sum\limits_{K\subset S_{ij}}\nu_K^L b_K(\varphi_j, \varphi_i)\\
      & \le & \int\limits_{S_{ij}}\left(\mathbf{\Omega}\cdot\nabla\varphi_j +
		\sigma_t\varphi_j\right)\varphi_i d\mathbf{x} -\left|\int\limits_{S_{ij}}
      \left(\mathbf{\Omega}\cdot\nabla\varphi_j +
		\sigma_t\varphi_j\right)\varphi_i d\mathbf{x}\right|\\
      & \le & 0.\qed
\end{eqnarray*}
\end{proof}

\begin{lemma}\label{diagonalpositive}
   The diagonal elements  of the linear system matrix are non-negative: $A^L_{i,i}\ge 0$.
\end{lemma}
\begin{proof}
\[
	A^L_{i,i} = \int\limits_{S_{i}}\left(\nabla\cdot\frac{\mathbf{\Omega}\varphi_i^2}{2} +
		\sigma_t\varphi_i^2\right) d\mathbf{x} +
		\sum\limits_{K\subset S_{i}}\nu_K^L b_K(\varphi_i, \varphi_i).
\]
To prove that $A^L_{i,i}$ is non-negative, it is sufficient to prove that
each term in the above expression is non-negative. The non-negativity of
the interaction term and viscous term are obvious ($\sigma_t \ge 0, 
\, \nu_K^L\ge 0, \, b_K(\varphi_i, \varphi_i)>0$), but
the non-negativity of the divergence term is not necessarily obvious. On the interior of
the domain, the divergence term gives zero contribution because the divergence integral may
be transformed into a surface integral $\int\limits_{\partial S_{i}}
\mathbf{\Omega}\cdot\mathbf{n}\frac{\varphi_i^2}{2} d\mathbf{x}$
via the divergence theorem; one can then recognize that
the basis function $\varphi_i$ evaluates to zero on the boundary of its support $S_{i}$.\\
On the outflow boundary of the domain, the term $\mathbf{\Omega}\cdot\mathbf{n}
\frac{\varphi_i^2}{2}$ is positive because the $\mathbf{\Omega}\cdot\mathbf{n} >0$
for an outflow boundary. This quantity is of course negative for the inflow boundary,
but a Dirichlet boundary condition is strongly imposed on the incoming boundary, so
for degrees of freedom $i$ on the incoming boundary, $A^L_{i,i}$ will be set equal
to some positive value such as 1 with a corresponding incoming value
accounted for in the right hand side $\mathbf{b}$ of the linear system.\qed
\end{proof}

\newpage
\begin{lemma}
   The sum of all elements in a row $i$ is non-negative: $\sum\limits_j A^L_{i,j} \ge 0$.
\end{lemma}

\begin{proof}
Using the fact that $\sum\limits_j\varphi_j=1$,
\begin{eqnarray*}
	\sum\limits_j A^L_{i,j} & = & \sum\limits_j\int\limits_{S_{ij}}
      \left(\mathbf{\Omega}\cdot\nabla\varphi_j +
		\sigma_t\varphi_j\right)\varphi_i d\mathbf{x} +
		\sum\limits_j\sum\limits_{K\subset S_{ij}}\nu_K b_K(\varphi_j, \varphi_i)\\
		& = & \int\limits_{S_{i}}\left(\mathbf{\Omega}\cdot\nabla\sum\limits_j\varphi_j +
		\sigma_t\sum\limits_j\varphi_j\right)\varphi_i d\mathbf{x}\\
		& = & \int\limits_{S_{i}}\sigma_t\varphi_i d\mathbf{x}\\
		&\ge& 0.\qed
\end{eqnarray*}
\end{proof}

\begin{lemma}\label{diagonallydominant}
   $\mathbf{A^L}$ is strictly diagonally dominant:
   $\left|A^L_{i,i}\right| \ge \sum\limits_{j\ne i} \left|A^L_{i,j}\right|$.
\end{lemma}
\begin{proof}
Using the inequalities $\sum\limits_j A^L_{i,j} \ge 0$ and $A^L_{i,j}\le 0, j\ne i$,
it is proven that $\mathbf{A^L}$ is strictly diagonally dominant:
\begin{eqnarray*}
	\sum\limits_j A^L_{i,j} & \ge & 0\\
	\sum\limits_{j\ne i} A^L_{i,j} + A^L_{i,i} & \ge & 0\\
	\left|A^L_{i,i}\right| & \ge & \sum\limits_{j\ne i} -A^L_{i,j}\\
	\left|A^L_{i,i}\right| & \ge & \sum\limits_{j\ne i} \left|A^L_{i,j}\right|.\qed
\end{eqnarray*}
\end{proof}

\begin{lemma}
   $\mathbf{A^L}$ is an M-Matrix.
\end{lemma}
\begin{proof}
To prove that a matrix is an M-Matrix, it is sufficient to prove that:
\[
\left\{\begin{array}{l}
A^L_{i,j}\le 0, j\ne i\\
A^L_{i,i}\ge 0\\
\left|A^L_{i,i}\right| \ge \sum\limits_{j\ne i} \left|A^L_{i,j}\right|\\
\end{array}
\right.,
\]
which are given by Lemmas \ref{offdiagonalnegative}, \ref{diagonalpositive}, and
\ref{diagonallydominant}, respectively.\qed
\end{proof}

\begin{theorem}
The solution $\mathbf{\psi}$ is non-negative and satisfies the following maximum principle:
\begin{equation}\label{ss_max_principle}
   \left(1 - \frac{1}{A^L_{i,i}}\sum\limits_j A^L_{i,j}\right)U_{min,i}
      + \frac{b_i}{A^L_{i,i}}\le U_i
   \le \left(1 - \frac{1}{A^L_{i,i}}\sum\limits_j A^L_{i,j}\right)U_{max,i}
      + \frac{b_i}{A^L_{i,i}},
\end{equation}
where $U_{min,i} = \min\limits_{j\in \mathcal{I}(S_i)}U_j$, $U_{max,i}
= \max\limits_{j\in \mathcal{I}(S_i)}U_j$,
and $\mathcal{I}(S_i)$ is the set of indices of degrees of freedom in the support
of degree of freedom $i$.
\end{theorem}
\begin{proof}
\begin{eqnarray*}
	\sum\limits_j A^L_{i,j}U_j & = & b_i\\
	A^L_{i,i}U_i & = & \sum\limits_{j\ne i} -A^L_{i,j}U_j + b_i\\
	A^L_{i,i}U_i & \le & \left(\sum\limits_{j\ne i} -A^L_{i,j}\right)U_{max,i} + b_i\\
	A^L_{i,i}(U_i - U_{max,i}) & \le & \left(-\sum\limits_j A^L_{i,j}\right)U_{max,i}
      + b_i\\
	U_i - U_{max,i} & \le & \left(-\frac{1}{A^L_{i,i}}\sum\limits_j A^L_{i,j}\right)
      U_{max,i} + \frac{b_i}{A^L_{i,i}}\\
	U_i & \le & \left(1 - \frac{1}{A^L_{i,i}}\sum\limits_j A^L_{i,j}\right)U_{max,i}
      + \frac{b_i}{A^L_{i,i}}
\end{eqnarray*}
A similar analysis is performed to prove the lower bound for $U_i$.\qed
\end{proof}
%================================================================================
\section{FCT Schemes}
%================================================================================
%--------------------------------------------------------------------------------
\subsection{Steady-State FCT Scheme}
%--------------------------------------------------------------------------------
In this section, the general FCT scheme for the steady-state case is presented.
The flux correction vector $\mathbf{f}$ is defined such that:
\begin{equation}\label{ss_f}
   \mathbf{A}^L\mathbf{U}^H = \mathbf{b} + \mathbf{f}.
\end{equation}
Decomposing the correction flux into $\mathbf{F}$ and applying a limiter
$\mathcal{L}$ gives
\begin{equation}\label{FCT_limited_ss}
   \mathbf{A}^L\mathbf{U}
      = \mathbf{b} + \mathcal{L}[\mathbf{F}].
\end{equation}
where $(\mathcal{L}[\mathbf{F}])_i = \mathcal{L}_{i,:}F_{i,:}^T
= \sum\limits_j \mathcal{L}_{i,j}F_{i,j}$ and $\mathbf{U}$
is the FCT solution.  The limiter $\mathcal{L}$ is
defined in Section \ref{L}.

Combining Equations \eqref{ss_f} and \eqref{steadystate_high}
gives the definition of $\mathbf{f}$:
\begin{equation}
   \mathbf{f} \equiv \left(\mathbf{D}^L-\mathbf{D}^H\right)\mathbf{U}^H.
\end{equation}
Since $\mathbf{D}^L-\mathbf{D}^H(t)$ is symmetric
and features zero row and column sums, a valid decomposition for $\mathbf{f}$,
called $\mathbf{F}$, is
\begin{equation}
   F_{i,j} = \left(D_{i,j}^L-D_{i,j}^H\right)\left(U_j^H - U_i^H\right).
\end{equation}
%================================================================================
\section{Explicit Transient Radiation Transport}
%================================================================================
%--------------------------------------------------------------------------------
\subsection{Deriving a Maximum Principle}
%--------------------------------------------------------------------------------
Here we seek to find a maximum principle for the transient problem, given by
Equation \ref{tr}.
Using explicit Euler for time stepping, the following equation relates the new and old
time step solutions for the low-order scheme:
\begin{equation}\label{newstep}
   U_i^L = U_i^n - \frac{\Delta t}{m_i}\sum\limits_j U_j^n A^L_{i,j}
      + \frac{\Delta t}{m_i}b_i,
\end{equation}
where $\Delta t$ is the time step size, $m_i$ is the $i$th element of the \emph{lumped} mass
matrix, i.e., $M^L_{i,i}$, $\mathbf{A}^L$ is the steady-state system matrix, and
$\mathbf{b}$ is the right-hand side.

\begin{theorem}
If the CFL-like condition
\begin{equation}\label{ex_CFL}
   \Delta t \leq \frac{m_i}{A_{i,i}^L}\quad\forall i
   \qquad\Longleftrightarrow\qquad
   1 - \frac{\Delta t}{m_i}A_{i,i}^L \geq 0\quad\forall i,
\end{equation}
then the low-order solution at the new time step, $\mathbf{U}^L$ satisfies the following maximum principle:
\begin{equation}\label{explicit_max_principle}
   W_i^-\leq U_i^L\leq W_i^+\quad\forall i,
\end{equation}
where
\begin{equation}\label{ex_bounds}
   W_i^\pm\equiv U_{\substack{\max\\\min},i}^n\left(1-\frac{\Delta t}{m_i}\sum\limits_j A^L_{i,j}\right)
      + \frac{\Delta t}{m_i}b_i,
\end{equation}
where $U_{\substack{\max\\\min},i}^n = \substack{\max\\\min\limits_{j\in \mathcal{I}(S_i)}}U_j^n$,
and $\mathcal{I}(S_i)$ is the set of indices of degrees of freedom in the
support of degree of freedom $i$.
\end{theorem}
\begin{proof}
Rearranging Equation \ref{newstep},
\[
   U_i^L = \left(1-\frac{\Delta t}{m_i}A^L_{i,i}\right)U_i^n - \frac{\Delta t}{m_i}
      \sum\limits_{j\ne i} U_j^n A^L_{i,j} + \frac{\Delta t}{m_i}b_i,
\]
The CFL-like condition in Equation \ref{ex_CFL} gives that $1-\frac{\Delta t}{m_i}A^L_{i,i} \ge 0$, and by
Lemma \ref{offdiagonalnegative}, it is known that the off-diagonal
elements $A^L_{i,j}, j\ne i$, are non-positive. Thus, the following inequality is
able to be applied:
\[
   U_i^L \le
   U_{\max,i}^n\left(1-\frac{\Delta t}{m_i}\sum\limits_j A^L_{i,j}\right)
      + \frac{\Delta t}{m_i}b_i,
\]
and similarly for the lower bound.\qed
\end{proof}
%================================================================================
\subsection{FCT}
%================================================================================
%================================================================================
\subsubsection{The Explicit System}
%================================================================================
Without strongly imposing Dirichlet boundary conditions, the high-order system is
\begin{equation}
   \mathbf{M}_C\frac{\mathbf{U}^H-\mathbf{U}^n}{\Delta t}
      = \mathbf{K}\mathbf{U}^n + \mathbf{b},
\end{equation}
\begin{equation}\label{high}
   \mathbf{M}_C\mathbf{U}^H = \mathbf{M}_C\mathbf{U}^n
      + \Delta t\mathbf{K}\mathbf{U}^n + \Delta t\mathbf{b} \equiv \mathbf{b}^H,
\end{equation}
and the corresponding low-order system is
\begin{equation}
   \mathbf{M}_L\frac{\mathbf{U}^L-\mathbf{U}^n}{\Delta t}
      = (\mathbf{K} + \mathbf{D})\mathbf{U}^n + \mathbf{b}.
\end{equation}
\begin{equation}\label{low}
   \mathbf{M}_L\mathbf{U}^L = \mathbf{M}_L\mathbf{U}^n
      + \Delta t(\mathbf{K} + \mathbf{D})\mathbf{U}^n + \Delta t\mathbf{b}
      \equiv \mathbf{b}^L.
\end{equation}

After strongly imposing Dirichlet boundary conditions, the linear
systems given by Equations \ref{high} and \ref{low} are modified:
\begin{equation}\label{highD}
   \tilde{\mathbf{M}}_C\mathbf{U}^H = \tilde{\mathbf{b}}^H,
\end{equation}
\begin{equation}\label{lowD}
   \tilde{\mathbf{M}}_L\mathbf{U}^L = \tilde{\mathbf{b}}^L.
\end{equation}
%--------------------------------------------------------------------------------
\subsubsection{Guermond's Flux Correction Algorithm}
%--------------------------------------------------------------------------------
The flux correction vector $\mathbf{f}$ is defined such that:
\begin{equation}\label{fc}
   \mathbf{M}_L\mathbf{U}^H = \mathbf{M}_L\mathbf{U}^L + \mathbf{f} \equiv \mathbf{b}^F,
\end{equation}
Substituting the solutions of Equations \ref{high} and \ref{low} gives the definition
of $\mathbf{f}$:
\begin{equation}\label{Fdef}
   \mathbf{f} \equiv \mathbf{M}_L\mathbf{M}_C^{-1}\mathbf{b}^H
      -\mathbf{M}_L\mathbf{M}_L^{-1}\mathbf{b}^L.
\end{equation}

\begin{lemma}
   Suppose that with strongly imposed Dirichlet boundary conditions, the
   correction step is defined such that the Dirichlet boundary conditions
   are strongly imposed on Equation \ref{fc}:
   \begin{equation}\label{fcD}
      \tilde{\mathbf{M}}_L\hat{\mathbf{U}}^H = \tilde{\mathbf{b}}^F,
   \end{equation}
   and the definition for the flux correction vector $\mathbf{f}$ from
   Equation \ref{Fdef} is used without modification. Then the solution
   to this correction step, $\hat{\mathbf{U}}^H$, will not yield the
   correct high-order solution $\mathbf{U}^H$ given by Equation \ref{highD}.
\end{lemma}
\begin{proof}
   The solution to Equation \ref{fcD} is
   \begin{equation}
      \hat{U}^H_i = \left\{
      \begin{array}{l l}
         g_i                & i\in\mathcal{D}\\
         \frac{1}{m_i}b^F_i & i\notin\mathcal{D}
      \end{array}
      \right.
   \end{equation}
   For $i\in\mathcal{D}$, $\hat{U}^H_i = U_i^H$ because equivalent Dirichlet
   conditions are imposed on Equation \ref{highD}. For $i\notin\mathcal{D}$,
   \begin{eqnarray}
      \hat{U}^H_i & = & \frac{1}{m_i}b^F_i\\
                  & = & \frac{1}{m_i}(m_i U_i^L + f_i)\qquad\mbox{from Equation \ref{fc}}\\
                  & = & U_i^L + \frac{1}{m_i}f_i\label{uhati}
   \end{eqnarray}
   \begin{eqnarray}
      f_i & = & (\mathbf{M}_L\mathbf{M}_C^{-1}\mathbf{b}^H)_i - b^L_i
         \qquad\mbox{from Equation \ref{Fdef}}\\
          & = & m_i(\mathbf{M}_C^{-1}\mathbf{b}^H)_i - b^L_i
   \end{eqnarray}
   Substituting into Equation \ref{uhati} gives
   \begin{eqnarray}
      \hat{U}^H_i & = & U_i^L + \frac{1}{m_i}(m_i(\mathbf{M}_C^{-1}\mathbf{b}^H)_i - b^L_i)\\
                  & = & U_i^L + (\mathbf{M}_C^{-1}\mathbf{b}^H)_i - \frac{1}{m_i}b^L_i.
   \end{eqnarray}
   Recognizing from Equation \ref{lowD} that $U_i^L = \frac{1}{m_i}b^L_i$ finally gives
   \begin{equation}
      \hat{U}^H_i = (\mathbf{M}_C^{-1}\mathbf{b}^H)_i\qquad i\notin\mathcal{D}.
   \end{equation}
   This can be recognized as the solution of Equation \ref{high}, which is not the
   solution to Equation \ref{highD}. Therefore the solution using this correction will yield
   the solution of the high-order system without Dirichlet boundary conditions strongly
   imposed, except at the Dirichlet nodes, where the solution will be forced to the
   Dirichlet values.
   \qed
\end{proof}
%--------------------------------------------------------------------------------
\subsubsection{Kuzmin's Flux Correction Algorithm}
%--------------------------------------------------------------------------------
The flux correction vector $\mathbf{f}$ is defined such that:
\begin{equation}\label{ex_kuzminfc}
   \mathbf{M}_L\mathbf{U}^H
      = (\mathbf{M}_L + \Delta t(\mathbf{K} + \mathbf{D}))\mathbf{U}^n
         + \Delta t\mathbf{b} + \mathbf{f} \equiv \mathbf{b}^F,
\end{equation}
Subtracting Equation \ref{high} from \ref{ex_kuzminfc} gives the definition of
$\mathbf{f}$:
\begin{equation}\label{ex_kuzminFdef}
   \mathbf{f} \equiv -(\mathbf{M}_C-\mathbf{M}_L)\Delta\mathbf{U}^H
      -\Delta t\mathbf{D}\mathbf{U}^n,
\end{equation}
where $\Delta\mathbf{U}^H = \mathbf{U}^H - \mathbf{U}^n$. Since
$\mathbf{M}_C-\mathbf{M}_L$ and $\mathbf{D}$ are symmetric
and feature zero row and column sums, a valid decomposition for $\mathbf{f}$,
called $\mathbf{F}$, is
\begin{equation}
   F_{i,j} = -m_{i,j}(\Delta U^H_j - \Delta U^H_i) - \Delta t D_{i,j}(U^n_j - U^n_i)
\end{equation}
where $m_{i,j}$ is the $i,j$th element of the consistent mass matrix.
Applying a limiter to Equation \ref{ex_kuzminfc} gives
\begin{equation}\label{ex_limited}
   \mathbf{M}_L\mathbf{U}^{n+1}
      = (\mathbf{M}_L + \Delta t(\mathbf{K} + \mathbf{D}))\mathbf{U}^n
         + \Delta t\mathbf{b} + \mathcal{L}[\mathbf{F}],
\end{equation}
where $(\mathcal{L}[\mathbf{F}])_i = \mathcal{L}_{i,:}F_{i,:}^T
= \sum\limits_j \mathcal{L}_{i,j}F_{i,j}$ and $\mathbf{U}^{n+1}$ is the FCT solution.

\begin{lemma}
   Suppose that with strongly imposed Dirichlet boundary conditions, the
   correction step is defined such that the Dirichlet boundary conditions
   are strongly imposed on Equation \ref{ex_kuzminfc}:
   \begin{equation}\label{ex_kuzminfcD}
      \tilde{\mathbf{M}}_L\hat{\mathbf{U}}^H
         = \tilde{\mathbf{b}}^F,
   \end{equation}
   and the definition for
   the flux correction vector $\mathbf{f}$ from
   Equation \ref{ex_kuzminFdef} is used without modification. Then the solution
   to this correction step, $\hat{\mathbf{U}}^H$, will yield the
   correct high-order solution $\mathbf{U}^H$ given by Equation \ref{highD}.
\end{lemma}
\begin{proof}
   For $i\in\mathcal{D}$, $\hat{U}^H_i = U^H_i = g_i$. For $i\notin\mathcal{D}$,
   we take the $i$th equation of the linear system given by Equation \ref{ex_kuzminfcD}:
   \begin{equation}\label{ex_i_eq}
      m_i\hat{U}^H_i
         = m_i U^n_i + \Delta t((\mathbf{K}+\mathbf{D})\mathbf{U}^n)_i + \Delta t b_i - (\mathbf{M}_C\Delta\mathbf{U}^H)_i
         + m_i(U^H_i-U^n_i) - \Delta t(\mathbf{D}\mathbf{U}^n)_i.
   \end{equation}
   Examining the $i$th equation of the linear system given by Equation \ref{highD}
   and solving for $\Delta t b_i$ gives
   \begin{equation}
      \Delta t b_i = (\mathbf{M}_C\Delta\mathbf{U}^H)_i
         - \Delta t(\mathbf{K}\mathbf{U}^n)_i.
   \end{equation}
   Substituting this back into Equation \ref{ex_i_eq} gives
   \begin{eqnarray}
      m_i\hat{U}^H_i & = & m_i U^H_i + \Delta t((\mathbf{K}+\mathbf{D})\mathbf{U}^n)_i
         - \Delta t(\mathbf{K}\mathbf{U}^n)_i - \Delta t(\mathbf{D}\mathbf{U}^n)_i\\
      m_i\hat{U}^H_i & = & m_i U^H_i\\
      \hat{U}^H_i & = & U^H_i
   \end{eqnarray}
   Therefore, $\hat{\mathbf{U}}^H = \mathbf{U}^H$.
   \qed
\end{proof}
%================================================================================
\subsubsection{Deriving the Limiting Coefficients}
%================================================================================
\begin{lemma}
   The definitions given by Equations \ref{P_defs}, \ref{R_defs}, and \ref{L_defs},
   and the following definitions yield maximum-principle preserving limiting coefficients,
   where $W_i^\pm$ are defined to be to be upper and lower bounds of the maximum
   principle given by Equation \ref{explicit_max_principle}:
   \begin{equation}
      W_i^+ \equiv U_{max,i}^n\left(1-\frac{\Delta t}{m_i}\sum\limits_j A^L_{i,j}\right)
      + \frac{\Delta t}{m_i}b_i,
   \end{equation}
   \begin{equation}
      W_i^- \equiv U_{min,i}^n\left(1-\frac{\Delta t}{m_i}\sum\limits_j A^L_{i,j}\right)
      + \frac{\Delta t}{m_i}b_i,
   \end{equation}
   \begin{equation}
      Q_i^\pm \equiv m_i(W_i^\pm-U_i^n) + \Delta t\sum\limits_j A_{i,j}^L U_j^n
         - \Delta t b_i,
   \end{equation}
\end{lemma}

\begin{proof}
   The properties given in the proof for Lemma \ref{ss_coef} for $P_i^\pm$,
   $Q_i^\pm$, $R_i^\pm$, and $\mathcal{L}_{i,j}$ hold with these definitions
   of $W_i^\pm$ and $Q_i^\pm$.
   The proof will be given for the upper bound. 
   Taking row $i$ of the linear system given by Equation \ref{ex_limited} gives
   \begin{gather*}
      m_i U_i^{n+1} = m_i U_i^n - \Delta t\sum\limits_j A_{i,j}^L U_j^n + \Delta t b_i
         + \sum\limits_j \mathcal{L}_{i,j}F_{i,j}\\
      m_i U_i^{n+1} - m_i U_i^n + \Delta t\sum\limits_j A_{i,j}^L U_j^n - \Delta t b_i
         = \sum\limits_j \mathcal{L}_{i,j}F_{i,j}
   \end{gather*}
   As in the proof of Lemma \ref{ss_coef}, the following inequality is found:
   \[
      \sum\limits_j \mathcal{L}_{i,j}F_{i,j} \leq Q_i^+
   \]
   \begin{eqnarray*}
      m_i U_i^{n+1} - m_i U_i^n + \Delta t\sum\limits_j A_{i,j}^L U_j^n - \Delta t b_i
      & = & \sum\limits_j \mathcal{L}_{i,j}F_{i,j}\\
      m_i U_i^{n+1} - m_i U_i^n + \Delta t\sum\limits_j A_{i,j}^L U_j^n - \Delta t b_i
      & \leq & Q_i^+\\
      m_i U_i^{n+1} - m_i U_i^n + \Delta t\sum\limits_j A_{i,j}^L U_j^n - \Delta t b_i
      & \leq & m_i(W_i^+-U_i^n) + \Delta t\sum\limits_j A_{i,j}^L U_j^n
         - \Delta t b_i\\
      U_i^{n+1} & \leq & W_i^+.
   \end{eqnarray*}
   The lower bound is proved similarly.
   \qed
\end{proof}
%================================================================================
\section{Implicit Transient Radiation Transport}
%================================================================================
%--------------------------------------------------------------------------------
\subsection{Deriving a Maximum Principle}
%--------------------------------------------------------------------------------
Here we seek to find a maximum principle for the transient problem, given by
Equation \ref{tr}.
Using implicit Euler for time stepping, the following equation relates the new and old
time step solutions:
\begin{equation}\label{im_newstep}
   U_i^{n+1} + \frac{\Delta t}{m_i}\sum\limits_j U_j^{n+1} A^L_{i,j} =
   U_i^n  + \frac{\Delta t}{m_i}b_i,
\end{equation}
where $\Delta t$ is the time step size, $m_i$ is the $i$th element of the \emph{lumped} mass
matrix, i.e., $M^L_{i,i}$, $\mathbf{A^L}$ is the steady-state system matrix, and
$\mathbf{b}$ is the right-hand side.

\begin{theorem}
The solution at the new time step, $U^{n+1}_i$ satisfies the following maximum principle:
\begin{equation}\label{implicit_max_principle}
   \frac{1}{1+\frac{\Delta t}{m_i}A_{i,i}}\left(U_i^n - \left(\frac{\Delta t}{m_i}
   \sum\limits_{j\ne i} A^L_{i,j}\right)U_{min,i}^{n+1} + \frac{\Delta t}{m_i}b_i\right)
   \le U_i^{n+1} \le
   \frac{1}{1+\frac{\Delta t}{m_i}A_{i,i}}\left(U_i^n - \left(\frac{\Delta t}{m_i}
   \sum\limits_{j\ne i} A^L_{i,j}\right)U_{max,i}^{n+1} + \frac{\Delta t}{m_i}b_i\right),
\end{equation}
where $U_{min,i}^{n+1} = \min\limits_{j\in \mathcal{I}(S_i)}U_j^{n+1}$, $U_{max,i}^{n+1} = \max\limits_{j\in \mathcal{I}(S_i)}U_j^{n+1}$,
and $\mathcal{I}(S_i)$ is the set of indices of degrees of freedom in the support of degree of freedom $i$.
\end{theorem}
\begin{proof}
Rearranging Equation \ref{im_newstep},
\[
   \left(1+\frac{\Delta t}{m_i}A_{i,i}\right)U_i^{n+1} + \frac{\Delta t}{m_i}
      \sum\limits_{j\ne i} U_j^{n+1} A^L_{i,j} = U_i^n  + \frac{\Delta t}{m_i}b_i.
\]
\[
   U_i^{n+1} = \frac{1}{1+\frac{\Delta t}{m_i}A_{i,i}}\left(U_i^n - \frac{\Delta t}{m_i}\sum\limits_{j\ne i} U_j^{n+1} A^L_{i,j} + \frac{\Delta t}{m_i}b_i\right).
\]
By Lemma \ref{offdiagonalnegative}, it is known that the off-diagonal
elements $A^L_{i,j}, j\ne i$, are non-positive. Thus, the following inequality is able to be applied:
\[
   U_i^{n+1} \le
      \frac{1}{1+\frac{\Delta t}{m_i}A_{i,i}}\left(U_i^n - \left(\frac{\Delta t}{m_i}\sum\limits_{j\ne i} A^L_{i,j}\right)U_{max,i}^{n+1} + \frac{\Delta t}{m_i}b_i\right),
\]
and similarly for the lower bound.\qed
\end{proof}
%================================================================================
\subsection{FCT}
%================================================================================
%================================================================================
\subsubsection{The Implicit System}
%================================================================================
Without strongly imposing Dirichlet boundary conditions, the high-order system is
\begin{equation}\label{im_high}
   \mathbf{A}^{tr,H}\mathbf{U}^H
      = \mathbf{M}_C\mathbf{U}^n + \Delta t\mathbf{b} \equiv \mathbf{b}^H,
\end{equation}
where $\mathbf{A}^{tr,H} \equiv \mathbf{M}_C-\Delta t\mathbf{K}$.
Strongly imposing Dirichlet boundary conditions on Equation \ref{im_high}
gives
\begin{equation}\label{im_highD}
   \tilde{\mathbf{A}}^{tr,H}\mathbf{U}^H
      = \tilde{\mathbf{b}}^H.
\end{equation}
The low-order system corresponding to Equation \ref{im_high} is
\begin{equation}\label{im_low}
   (\mathbf{M}_L-\Delta t(\mathbf{K}+\mathbf{D}))\mathbf{U}^L
      = \mathbf{M}_L\mathbf{U}^n + \Delta t\mathbf{b}.
\end{equation}
Note that the solution to this low-order system
is not computed in this Dr. Kuzmin's flux correction algorithm given in Section
\ref{im_kuzmin}; instead, this low-order system is just used
to define the correction step.
%--------------------------------------------------------------------------------
\subsubsection{Kuzmin's Implicit Flux Correction Algorithm}\label{im_kuzmin}
%--------------------------------------------------------------------------------
The flux correction vector $\mathbf{f}$ is defined such that:
\begin{equation}\label{im_kuzminfc}
   \mathbf{A}^{tr,F}\mathbf{U}^H
      = \mathbf{M}_L\mathbf{U}^n + \Delta t\mathbf{b} + \mathbf{f} \equiv \mathbf{b}^F,
\end{equation}
where $\mathbf{A}^{tr,F} \equiv \mathbf{M}_L-\Delta t(\mathbf{K}+\mathbf{D})$.
Subtracting Equation \ref{im_high} from \ref{im_kuzminfc} gives the definition of
$\mathbf{f}$:
\begin{equation}\label{im_kuzminFdef}
   \mathbf{f} \equiv -(\mathbf{M}_C-\mathbf{M}_L)\Delta\mathbf{U}^H
      -\Delta t\mathbf{D}\mathbf{U}^H,
\end{equation}
where $\Delta\mathbf{U}^H = \mathbf{U}^H - \mathbf{U}^n$. Since
$\mathbf{M}_C-\mathbf{M}_L$ and $\mathbf{D}$ are symmetric
and feature zero row and column sums, a valid decomposition for $\mathbf{f}$,
called $\mathbf{F}$, is
\begin{equation}
   F_{i,j} = -m_{i,j}(\Delta U^H_j - \Delta U^H_i) - \Delta t D_{i,j}(U^H_j - U^H_i)
\end{equation}
where $m_{i,j}$ is the $i,j$th element of the consistent mass matrix.
Applying a limiter to Equation \ref{im_kuzminfc} gives
\begin{equation}\label{im_limited}
   \mathbf{A}^{tr,F}\mathbf{U}^{n+1}
      = \mathbf{M}_L\mathbf{U}^n + \Delta t\mathbf{b} + \mathcal{L}[\mathbf{F}],
\end{equation}
where $(\mathcal{L}[\mathbf{F}])_i = \mathcal{L}_{i,:}F_{i,:}^T
= \sum\limits_j \mathcal{L}_{i,j}F_{i,j}$ and $\mathbf{U}^{n+1}$ is the FCT solution.

\begin{lemma}
   Suppose that with strongly imposed Dirichlet boundary conditions, the
   correction step is defined such that the Dirichlet boundary conditions
   are strongly imposed on Equation \ref{im_kuzminfc}:
   \begin{equation}\label{im_kuzminfcD}
      \tilde{\mathbf{A}}^{tr,F}\hat{\mathbf{U}}^H
         = \tilde{\mathbf{b}}^F,
   \end{equation}
   and the definition for
   the flux correction vector $\mathbf{f}$ from
   Equation \ref{im_kuzminFdef} is used without modification. Then the solution
   to this correction step, $\hat{\mathbf{U}}^H$, will yield the
   correct high-order solution $\mathbf{U}^H$ given by Equation \ref{im_highD}.
\end{lemma}
\begin{proof}
   For $i\in\mathcal{D}$, $\hat{U}^H_i = U^H_i = g_i$. For $i\notin\mathcal{D}$,
   we take the $i$th equation of the linear system given by Equation \ref{im_kuzminfcD}:
   \begin{equation}\label{i_eq}
      (\mathbf{A}^{tr,F}\hat{\mathbf{U}}^H)_i
         = m_i U^n_i + \Delta t b_i - (\mathbf{M}_C\Delta\mathbf{U}^H)_i
         + m_i(U^H_i-U^n_i) - \Delta t(\mathbf{D}\mathbf{U}^H)_i.
   \end{equation}
   Examining the $i$th equation of the linear system given by Equation \ref{im_highD}
   and solving for $\Delta t b_i$ gives
   \begin{equation}
      \Delta t b_i = (\mathbf{M}_C\Delta\mathbf{U}^H)_i
         - \Delta t(\mathbf{K}\mathbf{U}^H)_i.
   \end{equation}
   Substituting this back into Equation \ref{i_eq} gives
   \begin{eqnarray}
      (\mathbf{A}^{tr,F}\hat{\mathbf{U}}^H)_i
         & = & m_i U^H_i - \Delta t(\mathbf{K}\mathbf{U}^H)_i -
            \Delta t(\mathbf{D}\mathbf{U}^H)_i.\\
         & = & \left((\mathbf{M}_L - \Delta t(\mathbf{K}+\mathbf{D}))\mathbf{U}^H\right)_i\\
         & = &(\mathbf{A}^{tr,F}\mathbf{U}^H)_i.
   \end{eqnarray}
   The system of equations
   \begin{equation}
      \left\{
         \begin{array}{l l}
            \hat{U}^H_i = U^H_i & i\in\mathcal{D}\\
            (\mathbf{A}^{tr,F}\hat{\mathbf{U}}^H)_i = (\mathbf{A}^{tr,F}\mathbf{U}^H)_i
               & i\notin\mathcal{D}
         \end{array}
      \right.
   \end{equation}
   determines that $\hat{\mathbf{U}}^H$ is uniquely equal to $\mathbf{U}^H$.
   \qed
\end{proof}
%================================================================================
\subsubsection{Deriving the Limiting Coefficients}
%================================================================================
\begin{lemma}
   The definitions given by Equations \ref{P_defs}, \ref{R_defs}, and \ref{L_defs},
   and the following definitions yield maximum-principle preserving limiting coefficients,
   where $W_i^\pm$ are defined to be to be upper and lower bounds of the maximum
   principle given by Equation \ref{implicit_max_principle}:
   \begin{equation}
      W_i^+ \equiv \frac{1}{1+\frac{\Delta t}{m_i}A_{i,i}}\left(U_i^n
         - \left(\frac{\Delta t}{m_i}\sum\limits_{j\ne i} A^L_{i,j}\right)U_{max,i}^{L,n+1}
         + \frac{\Delta t}{m_i}b_i\right),
   \end{equation}
   \begin{equation}
      W_i^- \equiv \frac{1}{1+\frac{\Delta t}{m_i}A_{i,i}}\left(U_i^n
         - \left(\frac{\Delta t}{m_i}\sum\limits_{j\ne i} A^L_{i,j}\right)U_{min,i}^{L,n+1}
         + \frac{\Delta t}{m_i}b_i\right),
   \end{equation}
   \begin{equation}
      Q_i^\pm \equiv (m_i+\Delta t A_{i,i}^L)W_i^\pm + \Delta t
         \sum\limits_{j\ne i} A_{i,j}^L U_j^{n+1} - m_i U_i^n
         - \Delta t b_i,
   \end{equation}
\end{lemma}

\begin{proof}
   The properties given in the proof for Lemma \ref{ss_coef} for $P_i^\pm$,
   $Q_i^\pm$, $R_i^\pm$, and $\mathcal{L}_{i,j}$ hold with these definitions
   of $W_i^\pm$ and $Q_i^\pm$.
   The proof will be given for the upper bound. 
   As in the proof of Lemma \ref{ss_coef}, the following inequality is found:
   \[
      \sum\limits_j \mathcal{L}_{i,j}F_{i,j} \leq Q_i^+
   \]
   Taking row $i$ of the linear system given by Equation \ref{im_limited} gives
   \begin{eqnarray*}
     (m_i + \Delta t A^L{i,i})U_i^{n+1} + \Delta t\sum\limits_{j\ne i}A^L_{i,j}U_j^{n+1}
         - m_i U_i^n - \Delta t b_i & = & \sum\limits_j \mathcal{L}_{i,j}F_{i,j}\\
         & \leq & Q_i^+\\
         & \leq & (m_i+\Delta t A_{i,i}^L)W_i^+ + \Delta t
         \sum\limits_{j\ne i} A_{i,j}^L U_j^{n+1} - m_i U_i^n
         - \Delta t b_i\\
      U_i^{n+1} & \leq & W_i^+ 
   \end{eqnarray*}
   The lower bound is proved similarly.
   \qed
\end{proof}

%%--------------------------------------------------------------------------------
%\section{High-order viscosity+FCT: Implicit time discretization}
%%--------------------------------------------------------------------------------
%We use implicit Euler discretization. Let us write the low-order equations
%
%\begin{equation}
	%M^L (\psi^{L,n+1}-\psi^n) + \tau A^L\psi^{L,n+1} =\tau Q^{n+1}
%\end{equation}
%$M_L$ lumped mass matrix.
%
%High-order viscosity version:
%\begin{equation}
	%M^C (\psi^{H,n+1}-\psi^n) + \tau A^H\psi^{H,n+1} =\tau Q^{n+1}
%\end{equation}
%$M_C$ consistent mass matrix.
%
%We denote the dissipation operator by $D=A^L-A^H$.
%
%
%\begin{lemma}
%We have: 
%\[
%\left(M_L + \tau (A^H+D)\right) \psi^H = M_L \psi^n +  \tau Q  + \tau D \psi^H -(M_C-M_L)(\psi^H-\psi^n) 
%\]
%\end{lemma}
%\begin{proof}
%\begin{eqnarray*}
%\left(M_L + \tau (A^H+D)\right) \psi^H &=& \left(M_L + \tau (A^H)\right) \psi^H +M_C\psi^H -M_C\psi^H + \tau D \psi^H \\
%&=& M_L \psi^H +  \tau Q  + \tau D \psi^H -M_C(\psi^H-\psi^n) \\
%&=& M_L \psi^H +  \tau Q  + \tau D \psi^H -M_C(\psi^H-\psi^n) + M_L(\psi^n-\psi^n) \\
%&=& M_L \psi^n +  \tau Q  + \tau D \psi^H -M_C(\psi^H-\psi^n) + M_L(\psi^H-\psi^n) \\
%&=& M_L \psi^n +  \tau Q  + \tau D \psi^H -(M_C-M_L)(\psi^H-\psi^n) 
%\end{eqnarray*}
%\end{proof}
%
%
%Kuzmin's suggestion 
%\begin{enumerate}
%
%\item Do a high-order solve first:
%\begin{equation}
	%(M^C + \tau A^H)\psi^{H} =M_L\psi^n + \tau Q^{n+1}
%\end{equation}
%
%\item Use the above identity with a limiter $\mathcal{L}$
%\begin{equation}
%\left(M_L + \tau (A^H+D)\right) \psi = M_L \psi^n +  \tau Q  + \mathcal{L}\left[ \tau D \psi^H -(M_C-M_L)(\psi^H-\psi^n) \right]
%\end{equation}
%If $\mathcal{L}=1$, then $\psi=\psi^H$ by virtue of the above identity.\\
%If $\mathcal{L}=0$, then $\psi=\psi^L$.
%\end{enumerate}
%
%%--------------------------------------------------------------------------------
%\section{High-order viscosity+FCT: Steady state}
%%--------------------------------------------------------------------------------
%Let $\tau\rightarrow\infty$ in the above section. We obtain:
%
%
%Low-order viscosity:
%\begin{equation}
 %A^L\psi^{L} = (A^H+D)\psi^{L} = Q
%\end{equation}
%
%High-order viscosity version:
%\begin{equation}
 %A^H\psi^{H} = Q
%\end{equation}
%
%
%\begin{lemma}
%We have: 
%\[
%(A^H+D) \psi^H = Q  + D \psi^H 
%\]
%\end{lemma}
%\begin{proof}
%Obvious!
%\end{proof}
%
%
%Kuzmin's suggestion 
%\begin{enumerate}
%
%\item Do a high-order solve first:
%\begin{equation}
	%A^H\psi^{H} = Q
%\end{equation}
%
%\item Use the above identity with a limiter $\mathcal{L}$
%\begin{equation}
%(A^H+D) \psi =  Q  + \mathcal{L}\left[ D \psi^H \right]
%\end{equation}
%If $\mathcal{L}=1$, then $\psi=\psi^H$ by virtue of the above identity.\\
%If $\mathcal{L}=0$, then $\psi=\psi^L$.
%\end{enumerate}

\end{document}