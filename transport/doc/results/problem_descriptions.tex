\section{Test Problem Descriptions}
In this section, the test problem suite is described. These descriptions
include the domain, initial conditions, boundary conditions, and other
physical parameters.
%================================================================================
\subsection{Void-to-Absorber}\label{sec:void_to_absorber}
This problem examines the angular flux travelling in the $+x$ direction,
starting in a void and reaching a strong absorber region.
Table \ref{tab:void_to_absorber} summarizes the test parameters.

\begin{table}[h]\caption{Void-to-Absorber Test Problem Summary}
\label{tab:void_to_absorber}
\centering
\begin{tabular}{l l}\toprule
\emph{Parameter} & \emph{Value}\\\midrule
Domain & $\mathcal{D} = (0,1)^d$\\
Initial Conditions & $u_0(\x)=0$\\
Boundary Conditions & $u(\x,t)=1,\quad \x\in\partial\mathcal{D}^-,\quad t>0,
   \quad\partial\mathcal{D}^-=\{\x\in\partial\mathcal{D}:\mathbf{n}(\x)
   \cdot\mathbf{\Omega}<0\}$\\
Direction & $\mathbf{\Omega} = \mathbf{e}_x$\\
Cross Section & $\sigma(\x)=\left\{\begin{array}{c l}
   10, & \x\in(\frac{1}{2},1)^d\\
   0,  & \mbox{otherwise}\end{array}\right.$\\
Source & $q(\x,t)=0$\\
Speed & $v=1$\\
Exact Solution & $u(\x,t)=\left\{\begin{array}{l l}
   \left\{\begin{array}{l l}
      e^{-10(x-\frac{1}{2})}, & x\ge\frac{1}{2}, y\ge\frac{1}{2}, z\ge\frac{1}{2}\\
      1,                      & \mbox{otherwise}
   \end{array}\right., & x-t<0\\
   0, & \mbox{otherwise}
   \end{array}\right.$ \\
\bottomrule\end{tabular}
\end{table}
%================================================================================
\subsection{Skew Void-to-Absorber}
This problem is a more general case of the test problem described in
Section \ref{sec:void_to_absorber} in which the transport direction is
not necessarily the $+x$ direction but instead is any direction for which
$\Omega_i\ge 0,\forall i$.
Table \ref{tab:void_to_absorber_skew} summarizes the test parameters,
where the definition of $s$ is given below.

\begin{table}[h]\caption{Skew Void-to-Absorber Test Problem Summary}
\label{tab:void_to_absorber_skew}
\centering
\begin{tabular}{l l}\toprule
\emph{Parameter} & \emph{Value}\\\midrule
Domain & $\mathcal{D} = (0,1)^d$\\
Initial Conditions & $u_0(\x)=0$\\
Boundary Conditions & $u(\x,t)=1,\quad \x\in\partial\mathcal{D}^-,\quad t>0,
   \quad\partial\mathcal{D}^-=\{\x\in\partial\mathcal{D}:\mathbf{n}(\x)
   \cdot\mathbf{\Omega}<0\}$\\
Direction & $\mathbf{\Omega} = \left[\frac{1}{\sqrt{2}},\frac{1}{\sqrt{3}},
   \frac{1}{\sqrt{6}}\right]$\\
Cross Section & $\sigma(\x)=\left\{\begin{array}{c l}
   10, & \x\in(\frac{1}{2},1)^d\\
   0,  & \mbox{otherwise}\end{array}\right.$\\
Source & $q(\x,t)=0$\\
Speed & $v=1$\\
Exact Solution & $u(\x,t)=\left\{\begin{array}{l l}
   \left\{\begin{array}{l l}
      e^{-10s}, & x\ge\frac{1}{2}, y\ge\frac{1}{2}, z\ge\frac{1}{2}\\
      1,        & \mbox{otherwise}
   \end{array}\right., & \x-\mathbf{\Omega}t\notin\mathcal{D}\\
   0, & \mbox{otherwise}
   \end{array}\right.$ \\
\bottomrule\end{tabular}
\end{table}

The condition $\x-\mathbf{\Omega}t\notin\mathcal{D}$ is equivalent to the
following condition:
\[
   \x-\mathbf{\Omega}t\notin\mathcal{D} \Rightarrow
   \exists i: x_i-\Omega_i t < 0,
\]
where $i$ denotes a coordinate direction index $x$, $y$, or $z$.
The distance travelled in the absorber region, $s$, is computed
by first determining which plane segment of the absorber region
through which the line $\x-\mathbf{\Omega}t$ passes; the coordinate
direction normal to this plane is denoted by $i$ and the other
two by $j$ and $k$. This is determined as follows:
\[
   i: \frac{x_i-\frac{1}{2}}{\Omega_i} = \min\limits_j\left(
      \frac{x_j-\frac{1}{2}}{\Omega_j}\right).
\]
Then, $s$ is computed as follows:
\[
   s=\sqrt{s_i^2 + s_j^2 + s_k^2}, \quad
   s_i=x_i-\frac{1}{2}, \quad
   s_j=\frac{\Omega_j}{\Omega_i}s_i, \quad
   s_k=\frac{\Omega_k}{\Omega_i}s_i.
\]
%================================================================================
\subsection{Three-Region}\label{sec:three_region}
This is a 1-D problem that consists of a domain with 3 regions of differing
saturation values $\frac{q}{\sigma}$. This is used to test both reaction
terms and source terms.
Table \ref{tab:three_region} summarizes the test parameters.

\begin{table}[h]\caption{Three-Region Test Problem Summary}
\label{tab:three_region}
\centering
\begin{tabular}{l l}\toprule
\emph{Parameter} & \emph{Value}\\\midrule
Domain & $\mathcal{D} = (0,1)$\\
Initial Conditions & $u_0(\x)=0$\\
Boundary Conditions & $u(\x,t)=1,\quad \x\in\partial\mathcal{D}^-,\quad t>0,
   \quad\partial\mathcal{D}^-=\{\x\in\partial\mathcal{D}:\mathbf{n}(\x)
   \cdot\mathbf{\Omega}<0\}$\\
Direction & $\mathbf{\Omega} = \mathbf{e}_x$\\
Cross Section & $\sigma(\x)=\left\{\begin{array}{c l}
   1,  & x \leq 0.3\\
   10, & 0.3 < x \leq 0.6\\
   5,  & x > 0.6
   \end{array}\right.$\\
Source & $q(\x,t)=\left\{\begin{array}{c l}
   1,  & x \leq 0.3\\
   5,  & 0.3 < x \leq 0.6\\
   10, & x > 0.6
   \end{array}\right.$\\
Speed & $v=1$\\
Exact Solution & $u(\x,t)=\left\{\begin{array}{l l}
   \left\{\begin{array}{l l}
      e^{-10(x-\frac{1}{2})}, & x\ge\frac{1}{2}, y\ge\frac{1}{2}, z\ge\frac{1}{2}\\
      1,                      & \mbox{otherwise}
   \end{array}\right., & x-t<0\\
   0, & \mbox{otherwise}
   \end{array}\right.$ \\
\bottomrule\end{tabular}
\end{table}
%================================================================================
