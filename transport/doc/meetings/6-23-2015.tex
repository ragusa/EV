\section*{6-23-2015}

The following items were requested:
\begin{enumerate}

\item \textbf{Determine why Kuzmin needs defect correction if his limiting
coefficients are explicit}.
While answering the question of whether Kuzmin's implicit discrete maximum
principle bounds were equivalent to ours, the question arose of why Kuzmin
would need defect correction when his $Q_i^\pm$ are explicitly computed,
since the resulting system is linear.

\item \textbf{Try a near-void instead of an actual void in the
source-void-to-absorber test problem}.
The physical solution for the source-void-to-absorber problem is linear
in the void, followed by exponential decay. It was considered
suspicious that the numerical solution in this region without 
artificial dissipation results in spurious oscillations, so
it was suggested to try using very small $\sigma$ in this
region instead of $\sigma=0$.

\item \textbf{Try a three-region problem}.
It is desirable to see the solution for a problem with three
different saturation values $q/\sigma$.

\item \textbf{Try a two-region problem with equal saturation values
but different values for $q$ and $\sigma$}.
If the saturation value of $q/\sigma$ is reached in the first
region, one would like to ensure that if the second region has
the same saturation value $q/\sigma$, but different values of
$q$ and $\sigma$, then there should be no numerical artifacts at
the interface between the two regions.

\end{enumerate}
