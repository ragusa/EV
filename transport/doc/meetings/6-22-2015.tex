\section*{6-22-2015}

\begin{enumerate}
%--------------------------------------------------------------------------------
\incompleteitem{Obtain multidimensional results for explicit FCT.}

In progress.
%--------------------------------------------------------------------------------
\incompleteitem{Compare symmetric limiter with upwind limiter.}

The symmetric limiting coefficients have the following definition:
\begin{equation}
  L\ij = \left\{\begin{array}{l l}
    \min\{R_i^+,R_j^-\}, & F\ij > 0 \\
    \min\{R_i^-,R_j^+\}, & F\ij < 0
  \end{array}\right.,
\end{equation}
while the upwind-biased limiting coefficients have the following definition:
\begin{equation}
  L\ij = \left\{\begin{array}{l l}
    R_k^+, & F\ij > 0 \\
    R_k^-, & F\ij < 0
  \end{array}\right.,
\end{equation}
where $k$ is the upwind node of $i$ and $j$. It is noted here that, unlike
the symmetric limiting coefficients, the upwind-biased limiting coefficients
do not necessarily satisfy the discrete maximum principle. This is disproved
here by a counterexample. Recall that the discrete maximum principle is satisfied when
$Q_i^-\le \sum_j L\ij F\ij\le Q_i^+$. Suppose a node $i$ has a single
neighboring node $i-1$, and that this neighbor is upwind of $i$. For example, this is
the case for a 1-D domain on the outflow boundary. Thus there is a single
correction flux $F_{i,i-1}$: $\sum_j L\ij F\ij=L_{i,i-1} F_{i,i-1}$. Now
suppose that without limitation, this correction flux would exceed the
upper bound $Q_i^+$, i.e., $F_{i,i-1}>Q_i^+$. Suppose further that
$P_{i-1}^+\leq Q_{i-1}^+$. Then,
\[
  \sum\limits_j L\ij F\ij=
  L_{i,i-1} F_{i,i-1}=
  R_{i-1}^+ F_{i,i-1}=
  \min\left\{1,\frac{Q_{i-1}^+}{P_{i-1}^+}\right\} F_{i,i-1}=
  F_{i,i-1}>
  Q_i^+,
\]
which violates the DMP condition. However, this limiting coefficient has
yet to be tested for its effectiveness in practice.
%--------------------------------------------------------------------------------
\incompleteitem{Determine if converged steady-state results match transient results.}

In progress.
%--------------------------------------------------------------------------------
\incompleteitem{Determine how to evaluate analytic DMP bounds in multidimensional problems.}

In progress.
%--------------------------------------------------------------------------------
\incompleteitem{Compare time step size computed from $\dt\leq\frac{m_i}{A^L_{i,i}}$ to
that obtained by examining the eigenvalues of $(\M^L)^{-1}\A^L$.}

The condition $\dt\leq\frac{m_i}{A^L_{i,i}}$ results from the need to prove
positivity of the low-order system matrix diagonal entries. The suggested approach
of using the eigenvalues of $(\M^L)^{-1}\A^L$ is unclear. For a general scalar conservation
law
\[
  u_t + \f(u)_x = 0,
\]
the equation may be expressed using chain rule:
\[
  u_t + \frac{d\f}{du}u_x = 0,
\]
where $\frac{d\f}{du}$ is the characteristic speed \cite{toro}. In the case of a
discrete FEM system
\[
  \M\U_t + \A(\U)\U_x = 0,
\]
the characteristic speeds would indeed be given by the eigenvalues of $\M^{-1}\A(\U)$.
However, this equation does not correspond to the system in question:
\[
  \M\U_t + \B(\U)\U = 0,
\]
and thus the suggestion is unclear. Expressing the system in question in the
quasilinear form with the spatial derivative, i.e., expressing $\B(\U)\U$
as $\A(\U)\U_x$, is complicated by the presence of a reaction term and an
artificial diffusion term. For instance, the governing equation would be
\[
  u_t + \mathbf{v}u_x + \sigma u = (\nu u_x)_x,
\]
which in conservative form gives
\[
  \f(u) = \mathbf{v}u + \int\sigma u dx - \nu u_x,
\]
and thus computing $\f'(u)$ is unclear.

An alternative approach to computing a valid time step size for the high-order
scheme is therefore suggested here, at least in regards to the presence of the
reaction term, not the viscosity term: the presence of a reaction term should only decrease
the flux speed, a conservative time step size could be computed by ignoring
the reaction term. Of course, when entropy viscosity is used, the flux speed
is increased and thus the same reasoning cannot be applied. In general, a CFL-satisfying
time step size can be computed by satisfying the following condition:
\[
  \dt^{n+1}\leq\frac{h_{min}}{s_{max}^n},
\]
where $h_{min}$ is the smallest element diameter in the mesh, and $s_{max}^n$ is the
largest flux speed computed in the domain. Calling $s^{void}$ the flux speed
computed when ignoring the reaction term, the following is true, as discussed
before: $s^{void}\geq s$. Thus the usage of this flux yields a conservative
time step estimate:
\[
  \dt^{n+1}\leq\frac{h_{min}}{s_{max}^{void,n}}\leq\frac{h_{min}}{s_{max}^{n}}.
\]
%--------------------------------------------------------------------------------
\incompleteitem{In the source-void-to-absorber problem, ramp the source from zero instead of
instantaneously adding the source.}

In progress.
%--------------------------------------------------------------------------------
\completeitem{Determine if Kuzmin's implicit bounds are equivalent to our implicit bounds.}

In \cite{kuzmin_FCT}, Kuzmin defines the limited flux bounds as
\[
  Q_i^\pm=m_i\substack{\max_j\\\min_j}\substack{\max\\\min}
    \left\{0,\tilde{U_j}-\tilde{U_i}\right\},
\]
where $\tilde{\U}$ is the ``auxiliary'' solution, given by Equations (75) and (69)
in \cite{kuzmin_FCT}, which together give, in our notation (and adding a source term),
\[
  \M^L\frac{\tilde{\U}-\U^n}{(1-\theta)\dt} + \A^L\U^n = \b^n.
\]
This auxiliary solution can thus be interpreted as the low-order solution at time
$t^{n+1-\theta}$, or alternatively may be viewed as the ``explicit portion'' of
a low-order time step; for explicit Euler, this is the entire time step and thus
$\tilde{\U}=\U^{L,n+1}$, while for implicit Euler, this includes none of the step
and thus $\tilde{\U}=\U^{n}$. Thus the question of whether Kuzmin's implicit bounds are
equivalent to ours is immediately answered: \emph{they are not equivalent} because
our implicit bounds are implicit with the new solution, while Kuzmin's are
explicit.

It should be noted that at least in \cite{kuzmin_FCT}, Kuzmin's objective
appears to be to satisfy the positivity constraint given by Equation (83)
in \cite{kuzmin_FCT}, unlike our objective, which is to satisfy a discrete
maximum principle, which happens to have a lower bound greater than or
equal to zero.
%--------------------------------------------------------------------------------
\incompleteitem{Evaluate the benefit of Kuzmin's ``prelimiting'' step.}

Equation (84) of \cite{kuzmin_FCT} gives Kuzmin's prelimiting step:
\[
  F\ij := 0 \qquad \mbox{if}\quad F\ij \left(\tilde{U}_i-\tilde{U}_j\right)\leq 0,
\]

The prelimiting step has not yet been re-implemented.
%--------------------------------------------------------------------------------
\incompleteitem{For the defect correction scheme, try damping viscosity updates.}

The implicit scheme is nonlinear and thus requires some iteration; the defect
correction technique was used. It was noted that difficulties were noted in
iteratively computing the entropy viscosity solution because entropy viscosity
was oscillating between two states on each iteration. Thus a damping coefficient
$\lambda$ was applied to solution updates:
\[
  \U^{H,n+1,(k+1)} = \U^{H,n+1,(k)} + \lambda\Delta\U^{H,n+1,(k+1)}.
\]
This was found to give some success but still showed difficulties for some
problems. It was thus suggested to try instead dampening only the entropy viscosity updates:
\[
  \nu^{E,n+1,(k+1)} = \nu^{E,n+1,(k)} + \lambda\Delta\nu^{E,n+1,(k+1)}.
\]
%--------------------------------------------------------------------------------
\end{enumerate}
