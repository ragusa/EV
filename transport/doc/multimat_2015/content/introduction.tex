There are a number of positivity-preserving
schemes developed for finite volume discretizations based on slope
limiters or flux limiters~\cite{toro99}\cite{leveque02}, but these
techniques have proved difficult to extend to finite element
discretizations. Therefore, obtaining high-order, positivity-preserving
solutions has long remained a challenge for finite element methods, but
recently the flux-corrected transport (FCT) algorithm,
a positivity-preserving scheme initially developed by Boris and Book\cite{borisbook},
has been applied to continuous finite element discretizations
for conservation laws of the following form~\cite{kuzmin_book}:

\[
   \frac{\partial u}{\partial t} + \nabla\cdot\mathbf{f}(u) = 0.
\]

Here we generalize flux-corrected-transport concepts to
include equations with a positive reaction term $\sigma$ and external
source term $q$:

\begin{equation}\label{eq:tr}
  \frac{\partial u}{\partial t} + \nabla\cdot(\mathbf{v}u(\mathbf{x},t))
  + \sigma(\mathbf{x})u(\mathbf{x},t) = q(\mathbf{x},t),
\end{equation}

where $\mathbf{v}$ is a constant velocity field, as found in
radiation transport equations. In addition, a recently developed
high-order scheme based on the entropy viscosity work performed
by Guermond~\cite{guermond_secondorder} is applied. The
concept of entropy viscosity is to add artificial diffusion
according to local entropy production, thus avoiding unnecessarily
adding diffusion where it is not needed and thus deteriorating 
the solution more than necessary.

