Testing with each FEM basis function $\varphi_i(\mathbf{x})$ gives a linear system:

\begin{equation}\label{eq:semidiscrete}
      \mathbf{M}^C\frac{d\mathbf{U}}{dt}+\mathbf{A} \mathbf{U}(t) = \mathbf{b},
\end{equation}

To discretize Eqn.~\ref{eq:semidiscrete} in time, fully explicit
temporal discretization schemes are used, such as explicit Euler:

\begin{equation}\label{eq:exgalerkin}
   \mathbf{M}^C\frac{\mathbf{U}^{n+1}-\mathbf{U}^n}{\Delta t}
     + \mathbf{A}\mathbf{U}^n = \mathbf{b}^n.
\end{equation}

In this section, a graph-theoretic approach introduced by Guermond~\cite{guermond_firstorder}
is taken to define a monotonicity-preserving, positivity-preserving low-order
scheme that satisfies a discrete maximum principle. Specifically, the low-order operator
is obtained by adding the the following dissipative term (written below for  the $i$-th equation):

\begin{equation}\label{eq:viscousform}
\sum_{K \subset S_i} \sum_{j\in \mathcal{I}(K)} U_j\nu_K^L b_K(\varphi_j,\varphi_i) .
\end{equation}

\noindent
The following local bilinear form has been employed to define this low-order scheme:

\begin{equation}\label{eq:bilinearform}
      b_K(\varphi_j, \varphi_i) \equiv \left\{\begin{array}{l l}
         -\frac{1}{n_K - 1}|K| & i\ne j, \quad i,j\in \mathcal{I}(K),\\
         |K|                   & i = j,  \quad i,j\in \mathcal{I}(K),\\
         0                     & i\notin\mathcal{I}(K) \quad | \quad j\notin\mathcal{I}(K),
      \end{array}\right.
\end{equation}

\noindent
where $K$ is the cell, $|K|$ is the volume of cell $K$, $\mathcal{I}(K)\equiv \{j\in\{1,\ldots,N\}: |S_j\cap K|\ne 0\}$
is the set of indices corresponding to degrees of freedom in
the support of cell $K$, and $n_K \equiv \mbox{card}(\mathcal{I}(K))$.
The local low-order viscosity is also defined:

\begin{equation}
   \nu_K^L \equiv \max\limits_{i\ne j\in \mathcal{I}(K)}\frac{\max(0,A_{i,j})}
      {-\sum\limits_{T\subset S_{i,j}} b_T(\varphi_j, \varphi_i)},
\end{equation}

\noindent
where $A_{i,j}$ is the $i,j$th entry of the Galerkin steady-state
matrix given by Eqn.~\ref{eq:Aij}.
Adding the viscous form of Eqn.~\ref{eq:viscousform} to the Galerkin scheme given in
Eqn.~\ref{eq:exgalerkin} gives the linear system for the low-order scheme

\begin{equation}\label{eq:loworderscheme}
   \mathbf{M}^L\frac{\mathbf{U}^{L,n+1}-\mathbf{U}^n}{\Delta t}
      +\mathbf{A}^L\mathbf{U}^n = \mathbf{b},
\end{equation}

\noindent
where $\mathbf{M}^L$ is the lumped mass matrix,
$\mathbf{U}^{L,n+1}$ is the low-order solution at time $t^{n+1}$, and
$\mathbf{A}^L = \mathbf{A} + \mathbf{D}^L$ is the low-order steady-state matrix,
which is the Galerkin steady-state matrix $\mathbf{A}$ plus a low-order
artificial diffusion matrix $\mathbf{D}^L$, defined as follows:

\begin{equation}\label{eq:loworderD}
   D^L_{i,j} = \sum\limits_{K\subset S_{i,j}}\nu_K^L b_K(\varphi_j,\varphi_i).
\end{equation}

\noindent
If the CFL condition $\Delta t \leq \frac{m_i}{A_{i,i}^L}$
is satisfied for all $i$, where the shorthand $m_i\equiv M^L_{i,i}$ applies, then the explicit
low-order scheme given in Eqn. \ref{eq:loworderscheme} satisfies the following
discrete maximum principle:

\begin{equation}\label{eq:dmp}
   U_{\min,i}^n\left(1-\frac{\Delta t}{m_i}
      \sum\limits_j A^L_{i,j}\right)
      + \frac{\Delta t}{m_i}b_i\leq
   U_i^{L,n+1}\leq
   U_{\max,i}^n\left(1-\frac{\Delta t}{m_i}
      \sum\limits_j A^L_{i,j}\right)
      + \frac{\Delta t}{m_i}b_i\quad\forall i,
\end{equation}

\noindent
where $U_{\min,i}^n = \min\limits_{j\in \mathcal{I}(S_i)}U_j^n$,
$U_{\max,i}^n = \max\limits_{j\in \mathcal{I}(S_i)}U_j^n$
and $\mathcal{I}(S_i)$ is the set of indices of degrees of freedom in the
support of degree of freedom $i$. The proof of this discrete maximum
principle is given in Appendix \ref{ap:dmp}.

%---------------------------------------------------------------------
\subsection{High-Order Scheme}\label{sec:highorder}

In this section, a high-order scheme based on the concept of entropy
viscosity introduced by Guermond~\cite{guermond_ev} is described, which by itself
is not guaranteed to be monotonicity-preserving or positivity-preserving;
however, when used as a component of the flux-corrected transport scheme given in
the following section, the resulting scheme is positivity-preserving
and satisfies a discrete maximum principle. Alternatively, one could use
the inviscid Galerkin scheme given in Eqn.~\ref{eq:exgalerkin} as
the high-order scheme component in the flux-corrected transport algorithm,
as has been done in the past for FEM-FCT~\cite{kuzmin_book}; however,
in practice it has been found that the FCT solution may still
contain small oscillations and effects that deteriorate the quality of the solution.

This scheme uses the high-order entropy viscosity scheme given by
Guermond~\cite{guermond_secondorder}.
One first decides upon a convex entropy functional $E(u)$ such as $E(u)=\frac{1}{2}u^2$.
The entropy viscosity is designed to add viscosity in regions of entropy
production, such as in shocks or steep gradients, and avoid adding
viscosity elsewhere. This is achieved by computing the entropy viscosity
with an entropy residual:

\begin{equation}
   \nu^{E,n}_K = \frac{c_E R_K^n(u_h^n,u_h^{n-1})
      + c_J\max\limits_{F\in\partial K}J_F(u_h^n)}
      {\|E(u_h^n)-\bar{E}(u_h^n)\|_{L^\infty(\mathcal{D})}},
\end{equation}

\noindent
where $R_K^n(u_h^n,u_h^{n-1})$ is the entropy residual, $J_F(u_h^n)$
is the jump in entropy flux across face $F$ of cell $K$, $\bar{E}(u_h^n)$ is the average
entropy over the domain, and $c_E$ and $c_J$ are tunable normalization
parameters, usually $\sim 1$.
The entropy residual evaluated with explicit Euler is the following:

\begin{equation}
    R_K^n(u_h^n,u_h^{n-1}) = \left\|\frac{E(u_h^n)-E(u_h^{n-1})}{\Delta t^n}
      + \left.\frac{dE}{du}\right|_{u_h^n}\left[\mathbf{\Omega}\cdot\nabla u_h^n
      + \sigma u_h^n
      - q \right]\right\|_{L^\infty(K)},
\end{equation}

\noindent
where the $L^\infty(K)$ norm is approximated as the maximum of the norm operand evaluated
at each quadrature point on $K$.
The entropy flux jumps are also computed on each face $F$ on the boundary of $K$:

\begin{equation}
   J_F(u_h^n) = \|\mathbf{\Omega}\cdot
      \mathbf{n}_F[\![\partial_n E(u_h^n)]\!]\|_{L^\infty(F)},
\end{equation}

\noindent
where $\mathbf{n}_F$ is the outward unit vector for face $F$,
$[\![\cdot]\!]$ denotes the jump in flux, and
the $L^\infty(F)$ norm is approximated as the maximum of the norm operand evaluated
at each quadrature point on $F$.

Finally, the high-order viscosity $\nu^{H,n}_K = \min(\nu^{L}_K,\nu^{E,n}_K)$ at
time $t^n$ is computed as the minimum
of the low-order viscosity $\nu^{L}_K$ defined in Section \ref{sec:loworder} and
the entropy viscosity $\nu^{E,n}_K$. The high-order counterpart of the low-order
artificial diffusion matrix defined in Eqn.~\ref{eq:loworderD} uses the high-order viscosity
$\nu_K^{H,n}$ instead of the low-order viscosity:

\begin{equation}
   D^{H,n}_{i,j} = \sum\limits_{K\subset S_{i,j}}\nu_K^{H,n} b_K(\varphi_j,\varphi_i).
\end{equation}

\noindent
Similarly to the low-order scheme, a high-order steady-state matrix
$\mathbf{A}^{H,n} = \mathbf{A} + \mathbf{D}^{H,n}$ is
defined to be the sum of the Galerkin steady-state matrix $\mathbf{A}$ and the
high-order artificial diffusion matrix $\mathbf{D}^{H,n}$,
and the high-order scheme is the following:

\begin{equation}\label{eq:highorderscheme}
   \mathbf{M}^C\frac{\mathbf{U}^{H,n+1}-\mathbf{U}^n}{\Delta t}
      +\mathbf{A}^{H,n}\mathbf{U}^n = \mathbf{b},
\end{equation}

\noindent
where $\mathbf{U}^{H,n+1}$ denotes the high-order solution at time $t^{n+1}$.

%---------------------------------------------------------------------
\subsection{Flux-Corrected Transport Scheme}\label{sec:fct}

This section blends the low-order and high-order schemes given
in Sections \ref{sec:loworder} and \ref{sec:highorder}, respectively, to produce a scheme
that is high-order, positivity-preserving, DMP-satisfying,
and in practice, free of spurious oscillations.

The crux of the flux-corrected transport algorithm is that an
antidiffusive flux $\mathbf{f}$ is defined that corrects the
low-order scheme to produce the high-order scheme solution:

\begin{equation}\label{eq:fct_full}
   \mathbf{M}^L\frac{\mathbf{U}^{H,n+1}-\mathbf{U}^n}{\Delta t}
      + \mathbf{A}^L\mathbf{U}^n
      = \mathbf{b} + \mathbf{f}.
\end{equation}

\noindent
Combining Eqns. \ref{eq:fct_full} and \ref{eq:highorderscheme}
yields the definition of $\mathbf{f}$:

\begin{equation}\label{gtf}
   \mathbf{f} \equiv -(\mathbf{M}^C-\mathbf{M}^L)
      \frac{\mathbf{U}^{H,n+1}-\mathbf{U}^n}{\Delta t}
      +(\mathbf{D}^L-\mathbf{D}^H)\mathbf{U}^n.
\end{equation}
It is easy to note that with the definition of $\mathbf{f}$ given in Eqn.~\ref{gtf},
Eqn.~\ref{eq:fct_full} yields the high-order solution. 

\noindent
However, the antidiffusive flux is added in a limited fashion
such that physically-motivated
solution bounds are not violated. In this paper, these bounds
are chosen to be the bounds of the low-order scheme discrete
maximum principle, given by Eqn.~\ref{eq:dmp}. With the
low-order and high-order schemes given in this paper, it
is possible to decompose the antidiffusive flux going into
node $i$, $f_i$, into a combination of internodal antidiffusive
fluxes $F_{i,j}$ such that $f_i = \sum\limits_j F_{i,j}$.
Since $\mathbf{M}^C-\mathbf{M}^L$ and $\mathbf{D}^L-\mathbf{D}^H$ are symmetric
and feature zero row and column sums, a valid decomposition for $\mathbf{f}$ is

\begin{equation}
   F_{i,j} = -M_{i,j}^C(\frac{U^{H,n+1}_j-U^n_j}{\Delta t} - \frac{U^{H,n+1}_i-U^n_i}{\Delta t})
   + (D_{i,j}^L-D_{i,j}^H)(U^n_j - U^n_i).
\end{equation}

\noindent
Finally, the FCT scheme is the following, where the operator
$\mathcal{L}$ denotes the limiter operation:

\begin{equation}\label{eq:FCTscheme}
   \mathbf{M}^L\frac{\mathbf{U}^{n+1}-\mathbf{U}^n}{\Delta t}
      + \mathbf{A}^L\mathbf{U}^n
      = \mathbf{b} + \mathcal{L}[\mathbf{F}],
\end{equation}

\noindent
where $(\mathcal{L}[\mathbf{F}])_i = \sum\limits_j \alpha_{i,j}F_{i,j}$
and $\mathbf{U}^{n+1}$ is the FCT solution at time $t^{n+1}$. The
limiting coefficients $\alpha_{i,j}$ are given by the multidimensional
limiter of Zalesak~\cite{zalesak}:

\begin{equation}\label{eq:P_defs}
   P_i^+ \equiv \sum\limits_j\max(0,F_{i,j}) \qquad
   P_i^- \equiv \sum\limits_j\min(0,F_{i,j}),
\end{equation}
\begin{equation}\label{eq:Q_defs}
      Q_i^\pm \equiv m_i\frac{W_i^\pm-U_i^n}{\Delta t}
      + \sum\limits_j A_{i,j}^L U_j^n - b_i,
\end{equation}
\begin{equation}\label{eq:R_defs}
   R_i^\pm \equiv\left\{
      \begin{array}{l l}
         1                                          & P_i^\pm = 0\\
         \min\left(1,\frac{Q_i^\pm}{P_i^\pm}\right) & P_i^\pm \ne 0
      \end{array}
      \right.,
\end{equation}
\begin{equation}\label{eq:L_defs}
   \alpha_{i,j} \equiv\left\{
      \begin{array}{l l}
         \min(R_i^+,R_j^-) & F_{i,j} \geq 0\\
         \min(R_i^-,R_j^+) & F_{i,j} < 0
      \end{array}
      \right.,
\end{equation}

\noindent
where $W_i^\pm$ are the upper and lower discrete maximum principle bounds
given in Eqn.~\ref{eq:dmp}. The proof that this definition of limiting
coefficients $\alpha_{i,j}$
satisfies the discrete maximum principle is given in Appendix \ref{ap:fct_dmp}.
Note that the symmetry of the limited coefficients $\alpha_{i,j}=\alpha_{j,i}$ and
antisymmetric correction fluxes $F_{i,j}=-F_{j,i}$ make the FCT scheme conservative, since
the FCT scheme is merely the low-order scheme plus some equal-and-opposite source
terms.
