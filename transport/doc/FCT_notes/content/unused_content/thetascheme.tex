%================================================================================
\section{Transient Radiation Transport Using the $\theta$-Scheme}
%================================================================================
%--------------------------------------------------------------------------------
\subsection{Deriving a Local Discrete Maximum Principle}
%--------------------------------------------------------------------------------
Here we seek to find a local discrete maximum principle for the transient problem, given by
Equation \ref{tr}.
Using the $\theta$-scheme for time stepping, the following equation relates the new and old
time step solutions for the low-order scheme, with solution $\mathbf{U}^L$:
\begin{equation}
   m_i\frac{U_i^L-U_i^n}{\Delta t} = -\sum\limits_j A^L_{i,j}
    \left(\theta U_j^L + (1-\theta) U_j^n\right) + b_i,
\end{equation}
or
\begin{equation}\label{theta_newstep}
   \left(1+\frac{\theta\Delta t}{m_i}A_{i,i}^L\right)U_i^L
    = \left(1-\frac{(1-\theta)\Delta t}{m_i}A_{i,i}^L\right)U_i^n
    -\frac{\Delta t}{m_i}\sum\limits_{j\ne i}A_{i,j}^L
     \left(\theta U_j^L + (1-\theta) U_j^n\right)
    +\frac{\Delta t}{m_i}b_i,
\end{equation}
where $\Delta t$ is the time step size, $m_i$ is the $i$th element of the \emph{lumped} mass
matrix, i.e., $M^L_{i,i}$, $\mathbf{A}^L$ is the steady-state system matrix, and
$\mathbf{b}$ is the right-hand side.

\begin{theorem}
If the CFL-like condition
\begin{equation}\label{theta_CFL}
   \Delta t \leq \frac{m_i}{(1-\theta)A_{i,i}^L}\quad\forall i
   \qquad\Longleftrightarrow\qquad
   1 - \frac{(1-\theta)\Delta t}{m_i}A_{i,i}^L \geq 0\quad\forall i
\end{equation}
is satisfied, then
the low-order solution at the new time step, $\mathbf{U}^L$, satisfies the following maximum principle:
\begin{equation}\label{theta_max_principle}
   W_i^- \le U_i^L \le W_i^+\quad\forall i,
\end{equation}
where
\begin{equation}\label{theta_W}
   W_i^\pm \equiv \frac{1}{1+\frac{\theta\Delta t}{m_i}A_{i,i}^L}
     \left[\left(1 - \frac{(1-\theta)
     \Delta t}{m_i}\sum\limits_j A_{i,j}^L\right)U_{\substack{\max\\\min},i}^n
    -\frac{\theta\Delta t}{m_i}\sum\limits_{j\ne i}A_{i,j}^L
     U_{\substack{\max\\\min},i}^L
    +\frac{\Delta t}{m_i}b_i\right],
\end{equation}
where $U_{\substack{\max\\\min},i}^n = \substack{\max\\\min\limits_{j\in \mathcal{I}(S_i)}}U_j^n$,
$U_{\substack{\max\\\min},i}^L = \substack{\max\\\min\limits_{j\in \mathcal{I}(S_i)}}U_j^L$,
and $\mathcal{I}(S_i)$ is the set of indices of degrees of freedom in the
support of degree of freedom $i$.
\end{theorem}
\begin{proof}
Rearranging Equation \ref{theta_newstep},
\[
   U_i^L = \frac{1}{1+\frac{\theta\Delta t}{m_i}A_{i,i}^L}
     \left[\left(1 - \frac{(1-\theta)\Delta t}{m_i}A_{i,i}^L\right)U_i^n
    -\frac{\Delta t}{m_i}\sum\limits_{j\ne i}A_{i,j}^L
     \left(\theta U_j^L
    +(1-\theta) U_j^n\right)
    +\frac{\Delta t}{m_i}b_i\right].
\]
By Lemma \ref{offdiagonalnegative}, it is known that the off-diagonal
elements $A^L_{i,j}, j\ne i$, are non-positive, and by Equation \ref{theta_CFL}
the term $1 - \frac{(1-\theta)\Delta t}{m_i}A_{i,i}^L$ is non-negative.
Thus, the following inequality is able to be applied:
\[
   U_i^L \le
     \frac{1}{1+\frac{\theta\Delta t}{m_i}A_{i,i}^L}
     \left[\left(1 - \frac{(1-\theta)
     \Delta t}{m_i}\sum\limits_j A_{i,j}^L\right)U_{\max,i}^n
    -\frac{\theta\Delta t}{m_i}\sum\limits_{j\ne i}A_{i,j}^L
     U_{\max,i}^L
    +\frac{\Delta t}{m_i}b_i\right],
\]
and similarly for the lower bound.\qed
\end{proof}