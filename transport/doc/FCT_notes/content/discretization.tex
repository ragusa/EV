\section{Discretization}
%================================================================================
\subsection{Spatial Discretization}\label{galerkindef}
%================================================================================
The continuous Galerkin finite element method is used for spatial discretization.
The numerical solution is thus
\begin{equation}
   \psi(\mathbf{x},t) = \sum\limits_{j=1}^N U_j(t) \varphi_j(\mathbf{x}),
\end{equation}
where $N$ is the number of degrees of freedom in the finite element expansion,
$U_j(t)$ are nodal solution values at time $t$, and $\varphi_j(\mathbf{x})$
are linear basis functions. Multiplying Equation \eqref{tr} by $c$ and testing
with basis function $\varphi_i(\mathbf{x})$ gives
\begin{equation}
   \int\limits_{S_i}\frac{\partial \psi}{\partial t}\varphi_i(\mathbf{x}) d\mathbf{x}
      + c\int\limits_{S_i}\left(\mathbf{\Omega}\cdot\nabla\psi(\mathbf{x},t)
      + \sigma(\mathbf{x})\psi(\mathbf{x},t)\right)\varphi_i(\mathbf{x}) d\mathbf{x}
      = c\int\limits_{S_i} q(\mathbf{x},t) \varphi_i(\mathbf{x}) d\mathbf{x},
\end{equation}
where $S_i$ is the support of $\varphi_i(\mathbf{x})$.
This yields a linear system
\begin{equation}\label{semidiscrete}
   \mathbf{M}^C\frac{d\mathbf{U}}{dt}+\mathbf{A} \mathbf{U}(t) = \mathbf{b},
\end{equation}
where $\mathbf{A}$ is the steady-state system matrix and $\mathbf{U}$ is the
Galerkin solution. The elements of $\mathbf{A}$ are
\begin{equation}\label{Aij}
	A_{i,j} \equiv c\int\limits_{S_{i,j}}\left(
      \mathbf{\Omega}\cdot\nabla\varphi_j(\mathbf{x}) +
		\sigma(\mathbf{x})\varphi_j(\mathbf{x})\right)\varphi_i(\mathbf{x}) d\mathbf{x},
\end{equation}
where $S_{i,j}$ is the dual support of $\varphi_i(\mathbf{x})$ and $\varphi_j(\mathbf{x})$.
The elements of $\mathbf{b}$ are
\begin{equation}
	b_i \equiv c\int\limits_{S_i} q(\mathbf{x},t)\varphi_i(\mathbf{x}) d\mathbf{x}.
\end{equation}
$\mathbf{M}^C$ is the consistent mass matrix, which has the entries
\begin{equation}\label{massmatrix}
	m_{i,j} \equiv \int\limits_{S_{i,j}}
   \varphi_j(\mathbf{x})\varphi_i(\mathbf{x}) d\mathbf{x}.
\end{equation}
%================================================================================
\subsection{Temporal Discretization}
%================================================================================
\subsubsection{Explicit Euler}
%================================================================================
Considering a time step from time $t^n$ to time $t^{n+1}$ with time step size
$\Delta t\equiv t^{n+1}-t^n$, the semi-discrete equation given by Equation
\eqref{semidiscrete} is discretized using the explicit Euler:
\begin{equation}\label{exgalerkin}
   \mathbf{M}^C\frac{\mathbf{U}^{n+1}-\mathbf{U}^n}{\Delta t}
     + \mathbf{A}\mathbf{U}^n = \mathbf{b},
\end{equation}
where $\mathbf{U}^{n+1}$ is the Galerkin solution at time $t^{n+1}$.
%================================================================================
\subsubsection{Strong Stability-Preserving Runge Kutta Methods}\label{ssprk}
%================================================================================
Strong Stability-Preserving Runge Kutta (SSPRK) methods are a class of multistage
Runge Kutta methods that offer high-order accuracy while preserving stability.
The subset of SSPRK methods considered in this research conform to the
following form for a given time step $t^n\rightarrow t^{n+1}$,
where $\hat{\mathbf{U}}^i$ denotes the $i$th stage solution:
\begin{equation}
   \hat{\mathbf{U}}^i = \alpha_i\hat{\mathbf{U}}^0
   + \beta_i\,\mbox{step}(\hat{\mathbf{U}}^{i-1},t^n+c_i\Delta t),
\end{equation}
where $\Delta t=t^{n+1}-t^n$, $\hat{\mathbf{U}}^0=\mathbf{U}^n$,
and $\mbox{step}(\mathbf{U},t)$
is a function that returns the solution $\mathbf{U}^{new}$ to the linear system
\begin{equation}
   \mathbf{M}\mathbf{U}^{new} = \mathbf{M}\mathbf{U}
   + \Delta t(\mathbf{b}(t) - \mathbf{A}(t)\mathbf{U}),
\end{equation}
where $\mathbf{M}$ is the mass matrix, $\mathbf{b}(t)$ is the steady-state
right hand side vector, and $\mathbf{A}(t)$ is the steady-state matrix.
Therefore the function performs a time step that is similar to an
explicit Euler step, except that the steady-state residual is not
necessarily evaluated at time $t^n$. The new solution $\mathbf{U}^{n+1}$
is the final stage solution:
\begin{equation}
   \mathbf{U}^{n+1} = \hat{\mathbf{U}}^s,
\end{equation}
where $s$ is the number of stages for the scheme.

The Explicit Euler scheme can be expressed as a 1-stage SSPRK method with
with the following coefficients:
\begin{equation}
   \alpha_1 = 0\qquad\beta_1 = 1\qquad c_1 = 0,
\end{equation}
and the 3-stage, 3rd-order accurate SSPRK scheme (also known as the Shu-Osher
scheme), has the following coefficients:
\begin{equation}
   \alpha = \left[\begin{array}{c}0\\\frac{3}{4}\\\frac{1}{3}\end{array}\right]
   \qquad\beta = \left[\begin{array}{c}1\\\frac{1}{4}\\\frac{2}{3}\end{array}\right]
   \qquad c = \left[\begin{array}{c}0\\1\\\frac{1}{2}\end{array}\right].
\end{equation}

%================================================================================