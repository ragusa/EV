\section{Nonlinear Iteration}
%================================================================================
\subsection{Steady-State High-Order Scheme}
%--------------------------------------------------------------------------------
The high-order steady-state scheme
\begin{equation}
   \A^H\U = \b
\end{equation}
is in general nonlinear since the high-order steady-state matrix is
$\A^H = \A + \D^H$, where $\D^H$ may be a function of the solution, as
it is in the case of the entropy viscosity high-order scheme.
This section presents a defect correction scheme to solve this
nonlinear system iteratively.

If the system is modified to strongly impose Dirichlet
boundary conditions, let it be denoted by
\begin{equation}
   \tilde{\A}^H\U = \tilde{\b} \eqc
\end{equation}
where the tildes denote modification to the vector and matrix.
Each new iterate of the solution is computed as
\begin{equation}
   \U^{(k+1)} = \U^{(k)} + \alpha\Delta\U^{(k+1)} \eqc
\end{equation}
where
\begin{equation}
   \Delta\U^{(k+1)} = [\tilde{\A}^{H,(k)}]^{-1}\r^{(k)} \eqc
\end{equation}
and the steady-state residual as is defined as
\begin{equation}\label{ssres_ss}
   \r^{(k)} = \tilde{\b} - \tilde{\A}^{H,(k)}\U^{(k)} \eqc
\end{equation}
and $\alpha$ is a relaxation parameter. Typically, $\alpha\le 1$,
and if $\alpha=1$, then this scheme is equivalent to fixed point iteration.

%--------------------------------------------------------------------------------
\subsection{Theta High-Order Scheme}
%--------------------------------------------------------------------------------
The high order theta scheme is the following:
\begin{equation}
   \M^C\frac{\U^{n+1}-\U^n}{\dt^{n+1}} + (1-\theta)\A^{H,n}\U^n
   + \theta\A^{H,n+1}\U^{n+1} = (1-\theta)\b^n + \theta\b^{n+1} \eqp
\end{equation}
Rearranging gives
\begin{equation}\label{thetahigh_unmod}
   \B\U^{n+1} = \M^C\U^n
   + (1-\theta)\dt^{n+1}\r^n + \theta\dt^{n+1}\b^{n+1} \eqc
\end{equation}
where $\B\equiv \M^C+\theta\dt^{n+1}\A^{H,n+1}$, and the steady-state
residual is defined as
\begin{equation}
   \r^n = \b^n - \A^{H,n}\U^n \eqp
\end{equation}
Note that this differs from the definition given by Equation \eqref{ssres_ss},
where the residual uses the strongly Dirichlet-modified system matrix and right
hand side. Denoting the right hand side of Equation \eqref{thetahigh_unmod} as
\begin{equation}
   \s = \M^C\U^n + (1-\theta)\dt^{n+1}\r^n + \theta\dt^{n+1}\b^{n+1}
\end{equation}
and modifying for Dirichlet boundary conditions gives
\begin{equation}
   \tilde{\B}^C\U^{n+1} = \tilde{\s}.
\end{equation}
This system is nonlinear, so defect correction is used to iteratively
compute the solution:
\begin{equation}
   \U^{n+1,(k+1)} = \U^{n+1,(k)} + \alpha\Delta\U^{n+1,(k+1)} \eqc
\end{equation}
where the change in solution is computed as
\begin{equation}
   \Delta\U^{n+1,(k+1)} = [\tilde{B}^{(s)}]^{-1}\tilde{\s} - \U^{n+1,(k)} \eqc
\end{equation}
and again, $\alpha$ is a relaxation parameter.

%--------------------------------------------------------------------------------
\subsection{Steady-State FCT Scheme}
%--------------------------------------------------------------------------------
The steady-state FCT scheme, given by Equation \eqref{FCT_limited_ss}, is
\begin{equation}
   \A^L\U = \b + \LF \equiv \b^F \eqp
\end{equation}
The first step is to perform a nonlinear solve to get the high-order
solution $\U^H$, since the flux corrections $\F$ depend on the high-order
solution. The FCT step is also nonlinear since the limiting coefficients
are nonlinear, so a defect correction scheme is used, which will
now be described.
Strongly imposing Dirichlet boundary conditions,
\begin{equation}
   \tilde{\A}^L\U = \tilde{\b}^F \eqp
\end{equation}
Defining the steady-state residual as
\begin{equation}
   \r^{(k)} = \tilde{\b}^{F,(k)} - \tilde{\A}^{L}\U^{(k)} \eqc
\end{equation}
updates to the solution are computed as
\begin{equation}
   \Delta\U^{(k+1)} = [\tilde{\A}^{L,(k)}]^{-1}\r^{(k)} \eqp
\end{equation}
The update is performed with a relaxation parameter $\alpha$:
\begin{equation}
   \U^{(k+1)} = \U^{(k)} + \alpha\Delta\U^{(k+1)} \eqp
\end{equation}

%--------------------------------------------------------------------------------
\subsection{Theta FCT Scheme}
%--------------------------------------------------------------------------------
Discretizing Equation \eqref{semidiscrete_limited} using the theta scheme
given in Section \ref{theta} gives
\begin{equation}
   \M^L\frac{\U^{n+1}-\U^n}{\dt^{n+1}} + (1-\theta)\A^L\U^{n}
   + \theta\A^L\U^{n+1} = (1-\theta)\b^n + \theta\b^{n+1}
   + \LF \eqp
\end{equation}
Rearranging gives
\begin{equation}\label{thetaFCT_unmod}
   \B\U^{n+1} = \M^L\U^n + (1-\theta)\dt^{n+1}\r^n + \theta\dt^{n+1}\b^{n+1}
   + \dt^{n+1}\LF \eqc
\end{equation}
where $\B\equiv \M^L + \theta\dt^{n+1}\A^L$, and the steady-state residual
is defined as
\begin{equation}
   \r^n = \b^n - \A^L\U^n \eqp
\end{equation}
Note that this differs from the definition given by Equation \eqref{ssres_ss},
where the residual uses the strongly Dirichlet-modified system matrix and right
hand side. Denoting the right hand side of Equation \eqref{thetaFCT_unmod} as
\begin{equation}
   \s = \M^L\U^n + (1-\theta)\dt^{n+1}\r^n + \theta\dt^{n+1}\b^{n+1} + \dt^{n+1}\LF
\end{equation}
and modifying for Dirichlet boundary conditions gives
\begin{equation}
   \tilde{\B}\U^{n+1} = \tilde{\s}.
\end{equation}
This system is nonlinear, so defect correction is used to iteratively
compute the solution:
\begin{equation}
   \U^{n+1,(k+1)} = \U^{n+1,(k)} + \alpha\Delta\U^{n+1,(k+1)} \eqc
\end{equation}
where the change in solution is computed as
\begin{equation}
   \Delta\U^{n+1,(k+1)} = [\tilde{\B}]^{-1}\tilde{\s}^{(k)} - \U^{n+1,(k)} \eqc
\end{equation}
and again, $\alpha$ is a relaxation parameter.


