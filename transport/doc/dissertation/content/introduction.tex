%================================================================================
\section{Introduction}
%================================================================================
The radiation transport equation is the following:
\begin{equation}\label{tr}
	\frac{1}{c}\frac{\partial \psi}{\partial t} + \mathbf{\Omega}\cdot\nabla\psi(\mathbf{x},t)
      + \sigma(\mathbf{x})\psi(\mathbf{x},t) = q(\mathbf{x},t),
\end{equation}
where $\psi(\mathbf{x},t)$ is the angular flux at position $\mathbf{x}$ and time
$t$ in direction $\mathbf{\Omega}$, $c$ is the transport speed, $\sigma(\mathbf{x})$
is the \emph{macroscopic} cross-section, and $q(\mathbf{x},t)$ is the
total source (extraneous plus scattering).
The problem definition is completed with an incoming flux boundary condition:
\begin{equation}
   \psi(\mathbf{x}) = \psi^{inc}(\mathbf{x})  \quad \forall \mathbf{x}\in \partial V^-,
      \quad \partial V^- = \{\mathbf{x}\in\partial V: \mathbf{\Omega}\cdot\mathbf{n}(\mathbf{x})<0\}.
\end{equation}
For transient problems, the following initial condition applies:
\begin{equation}
   \psi(\mathbf{x},t) = \psi^0(\mathbf{x})  \quad \forall \mathbf{x}\in V.
\end{equation}