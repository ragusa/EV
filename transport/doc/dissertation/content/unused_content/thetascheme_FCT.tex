%================================================================================
\subsection{FCT}
%================================================================================
%================================================================================
\subsubsection{The $\theta$-Scheme System}
%================================================================================
Without strongly imposing Dirichlet boundary conditions, the high-order system is
\begin{equation}\label{theta_high}
   \mathbf{A}^{tr,H}\mathbf{U}^H
      = (\mathbf{M}_C-(1-\theta)\Delta t\mathbf{A}^H)\mathbf{U}^n
      + \Delta t\mathbf{b} \equiv \mathbf{b}^H,
\end{equation}
where $\mathbf{A}^{tr,H} \equiv \mathbf{M}_C+\theta\Delta t\mathbf{A}^H$.
Strongly imposing Dirichlet boundary conditions on Equation \ref{theta_high}
gives
\begin{equation}\label{theta_highD}
   \tilde{\mathbf{A}}^{tr,H}\mathbf{U}^H
      = \tilde{\mathbf{b}}^H.
\end{equation}
The low-order system corresponding to Equation \ref{theta_high} is
\begin{equation}\label{theta_low}
   \mathbf{A}^{tr,L}\mathbf{U}^L
      = (\mathbf{M}_L-(1-\theta)\Delta t\mathbf{A}^L)\mathbf{U}^n
      + \Delta t\mathbf{b} \equiv \mathbf{b}^L,
\end{equation}
where $\mathbf{A}^{tr,L} \equiv \mathbf{M}_L+\theta\Delta t\mathbf{A}^L$.
Note that the solution to this low-order system
is not computed in this Dr. Kuzmin's flux correction algorithm given in Section
\ref{theta_kuzmin}; instead, this low-order system is just used
to define the correction step.
%--------------------------------------------------------------------------------
\subsubsection{Flux Correction Algorithm}\label{theta_kuzmin}
%--------------------------------------------------------------------------------
The flux correction vector $\mathbf{f}$ is defined such that:
\begin{equation}\label{theta_kuzminfc}
   \mathbf{A}^{tr,L}\mathbf{U}^H
      = (\mathbf{M}_L-(1-\theta)\Delta t\mathbf{A}^L)\mathbf{U}^n
      + \Delta t\mathbf{b} + \mathbf{f} \equiv \mathbf{b}^F.
\end{equation}
Subtracting Equation \ref{theta_high} from \ref{theta_kuzminfc} gives the definition of
$\mathbf{f}$:
\begin{equation}\label{theta_kuzminFdef}
   \mathbf{f} \equiv -(\mathbf{M}_C-\mathbf{M}_L)\Delta\mathbf{U}^H
      -\Delta t\mathbf{D}\left(\theta\mathbf{U}^H
      +(1-\theta)\mathbf{U}^n\right),
\end{equation}
where $\Delta\mathbf{U}^H = \mathbf{U}^H - \mathbf{U}^n$. Since
$\mathbf{M}_C-\mathbf{M}_L$ and $\mathbf{D}$ are symmetric
and feature zero row and column sums, a valid decomposition for $\mathbf{f}$,
called $\mathbf{F}$, is
\begin{equation}
   F_{i,j} = -m_{i,j}(\Delta U^H_j - \Delta U^H_i)
    - \Delta t D_{i,j}\left(\theta(U^H_j - U^H_i) + (1-\theta)(U^n_j - U^n_i)\right)
\end{equation}
where $m_{i,j}$ is the $i,j$th element of the consistent mass matrix.
Applying a limiter to Equation \ref{theta_kuzminfc} gives
\begin{equation}\label{theta_limited}
   \mathbf{A}^{tr,L}\mathbf{U}^{n+1}
      = (\mathbf{M}_L-(1-\theta)\Delta t\mathbf{A}^L)\mathbf{U}^n + \Delta t\mathbf{b} + \mathcal{L}[\mathbf{F}],
\end{equation}
where $(\mathcal{L}[\mathbf{F}])_i = \mathcal{L}_{i,:}F_{i,:}^T
= \sum\limits_j \mathcal{L}_{i,j}F_{i,j}$ and $\mathbf{U}^{n+1}$ is the FCT solution.

\begin{lemma}
   Suppose that with strongly imposed Dirichlet boundary conditions, the
   correction step is defined such that the Dirichlet boundary conditions
   are strongly imposed on Equation \ref{theta_kuzminfc}:
   \begin{equation}\label{theta_kuzminfcD}
      \tilde{\mathbf{A}}^{tr,L}\hat{\mathbf{U}}^H
         = \tilde{\mathbf{b}}^F,
   \end{equation}
   and the definition for
   the flux correction vector $\mathbf{f}$ from
   Equation \ref{theta_kuzminFdef} is used without modification. Then the solution
   to this correction step, $\hat{\mathbf{U}}^H$, will yield the
   correct high-order solution $\mathbf{U}^H$ given by Equation \ref{theta_highD}.
\end{lemma}
\begin{proof}
   For $i\in\mathcal{D}$, $\hat{U}^H_i = U^H_i = g_i$. For $i\notin\mathcal{D}$,
   we take the $i$th equation of the linear system given by Equation \ref{theta_kuzminfcD}:
   \begin{equation}\label{i_eq}
      (\mathbf{A}^{tr,L}\hat{\mathbf{U}}^H)_i
         = m_i U^n_i - (1-\theta)\Delta t(\mathbf{A}^L\mathbf{U}^n)_i
         + \Delta t b_i - (\mathbf{M}_C\Delta\mathbf{U}^H)_i
         + m_i(U^H_i-U^n_i) - \theta\Delta t(\mathbf{D}\mathbf{U}^H)_i
         - (1-\theta)\Delta t(\mathbf{D}\mathbf{U}^n)_i.
   \end{equation}
   Examining the $i$th equation of the linear system given by Equation \ref{theta_highD}
   and solving for $\Delta t b_i$ gives
   \begin{equation}
      \Delta t b_i = (\mathbf{M}_C\Delta\mathbf{U}^H)_i
         + \theta\Delta t(\mathbf{A}^H\mathbf{U}^H)_i
         + (1-\theta)\Delta t(\mathbf{A}^H\mathbf{U}^n)_i.
   \end{equation}
   Substituting this back into Equation \ref{i_eq} gives
   \begin{eqnarray}
      (\mathbf{A}^{tr,L}\hat{\mathbf{U}}^H)_i
         & = & \theta\Delta t(\mathbf{A}^H\mathbf{U}^H)_i
         + m_i U^H_i - \theta\Delta t(\mathbf{D}\mathbf{U}^H)_i\\
         & = & \left((\mathbf{M}_L+\theta\Delta t\mathbf{A}^L)\mathbf{U}^H\right)_i\\
         & = &(\mathbf{A}^{tr,L}\mathbf{U}^H)_i.
   \end{eqnarray}
   The system of equations
   \begin{equation}
      \left\{
         \begin{array}{l l}
            \hat{U}^H_i = U^H_i & i\in\mathcal{D}\\
            (\mathbf{A}^{tr,L}\hat{\mathbf{U}}^H)_i = (\mathbf{A}^{tr,L}\mathbf{U}^H)_i
               & i\notin\mathcal{D}
         \end{array}
      \right.
   \end{equation}
   determines that $\hat{\mathbf{U}}^H$ is uniquely equal to $\mathbf{U}^H$.
   \qed
\end{proof}
%================================================================================
\subsubsection{Deriving the Limiting Coefficients}
%================================================================================
\begin{lemma}
   The definitions given by Equations \ref{P_defs}, \ref{R_defs}, \ref{L_defs},
   and \ref{theta_W},
   and the following definition yield maximum-principle preserving limiting coefficients:
   \begin{equation}
      Q_i^\pm \equiv (m_i+\theta\Delta t A_{i,i}^L)W_i^\pm
       - (m_i-(1-\theta)\Delta t A_{i,i}^L)U_i^n
       + \Delta t\sum\limits_{j\ne i} A_{i,j}^L (\theta U_j^{n+1}+(1-\theta)U_j^n)
       - \Delta t b_i.
   \end{equation}
\end{lemma}

\begin{proof}
   The properties given in the proof for Lemma \ref{ss_coef} for $P_i^\pm$,
   $Q_i^\pm$, $R_i^\pm$, and $\mathcal{L}_{i,j}$ hold with these definitions
   of $W_i^\pm$ and $Q_i^\pm$.
   The proof will be given for the upper bound. 
   As in the proof of Lemma \ref{ss_coef}, the following inequality is found:
   \[
      \sum\limits_j \mathcal{L}_{i,j}F_{i,j} \leq Q_i^+
   \]
   Taking row $i$ of the linear system given by Equation \ref{theta_limited} gives
   \begin{multline*}
     (m_i+\theta\Delta t A_{i,i}^L)U_i^{n+1}
       - (m_i-(1-\theta)\Delta t A_{i,i}^L)U_i^n
       + \Delta t\sum\limits_{j\ne i} A_{i,j}^L (\theta U_j^{n+1}+(1-\theta)U_j^n)
       - \Delta t b_i = \sum\limits_j \mathcal{L}_{i,j}F_{i,j}\\
     \leq Q_i^+ = (m_i+\theta\Delta t A_{i,i}^L)W_i^+
       - (m_i-(1-\theta)\Delta t A_{i,i}^L)U_i^n
       + \Delta t\sum\limits_{j\ne i} A_{i,j}^L (\theta U_j^{n+1}+(1-\theta)U_j^n)
       - \Delta t b_i\\
      U_i^{n+1} \leq W_i^+ 
   \end{multline*}
   The lower bound is proved similarly.
   \qed
\end{proof}