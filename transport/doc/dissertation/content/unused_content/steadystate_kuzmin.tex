%================================================================================
\section{Steady-State Radiation Transport}
%================================================================================
%--------------------------------------------------------------------------------
\subsection{Deriving a Maximum Principle}
%--------------------------------------------------------------------------------
Applying the Galerkin finite element method to Equation \ref{ss} and adding a
viscous bilinear form gives a linear system $\mathbf{A}^H \mathbf{U} = \mathbf{b}$.
The superscript $H$ denotes the high-order matrix. 
The $i$-th equation is the following:
\begin{equation}
	\sum\limits_{K\subset S_i}\int\limits_K\left(\mathbf{\Omega}\cdot\nabla\psi
      + \sigma_t\psi\right)\varphi_i d\mathbf{x}
      = \sum\limits_{K\subset S_i}\int\limits_K q \varphi_i d\mathbf{x} = b_i.
\end{equation}

\begin{lemma}
The coefficients of the resulting linear system are given by the following equation:
\begin{equation}\label{Aij}
	A^H_{i,j} = \int\limits_{S_{ij}}\left(\mathbf{\Omega}\cdot\nabla\varphi_j +
		\sigma_t\varphi_j\right)\varphi_i d\mathbf{x}.
\end{equation}
\end{lemma}

\begin{proof}
Expanding the solution gives
\[
	\sum\limits_{K\subset S_i}\int\limits_K\left(\mathbf{\Omega}\cdot
      \nabla\sum\limits_{j\in \mathcal{I}(K)}U_j\varphi_j +
		\sigma_t\sum\limits_{j\in \mathcal{I}(K)}U_j\varphi_j\right)\varphi_i d\mathbf{x},
\]
and rearranging gives
\[
	\sum\limits_{j\in \mathcal{I}(S_{i})}U_j\sum\limits_{K\subset S_{ij}}
      \int\limits_K\left(\mathbf{\Omega}\cdot\nabla\varphi_j +
		\sigma_t\varphi_j\right)\varphi_i d\mathbf{x} = b_i
\]
Thus the elements of the linear system matrix are:
\begin{eqnarray*}
	A^H_{i,j} & = & \sum\limits_{K\subset S_{ij}}\int\limits_K
      \left(\mathbf{\Omega}\cdot\nabla\varphi_j +
		\sigma_t\varphi_j\right)\varphi_i d\mathbf{x}\\
  & = & \int\limits_{S_{ij}}\left(\mathbf{\Omega}\cdot\nabla\varphi_j +
		\sigma_t\varphi_j\right)\varphi_i d\mathbf{x}.\qed
\end{eqnarray*}
\end{proof}

The high-order matrix $\mathbf{A}^H$ is split into an advection matrix $\mathbf{K}^H$ and a reaction
matrix $\mathbf{R}^H$:
\begin{equation}
   \mathbf{A}^H = \mathbf{K}^H + \mathbf{R}^H.
\end{equation}
\begin{definition}
   The low-order advection matrix $\mathbf{K}^H$ is defined as
   \begin{equation}
      \mathbf{K}^L \equiv \mathbf{K}^H - \mathbf{D},
   \end{equation}
   where $\mathbf{D}$ is an artificial diffusion operator:
   \begin{equation}
      D_{i,j} = \max(0,K_{i,j}^H,K_{j,i}^H)\quad j\ne i,\qquad D_{i,i} = -\sum\limits_{j\ne i} D_{i,j}.
   \end{equation}
\end{definition}
\begin{definition}
   The low-order reaction matrix $\mathbf{R}^L$ is formed by lumping the consistent, high-order
   reaction matrix $\mathbf{R}^H$:
   \begin{equation}
      R_{i,j}^L = \left\{\begin{array}{c c}
         \sum\limits_k R_{i,k}^H & j = i\\
         0 & j\ne i
      \end{array}\right.
   \end{equation}
\end{definition}
The low-order system matrix is
\begin{equation}
   \mathbf{A}^L = \mathbf{K}^L + \mathbf{R}^L.
\end{equation}

\newpage
\begin{lemma}\label{offdiagonalnegative}
   The off-diagonal elements of the low-order system matrix are non-positive:
   $A^L_{i,j}\le 0, j\ne i$.
\end{lemma}
\begin{proof}
The off-diagonal elements of $\mathbf{R}^L$ are all zero, so the off-diagonal elements of the
low-order system matrix are
\begin{eqnarray*}
	A^L_{i,j} & = & K_{i,j}^L\\
             & = & K_{i,j}^H - D_{i,j}\\
             & = & K_{i,j}^H - \max(0,K_{i,j}^H,K_{j,i}^H)
\end{eqnarray*}
The non-positivity of the off-diagonal coefficients is proven by examining the three possible
outputs of the $\max()$ function:

\begin{tabular}{l l}
   case $D_{i,j}=0$:         & $K^L_{i,j} = K^H_{i,j} \leq 0$ because $D_{i,j}=0\geq K^H_{i,j}$.\\
   case $D_{i,j}=K^H_{i,j}$: & $K^L_{i,j} = 0$.\\
   case $D_{i,j}=K^H_{j,i}$: & $K^L_{i,j} = K^H_{i,j} - K^H_{j,i} \leq 0$ because $K^H_{j,i} \geq K^H_{i,j}$.\qed
\end{tabular}
\end{proof}

\begin{lemma}\label{diagonalpositive}
   The diagonal elements  of the linear system matrix are non-negative: $A^L_{i,i}\ge 0$.
\end{lemma}
\begin{proof}
The diagonal elements  of the low-order system matrix are
\[
	A^L_{i,i} = \int\limits_{S_{i}}\left(\nabla\cdot\frac{\mathbf{\Omega}\varphi_i^2}{2} +
		\sigma_t\varphi_i^2\right) d\mathbf{x} - D_{i,i}.
\]
To prove that $A^L_{i,i}$ is non-negative, it is sufficient to prove that
each term in the above expression is non-negative. The non-negativity of
the interaction term and viscous term are obvious ($\sigma_t \ge 0$, $-D_{i,i}\geq 0$), but
the non-negativity of the divergence term is not necessarily obvious. On the interior of
the domain, the divergence term gives zero contribution because the divergence integral may
be transformed into a surface integral $\int\limits_{\partial S_{i}}
\mathbf{\Omega}\cdot\mathbf{n}\frac{\varphi_i^2}{2} d\mathbf{x}$
via the divergence theorem; one can then recognize that
the basis function $\varphi_i$ evaluates to zero on the boundary of its support $S_{i}$.\\
On the outflow boundary of the domain, the term $\mathbf{\Omega}\cdot\mathbf{n}
\frac{\varphi_i^2}{2}$ is positive because $\mathbf{\Omega}\cdot\mathbf{n} >0$
for an outflow boundary. This quantity is of course negative for the inflow boundary,
but a Dirichlet boundary condition is strongly imposed on the incoming boundary, so
for degrees of freedom $i$ on the incoming boundary, $A^L_{i,i}$ will be set equal
to some positive value such as 1 with a corresponding incoming value
accounted for in the right hand side $\mathbf{b}$ of the linear system.\qed
\end{proof}

\newpage
\begin{lemma}
   The sum of all elements in a row $i$ is non-negative: $\sum\limits_j A^L_{i,j} \ge 0$.
\end{lemma}

\begin{proof}
Using the fact that $\sum\limits_j\varphi_j=1$,
\begin{eqnarray*}
	\sum\limits_j A^L_{i,j} & = & \sum\limits_j\int\limits_{S_{ij}}
      \left(\mathbf{\Omega}\cdot\nabla\varphi_j +
		\sigma_t\varphi_j\right)\varphi_i d\mathbf{x} +
		D_{i,i} + \sum\limits_{j\neq i}D_{i,j}\\
		& = & \int\limits_{S_{i}}\left(\mathbf{\Omega}\cdot\nabla\sum\limits_j\varphi_j +
		\sigma_t\sum\limits_j\varphi_j\right)\varphi_i d\mathbf{x}\\
		& = & \int\limits_{S_{i}}\sigma_t\varphi_i d\mathbf{x}\\
		&\ge& 0.\qed
\end{eqnarray*}
\end{proof}

\begin{lemma}\label{diagonallydominant}
   $\mathbf{A}^L$ is strictly diagonally dominant:
   $\left|A^L_{i,i}\right| \ge \sum\limits_{j\ne i} \left|A^L_{i,j}\right|$.
\end{lemma}
\begin{proof}
Using the inequalities $\sum\limits_j A^L_{i,j} \ge 0$ and $A^L_{i,j}\le 0, j\ne i$,
it is proven that $\mathbf{A}^L$ is strictly diagonally dominant:
\begin{eqnarray*}
	\sum\limits_j A^L_{i,j} & \ge & 0\\
	\sum\limits_{j\ne i} A^L_{i,j} + A^L_{i,i} & \ge & 0\\
	\left|A^L_{i,i}\right| & \ge & \sum\limits_{j\ne i} -A^L_{i,j}\\
	\left|A^L_{i,i}\right| & \ge & \sum\limits_{j\ne i} \left|A^L_{i,j}\right|.\qed
\end{eqnarray*}
\end{proof}

\begin{lemma}
   $\mathbf{A}^L$ is an M-Matrix.
\end{lemma}
\begin{proof}
To prove that a matrix is an M-Matrix, it is sufficient to prove that:
\[
\left\{\begin{array}{l}
A^L_{i,j}\le 0, j\ne i\\
A^L_{i,i}\ge 0\\
\left|A^L_{i,i}\right| \ge \sum\limits_{j\ne i} \left|A^L_{i,j}\right|\\
\end{array}
\right.,
\]
which are given by Lemmas \ref{offdiagonalnegative}, \ref{diagonalpositive}, and
\ref{diagonallydominant}, respectively.\qed
\end{proof}

\begin{theorem}
The low-order solution $\mathbf{U}^L$ satisfies the following maximum principle:
\begin{equation}\label{ss_max_principle}
   W_i^-\leq U_i^L\leq W_i^+\quad\forall i,
\end{equation}
where
\begin{equation}
   W_i^\pm \equiv -\frac{1}{A^L_{i,i}}\sum\limits_{j\ne i} A^L_{i,j}
      U_{\substack{\max\\\min},i}^L + \frac{b_i}{A^L_{i,i}},
\end{equation}
where $U_{\min,i}^L = \min\limits_{j\in \mathcal{I}(S_i)}U_j^L$, $U_{\max,i}^L
= \max\limits_{j\in \mathcal{I}(S_i)}U_j^L$,
and $\mathcal{I}(S_i)$ is the set of indices of degrees of freedom in the support
of degree of freedom $i$.
\end{theorem}
\begin{proof}
\begin{eqnarray*}
	\sum\limits_j A^L_{i,j}U_j^L & = & b_i\\
	A^L_{i,i}U_i^L & = & \sum\limits_{j\ne i} -A^L_{i,j}U_j^L + b_i\\
	A^L_{i,i}U_i^L & \le & \left(\sum\limits_{j\ne i} -A^L_{i,j}\right)U_{\max,i}^L + b_i\\
   U_i^L & \le & -\frac{1}{A^L_{i,i}}\sum\limits_{j\ne i} A^L_{i,j}U_{\max,i}^L
      + \frac{b_i}{A^L_{i,i}}
\end{eqnarray*}
A similar analysis is performed to prove the lower bound for $U_i^L$.\qed
\end{proof}