\section{Discretization}
%================================================================================
\subsection{Spatial Discretization}\label{galerkindef}
%================================================================================
The continuous Galerkin finite element method is used for spatial discretization.
The numerical solution is thus
\begin{equation}
   \psi(\x,t) = \sum\limits_{j=1}^N U_j(t) \varphi_j(\x),
\end{equation}
where $N$ is the number of degrees of freedom in the finite element expansion,
$U_j(t)$ are nodal solution values at time $t$, and $\varphi_j(\x)$
are linear basis functions. Multiplying Equation \eqref{tr} by $c$ and testing
with basis function $\varphi_i(\x)$ gives
\begin{equation}
   \int\limits_{S_i}\frac{\partial \psi}{\partial t}\varphi_i(\x) d\x
      + c\int\limits_{S_i}\left(\mathbf{\Omega}\cdot\nabla\psi(\x,t)
      + \sigma(\x)\psi(\x,t)\right)\varphi_i(\x) d\x
      = c\int\limits_{S_i} q(\x,t) \varphi_i(\x) d\x,
\end{equation}
where $S_i$ is the support of $\varphi_i(\x)$.
This yields a linear system
\begin{equation}\label{semidiscrete}
   \M^C\frac{d\U}{dt}+\A \U(t) = \b(t),
\end{equation}
where $\A$ is the steady-state system matrix and $\U$ is the
Galerkin solution. The elements of $\A$ are
\begin{equation}\label{Aij}
	A_{i,j} \equiv c\int\limits_{S_{i,j}}\left(
      \mathbf{\Omega}\cdot\nabla\varphi_j(\x) +
		\sigma(\x)\varphi_j(\x)\right)\varphi_i(\x) d\x,
\end{equation}
where $S_{i,j}$ is the dual support of $\varphi_i(\x)$ and $\varphi_j(\x)$.
The elements of $\b(t)$ are
\begin{equation}
	b_i(t) \equiv c\int\limits_{S_i} q(\x,t)\varphi_i(\x) d\x.
\end{equation}
$\M^C$ is the consistent mass matrix, which has the entries
\begin{equation}\label{massmatrix}
	m_{i,j} \equiv \int\limits_{S_{i,j}}
   \varphi_j(\x)\varphi_i(\x) d\x.
\end{equation}
Similarly, for the steady-state case, the linear system is
\begin{equation}
  \A\U = \b.
\end{equation}
%================================================================================
\subsection{Temporal Discretization}
%================================================================================
\subsubsection{Explicit Euler}
%================================================================================
Considering a time step from time $t^n$ to time $t^{n+1}$ with time step size
$\Delta t\equiv t^{n+1}-t^n$, the semi-discrete equation given by Equation
\eqref{semidiscrete} is discretized using the explicit Euler:
\begin{equation}\label{galerkin_FE}
   \M^C\frac{\U^{n+1}-\U^n}{\dt} + \A\U^n = \b^n,
\end{equation}
where $\U^{n+1}$ is the Galerkin solution at time $t^{n+1}$.
%================================================================================
\subsubsection{Strong Stability-Preserving Explicit Runge Kutta Methods}
\label{ssprk}
%================================================================================
Strong Stability-Preserving Runge Kutta (SSPRK) methods are a class of multistage
Runge Kutta methods that offer high-order accuracy while preserving stability.
The subset of SSPRK methods considered in this research conform to the
following form for a given time step $t^n\rightarrow t^{n+1}$,
where $\hat{\U}^i$ denotes the $i$th stage solution:
\begin{equation}
   \hat{\U}^i = \alpha_i\hat{\U}^0
   + \beta_i\,\mbox{step}(\hat{\U}^{i-1},t^n+c_i\Delta t),
\end{equation}
where $\Delta t=t^{n+1}-t^n$, $\hat{\U}^0=\U^n$,
and $\mbox{step}(\U,t)$
is a function that returns the solution $\U^{new}$ to the linear system
\begin{equation}
   \M\U^{new} = \M\U
   + \Delta t(\b(t) - \A(t)\U),
\end{equation}
where $\M$ is the mass matrix, $\b(t)$ is the steady-state
right hand side vector, and $\A(t)$ is the steady-state matrix.
Therefore the function performs a time step that is similar to an
explicit Euler step, except that the steady-state residual is not
necessarily evaluated at time $t^n$. The new solution $\U^{n+1}$
is the final stage solution:
\begin{equation}
   \U^{n+1} = \hat{\U}^s,
\end{equation}
where $s$ is the number of stages for the scheme.

The Explicit Euler scheme can be expressed as a 1-stage SSPRK method with
with the following coefficients:
\begin{equation}
   \alpha_1 = 0\qquad\beta_1 = 1\qquad c_1 = 0,
\end{equation}
and the 3-stage, 3rd-order accurate SSPRK scheme (also known as the Shu-Osher
scheme), has the following coefficients:
\begin{equation}
   \alpha = \left[\begin{array}{c}0\\\frac{3}{4}\\\frac{1}{3}\end{array}\right]
   \qquad\beta = \left[\begin{array}{c}1\\\frac{1}{4}\\\frac{2}{3}\end{array}\right]
   \qquad c = \left[\begin{array}{c}0\\1\\\frac{1}{2}\end{array}\right].
\end{equation}
%================================================================================
\subsubsection{Theta Scheme}\label{theta}
%================================================================================
Discretizing Equation \eqref{semidiscrete} with the $\theta$ scheme gives
\begin{equation}\label{galerkin_theta}
  \M^C\frac{\U^{n+1}-\U^n}{\dt}
  + (1-\theta)\A\U^n + \theta\A\U^{n+1}
  = (1-\theta)\b^n + \theta\b^{n+1}.
\end{equation}
Note that $\theta=0$ corresponds to the explicit Euler method,
$\theta=1$ corresponds to the implicit Euler method, and $\theta=\frac{1}{2}$
corresponds to the Crank-Nicolson method.

%================================================================================
