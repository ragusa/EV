%--------------------------------------------------------------------------------
\section{Problem Formulation}
%--------------------------------------------------------------------------------
The main focus of this research is on radiation transport;
however, most of the analysis performed is valid
for any general conservation law of the following form:
\begin{equation}\label{eq:cons_law}
   \pd{u}{t} + \nabla\cdot\g(u) + \sigma\xt u\xt = q\xt \eqc
\end{equation}
where $u\xt$ is a general scalar conserved quantity at position $\x$
and time $t$, $\g(u)$ is a general flux function,
$\sigma\xt$ is a reaction term, and $q\xt$ is a source
term. This notation will be used throughout this document to keep
the analysis as general as possible; radiation transport notation
will only be adopted when assumptions are needed. This is often the case when
the assumption of a constant, linear flux function $\g(u)$ is needed.
In this case, the constant velocity field is denoted by $\v$: $\g(u)=\v u$.

For radiation transport, the conservation law equation is the following:
\begin{equation}\label{eq:rad_transport}
  \frac{1}{v}\ppt{\psi} + \O\cdot\nabla\psi(\x,t)
  + \Sigma(\x)\psi(\x,t) = Q(\x,t) \eqc
\end{equation}
where $\psi(\x,t)$ is the angular flux in direction $\O$,
$v$ is the transport speed, $\Sigma(\x)$
is the macroscopic cross-section, and $Q(\x,t)$ is the
total source (extraneous plus scattering).

To complete the problem formulation, one must provide boundary
conditions and, for transient problems, intitial conditions:
\begin{equation}
   u(\x,t) = u^0(\x)  \quad \forall \x\in\dom \eqp
\end{equation}
Boundary conditions will depend on the chosen conservation law and
the particular problem. 
For radiation transport, and any conservation law for which the velocity field
is constant, a well-posed problem can be completed with an incoming flux
boundary condition:
\begin{equation}
   u\xt = u^{inc}\xt  \quad \forall \x\in \partial\dom^-,
     \quad \partial\dom^- = \{\x\in\partial\dom:
     \v\cdot\mathbf{n}(\x)<0\} \eqp
\end{equation}
